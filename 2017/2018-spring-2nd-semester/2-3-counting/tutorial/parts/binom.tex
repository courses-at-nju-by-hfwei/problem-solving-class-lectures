% file: parts/binom.tex

%%%%%%%%%%%%%%%
\begin{frame}{}
  \begin{exampleblock}{Computing $\binom{n}{k}$ (CS $1.5:14$)}
    % file: algs/binom-recursive.tex

\begin{algorithm}[H]
  \caption{Computing $\binom{n}{k}$.}
  \begin{algorithmic}[1]
    \Procedure{Binom}{$n, k$} \Comment{Required: $n \ge k \ge 0$}
      \If{$k = 0 \lor n = k$}
	\State \Return 1
      \EndIf

      \State \Return $\Call{Binom}{n-1, k} + \Call{Binom}{n-1, k-1}$
    \EndProcedure
  \end{algorithmic}
\end{algorithm}

  \end{exampleblock}

  \pause
  \fignocaption{width = 0.60\textwidth}{figs/binom-4-2}
\end{frame}
%%%%%%%%%%%%%%%

%%%%%%%%%%%%%%%
\begin{frame}{}
  % file: algs/binom-recursive.tex

\begin{algorithm}[H]
  \caption{Computing $\binom{n}{k}$.}
  \begin{algorithmic}[1]
    \Procedure{Binom}{$n, k$} \Comment{Required: $n \ge k \ge 0$}
      \If{$k = 0 \lor n = k$}
	\State \Return 1
      \EndIf

      \State \Return $\Call{Binom}{n-1, k} + \Call{Binom}{n-1, k-1}$
    \EndProcedure
  \end{algorithmic}
\end{algorithm}


  \begin{enumerate}[(i)]
    \pause
    \item \teal{\# of ``+'':}
      \pause
      \[
	A(n,k) = 1 + A(n-1, k) + A(n-1, k-1)
      \]
    \pause
    \vspace{-0.60cm}
    \item \teal{\# of recursive calls of \textsc{Binom}:}
      \pause
      \[
	R(n,k) = 2 + R(n-1, k) + R(n-1, k-1)
      \]
  \end{enumerate}

  \pause
  \[
    \boxed{T(n,k) = \left\{\begin{array}{lr}
      \uncover<7->{\red{0}, & k = 0 \lor n = k} \\
      T(n-1, k) + T(n-1, k-1) + c, & \uncover<7->{\text{\it o.w.}}
    \end{array}\right.}
  \]
\end{frame}
%%%%%%%%%%%%%%%

%%%%%%%%%%%%%%%
\begin{frame}{}
  \[
    \boxed{T(n,k) = \left\{\begin{array}{lr}
      0, & k = 0 \lor n = k \\
      T(n-1, k) + T(n-1, k-1) + c, & \text{\it o.w.}
    \end{array}\right.}
  \]

  \pause
  \[
    T(n,k) = T(n-1, k) + T(n-1, k-1) \implies T(n,k) = \pause \red{\alpha} \binom{n}{k}
  \]

  \pause
  \vspace{-0.50cm}
  \[
    \boxed{\teal{T(n,k) = \alpha \binom{n}{k} + \beta}}
  \]

  \pause
  \[
    \alpha \binom{n}{k} + \beta = \alpha \binom{n-1}{k} + \beta + \alpha \binom{n-1}{k-1} + \beta + c \implies \beta = -c
  \]

  \pause
  \[
    \alpha \binom{n}{0} - c = 0, \quad \alpha \binom{n}{n} - c = 0 \implies \alpha = c
  \]
  % \[
  %   \left.\begin{array}{lr}
  %     \alpha \binom{n}{0} - c = 0 \\[7pt]
  %   \alpha \binom{n}{n} - c = 0
  %   \end{array}\right\} \implies \alpha = c
  % \]

  \pause
  \[
    \boxed{\red{T(n,k) = c \binom{n}{k} - c}}
  \]
\end{frame}
%%%%%%%%%%%%%%%

%%%%%%%%%%%%%%%
\begin{frame}{}
  \fignocaption{width = 0.60\textwidth}{figs/pascal}

  \pause
  \centerline{\large \red{$Q:$} How to calculate $\binom{5}{3}$?}
\end{frame}
%%%%%%%%%%%%%%%

%%%%%%%%%%%%%%%
\begin{frame}{}
  % file: algs/dp/binom-dp.tex

\begin{algorithm}[H]
  % \caption{Computing $\binom{n}{k}$.}
  \begin{algorithmic}[1]
    \Procedure{Binom}{$n, k$} \Comment{Required: $n \ge k \ge 0$}
      \For{$i \gets 0 \;\text{\bf to}\; n$}
	\State $B[i][0] \gets 1$
	\State $B[i][i] \gets 1$
      \EndFor

      \For{$i \gets 2 \;\text{\bf to}\; n$}
	\For{$j \gets 1 \;\text{\bf to}\; k$}
	  \State $B[n][k] \gets B[n-1][k] + B[n-1][k-1]$
	\EndFor
      \EndFor
      \State \Return $B[n][k]$
    \EndProcedure
  \end{algorithmic}
\end{algorithm}


  \pause
  \[
    (n-k+1) + (k) + k (n-k) = nk - k^2 + n + 1
  \]
\end{frame}
%%%%%%%%%%%%%%%
