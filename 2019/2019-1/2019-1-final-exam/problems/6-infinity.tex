% 6-infinity.tex

%%%%%%%%%%%%%%%
\begin{frame}{}
  \begin{exampleblock}{Problem 6: Infinity}
    请证明: $\mathbb{N}$ 的所有\red{有限}子集所组成的集合是可数的。
  \end{exampleblock}

  \pause
  \[
    X \triangleq \Big\{A \subseteq \mathbb{N} \mid \exists n \in \mathbb{N}: |A| = n\Big\}
  \]

  \pause
  \[
    \red{\boxed{f: X \xrightarrow{1-1} \mathbb{N}}}
    \pause \qquad f: A \mapsto \red{?}
  \]

  \pause
  \vspace{-0.30cm}
  \[
    A = \set{0, 2, 1, 7} \qquad \pause a = \purple{111}0000\purple{1}00000\cdots
  \]

  \pause
  \vspace{-0.30cm}
  \[
    a = a_0 a_1 \cdots \qquad \pause
    a_{k} = \begin{cases}
      1, & k \in A \\
      0, & k \notin A
    \end{cases}
  \]

  \pause
  \vspace{-0.30cm}
  \[
    \blue{\boxed{f(A) = \sum_{k=0}^{\red{\infty}} a_{k} 2^{k} \in \mathbb{N}}}
  \]
\end{frame}
%%%%%%%%%%%%%%%

%%%%%%%%%%%%%%%
\begin{frame}{}
  \begin{exampleblock}{Problem 6: Infinity}
    $\mathbb{N}$ 的\red{所有子集}所组成的集合是\red{不可数}的。
  \end{exampleblock}

  \pause
  \[
    |2^{\mathbb{N}}| = 2^{\aleph_{0}} = \mathfrak{c} = |\mathbb{R}|
  \]
  
  \pause
  \fig{width = 0.40\textwidth}{figs/cantor-monumento}

  \pause
  \vspace{-0.30cm}
  \[
    \blue{\boxed{f(A) = \sum_{k=0}^{\red{\infty}} a_{k} 2^{k}}} \text{ may} \notin \mathbb{N}
  \]
\end{frame}
%%%%%%%%%%%%%%%
