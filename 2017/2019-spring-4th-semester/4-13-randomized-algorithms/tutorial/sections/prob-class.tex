% prob-class.tex

%%%%%%%%%%%%%%%%%%%%
\begin{frame}
  \begin{definition}[$ZPP$: Zero-error Probabilistic Polynomial Time]
	\[
	  L \in ZPP
	\]
	\[
	  \iff
	\]
	\[
	  \exists A \text{ (\it probabilistic polynomial-time algorithm)}: 
	\]
	\[
	  Pr\Big(A(x) = L(x)\Big) \ge \frac{1}{2}
	\]
	\[
	  Prob\Big(A(x) = ?\Big) = 1 - Pr\Big(A(x) = L(x)\Big) \le \frac{1}{2}
	\]
  \end{definition}

  \pause
  \begin{center}
	\red{\large $Q:$ Why $1/2$?} \pause
	\[
	  \blue{ZPP_{\delta}: ZPP_{1/3} = ZPP_{1/2} = ZPP_{2/3}}
	\]
  \end{center}
\end{frame}
%%%%%%%%%%%%%%%%%%%%

%%%%%%%%%%%%%%%%%%%%
\begin{frame}
  \[
	\blue{L \in ZPP_{\delta}}
  \]

  \pause
  \[
	A^{(k)}: \text{Repeat } A\; k \text{ times independently}
  \]

  \pause
  \begin{center}
	\purple{Output the non-``?'' value if any; Otherwise, output ``?''}
  \end{center}

  \pause
  \[
	\violet{L \in ZPP_{\alpha} \text{ for some } \alpha}
  \]

  \pause
  \[
	Pr\Big(A^{(k)}(x) = L(x)\Big) = 1 - Pr\Big(A^{(k)}(x) = ?\Big) \ge 1 - (1 - \delta)^{k}
  \]

  \pause
  \[
	\red{L \in ZPP_{1 - (1-\delta)^{k}}}
  \]
\end{frame}
%%%%%%%%%%%%%%%%%%%%

%%%%%%%%%%%%%%%%%%%%
\begin{frame}
  \begin{definition}[$RP$: Randomized Polynomial time (One-Sided Error)]
	\[
	  L \in RP
	\]
	\[
	  \iff
	\]
	\[
	  \exists A \text{ (\it probabilistic polynomial-time algorithm)}: 
	\]
	\[
	  x \in L \implies Pr\Big(A(x) = 1\Big) \ge \frac{1}{2}
	\]
	\[
	  x \notin L \implies Pr\Big(A(x) = 0\Big) = 1
	\]
  \end{definition}

  \begin{center}
	\red{\large $Q:$ Why $1/2$?} \pause
	\[
	  \blue{RP_{\delta}: RP_{1/3} = RP_{1/2} = RP_{2/3}}
	\]
  \end{center}
\end{frame}
%%%%%%%%%%%%%%%%%%%%

%%%%%%%%%%%%%%%%%%%%
\begin{frame}
  \[
	\blue{L \in RP_{\delta}}
  \]

  \pause
  \[
	A^{(k)}: \text{Repeat } A\; k \text{ times independently}
  \]

  \pause
  \begin{center}
	\purple{Accept $x$ iff any of the $k$ runs accepts}
  \end{center}

  \pause
  \[
	\violet{L \in RP_{\alpha} \text{ for some } \alpha}
  \]

  \pause
  \[
	Pr\Big(x \in L \land A^{(k)}(x) = 1\Big) = 1 - Pr\Big(x \in L \land A^{(k)}(x) = 0\Big) \ge 1- (1-\delta)^{k}
  \]

  \pause
  \[
	\red{L \in RP_{1-(1-\delta)^{k}}}
  \]
\end{frame}
%%%%%%%%%%%%%%%%%%%%

%%%%%%%%%%%%%%%%%%%%
\begin{frame}
  \begin{definition}[$BPP$: Bounded-error Probabilistic Polynomial time (Two-Sided Error)]
	\[
	  L \in BPP
	\]
	\[
	  \iff
	\]
	\[
	  \exists A \text{ (\it probabilistic polynomial-time algorithm)}: 
	\]
	\[
	  \exists \epsilon, 0 < \epsilon \le 1/2: Pr\Big(A(x) = L(x)\Big) \ge \frac{1}{2} + \epsilon
	\]
  \end{definition}

  \pause
  \begin{center}
	\red{\large $Q:$ Why $1/2$?} \\[8pt] \pause
	\red{\large $Q:$ Why $\epsilon$?}
  \end{center}
\end{frame}
%%%%%%%%%%%%%%%%%%%%

%%%%%%%%%%%%%%%%%%%%
\begin{frame}
  \[
	\blue{L \in BPP_{p \triangleq (\frac{1}{2} + \delta)}}
  \]

  \pause
  \[
	A^{(k)}: \text{Repeat } A\; k \text{ times independently}
  \]

  \pause
  \begin{center}
	\purple{Output the ``majority'' ($\# \ge \lceil k/2 \rceil$) value}
  \end{center}

  \pause
  \[
	\violet{L \in BPP_{\alpha} \text{ for some } \alpha}
  \]

  \pause
  \[
	Pr\Big(A^{(k)}(x) = L(x)\Big) \ge 1 - \sum_{i=0}^{\lfloor k/2 \rfloor} \binom{t}{i} p^i (1-p)^{k-i} > 1 - \frac{1}{2}(1-4\delta^2)^{k/2}
  \]

  \pause
  \[
	\red{L \in BPP_{1 - \epsilon}} \implies k \ge \frac{2 \ln 2\epsilon}{\ln (1-4\delta^2)}
  \]
\end{frame}
%%%%%%%%%%%%%%%%%%%%

%%%%%%%%%%%%%%%%%%%%
\begin{frame}
  \begin{definition}[$BPP$: Bounded-error Probabilistic Polynomial time (Two-Sided Error)]
	\[
	  L \in BPP
	\]
	\[
	  \iff
	\]
	\[
	  \exists A \text{ (\it probabilistic polynomial-time algorithm)}: 
	\]
	\[
	  \exists \epsilon, 0 < \epsilon \le 1/2: Pr\Big(A(x) = L(x)\Big) \ge \frac{1}{2} + \epsilon
	\]
  \end{definition}

  \begin{center}
	\red{\large $Q:$ Why $\epsilon$?}
  \end{center}

  \pause
  \[
	\blue{Q: \text{What about } Pr\Big(A(x) = L(x)\Big) > \frac{1}{2}?}
  \]

  \pause
  \vspace{-0.50cm}
  \[
	\blue{Q: \text{What about } Pr\Big(A(x) = L(x)\Big) \ge \frac{1}{2} + n^{-c} \text{ for some constant } c?}
  \]
\end{frame}
%%%%%%%%%%%%%%%%%%%%

%%%%%%%%%%%%%%%%%%%%
\begin{frame}
  \[
	Pr\Big(A(x) = L(x)\Big) \ge \frac{1}{2} + n^{-c} \text{ for some constant } c
  \]

  \pause
  \[
	\blue{L \in BPP_{p \triangleq (\frac{1}{2} + n^{-c})}}
  \]

  \pause
  \[
	A^{(k)}: \text{Repeat } A\; k \text{ times independently}
  \]

  \pause
  \begin{center}
	\purple{Output the ``majority'' ($\# \ge \lceil k/2 \rceil$) of $x_1, x_2, \ldots, x_k$}
  \end{center}

  \pause
  \[
	\violet{L \in BPP_{\alpha} \text{ for some } \alpha}
  \]
\end{frame}
%%%%%%%%%%%%%%%%%%%%

%%%%%%%%%%%%%%%%%%%%
\begin{frame}
  \centerline{Indicator random variables} 
  \[
	X_i = \begin{cases}
	  1, & x_i = L(x) \\
	  0, & \text{otherwise}
	\end{cases}
  \]

  \pause
  \[
	\purple{X = \sum_{i=1}^{k} X_i}
  \]

  \pause
  \[
	Pr\Big(X \ge \frac{1}{2} k \Big) \ge \cdots
  \]

  \pause
  \[
	\blue{\boxed{\forall 0 < \delta < 1: Pr\Big(X < (1-\delta)pk\Big) < e^{-\frac{\delta^2}{3}pk}}}
  \]

  \pause
  \[
	\teal{\text{Fix } \delta = 1 - \frac{1}{2p}}
  \]

  \pause
  \[
	Pr\Big(X \ge \frac{1}{2} k \Big) \ge 1 - e^{-\frac{k}{3n^c}}
  \]
\end{frame}
%%%%%%%%%%%%%%%%%%%%
\begin{frame}
  \[
	Pr\Big(X \ge \frac{1}{2} k \Big) \ge 1 - e^{-\frac{k}{3n^c}}
  \]

  \pause
  \[
	\teal{\text{Choose } k = 3n^{c+d} \text{ for some constant } d}
  \]

  \pause
  \[
	Pr\Big(X \ge \frac{1}{2} k \Big) \ge 1 - e^{-n^{d}}
  \]

  \pause
  \[
	\red{L \in BPP_{1 - e^{-n^d}}}
  \]

  \pause
  \[
	\red{\boxed{\forall \text{ constant } c, d > 0: BPP_{\frac{1}{2} + \frac{1}{n^c}} = BPP_{1-\frac{1}{e^{n^d}}}}}
  \]
\end{frame}
%%%%%%%%%%%%%%%%%%%%

%%%%%%%%%%%%%%%%%%%%
\begin{frame}
  \begin{definition}[$PP$: Probabilistic Polynomial time (Unbounded Error)]
	\[
	  L \in BPP
	\]
	\[
	  \iff
	\]
	\[
	  \exists A \text{ (\it probabilistic polynomial-time algorithm)}: 
	\]
	\[
	  Pr\Big(A(x) = L(x)\Big) > \frac{1}{2}
	\]
  \end{definition}

  \pause
  \[
	\blue{\text{We want } Pr\Big(A^{(k)}(x) = L(x)\Big) \ge 1 - \delta}
  \]
  \[
	\red{k \text{ may be exponential of } n}
  \]

  \pause
  \[
	Pr\Big(A(x) = L(x)\Big) \ge \frac{1}{2} + \frac{1}{2^{n^{c}}} \text{ for some constant } c
  \]
\end{frame}
%%%%%%%%%%%%%%%%%%%%
