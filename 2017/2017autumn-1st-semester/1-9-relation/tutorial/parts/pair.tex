%%%%%%%%%%%%%%%
\begin{frame}{}
  \centerline{\LARGE Ordered Pair and Cartesian Product}

  \vspace{0.60cm}
  \fignocaption{width = 0.80\textwidth}{figs/cards}
\end{frame}
%%%%%%%%%%%%%%%

%%%%%%%%%%%%%%%
\begin{frame}{}
  \begin{exampleblock}{Definitions of $(a,b)$ and $A \times B$ (UD $9.16$)}
    \[
      (a, b) = \set{\set{a}, \set{a,b}}
    \]

    \[
      (a,b) = (x,y) \iff a = x \land b = y
    \]
  \end{exampleblock}

  \[
    \boxed{\set{\set{a}, \set{a,b}} = \set{\set{x}, \set{x,y}} \implies a = x \land b = y}
  \]

  \vspace{0.50cm}
  What is wrong with the following proof:
  \begin{columns}
    \column{0.45\textwidth}
      \begin{equation*}
	\begin{cases}
	  \set{a} &= \set{x} \\
	  \set{a,b} &= \set{x,y}
	\end{cases}
	\implies \begin{cases}
	    a = x \\
	    b = y
	  \end{cases}
      \end{equation*}
    \column{0.45\textwidth}
      \[
	\begin{cases}
	  \set{a} &= \set{x,y} \\
	  \set{a,b} &= \set{x}
	\end{cases}
	\implies \text{ no solution.}
      \]
  \end{columns}
\end{frame}
%%%%%%%%%%%%%%%

%%%%%%%%%%%%%%%
\begin{frame}{}
  \begin{exampleblock}{Definitions of $(a,b)$ and $A \times B$ (UD $9.16$)}
    \[
      (a, b) = \set{\set{a}, \set{a,b}}
    \]

    \[
      (a,b) = (x,y) \iff a = x \land b = y
    \]
  \end{exampleblock}

  \[
    \boxed{\set{\set{a}, \set{a,b}} = \set{\set{x}, \set{x,y}} \implies a = x \land b = y}
  \]

  \vspace{0.50cm}
  \begin{proof}
    \begin{columns}
      \column{0.45\textwidth}
        \[
	  \textsc{Case } a = b
	\]
      \column{0.45\textwidth}
        \[
	  \textsc{Case } a \neq b
	\]
    \end{columns}
  \end{proof}
\end{frame}
%%%%%%%%%%%%%%%

%%%%%%%%%%%%%%%
\begin{frame}{}
  \begin{exampleblock}{Definitions of $(a,b)$ and $A \times B$ (UD $9.16$)}
    \[
      (a, b) = \set{\set{a}, \set{a,b}}
    \]

    \[
      a \in A \land b \in B \implies (a, b) \in \ps{\ps{A \cup B}}
    \]

    \pause
    \[
      A \times B = \set{x \in \ps{\ps{A \cup B}} \mid \exists a \in A \,\exists b \in B : x = (a,b)}
    \]

    \[
      A \subseteq C \land B \subseteq D \implies A \times B \subseteq C \times D
    \]
  \end{exampleblock}
\end{frame}
%%%%%%%%%%%%%%%

%%%%%%%%%%%%%%%
\begin{frame}{}
  \begin{exampleblock}{(UD $9.13$)}
    \[
      A \times B \subseteq C \times D \;\red{\xRightarrow{\;\;?\;\;}}\; A \subseteq C \land B \subseteq D
    \]
  \end{exampleblock}

  \[
    \red{A = \emptyset}
  \]

  \[
    A \times B \subseteq C \times D \;\blue{\xRightarrow{A, B \neq \emptyset}}\; A \subseteq C \land B \subseteq D
  \]

  \centerline{\blue{By contradiction.}}
\end{frame}
%%%%%%%%%%%%%%%

%%%%%%%%%%%%%%%
\begin{frame}{}
  \begin{exampleblock}{Distributive Laws (UD $9.14$)}
    \begin{align*}
      A \times (B \cup C) &= (A \times B) \cup (A \times C) \\
      A \times (B \cap C) &= (A \times B) \cap (A \times C) \\
      A \times (B \setminus C) &= (A \times B) \setminus (A \times C)
    \end{align*}
  \end{exampleblock}

  \fignocaption{width = 0.60\textwidth}{figs/cp-distributive}
\end{frame}
%%%%%%%%%%%%%%%
