% polygon-diameter.tex

%%%%%%%%%%%%%%%
\begin{frame}{}
  \begin{center}
    \teal{\Large Convex Polygon Diameter}
  \end{center}

  \fig{width = 0.60\textwidth}{figs/convex-diameter-alg}
\end{frame}
%%%%%%%%%%%%%%%

%%%%%%%%%%%%%%%
% \begin{frame}{}
%   \begin{exampleblock}{Convex Polygon Diameter (DH $6.8$)}
%     Show that the ``Convex Polygon Diameter'' algorithm is of \red{\bf linear-time} complexity.
%   \end{exampleblock}
% 
%   \begin{center}
%     \red{$Q:$} Linear-time of \red{\bf WHAT}? \\[5pt] \pause
%     \blue{$A:$} Linear-time of \blue{\bf the size of input} \\[12pt] \pause
% 
%     \red{$Q:$} What is the input? \\[5pt] \pause
%     \blue{$A:$} A convex polygon \uncover<5->{\teal{represented by $n$ vertices}} \\[12pt]
% 
%     \uncover<6->{
%       \red{$Q:$} What are the critical operations? \\[5pt] \pause
%     }
%     \uncover<7->{
%       \blue{$A:$} $d(p_1, p_2) = \sqrt{(x_1 - x_2)^2 + (y_1 - y_2)^2}$
%     }
%   \end{center}
% 
%   \uncover<8->{
%     \[
%       \Theta(c \cdot n) = \Theta(n)
%     \]
%   }
% \end{frame}
%%%%%%%%%%%%%%%

%%%%%%%%%%%%%%%
\begin{frame}{}
  \begin{center}
    {\teal{\Large Rotating Caliper}}
  \end{center}
  \vspace{-0.30cm}
  \fig{width = 0.50\textwidth}{figs/caliper}

  \vspace{-0.30cm}
  \begin{columns}
    \pause
    \column{0.50\textwidth}
      \fig{width = 0.40\textwidth}{figs/shamos}
      \begin{center}
        \href{http://euro.ecom.cmu.edu/people/faculty/mshamos/1978ShamosThesis.pdf}{\teal{``Computational Geometry''}} \\
        Ph.D Thesis, Michael Shamos, 1978
      \end{center}
    \pause
    \column{0.50\textwidth}
      \fig{width = 0.40\textwidth}{figs/toussaint}
      \begin{center}
        \href{https://www.cs.swarthmore.edu/~adanner/cs97/s08/pdf/calipers.pdf}{\teal{``Solving Geometric Problems with the Rotating Calipers''}}, 1983
      \end{center}
  \end{columns}
\end{frame}
%%%%%%%%%%%%%%%

%%%%%%%%%%%%%%%
\begin{frame}{}
  \begin{center}
    {\teal{\Huge Correctness}}
  \end{center}

  \vspace{0.30cm}
  \fig{width = 0.60\textwidth}{figs/wait-why}
\end{frame}
%%%%%%%%%%%%%%%

%%%%%%%%%%%%%%%
\begin{frame}{}
  \begin{theorem}[DH 4-8]
    % For a convex polygon, a pair of vertices determine the diameter.
    If $AB$ is a diameter of a convex polygon $P$, then $A$ and $B$ are vertices.
  \end{theorem}

  \pause
  % \fig{width = 0.45\textwidth}{figs/convex-polygon-vertex-diameter}
  \begin{center}
    % convex-polygon-vertex-diameter.tex

\begin{tikzpicture}
  \coordinate (1) at (1,2); 
  \coordinate (2) at (4, 1.5); 
  \coordinate (3) at (5,0); 
  \coordinate (4) at (3.5, -0.8); 
  \coordinate (5) at (1.5, -1);
  \coordinate (6) at (0,0);

  \path[draw] (6) node[left] {6}
    -- (5) node[below] {5}
    -- (4) node[below] {4}
    -- (3) node[right] {3}
    -- (2) node[above] {2}
    -- (1) node[above] {1}
    -- cycle;

  % A -- B
  \coordinate (A) at ($(1)!0.4!(2)$);
  \coordinate (B) at ($(3)!0.4!(4)$);
  \draw[thick, blue] (A) node[above] {$A$} -- (B) node[below] {$B$};

  \pause
  % B -- 6
  \draw[thick, dashed, red, shorten >= -2cm, shorten <= -0.5cm] (A) -- ($(A)!1cm!-90:(B)$);
  \draw[thick, teal] (B) -- (6);

  \pause
  % 6 -- 3
  \draw[thick, dashed, red, shorten >= 2cm, shorten <= -1.0cm] (B) -- ($(B)!1cm!-90:(A)$);
  \draw[thick, purple] (6) -- (3);
\end{tikzpicture}
  \end{center}

  \uncover<4->{
    \begin{center}
      \red{\large BUT, we have {\it not} enumerated {\it all} pairs of vertices.} \\[10pt] \pause
      \teal{We have enumerated {\it all antipodals.}} \\
    \end{center}}
\end{frame}
%%%%%%%%%%%%%%%

%%%%%%%%%%%%%%%
\begin{frame}{}
  \begin{definition}[Line of Support]
    A line $L$ is a \red{\it line of support} of a convex polygon $P$ if
    \[
      L \cap P = \text{ a vertex/an edge of } P.
    \]
  \end{definition}

  \fig{width = 0.30\textwidth}{figs/supporting-line}

  \pause
  \vspace{0.20cm}
  \begin{center}
    {$L \cap P \neq \emptyset$ \qquad $P$ lies entirely on one side of $L$.}
  \end{center}
\end{frame}
%%%%%%%%%%%%%%%

%%%%%%%%%%%%%%%
\begin{frame}{}
  \begin{definition}[Antipodal]
    An \red{\it antipodal} is a pair of points that admits parallel supporting lines.
  \end{definition}

  \fig{width = 0.60\textwidth}{figs/convex-diameter-alg}

  \pause
  \begin{center}
    {\teal{\large We have enumerated {\it all} antipodals by {\it rotating} through all angles.}}
  \end{center}
\end{frame}
%%%%%%%%%%%%%%%

%%%%%%%%%%%%%%%
\begin{frame}{}
  \begin{theorem}[We Won't Miss the Diameter]
    If $AB$ is a diameter of a convex polygon $P$, then $AB$ is an antipodal.
  \end{theorem}

  \vspace{0.30cm}
  \pause
  % \fig{width = 0.40\textwidth}{figs/convex-polygon-antipodal-diameter}
  \begin{center}
    % convex-polygon-antipodal-diameter.tex

\begin{tikzpicture}
  \coordinate (1) at (1,2); 
  \coordinate (2) at (4, 1.5); 
  \coordinate (3) at (5,0); 
  \coordinate (4) at (3.5, -0.8); 
  \coordinate (5) at (1.5, -1);
  \coordinate (6) at (0,0);

  \path[draw] (6) node[left] {6}
    -- (5) node[below] {5}
    -- (4) node[below] {4}
    -- (3) node[right] {3}
    -- (2) node[above] {2}
    -- (1) node[above] {1}
    -- cycle;

  % A -- B
  \draw[thick, blue] (3) node[below] {$A$} -- (6) node[below] {$B$};

  \pause
  % l_A, l_B
  \draw[thick, dashed, red, shorten >= -1.5cm, shorten <= -2.3cm] 
    (3) -- node[above = 2.0cm]{$l_{A} \perp AB$} ($(3)!0.5cm!-90:(6)$);
  \draw[thick, dashed, red, shorten >= -1cm, shorten <= -2cm] 
    (6) -- node[above = 2.5cm]{$l_{B} \perp AB$} ($(6)!0.5cm!-90:(3)$);
  
  \pause
  \node (x) [above right = 1.5cm and 0.2cm of 3, teal] {$X$};

  \pause
  \draw[thick, teal] (6) -- (x);
\end{tikzpicture}
  \end{center}

  \vspace{0.30cm}
  \uncover<3->{
    \begin{center}
      {$L \cap P \neq \emptyset$ \qquad $P$ lies entirely on one side of $L$.}
    \end{center}
  }
\end{frame}
%%%%%%%%%%%%%%%

%%%%%%%%%%%%%%%
\begin{frame}{}
  \begin{exampleblock}{Finding the Closest Pair of Points (Additional: DH 4-10)}
    \fig{width = 0.30\textwidth}{figs/closest-pair-plane}
  \end{exampleblock}

  \pause
  \begin{proof}[A Classical and Beautiful Divide-Conquer\only<3->{\red{-Combine}} Algorithm:]
    \fig{width = 0.35\textwidth}{figs/try}
    \vspace{-0.60cm}
    \begin{center}
      {\teal{Section $33.4$, CLRS}}
    \end{center}
  \end{proof}
\end{frame}
%%%%%%%%%%%%%%%