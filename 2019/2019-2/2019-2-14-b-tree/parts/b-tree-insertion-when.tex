% b-tree-insertion-when.tex

%%%%%%%%%%%%%%%
\begin{frame}{}
  \begin{enumerate}[(I)]
    \setcounter{enumi}{1}
    \centering
    \item \red{When does each node $s \in S$ \textsc{split}?}
  \end{enumerate}

  \[
    T_{s} = \langle s_{1}, s_{2}, \dots \rangle
  \]

  \pause
  \vspace{0.80cm}
  \begin{center}
    Let's focus the \red{rightmost} node first, denoted $A$.
  \end{center}
\end{frame}
%%%%%%%%%%%%%%%

%%%%%%%%%%%%%%%
\begin{frame}{}
  \fig{width = 0.10\textwidth}{figs/t0}
  
  \begin{columns}
    \column{0.33\textwidth}
      \fig{width = 0.40\textwidth}{figs/b-tree-1}
    \column{0.33\textwidth}
      \fig{width = 0.70\textwidth}{figs/b-tree-2}
    \column{0.33\textwidth}
      \fig{width = 0.80\textwidth}{figs/b-tree-3}
  \end{columns}
\end{frame}
%%%%%%%%%%%%%%%

%%%%%%%%%%%%%%%
\begin{frame}{}
  \begin{columns}
    \column{0.33\textwidth}
      \fig{width = 0.80\textwidth}{figs/b-tree-4}
    \column{0.33\textwidth}
      \fig{width = 0.90\textwidth}{figs/b-tree-5}
    \column{0.33\textwidth}
      \fig{width = 0.90\textwidth}{figs/b-tree-6}
  \end{columns}
\end{frame}
%%%%%%%%%%%%%%%

%%%%%%%%%%%%%%%
\begin{frame}{}
  \begin{columns}
    \column{0.50\textwidth}
      \fig{width = 0.80\textwidth}{figs/b-tree-7}
    \column{0.50\textwidth}
      \fig{width = 0.80\textwidth}{figs/b-tree-8}
  \end{columns}
\end{frame}
%%%%%%%%%%%%%%%

%%%%%%%%%%%%%%%
\begin{frame}{}
  \begin{columns}
    \column{0.50\textwidth}
      \fig{width = 0.80\textwidth}{figs/b-tree-9}
    \column{0.50\textwidth}
      \fig{width = 0.80\textwidth}{figs/b-tree-10}
  \end{columns}
\end{frame}
%%%%%%%%%%%%%%%

%%%%%%%%%%%%%%%
\begin{frame}{}
  \begin{columns}
    \column{0.50\textwidth}
      \fig{width = 0.90\textwidth}{figs/b-tree-11}
    \column{0.50\textwidth}
      \fig{width = 0.90\textwidth}{figs/b-tree-12}
  \end{columns}

  \pause
  \vspace{1.00cm}
  \[
    \red{A\; \textsc{split}: 4,\quad 6,\quad 8,\quad 10,\quad 12,\quad \dots}
  \]
  % \[
  %   \blue{A\; \textsc{full}: 3,\quad 5,\quad 7,\quad 9,\quad 11,\quad \dots}
  % \]
\end{frame}
%%%%%%%%%%%%%%%

%%%%%%%%%%%%%%%
\begin{frame}{}
  \begin{enumerate}[(I)]
    \setcounter{enumi}{1}
    \centering
    \item \red{When does each node \textsc{split}?}
  \end{enumerate}

  \pause
  \vspace{0.60cm}
  \begin{center}
    Let's consider the \red{parent of $A$}, denoted $B \triangleq p(A)$. \\[30pt] \pause
    \blue{Every time $A$ splits, $B$ obtains a new key.}
  \end{center}
\end{frame}
%%%%%%%%%%%%%%%

%%%%%%%%%%%%%%%
\begin{frame}{}
  \begin{columns}
    \column{0.25\textwidth}
      \fig{width = 0.40\textwidth}{figs/b-tree-1}
    \column{0.25\textwidth}
      \fig{width = 0.75\textwidth}{figs/b-tree-2}
    \column{0.25\textwidth}
      \fig{width = 0.85\textwidth}{figs/b-tree-3}
    \column{0.25\textwidth}
      \fig{width = 0.90\textwidth}{figs/b-tree-4}
  \end{columns}

  \pause
  \vspace{0.50cm}
  \begin{columns}
    \column{0.50\textwidth}
      \fig{width = 0.70\textwidth}{figs/b-tree-5}
    \column{0.50\textwidth}
      \fig{width = 0.70\textwidth}{figs/b-tree-6}
  \end{columns}

  \pause
  \vspace{0.50cm}
  \begin{columns}
    \column{0.50\textwidth}
      \fig{width = 0.90\textwidth}{figs/b-tree-7}
    \column{0.50\textwidth}
      \fig{width = 0.90\textwidth}{figs/b-tree-8}
  \end{columns}
\end{frame}
%%%%%%%%%%%%%%%

%%%%%%%%%%%%%%%
\begin{frame}{}
  \begin{columns}
    \column{0.50\textwidth}
      \fig{width = 0.80\textwidth}{figs/b-tree-9}
    \column{0.50\textwidth}
      \fig{width = 0.80\textwidth}{figs/b-tree-10}
  \end{columns}

  \pause
  \vspace{0.50cm}
  \begin{columns}
    \column{0.50\textwidth}
      \fig{width = 0.90\textwidth}{figs/b-tree-11}
    \column{0.50\textwidth}
      \fig{width = 0.90\textwidth}{figs/b-tree-12}
  \end{columns}
\end{frame}
%%%%%%%%%%%%%%%

%%%%%%%%%%%%%%%
\begin{frame}{}
  \begin{columns}
    \column{0.50\textwidth}
      \fig{width = 0.80\textwidth}{figs/b-tree-13}
    \column{0.50\textwidth}
      \fig{width = 0.80\textwidth}{figs/b-tree-14}
  \end{columns}

  \pause
  \vspace{0.50cm}
  \begin{columns}
    \column{0.50\textwidth}
      \fig{width = 1.00\textwidth}{figs/b-tree-15}
    \column{0.50\textwidth}
      \fig{width = 1.00\textwidth}{figs/b-tree-16}
  \end{columns}

  \pause
  \vspace{1.00cm}
  \[
    \red{A\; \textsc{split}: 4,\quad 6,\quad 8,\quad 10,\quad 12,\quad \dots}
  \]
  \[
    \red{B\; \textsc{split}: 9,\quad 13,\quad 17,\quad 21,\quad 25,\quad \dots}
  \]
  % \[
  %   \blue{B\; \textsc{full}: 8,\quad 12,\quad 16,\quad 20,\quad 24,\quad \dots}
  % \]
\end{frame}
%%%%%%%%%%%%%%%

%%%%%%%%%%%%%%%
% \begin{frame}{}
%   \[
%     \red{A\; \textsc{split}: 4,\quad 6,\quad 8,\quad 10,\quad 12,\quad \dots}
%   \]
%   % \[
%   %   \blue{A\; \textsc{full}: 3,\quad 5,\quad 7,\quad 9,\quad 11,\quad \dots}
%   % \]
% 
%   \vspace{1.00cm}
%   \[
%     \red{B\; \textsc{split}: 9,\quad 13,\quad 17,\quad 21,\quad 25,\quad \dots}
%   \]
%   % \[
%   %   \blue{B\; \textsc{full}: 8,\quad 12,\quad 16,\quad 20,\quad 24,\quad \dots}
%   % \]
% \end{frame}
%%%%%%%%%%%%%%%

%%%%%%%%%%%%%%%
\begin{frame}{}
  \begin{enumerate}[(I)]
    \setcounter{enumi}{1}
    \centering
    \item \red{When does each node \textsc{split}?}
  \end{enumerate}

  \pause
  \vspace{0.80cm}
  \begin{center}
    Let's consider the \red{parent of $B$}, denoted $C = p(B)$.
  \end{center}
\end{frame}
%%%%%%%%%%%%%%%

%%%%%%%%%%%%%%%
\begin{frame}{}
  \[
    \teal{A\; \textsc{split}: 4,\quad 6,\quad 8,\quad 10,\quad 12,\quad \dots}
  \]
  % \[
  %   \blue{A\; \textsc{full}: 3,\quad 5,\quad 7,\quad 9,\quad 11,\quad \dots}
  % \]
  \[
    \teal{B\; \textsc{split}: 9,\quad 13,\quad 17,\quad 21,\quad 25,\quad \dots}
  \]
  % \[
  %   \blue{B\; \textsc{full}: 8,\quad 12,\quad 16,\quad 20,\quad 24,\quad \dots}
  % \]
  \[
    \teal{C\; \textsc{split}: 18,\quad 26,\quad 34,\quad 42,\quad 50,\quad \dots}
  \]

  \pause
  \hrule
  \[
    A: 1 \qquad B: 2 \qquad C: 3 \qquad
  \]

  \[
    \red{T_{i}: \text{ the first time point the $i$-th node splits}}
  \]

  \pause
  \[
    T_{1} = 4
  \]
  
  \pause
  \[
    \blue{T_{i} = \underbrace{T_{i-1}}_{\text{its right child first split}} + 
        \underbrace{2 \times 2^{i-1}}_{\text{its right child split twice more}} + 
        \underbrace{1}_{\text{insert one more}}}
  \]

  \pause
  \[
    \red{T_{i} = 2^{i+1} + i - 1}
  \]
\end{frame}
%%%%%%%%%%%%%%%