% file: sections/matchings-covers.tex

%%%%%%%%%%%%%%%%%%%%
\begin{frame}{}
  \begin{theorem}[Hall's Theorem, $1935$; Theorem $8.3$]
    Let $G$ be a \red{bipartite graph} with partite sets $U$ and $W$ such that $r = |U| \le |W|$.

    $G$ contains a matching of cardinality $r$ $\iff$ $G$ satisfies \red{Hall's Condition}:

    \[
      \forall X \subseteq U: \big\lvert N(X) \big\rvert \ge \big\lvert X \big\rvert
    \]
  \end{theorem}

  \pause
  \begin{center}
    \purple{TONCAS} \\[5pt]
    \teal{(The Obvious Necessary Conditions are Also Sufficient)} \\[8pt]
  \end{center}

  \pause
  \vspace{-0.30cm} 
  \fig{width = 0.20\textwidth}{figs/homer}
  \vspace{-0.30cm} 
  \centerline{\footnotesize Other TONCAS?}
\end{frame}
%%%%%%%%%%%%%%%%%%%%

%%%%%%%%%%%%%%%%%%%%
\begin{frame}{}
  \begin{exampleblock}{Perfect Matching on Trees (Problem $8.5$)}
    \begin{center}
      Prove that every tree $T$ has $\le 1$ perfect matching.
    \end{center}
  \end{exampleblock}

  \pause
  \vspace{0.60cm}
  \begin{columns}
    \column{0.50\textwidth}
      \fig{width = 0.50\textwidth}{figs/hui}
    \column{0.50\textwidth}
      \begin{CJK*}{UTF8}{gbsn}
	\centerline{``这题有四样证法, 你知道吗?''}
      \end{CJK*}
      \vspace{-0.30cm}
      \fig{width = 0.60\textwidth}{figs/kong-hui}
  \end{columns}
\end{frame}
%%%%%%%%%%%%%%%%%%%%

%%%%%%%%%%%%%%%%%%%%
\begin{frame}{}
  \begin{exampleblock}{Perfect Matching on Trees (Problem $8.5$)}
    \begin{center}
      Prove that every tree $T$ has $\le 1$ perfect matching.
    \end{center}
  \end{exampleblock}

  \pause
  \begin{center}
    \red{By strong mathematical induction on the order $n$ of trees.} \\[6pt] \pause
    \blue{Inductive step: Consider a tree of order $n$.}
  \end{center}

  \pause
  \begin{columns}
    \column{0.60\textwidth}
      % file: algs/tree-perfect-matching.tex

\begin{algorithm}[H]
  % \caption{Perfect Matching on Tree}
  % \label{alg:perfect-matching-tree}
  \begin{algorithmic}[1]
    \If{$n$ is odd}
      \State \uncover<5->{\cyan{\# Perfect Matching $= 0$}}
    \Else	\Comment{\teal{$n$ is even}}
      \uncover<6->{
	\State Consider $G - r$ \Comment{\teal{$r:$ root of $G$}}
	\If{$k_o(G - r) > 1$}
	  \State \uncover<7->{\cyan{\# Perfect Matching $= 0$}}
	\Else	\Comment{\teal{$k_o(G - r) = 1$}}
	  \State \uncover<8->{\red{By Induction Hypothesis.}}
	\EndIf
      }
    \EndIf
  \end{algorithmic}
\end{algorithm}

  \end{columns}
\end{frame}
%%%%%%%%%%%%%%%%%%%%

%%%%%%%%%%%%%%%%%%%%
\begin{frame}{}
  \begin{exampleblock}{Perfect Matching on Trees (Problem $8.5$)}
    \begin{center}
      Prove that every tree $T$ has $\le 1$ perfect matching.
    \end{center}
  \end{exampleblock}

  \begin{center}
    \red{By strong mathematical induction on the order $n$ of trees.} \\[6pt]
    \blue{Inductive step: Consider a tree of order $n$.}
  \end{center}

  \pause
  % file: algs/tree-perfect-matching-leaf.tex

\begin{algorithm}[H]
  % \caption{Perfect Matching on Tree}
  % \label{alg:perfect-matching-tree-leaf}
  \begin{algorithmic}[1]
    \If{$T$ has no perfect matchings}
      \State \cyan{\# Perfect Matching $= 0$}
    \Else	\Comment{\teal{$T$ has perfect matchings}}
      \uncover<3->{
	\State Consider a leaf $v$
      }
      \uncover<4->{
	\State $v$ \purple{must} be matched with its parent $u$
      }
      \uncover<5->{
	\State \red{By Induction Hypothesis} on each component of $G - \set{u, v}$
      }
    \EndIf
  \end{algorithmic}
\end{algorithm}

\end{frame}
%%%%%%%%%%%%%%%%%%%%

%%%%%%%%%%%%%%%%%%%%
\begin{frame}{}
  \begin{exampleblock}{Perfect Matching on Trees (Problem $8.5$)}
    \begin{center}
      Prove that every tree $T$ has $\le 1$ perfect matching.
    \end{center}
  \end{exampleblock}

  \pause
  \begin{center}
    \red{By Contradiction.} \\[6pt]
    \teal{Suppose that there are two different perfect matchings $M$ and $M'$ on $T$.} \\[10pt] \pause
    \blue{$\exists v:$ $v$ is matched with different vertices in $M$ and $M'$.}
  \end{center}

  \pause
  \fig{width = 0.40\textwidth}{figs/tree-perfect-matching-diff-cycle}

  \pause
  \begin{center}
    \red{$Q:$ What about $u$ and $w$?} \\[8pt] \pause
    \blue{Contradiction: Cycle}
  \end{center}
\end{frame}
%%%%%%%%%%%%%%%%%%%%

%%%%%%%%%%%%%%%%%%%%
\begin{frame}{}
  \begin{exampleblock}{Perfect Matching on Trees (Problem $8.5$)}
    \begin{center}
      Prove that every tree $T$ has $\le 1$ perfect matching.
    \end{center}
  \end{exampleblock}

  \begin{center}
    \red{By Contradiction.} \\[6pt]
    \teal{Suppose that there are two different perfect matchings $M$ and $M'$ on $T$.}
  \end{center}

  \pause
  \vspace{-0.30cm}
  \[
    M \Delta M' = (M - M') \cup (M' - M)
  \]
  \begin{center}
    \blue{Consider the subgraph $\mathcal{M}$ with $V(T)$ and $M \Delta M'$.}
  \end{center}

  \begin{columns}
    \column{0.60\textwidth}
      \pause
      \begin{columns}
	\column{0.50\textwidth}
	  \fig{width = 0.60\textwidth}{figs/perfect-matching-diff-same-edge}
	  \vspace{-0.20cm}
	  \centerline{\teal{\textsc{Case I}}}
	\column{0.50\textwidth}
	  \fig{width = 0.70\textwidth}{figs/perfect-matching-diff-not-same-edge}
	  \vspace{-0.20cm}
	  \centerline{\teal{\textsc{Case II}}}
      \end{columns}
    \column{0.40\textwidth}
      \pause
      \[
	\forall v \in V(\mathcal{M}):
      \]
      \[
	\text{deg}(v) = 0 \lor \text{deg}(v) = 2
      \]
      \pause
      \vspace{-0.50cm}
      \[
	\purple{T \text{ is a tree } \implies \text{deg}(v) = 0}
      \]
      \pause
      \vspace{-0.50cm}
      \[
	\text{deg}(v) = 0 \implies \text{\teal{\textsc{ Case I}}}
      \]
  \end{columns}
\end{frame}
%%%%%%%%%%%%%%%%%%%%

%%%%%%%%%%%%%%%%%%%%
\begin{frame}{}
  \uncover<3->{
    \begin{theorem}[Gallai Identities, $1959$; Theorem $8.7$]
      If $G$ is graph without isolated vertices, then
      \[
	\alpha'(G) + \beta'(G) = n(G).
      \]
    \end{theorem}
  }

  \begin{align*}
    \alpha(G) 	& \onslide<2->{\text{\quad Maximum size of independent set}} \\[8pt]
    \beta(G) 	& \onslide<2->{\text{\quad Minimum size of vertex cover}} \\[8pt]
    \alpha'(G)  & \onslide<2->{\text{\quad Maximum size of matching}} \\[8pt]
    \beta'(G) 	& \onslide<2->{\text{\quad Minimum size of edge cover}}
  \end{align*}

  \uncover<4->{
    \begin{theorem}[Gallai Identities, $1959$; Theorem $8.8$]
      If $G$ is graph without isolated vertices, then
      \[
	\alpha(G) + \beta(G) = n(G).
      \]
    \end{theorem}
  }
\end{frame}
%%%%%%%%%%%%%%%%%%%%

%%%%%%%%%%%%%%%%%%%%
\begin{frame}{}
  \begin{exampleblock}{Matching and Edge Cover (Problem $8.14$)}
    A graph $G$ without isolated vertices has a perfect matching if and only if $\alpha'(G) = \beta'(G)$.
  \end{exampleblock}

  \begin{columns}
    \column{0.50\textwidth}
      \pause
      \begin{center}
	\teal{``$\implies$''}
      \end{center}

      \pause
      \vspace{-0.30cm}
      \begin{align*}
	G &\text{ has a perfect matching} \\[5pt]
	&\implies n \text{ is even } \land \alpha'(G) = n/2 \\[5pt]
	&\implies \beta'(G) = n/2 % \\[5pt]
	% &\implies \alpha'(G) = \beta'(G)
      \end{align*}
    \column{0.50\textwidth}
      \pause
      \begin{center}
	\teal{``$\Longleftarrow$''}
      \end{center}

      \pause
      \vspace{-0.30cm}
      \begin{align*}
	\alpha'&(G) = \beta'(G) \\[5pt]
	&\implies \alpha'(G) = n/2 \land n \text{ is even} \\[5pt]
	&\implies G \text{ has a perfect matching}
      \end{align*}
  \end{columns}
\end{frame}
%%%%%%%%%%%%%%%%%%%%

%%%%%%%%%%%%%%%%%%%%
\begin{frame}{}
  \uncover<2->{
    \begin{theorem}[K\"onig, $1931$; Egerv\'ary, $1931$]
      If $G$ is a \red{bipartite graph}, then
      \[
	\alpha'(G) = \beta(G).
      \]
    \end{theorem}
  }

  \begin{align*}
    \alpha(G) 	& \text{\quad Maximum size of independent set} \\[8pt]
    \beta(G) 	& \text{\quad Minimum size of vertex cover} \\[8pt]
    \alpha'(G)  & \text{\quad Maximum size of matching} \\[8pt]
    \beta'(G) 	& \text{\quad Minimum size of edge cover}
  \end{align*}

  \uncover<2->{
    \begin{theorem}[K\"onig, $1931$]
      If $G$ is a \red{bipartite graph}, then
      \[
	\alpha(G) = \beta'(G).
      \]
    \end{theorem}
  }
\end{frame}
%%%%%%%%%%%%%%%%%%%%

%%%%%%%%%%%%%%%%%%%%
\begin{frame}{}
  \begin{exampleblock}{Vertex Covering Number (Problem $8.16$)}
    If $G$ is a graph of order $n$, maximum degree $\Delta$ 
    and \textcolor<2->{red}{having no isolated vertices}, then 
    \[
      \beta(G) \ge \frac{n}{\Delta + 1}.
    \]
  \end{exampleblock}

  \begin{columns}
    \column{0.45\textwidth}
      \uncover<3->{
	\fig{width = 0.50\textwidth}{figs/isolated-vertex}

	\[
	  n = 3, \Delta = 1, \purple{\frac{n}{\Delta + 1} = \frac{3}{2}}, \beta = 1
	\]
      }
    \column{0.55\textwidth}
      \uncover<4->{
	\begin{center}
	  \red{By Contradiction: $\beta < \frac{n}{\Delta + 1}$.}
	\end{center}
      }
      \vspace{-0.20cm}
      \uncover<5->{
	\begin{align*}
	  \beta \cdot \Delta &< \frac{n \Delta}{\Delta + 1} \\
	  &= n - \frac{n}{\Delta + 1} \\
	  &\le n - 1
	\end{align*}
      }
      \vspace{-0.30cm}
      \uncover<6->{
	\centerline{\red{Contradiction: No isolated vertices.}}
      }
  \end{columns}
\end{frame}
%%%%%%%%%%%%%%%%%%%%

%%%%%%%%%%%%%%%%%%%%
\begin{frame}{}
  \begin{exampleblock}{Vertex Independence Number (Additional Problem)}
    If $G$ is a graph of order $n$, maximum degree $\Delta$, then
    \[
      \alpha(G) \ge \frac{n}{\Delta + 1}.
    \]
  \end{exampleblock}

  \pause
  \begin{center}
    \red{By Construction.} \\[6pt] \pause
    \teal{To construct an independent set $S$ with $|S| \ge \frac{n}{\Delta + 1}$.}
  \end{center}

  \pause
  \begin{columns}
    \column{0.50\textwidth}
      % file: algs/independent-set-construction.tex

\begin{algorithm}[H]
  % \caption{Independent Set Construction}
  % \label{alg:independent-set-construction}
  \begin{algorithmic}[1]
    \While{$|V(G) > 0|$}
      \State Choose $v \in V(G)$
      \State \red{$S \gets S \cup \set{v}$}
      \State $G \gets G - \set{v} - N(v)$
    \EndWhile
  \end{algorithmic}
\end{algorithm}

  \end{columns}
\end{frame}
%%%%%%%%%%%%%%%%%%%%
