% file: sections/overview.tex

%%%%%%%%%%%%%%%
\begin{frame}{}
  \begin{columns}
    \column{0.50\textwidth}
      \fignocaption{width = 0.45\textwidth}{figs/tarjan}{\centerline{\teal{Robert Tarjan}}}
    \column{0.50\textwidth}
      \fignocaption{width = 0.50\textwidth}{figs/hopcroft}{\centerline{\teal{John Hopcroft}}}
  \end{columns}

  \vspace{0.30cm}
  \begin{quote}
    \centering
    \purple{\large For fundamental achievements \\ in the design and analysis of algorithms and data structures.} \\[4pt]

    \hfill --- Turing Award, 1986
  \end{quote}
\end{frame}
%%%%%%%%%%%%%%%

%%%%%%%%%%%%%%%
\begin{frame}{}
  \fignocaption{width = 0.80\textwidth}{figs/tarjan-amortized-paper}

  \begin{quote}
    \centering
    \teal{``Amortized Computational Complexity'', 1985}
  \end{quote}
\end{frame}
%%%%%%%%%%%%%%%


%%%%%%%%%%%%%%%
\begin{frame}{}
  \begin{quote}
    \centering
    {\large
      \red{\Large Amortized analysis} is \\[6pt]
      \blue{an algorithm analysis technique} for \\[6pt]
      \purple{analyzing a sequence of operations} \\[6pt]
      \blue{irrespective of the input} to show that \\[6pt]
      \cyan{the average cost per operation} is small, even though \\[6pt]
      \teal{a single operation within the sequence might be expensive}.
    }
  \end{quote}
\end{frame}
%%%%%%%%%%%%%%%

%%%%%%%%%%%%%%%
\begin{frame}{}
  \begin{quote}
    \centering
    By \violet{averaging the cost per operation over a worst-case sequence}, \\[6pt]
    \red{\large amortized analysis} can yield a time complexity that is \\[6pt]
    more \purple{robust} than \purple{average-case analysis}, since \\[6pt]
    its \cyan{probabilistic assumptions on inputs} may be false, \\[6pt]
    and more \purple{realistic} than \purple{worst-case analysis}, since it may be \\[6pt]
    \cyan{impossible for every operation to take the worst-case time}, \\[6pt]
    \brown{as occurs often in manipulation of data structures}.
  \end{quote}
\end{frame}
%%%%%%%%%%%%%%%

%%%%%%%%%%%%%%%
\begin{frame}{}
  \begin{columns}
    \column{0.50\textwidth}
      \fignocaption{width = 0.60\textwidth}{figs/three-roads}
      \begin{enumerate}[(I)]
	\setlength{\itemsep}{3pt}
	\item Summation Method
	\item Accounting Method
	\item Potential Method
      \end{enumerate}
    \column{0.50\textwidth}
      \fignocaption{width = 0.70\textwidth}{figs/example}
      \centerline{Dynamic Tables}

      \pause
      \begin{center}
	``Move-to-Front'' List \\[6pt]
	Splay Tree
      \end{center}
  \end{columns}
\end{frame}
%%%%%%%%%%%%%%%
