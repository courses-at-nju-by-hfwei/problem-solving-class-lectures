%%%%%%%%%%%%%%%
\begin{frame}{为什么列书单?}
\end{frame}
%%%%%%%%%%%%%%%

%%%%%%%%%%%%%%%
\begin{frame}{如何读?}
  \begin{center}
    不要指望一遍就能读懂\\[10pt]

    交叉阅读, 多角度思考\\[10pt]

    三五人同读一本书\\
  \end{center}
\end{frame}
%%%%%%%%%%%%%%%

%%%%%%%%%%%%%%%
\begin{frame}{}
  \begin{columns}
    \column{0.45\textwidth}
      \fignocaption{width = 0.85\textwidth}{figs/papadimitrous}
    \column{0.45\textwidth}
      \fignocaption{width = 0.65\textwidth}{figs/algorithms-papa}
  \end{columns}

  \vspace{0.80cm}
  \centerline{言简意赅、观点独到}
\end{frame}
%%%%%%%%%%%%%%%

%%%%%%%%%%%%%%%
\begin{frame}{}
  \begin{columns}
    \column{0.45\textwidth}
      \fignocaption{width = 0.85\textwidth}{figs/jon-kleinberg}
    \column{0.45\textwidth}
      \fignocaption{width = 0.75\textwidth}{figs/algorithm-design-china}
  \end{columns}

  \vspace{0.10cm}
  \begin{center}
    侧重算法设计思路、详略得当 \\[6pt]
    有别的书籍里找不到的算法题目 \\[10pt]

    \href{http://www.cs.princeton.edu/~wayne/kleinberg-tardos/}{Pretty slides}
  \end{center}
\end{frame}
%%%%%%%%%%%%%%%

%%%%%%%%%%%%%%%
\begin{frame}{}
  \begin{columns}
    \column{0.45\textwidth}
      \fignocaption{width = 0.80\textwidth}{figs/steven-skiena}
    \column{0.45\textwidth}
      \fignocaption{width = 0.70\textwidth}{figs/algorithm-design-manual-china}
  \end{columns}

  \vspace{0.50cm}
  \begin{center}
    我最喜欢读每一章的 War Story,尤其是第九章的那两个。\\[10pt]

    \href{http://www3.cs.stonybrook.edu/~algorith/video-lectures/}{Skiena's Algorithms Lectures}
  \end{center}
\end{frame}
%%%%%%%%%%%%%%%

%%%%%%%%%%%%%%%
\begin{frame}{}
  \begin{columns}
    \column{0.50\textwidth}
      \fignocaption{width = 0.60\textwidth}{figs/jeffe}
    \column{0.50\textwidth}
      \fignocaption{width = 0.70\textwidth}{figs/algorithms-jeffe}
      \vspace{-0.30cm}
      {\centerline{\href{http://jeffe.cs.illinois.edu/teaching/algorithms/all-algorithms.pdf}{讲义}。自由而亲切。}}
  \end{columns}

  \vspace{0.30cm}
  \begin{center}
    \href{https://recordings.engineering.illinois.edu:8443/ess/portal/section/d6d800b7-eb5c-4420-8175-10becaf2d25d}{CS 473: Algorithms (Spring 2017)} by Jeff Erickson
  \end{center}
\end{frame}
%%%%%%%%%%%%%%%

%%%%%%%%%%%%%%%
\begin{frame}{}
  \begin{columns}
    \column{0.45\textwidth}
      \fignocaption{width = 0.85\textwidth}{figs/sedgewick-youtube}
    \column{0.45\textwidth}
      \fignocaption{width = 0.70\textwidth}{figs/algorithms-4th}
  \end{columns}

  \vspace{0.50cm}
  \begin{center}
    包含大量图例, 提供可运行代码 (Java) 及实验数据 \\[8pt]
    在这里,算法是一门``实验科学''。\\[10pt]

    \href{https://www.youtube.com/playlist?list=PLxc4gS-\_A5VDXUIOPkJkwQKYiT2T1t0I8}{Algorithms and Data Structures (I)}
  \end{center}
\end{frame}
%%%%%%%%%%%%%%%

%%%%%%%%%%%%%%%
\begin{frame}{}
  \begin{columns}
    \column{0.45\textwidth}
      \fignocaption{width = 0.90\textwidth}{figs/leiserson}
    \column{0.45\textwidth}
      \fignocaption{width = 0.70\textwidth}{figs/erik-demaine-teaching.jpeg}
  \end{columns}

  \vspace{0.50cm}
  \begin{center}
    ``Algorithms''\\[4pt]
    \href{http://open.163.com/special/opencourse/algorithms.html}{MIT 6.046/18.410, Fall 2005}\\[3pt]
    \href{https://ocw.mit.edu/courses/electrical-engineering-and-computer-science/6-006-introduction-to-algorithms-fall-2011/lecture-videos/}{MIT 6.006, Fall 2011}\\[10pt]
  \end{center}
\end{frame}
%%%%%%%%%%%%%%%

%%%%%%%%%%%%%%%
\begin{frame}{}
  \fignocaption{width = 0.30\textwidth}{figs/concrete-mathematics}

  \begin{center}
    \href{http://www3.cs.stonybrook.edu/~algorith/math-video/}{Skiena's CSE 547 Discrete Mathematics Lectures}
  \end{center}
\end{frame}
%%%%%%%%%%%%%%%

%%%%%%%%%%%%%%%
\begin{frame}{}
  \begin{columns}
    \column{0.45\textwidth}
      \fignocaption{width = 0.85\textwidth}{figs/sedgewick-teaching}
    \column{0.45\textwidth}
      \fignocaption{width = 0.70\textwidth}{figs/AoA}
  \end{columns}

  \vspace{0.50cm}
  \begin{center}
    关注算法分析技术; 有一定难度
  \end{center}
\end{frame}
%%%%%%%%%%%%%%%