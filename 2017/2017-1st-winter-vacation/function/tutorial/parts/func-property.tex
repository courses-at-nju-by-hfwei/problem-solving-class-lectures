%%%%%%%%%%%%%%%
\begin{frame}{}
  \begin{center}
    {\LARGE Special Functions (\Large \emph{-jectivity})}
  \end{center}
\end{frame}
%%%%%%%%%%%%%%%

%%%%%%%%%%%%%%%
\begin{frame}{}
  \begin{definition}[Injective (one-to-one; 1-1) 单射函数]
    \[
      f: A \to B \qquad \uncover<2->{f: A \red{\;\rightarrowtail\;} B}
    \]

    \vspace{-0.60cm}
    \[
      \forall a_1, a_2 \in A: a_1 \neq a_2 \implies f(a_1) \neq f(a_2)
    \]
  \end{definition}

  \vspace{0.50cm}
  \uncover<3->{
    \begin{alertblock}{For Proof:}
      \begin{itemize}
	\item To prove that $f$ \blue{\it is} 1-1:
	  \[
	    \forall a_1, a_2 \in A: f(a_1) = f(a_2) \implies a_1 = a_2
	  \]
	\item<4-> To show that $f$ \blue{\it is not} 1-1:
	  \[
	    \red{\exists} a_1, a_2 \in A: a_1 \neq a_2 \land f(a_1) = f(a_2)
	  \]
      \end{itemize}
    \end{alertblock}
  }
\end{frame}
%%%%%%%%%%%%%%%

%%%%%%%%%%%%%%%
\begin{frame}{}
  \begin{definition}[Surjective (onto) 满射函数]
    \[
      f: A \to B \qquad \uncover<2->{f: A \red{\;\twoheadrightarrow\;} B}
    \]

    \vspace{-0.60cm}
    \[
      \ran{f} = B
    \]
  \end{definition}

  \vspace{0.50cm}
  \uncover<3->{
    \begin{alertblock}{For Proof:}
      \begin{itemize}
	\item To prove that $f$ \blue{\it is} onto:
	  \[
	    \forall b \in B \; \Big(\red{\exists} a \in A: f(a) = b \Big)
	  \]
	\item<4-> To show that $f$ \blue{\it is not} onto:
	  \[
	    \red{\exists} b \in B\; \Big(\red{\forall} a \in A: f(a) \neq b \Big)
	  \]
	\end{itemize}
    \end{alertblock}
  }
\end{frame}
%%%%%%%%%%%%%%%

%%%%%%%%%%%%%%%
\input{parts/cantor-theorem}
%%%%%%%%%%%%%%%

%%%%%%%%%%%%%%%
\begin{frame}{}
  \begin{definition}[Bijective (one-to-one correspondence) 一一对应]
    \[
      f: A \to B \qquad f: A \red{\;\xleftrightarrow[onto]{1-1}\;} B
    \]

    \vspace{0.50cm}
    \centerline{1-1 \& onto}
  \end{definition}
\end{frame}
%%%%%%%%%%%%%%%

%%%%%%%%%%%%%%%
\begin{frame}{}
  \begin{exampleblock}{Problem $14.12$}
    \[
      a, b, c, d \in \real{},\; a < b,\; c < d
    \]

    \vspace{0.30cm}
    \centerline{Define a bijective function:}
    \vspace{0.10cm}

    \[
      f: [a,b] \xleftrightarrow[onto]{1-1} [c,d]
    \]
  \end{exampleblock}

  \begin{proof}[Answer]
    \[
      f(x) = c + \frac{d-c}{b-a}(x-a)
    \]
  \end{proof}
\end{frame}
%%%%%%%%%%%%%%%
