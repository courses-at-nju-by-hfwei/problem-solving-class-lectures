% b-tree-bst.tex

%%%%%%%%%%%%%%%
\begin{frame}{}
  \begin{exampleblock}{Minimum (TC 18.2-3)}
    Explain how to find the \red{minimum} key stored in a $B$-tree. 
  \end{exampleblock}

  \fig{width = 0.75\textwidth}{figs/b-tree-example}

  \pause
  \vspace{0.30cm}
  \begin{center}
    \blue{the \red{leftmost key} in the leftmost node}
  \end{center}
\end{frame}
%%%%%%%%%%%%%%%

%%%%%%%%%%%%%%%
\begin{frame}{}
  \begin{exampleblock}{Predecessor (TC 18.2-3)}
    Explain how to find the \red{predecessor of a given key $(x, i)$} stored in a $B$-tree.
  \end{exampleblock}

  \pause
  \fig{width = 1.00\textwidth}{figs/b-tree-37-char}
  \begin{center}
    \teal{$B$-tree with 37 characters.}
  \end{center}

  \pause
  \[
    \red{x.\leaf = 0}
  \]

  \pause
  \vspace{-0.30cm}
  \[
    \blue{H \qquad N} \qquad \purple{S \qquad V}
  \]

  \pause
  \vspace{0.20cm}
  \begin{center}
    find the \red{rightmost key} in $x.c_{i}$
  \end{center}
\end{frame}
%%%%%%%%%%%%%%%

%%%%%%%%%%%%%%%
\begin{frame}{}
  \[
    (x, i)
  \]
  \fig{width = 1.00\textwidth}{figs/b-tree-37-char}

  \[
    \red{x.\leaf = 1}
  \]

  \vspace{-0.30cm}
  \[
    \blue{P \qquad U \qquad} \teal{T\qquad O \qquad} \purple{A}
  \]

  \pause
  \[
    i \ge 2 \implies (x, i-1)
  \]
  \[
    i = 1 \implies \text{ find $(y, j)$ such that $x$ is the \red{leftmost key} in $y.c_{j + 1}$}
  \]

  \pause
  \begin{center}
    $A$ is the \purple{only} key which has no predecessor.
  \end{center}
\end{frame}
%%%%%%%%%%%%%%%

%%%%%%%%%%%%%%%
\begin{frame}{}
  % b-tree-predecessor.tex

\begin{algorithm}[H]
% \caption{Predecessor of a given key in $B$-tree.}
% \label{b-tree-predecessor}
\begin{algorithmic}[1]
  \Procedure{B-Tree-Predecessor}{$T, x, i$} \Comment{\purple{$x.key_{i}$ in $B$-tree $T$}}
    \If{$x.\leaf = 0$}
      \State \Return the rightmost key in $x.c_{i}$
    \pause
    \ElsIf{$i \ge 2$} \Comment{$x.\leaf = 1$}
      \State \Return $(x, i-1)$
    \pause
    \Else \Comment{\red{$x.\leaf = 1 \land i = 1$}}
      \State $y \gets x.p$
      \While{\red{$y \neq T.root \land x = y.c_{1}$}}
        \Comment{\blue{exit: $y = T.root \lor x \neq y.c_{1}$}}
        \State $x \gets y$
        \State $y \gets y.p$
      \EndWhile

      \pause
      \If{\red{$x = y.c_{1}$}} \Comment{\teal{$y = T.root \land x = y.c_{1}$}}
        \State \Return ``no predecessor''
      \pause
      \Else \Comment{\teal{$x \neq y.c_{1}$}}
        \State $j \gets 2$
        \While{$y.c_{j} \neq x$}
            \State $j \gets j + 1$
        \EndWhile
        \State \Return $(y, j-1)$ \Comment{\purple{$x = y.c_{j}$}}
      \EndIf
    \EndIf
  \EndProcedure
\end{algorithmic}
\end{algorithm}
\end{frame}
%%%%%%%%%%%%%%%