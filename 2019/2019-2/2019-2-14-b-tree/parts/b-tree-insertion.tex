% b-tree-insertion.tex

%%%%%%%%%%%%%%%
\begin{frame}{}
  \begin{exampleblock}{Insertion (TC 18.2-4$\;^{\red{\star}}$)}
    \begin{center}
      Suppose that we insert the keys \blue{$\set{1, 2, \dots, n}$} \red{in increasing order} \\[5pt]
      into an \purple{empty B-tree} with minimum degree 2. \\[6pt]
      How many nodes, \teal{denoted $X_{n}$}, does the final B-tree have?
    \end{center}
  \end{exampleblock}

  \pause
  \vspace{0.60cm}
  \fig{width = 0.10\textwidth}{figs/t0}
  \[
    \purple{X_{0} = 1}
  \]
\end{frame}
%%%%%%%%%%%%%%%

%%%%%%%%%%%%%%%
\begin{frame}{}
  \begin{center}
    By \blue{Yangjing Dong} (June 2018) \\[20pt]
    \href{https://maxmute.com/TC18.2-4.html}{\teal{https://maxmute.com/TC18.2-4.html}}
  \end{center}
\end{frame}
%%%%%%%%%%%%%%%

%%%%%%%%%%%%%%%
\begin{frame}{}
  \begin{center}
    Only \red{\textsc{split}} can create new nodes.
  \end{center}

  \pause
  \begin{columns}
    \column{0.50\textwidth}
      \fig{width = 0.90\textwidth}{figs/b-tree-root-split}

      \begin{center}
        root \textsc{split}
      \end{center}
      \[
        \red{+2}
      \]
    \pause
    \column{0.50\textwidth}
      \fig{width = 0.90\textwidth}{figs/b-tree-non-root-split}

      \begin{center}
        non-root \textsc{split}
      \end{center}
      \[
        \red{+1}
      \]
  \end{columns}
\end{frame}
%%%%%%%%%%%%%%%

%%%%%%%%%%%%%%%
\begin{frame}{}
  \begin{enumerate}[(I)]
    \centering
    \item Which nodes will \textsc{split}? \quad \teal{$S$} \\[15pt]
    \item \textcolor<3>{red}{When does each node \teal{$s \in S$} \textsc{split}?} 
      \quad \teal{$T_{s} = \set{s_{1}, s_{2}, \dots}$} \\[15pt]
    \item How does it \textsc{split}, as a root or a non-root?
      \quad \teal{$T_{s} = s_{R} \uplus s_{NR}$}
  \end{enumerate}

  \pause
  \vspace{0.60cm}
  \[
    X_{n} = 1 + \sum_{s \in S} \Big( 2\; |s_{R}| + |s_{NR}| \Big)
  \]
\end{frame}
%%%%%%%%%%%%%%%
% b-tree-insertion-which.tex

%%%%%%%%%%%%%%%
\begin{frame}{}
  \[
    X_{n} = 1 + \sum_{\red{s \in S}} \Big( 2\; |s_{R}| + |s_{NR}| \Big)
  \]
  
  \begin{enumerate}[(I)]
    \centering
    \item Which nodes will \textsc{split}? \\[15pt]
  \end{enumerate}

  \pause
  \fig{width = 0.95\textwidth}{figs/b-tree-30}
  \[
    1, 2, \dots, 30
  \]

  \[
    \red{S = \Big\{ \text{the nodes in the \blue{rightmost chain}} \Big\}}
  \]
\end{frame}
%%%%%%%%%%%%%%%
% b-tree-insertion-when.tex

%%%%%%%%%%%%%%%
\begin{frame}{}
  \begin{enumerate}[(I)]
    \setcounter{enumi}{1}
    \centering
    \item \red{When does each node $s \in S$ \textsc{split}?}
  \end{enumerate}

  \[
    T_{s} = \langle s_{1}, s_{2}, \dots \rangle
  \]

  \pause
  \vspace{0.80cm}
  \begin{center}
    Let's focus the \red{rightmost} node first, denoted $A$.
  \end{center}
\end{frame}
%%%%%%%%%%%%%%%

%%%%%%%%%%%%%%%
\begin{frame}{}
  \fig{width = 0.10\textwidth}{figs/t0}
  
  \begin{columns}
    \column{0.33\textwidth}
      \fig{width = 0.40\textwidth}{figs/b-tree-1}
    \column{0.33\textwidth}
      \fig{width = 0.70\textwidth}{figs/b-tree-2}
    \column{0.33\textwidth}
      \fig{width = 0.80\textwidth}{figs/b-tree-3}
  \end{columns}
\end{frame}
%%%%%%%%%%%%%%%

%%%%%%%%%%%%%%%
\begin{frame}{}
  \begin{columns}
    \column{0.33\textwidth}
      \fig{width = 0.80\textwidth}{figs/b-tree-4}
    \column{0.33\textwidth}
      \fig{width = 0.90\textwidth}{figs/b-tree-5}
    \column{0.33\textwidth}
      \fig{width = 0.90\textwidth}{figs/b-tree-6}
  \end{columns}
\end{frame}
%%%%%%%%%%%%%%%

%%%%%%%%%%%%%%%
\begin{frame}{}
  \begin{columns}
    \column{0.50\textwidth}
      \fig{width = 0.80\textwidth}{figs/b-tree-7}
    \column{0.50\textwidth}
      \fig{width = 0.80\textwidth}{figs/b-tree-8}
  \end{columns}
\end{frame}
%%%%%%%%%%%%%%%

%%%%%%%%%%%%%%%
\begin{frame}{}
  \begin{columns}
    \column{0.50\textwidth}
      \fig{width = 0.80\textwidth}{figs/b-tree-9}
    \column{0.50\textwidth}
      \fig{width = 0.80\textwidth}{figs/b-tree-10}
  \end{columns}
\end{frame}
%%%%%%%%%%%%%%%

%%%%%%%%%%%%%%%
\begin{frame}{}
  \begin{columns}
    \column{0.50\textwidth}
      \fig{width = 0.90\textwidth}{figs/b-tree-11}
    \column{0.50\textwidth}
      \fig{width = 0.90\textwidth}{figs/b-tree-12}
  \end{columns}

  \pause
  \vspace{1.00cm}
  \[
    \red{A\; \textsc{split}: 4,\quad 6,\quad 8,\quad 10,\quad 12,\quad \dots}
  \]
  % \[
  %   \blue{A\; \textsc{full}: 3,\quad 5,\quad 7,\quad 9,\quad 11,\quad \dots}
  % \]
\end{frame}
%%%%%%%%%%%%%%%

%%%%%%%%%%%%%%%
\begin{frame}{}
  \begin{enumerate}[(I)]
    \setcounter{enumi}{1}
    \centering
    \item \red{When does each node \textsc{split}?}
  \end{enumerate}

  \pause
  \vspace{0.60cm}
  \begin{center}
    Let's consider the \red{parent of $A$}, denoted $B \triangleq p(A)$. \\[30pt] \pause
    \blue{Every time $A$ splits, $B$ obtains a new key.}
  \end{center}
\end{frame}
%%%%%%%%%%%%%%%

%%%%%%%%%%%%%%%
\begin{frame}{}
  \begin{columns}
    \column{0.25\textwidth}
      \fig{width = 0.40\textwidth}{figs/b-tree-1}
    \column{0.25\textwidth}
      \fig{width = 0.75\textwidth}{figs/b-tree-2}
    \column{0.25\textwidth}
      \fig{width = 0.85\textwidth}{figs/b-tree-3}
    \column{0.25\textwidth}
      \fig{width = 0.90\textwidth}{figs/b-tree-4}
  \end{columns}

  \pause
  \vspace{0.50cm}
  \begin{columns}
    \column{0.50\textwidth}
      \fig{width = 0.70\textwidth}{figs/b-tree-5}
    \column{0.50\textwidth}
      \fig{width = 0.70\textwidth}{figs/b-tree-6}
  \end{columns}

  \pause
  \vspace{0.50cm}
  \begin{columns}
    \column{0.50\textwidth}
      \fig{width = 0.90\textwidth}{figs/b-tree-7}
    \column{0.50\textwidth}
      \fig{width = 0.90\textwidth}{figs/b-tree-8}
  \end{columns}
\end{frame}
%%%%%%%%%%%%%%%

%%%%%%%%%%%%%%%
\begin{frame}{}
  \begin{columns}
    \column{0.50\textwidth}
      \fig{width = 0.80\textwidth}{figs/b-tree-9}
    \column{0.50\textwidth}
      \fig{width = 0.80\textwidth}{figs/b-tree-10}
  \end{columns}

  \pause
  \vspace{0.50cm}
  \begin{columns}
    \column{0.50\textwidth}
      \fig{width = 0.90\textwidth}{figs/b-tree-11}
    \column{0.50\textwidth}
      \fig{width = 0.90\textwidth}{figs/b-tree-12}
  \end{columns}
\end{frame}
%%%%%%%%%%%%%%%

%%%%%%%%%%%%%%%
\begin{frame}{}
  \begin{columns}
    \column{0.50\textwidth}
      \fig{width = 0.80\textwidth}{figs/b-tree-13}
    \column{0.50\textwidth}
      \fig{width = 0.80\textwidth}{figs/b-tree-14}
  \end{columns}

  \pause
  \vspace{0.50cm}
  \begin{columns}
    \column{0.50\textwidth}
      \fig{width = 1.00\textwidth}{figs/b-tree-15}
    \column{0.50\textwidth}
      \fig{width = 1.00\textwidth}{figs/b-tree-16}
  \end{columns}

  \pause
  \vspace{1.00cm}
  \[
    \red{A\; \textsc{split}: 4,\quad 6,\quad 8,\quad 10,\quad 12,\quad \dots}
  \]
  \[
    \red{B\; \textsc{split}: 9,\quad 13,\quad 17,\quad 21,\quad 25,\quad \dots}
  \]
  % \[
  %   \blue{B\; \textsc{full}: 8,\quad 12,\quad 16,\quad 20,\quad 24,\quad \dots}
  % \]
\end{frame}
%%%%%%%%%%%%%%%

%%%%%%%%%%%%%%%
% \begin{frame}{}
%   \[
%     \red{A\; \textsc{split}: 4,\quad 6,\quad 8,\quad 10,\quad 12,\quad \dots}
%   \]
%   % \[
%   %   \blue{A\; \textsc{full}: 3,\quad 5,\quad 7,\quad 9,\quad 11,\quad \dots}
%   % \]
% 
%   \vspace{1.00cm}
%   \[
%     \red{B\; \textsc{split}: 9,\quad 13,\quad 17,\quad 21,\quad 25,\quad \dots}
%   \]
%   % \[
%   %   \blue{B\; \textsc{full}: 8,\quad 12,\quad 16,\quad 20,\quad 24,\quad \dots}
%   % \]
% \end{frame}
%%%%%%%%%%%%%%%

%%%%%%%%%%%%%%%
\begin{frame}{}
  \begin{enumerate}[(I)]
    \setcounter{enumi}{1}
    \centering
    \item \red{When does each node \textsc{split}?}
  \end{enumerate}

  \pause
  \vspace{0.80cm}
  \begin{center}
    Let's consider the \red{parent of $B$}, denoted $C = p(B)$.
  \end{center}
\end{frame}
%%%%%%%%%%%%%%%

%%%%%%%%%%%%%%%
\begin{frame}{}
  \[
    \teal{A\; \textsc{split}: 4,\quad 6,\quad 8,\quad 10,\quad 12,\quad \dots}
  \]
  % \[
  %   \blue{A\; \textsc{full}: 3,\quad 5,\quad 7,\quad 9,\quad 11,\quad \dots}
  % \]
  \[
    \teal{B\; \textsc{split}: 9,\quad 13,\quad 17,\quad 21,\quad 25,\quad \dots}
  \]
  % \[
  %   \blue{B\; \textsc{full}: 8,\quad 12,\quad 16,\quad 20,\quad 24,\quad \dots}
  % \]
  \[
    \teal{C\; \textsc{split}: 18,\quad 26,\quad 34,\quad 42,\quad 50,\quad \dots}
  \]

  \pause
  \hrule
  \[
    A: 1 \qquad B: 2 \qquad C: 3 \qquad
  \]

  \[
    \red{T_{i}: \text{ the first time point the $i$-th node splits}}
  \]

  \pause
  \[
    T_{1} = 4
  \]
  
  \pause
  \[
    \blue{T_{i} = \underbrace{T_{i-1}}_{\text{its right child first split}} + 
        \underbrace{2 \times 2^{i-1}}_{\text{its right child split twice more}} + 
        \underbrace{1}_{\text{insert one more}}}
  \]

  \pause
  \[
    \red{T_{i} = 2^{i+1} + i - 1}
  \]
\end{frame}
%%%%%%%%%%%%%%%
% b-tree-insertion-how.tex

%%%%%%%%%%%%%%%
\begin{frame}{}
  \[
    X_{n} = 1 + \sum_{s \in S} \Big( 2\; \red{|s_{R}|} + \red{|s_{NR}|} \Big)
  \]
  \[
    \red{(T_{s} = s_{R} \uplus s_{NR})}
  \]

  \begin{enumerate}[(I)]
    \setcounter{enumi}{2}
    \centering
    \item How does it \textsc{split}, as a root or a non-root?
  \end{enumerate}

  \pause
  \[
    \blue{s_{R} = \set{s_{1}} \qquad s_{NR} = \set{s_{2}, s_{3}, \dots}}
  \]
  \[
    \blue{|s_{R}| = 1 \qquad |s_{NR}| = |T_{s}| - 1}
  \]

  \pause
  \[
    X_{n} = 1 + \sum_{s \in S} \Big(2 + |T_{s}| - 1\Big) = 1 + \sum_{s \in S} \Big( |T_{s}| + 1 \Big)
  \]
\end{frame}
%%%%%%%%%%%%%%%
%%%%%%%%%%%%%%%

%%%%%%%%%%%%%%%
\begin{frame}{}
  \[
    \blue{X_{n} = 1 + \sum_{s \in S} \Big( |T_{s}| + 1 \Big)} \qquad
    \red{T_{i} = 2^{i+1} + i - 1}
  \]

  \pause
  \begin{align*}
    \onslide<3->{X_{n} &= 1 + \sum_{i = 1}^{\infty} \teal{[T_{i} \le n]} 
      \Big( \Big(\left\lfloor \frac{n - T_{i}}{2^{i}} \right\rfloor + 1 \Big) + 1 \Big) \\}
          \onslide<4->{&= 1 + \sum_{i = 1}^{\infty} \teal{[T_{i} \le n]} 
      \Big( \left\lfloor \frac{n - T_{i}}{2^{i}} \right\rfloor + 2 \Big) \\}
          \onslide<5->{&= 1 + \sum_{i = 1}^{\infty} \teal{[T_{i} \le n]} 
      \Big( \left\lfloor \frac{n - 2^{i+1} - i + 1}{2^{i}} \right\rfloor \Big) \\}
          \onslide<6->{&= 1 + \sum_{\substack{i \ge 1 \\ 2^{i+1} + i - 1 \le n}} 
      \left\lfloor \frac{n - i + 1}{2^{i}} \right\rfloor \\}
          \onslide<7->{&= 1 + \sum_{\substack{\red{i \ge 0} \\ 2^{i+2} + i \le n}} 
      \left\lfloor \frac{n - i}{2^{i + 1}} \right\rfloor}
  \end{align*}
\end{frame}
%%%%%%%%%%%%%%%