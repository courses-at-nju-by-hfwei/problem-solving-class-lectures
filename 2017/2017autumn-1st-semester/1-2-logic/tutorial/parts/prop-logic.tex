%%%%%%%%%%%%%%%
\begin{frame}{}
  \centerline{\LARGE 命题逻辑部分习题选讲}
\end{frame}
%%%%%%%%%%%%%%%

%%%%%%%%%%%%%%%
\begin{frame}{}
  \begin{exampleblock}{题目 2.1: 前提、结论}
    \begin{center}
      if \\[8pt]
      whenever \\[12pt]
      \red{\large only if} (只有 $\cdots$, 才 $\cdots$; 除非 $\cdots$)
    \end{center}
  \end{exampleblock}

  \pause
  \vspace{0.80cm}
  \begin{quote}
    \centerline{只有男足夺冠了/游戏打通关了,我才能安心学习。}
  \end{quote}

  % \pause
  % \begin{quote}
  %   \centerline{要想人不知,除非己莫为。}
  % \end{quote}
\end{frame}
%%%%%%%%%%%%%%%

%%%%%%%%%%%%%%%
\begin{frame}{}
  \begin{exampleblock}{题目 2.5:命题逻辑中的语义}
    \[
      (P \to (\lnot R \lor Q)) \land R
    \]

    \vspace{0.40cm}
    \centerline{\red{\large 真值表 (truth table)}~\footnote{``T/F'' 是元语言中的概念,不是命题逻辑中的概念。}}
  \end{exampleblock}

  \pause

  \[
    p \to q
  \]

  \begin{quote}
    \centerline{如果男足夺冠了/游戏打通关了,我就安心学习。}
  \end{quote}

  \pause
  \begin{exampleblock}{题目 2.8:运算优先级}
    \[
      P \land Q \lor R
    \]
  \end{exampleblock}
\end{frame}
%%%%%%%%%%%%%%%

%%%%%%%%%%%%%%%
\begin{frame}{}
  \begin{exampleblock}{题目 2.6:否定}
    If the stars are green or white horse is shining,
    then the world is eleven feet wide.
  \end{exampleblock}

  \vspace{0.50cm}

  \centerline{\red{\large (Yes or No?)}}
  The stars are green, the white horse is shining, but the world is not eleven feet wide.
\end{frame}
%%%%%%%%%%%%%%%

%%%%%%%%%%%%%%%
\begin{frame}{}
  \begin{exampleblock}{题目 2.7:永真式 (Tautology)}
    \begin{enumerate}
      \item $\lnot(\lnot P)$
      \item $\lnot(P \lor Q)$
      \item $\lnot(P \land Q)$
      \item $P \to Q$
    \end{enumerate}
  \end{exampleblock}

  \vspace{0.30cm}
  \begin{columns}
    \column{0.40\textwidth}
      \begin{enumerate}
	\item $P$
      \end{enumerate}
    \pause
    \column{0.40\textwidth}
      \begin{itemize}
	\item De Morgan's Law
	\item Implication
	  \[
	    (P \to Q) \leftrightarrow (\lnot P \lor Q)
	  \]
      \end{itemize}
  \end{columns}

  \pause
  \vspace{0.30cm}
  \begin{exampleblock}{题目 2.10:等价}
    ``It snows or it is not sunny.''
  \end{exampleblock}
\end{frame}
%%%%%%%%%%%%%%%

%%%%%%%%%%%%%%%
\begin{frame}{}
  \begin{exampleblock}{题目 3.2:逆否命题与逆命题}
    \[
      P \to Q
    \]
  \end{exampleblock}

  \[
    (P \to Q) \leftrightarrow \red{(?)}
  \]
\end{frame}
%%%%%%%%%%%%%%%

%%%%%%%%%%%%%%%
\begin{frame}{}
  \begin{exampleblock}{题目 3.6:Breakfast}
    Matilda always eats at least one of the following for breakfast: \\
    \begin{itemize}
      \item cereal, bread, or yogurt. 
    \end{itemize}
    On Monday, she is especially picky.

    \begin{itemize}
      \item If she eats cereal and bread, she also eats yogurt.
      \item If she eats bread or yogurt, she also eats cereal.
      \item She never eats both cereal and yogurt.
      \item She always eats bread or cereal.
    \end{itemize}

    Can you say what Matilda eats on Monday? If so, what does she eat?
  \end{exampleblock}
\end{frame}
%%%%%%%%%%%%%%%

%%%%%%%%%%%%%%%
\begin{frame}{}
  引入命题符号:你觉得这有什么问题吗?
  \begin{columns}
    \column{0.50\textwidth}
      \begin{description}
	\item[$A:$] Cereal
	\item[$B:$] Bread
	\item[$C:$] Yogurt
      \end{description}
    \column{0.50\textwidth}
      \begin{description}
	\item[$P:$] Cereal
	\item[$Q:$] Bread
	\item[$R:$] Yogurt
      \end{description}
  \end{columns}

  \vspace{0.30cm}
  \pause
  \fignocaption{width = 0.50\textwidth}{figs/text-color-puzzle.jpg}
\end{frame}
%%%%%%%%%%%%%%%

%%%%%%%%%%%%%%%
\begin{frame}{}
  一个有效的推理:你觉得这有什么问题吗?
  
  \fignocaption{width = 0.70\textwidth}{figs/breakfast-reasoning}
\end{frame}
%%%%%%%%%%%%%%%

%%%%%%%%%%%%%%%
\begin{frame}{}
  \begin{quote}
    \centerline{\red{\LARGE Let us calculate [calculemus].}}
  \end{quote}
\end{frame}
%%%%%%%%%%%%%%%

%%%%%%%%%%%%%%%
\begin{frame}{}
  \begin{exampleblock}{题目 3.7:四类命题}
    \begin{enumerate}
      \item 原命题
      \item 逆否命题
      \item 逆命题
      \item 否命题
    \end{enumerate}
  \end{exampleblock}

  \vspace{0.50cm}
  \centerline{原命题与其逆否命题等价。}
  \pause
  \vspace{0.50cm}
  \centerline{\red{\large 问题:}逆命题与否命题等价吗?}
\end{frame}
%%%%%%%%%%%%%%%

%%%%%%%%%%%%%%%
\begin{frame}{}
  \begin{exampleblock}{题目 3.11:利用逆否命题作证明}
    Prove that if $x$ is odd, then $\sqrt{2x}$ is not an integer.
  \end{exampleblock}
\end{frame}
%%%%%%%%%%%%%%%
