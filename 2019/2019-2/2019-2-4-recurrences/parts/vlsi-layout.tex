%%%%%%%%%%%%%%%
\begin{frame}{}
  \begin{exampleblock}{Problem (Area-Efficient VLSI Layout)}
    Embed a \red{complete binary tree} of $n$ nodes into a grid with minimum \purple{area}. \\[2pt]
    \begin{itemize}
      \item Complete binary tree circuit of 
	\[
	  \#\text{layer} = 3,5,7,\ldots
	\]
      \item Vertex on grid; no crossing edges
      \item Area:
	\[
	  \underbrace{A(n)}_\text{\teal{area}} = \underbrace{H(n)}_{\text{\teal{height}}} \times \underbrace{W(n)}_{\text{\teal{width}}}
	\]
    \end{itemize}
  \end{exampleblock}

  \pause
  \fig{width = 0.25\textwidth}{figs/vlsi-chip}
\end{frame}
%%%%%%%%%%%%%%%

%%%%%%%%%%%%%%%
\begin{frame}{}
  \fig{width = 0.80\textwidth}{figs/vlsi-binary-tree}

  \pause
  \[ 
    H(n) = H(\frac{n}{2}) + \Theta(1) = \Theta(\log n) 
  \]

  \pause
  \[ 
    W(n) = \red{2}W(\frac{n}{2}) + \Theta(1) = \Theta(n) 
  \]

  \pause
  \[ 
    \boxed{\teal{A(n) = \Theta(n \log n)}}
  \]
\end{frame}
%%%%%%%%%%%%%%%

%%%%%%%%%%%%%%%
\begin{frame}{}
  \[ 
    \boxed{\red{Q: \boxed{H(n)} \times \boxed{W(n)} = n}}
  \]
  
  \pause
  \[
    1 \times n
  \]

  \pause
  \[
    \frac{n}{\log n} \times \log n
  \]

  \pause
  \[
    \boxed{\red{\sqrt{n} \times \sqrt{n}}}
  \]

  \pause
  \[
    H(n) = \Theta(\sqrt{n}),\; W(n) = \Theta(\sqrt{n}),\; A(n) = \Theta(n)
  \]

  \pause
  \[
    H(n) = \Box H(\frac{n}{\Box}) + O(\Box) 
    % H(n) = 2 H(\frac{n}{4}) + O(n^{\frac{1}{2} - \epsilon})
  \]

  \pause
  \[ 
    \boxed{\red{H(n) = 2H(\frac{n}{4}) + \Theta(1)}}
  \]
\end{frame}
%%%%%%%%%%%%%%%

%%%%%%%%%%%%%%%
\begin{frame}{}
  \fig{width = 0.45\textwidth}{figs/h-layout}

  \pause
  \centerline{\teal{\Large $H$-layout}}
\end{frame}
%%%%%%%%%%%%%%%

%%%%%%%%%%%%%%%
\begin{frame}{}
  \fig{width = 0.50\textwidth}{figs/h-layout-2}

  \begin{center}
    {\it ``VLSI Theory and Parallel Supercomputing''}, Charles E. Leiserson, 1989.
  \end{center}
\end{frame}
%%%%%%%%%%%%%%%
