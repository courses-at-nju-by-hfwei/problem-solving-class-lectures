% 2-pmi.tex

%%%%%%%%%%%%%%%
\begin{frame}{}
  \begin{exampleblock}{Problem 2: Mathematical Induction}
    令 $n, k \in \mathbb{N}^{+}$, 且 $n \ge k$。
    请使用\red{数学归纳法}证明 
    \[
      \frac{n!}{k! (n-k)!}
    \]
    是整数。
  \end{exampleblock}
\end{frame}
%%%%%%%%%%%%%%%

%%%%%%%%%%%%%%%
\begin{frame}{}
  \begin{exampleblock}{Problem 2: Mathematical Induction (Original Version)}
    令 $n, k \in \mathbb{N}^{+}$, 且 $n \ge k$。
    \red{请证明} 
    \[
      \frac{n!}{k! (n-k)!}
    \]
    是整数。
  \end{exampleblock}

  \pause
  \[
    {n \choose k} \triangleq \frac{n!}{k! (n-k)!}
  \]

  \pause
  \begin{center}
    \red{Combinatorial Interpretation}  \\[15pt] \pause

    \blue{\red{\# of ways} to choose a $k$-element subset from an $n$-element set}
  \end{center}
\end{frame}
%%%%%%%%%%%%%%%

%%%%%%%%%%%%%%%
\begin{frame}{}
  \begin{exampleblock}{Problem 2: Mathematical Induction}
    令 $n, k \in \mathbb{N}^{+}$, 且 $n \ge k$。
    请使用\red{数学归纳法}证明 
    \[
      \frac{n!}{k! (n-k)!}
    \]
    是整数。
  \end{exampleblock}

  \pause
  \vspace{0.30cm}
  \begin{center}
    \uncover<3->{\red{\fbox{By induction on $\ldots$}}} \\[15pt]

    Base Case \\[8pt]
    Induction Hypothesis \\[8pt]
    Induction Step \\[15pt]
    % \fig{width = 0.70\textwidth}{figs/good-job}

    \uncover<4->{\red{\fbox{$n$ {\it vs.} $k$}}}
  \end{center}
\end{frame}
%%%%%%%%%%%%%%%

%%%%%%%%%%%%%%%
\begin{frame}{}
  \[
    {n \choose k} \triangleq \frac{n!}{k! (n-k)!}
  \]

  \pause
  \vspace{0.60cm}
  \[
    {n+1 \choose k} = {n \choose k} + {n \choose k-1}
  \]

  \pause
  \vspace{0.30cm}
  \begin{center}
    \red{\fbox{By induction on $n$.}}
  \end{center}

  \pause
  \[
    P(n) \triangleq \forall k \le n: {n \choose k} \in \mathbb{N}
  \]

  \pause
  \[
    \forall n \in \mathbb{N}^{+}: P(n)
  \]
\end{frame}
%%%%%%%%%%%%%%%

%%%%%%%%%%%%%%%
\begin{frame}
  \[
    {n \choose k} \triangleq \frac{n!}{k! (n-k)!}\; \red{\in \mathbb{N}}
  \]
  \[
    P(n) \triangleq \forall k \le n: {n \choose k} \in \mathbb{N}
  \]
  \begin{center}
    \red{\fbox{By induction on $n$.}}
  \end{center}

  \pause
  \vspace{0.30cm}
  \begin{columns}
    \column{0.40\textwidth}
      \begin{center}
	Base Case: $P(1)$
      \end{center}
      \[
	{1 \choose 1} = 1 \;\red{\in \mathbb{N}}
      \]
    \column{0.60\textwidth}
      \begin{center}
	Induction Hypothesis: Suppose that
      \end{center}
      \[
	P(n) \triangleq \forall k \le n: {n \choose k} \in \mathbb{N}
      \]
  \end{columns}
\end{frame}
%%%%%%%%%%%%%%%

%%%%%%%%%%%%%%%
\begin{frame}{}
  \begin{center}
    Induction Step: Consider $P(n+1)$.
    \[
      P(n+1) \triangleq \forall k \le n+1: {n+1 \choose k} \in \mathbb{N}
    \]

    \pause
    \[
      \forall k \le n+1: {n+1 \choose k} = \red{{n \choose k}} + {n \choose k-1}
    \]

    \pause
    \hrule
    \begin{columns}
      \column{0.50\textwidth}
	\begin{center}
	  \teal{\textsc{Case I}}
	  \[
	    k = n+1
	  \]
	  \[
	    {n+1 \choose n+1} = 1\; \red{\in \mathbb{N}}
	  \]
	\end{center}
      \column{0.50\textwidth}
        \pause
	\begin{center}
	  \teal{\textsc{Case II}}
	  \[
	    k \le n
	  \]
	  \[
	    {n+1 \choose k} = {n \choose k} + {n \choose k-1}\; \red{\in \mathbb{N}}
	  \]
	\end{center}
    \end{columns}
  \end{center}
\end{frame}
%%%%%%%%%%%%%%%

%%%%%%%%%%%%%%%
\begin{frame}
  \fig{width = 0.30\textwidth}{figs/concrete-mathematics}

  \begin{center}
    Chapter 5: Binomial Coefficients
  \end{center}
\end{frame}
%%%%%%%%%%%%%%%