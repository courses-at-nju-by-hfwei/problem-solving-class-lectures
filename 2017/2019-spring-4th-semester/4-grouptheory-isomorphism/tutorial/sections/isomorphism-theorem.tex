% isomorphism-theorem.tex

%%%%%%%%%%%%%%%%%%%%
\begin{frame}
  \fig{width = 0.30\textwidth}{figs/Noether}
  \centerline{\teal{Emmy Noether ($1882 \sim 1935$)}}
\end{frame}
%%%%%%%%%%%%%%%%%%%%

%%%%%%%%%%%%%%%%%%%%
\begin{frame}
  \begin{theorem}[The First Isomorphism Theorem]
	\[
	  \psi: G \to H \red{\implies} \frac{G}{\text{Ker } \psi} \cong \psi(G)
	\]
  \end{theorem}
  
  \begin{theorem}[The Second Isomorphism Theorem]
	\[
	  H \le G, N \triangleleft G \red{\implies} \frac{H}{H \cap N} \cong \frac{HN}{N} 
	\]
  \end{theorem}

  \begin{theorem}[The Third Isomorphism Theorem]
	\[
	  N \triangleleft H \triangleleft G \red{\implies} \frac{G}{H} \cong \frac{G/N}{H/N}
	\]
  \end{theorem}

  \begin{theorem}[The Fourth Isomorphism Theorem \purple{(Correspondence)}]
	\vspace{-0.50cm}
	\begin{gather*}
	  N \triangleleft G \red{\implies} \\
	  \big\{\text{\blue{(normal)} subgroups of $G$ containing $N$}\big\} 
	  \leftrightarrow \big\{\text{\blue{(normal)} subgroups of $G/N$}\big\}
	\end{gather*}
  \end{theorem}
\end{frame}
%%%%%%%%%%%%%%%%%%%%

%%%%%%%%%%%%%%%%%%%%
\begin{frame}
  \begin{theorem}[The Fourth Isomorphism Theorem \purple{(Correspondence)}]
	\vspace{-0.50cm}
	\begin{gather*}
	  N \triangleleft G \red{\implies} \\
	  \big\{\text{\blue{(normal)} subgroups of $G$ containing $N$}\big\} 
	  \leftrightarrow \big\{\text{\blue{(normal)} subgroups of $G/N$}\big\}
	\end{gather*}
  \end{theorem}

  \pause
  \fig{width = 0.35\textwidth}{figs/4th-iso}
\end{frame}
%%%%%%%%%%%%%%%%%%%%

%%%%%%%%%%%%%%%%%%%%
\begin{frame}
  \begin{theorem}[The First Isomorphism Theorem]
	\[
	  \psi: G \to H \red{\implies} \frac{G}{\text{Ker } \psi} \cong \psi(G)
	\]
  \end{theorem}

  \pause
  \begin{center}
	\red{\large $Q:$ \it What if $\psi$ is injective?} \pause
	\[
	  G \cong \psi(G)
	\]
	\pause
	\red{\large $Q:$ \it How to decide whether $\psi$ is injective or not?}
  \end{center}

  \pause
  \begin{theorem}[$\text{Ker } \psi$ and Injectivity]
	\[
	  \psi: G \to H \text{ is injective } \red{\iff} \kernel \psi = \set{e_G}
	\]
  \end{theorem}

  \pause
  \vspace{-0.80cm}
  \begin{center}
	\[
	  \violet{\boxed{\frac{G}{\kernel \psi}: \text{Quotient } G \text{ out by } \kernel \psi}}
	\]
  \end{center}
\end{frame}
%%%%%%%%%%%%%%%%%%%%

%%%%%%%%%%%%%%%%%%%%
\begin{frame}
  \begin{columns}
	\column{0.60\textwidth}
	  \begin{align*}
		\rho_{1} = (2\; 3\; 4)\quad \rho_{1}^{2} = (2\; 4\; 3) \\[3pt]
		\rho_{2} = (1\; 3\; 4)\quad \rho_{2}^{2} = (1\; 4\; 3) \\[3pt]
		\rho_{3} = (1\; 2\; 4)\quad \rho_{3}^{2} = (1\; 4\; 2) \\[3pt]
		\rho_{4} = (1\; 2\; 3)\quad \rho_{4}^{2} = (1\; 3\; 2)
	  \end{align*}
	\column{0.40\textwidth}
	  \begin{align*}
		r_1 = (1\; 4) (2\; 3) \\[6pt]
		r_2 = (1\; 2) (3\; 4) \\[6pt]
		r_3 = (1\; 3) (2\; 4)
	  \end{align*}
  \end{columns}

  \vspace{0.60cm}
  \[
	\red{\boxed{\sym(T) \cong A_4 = \set{\text{id},\quad \underbrace{\text{3-cycle}}_{\# = 8},\quad \underbrace{\text{2-2-cycle}}_{\# = 3}}}}
  \]

  \pause
  \[
	\frac{A_4}{\set{1, r_1, r_2, r_3}} \cong C_3
  \]
\end{frame}
%%%%%%%%%%%%%%%%%%%%

%%%%%%%%%%%%%%%%%%%%
\begin{frame}
  \fig{width = 0.75\textwidth}{figs/iso-a4toc3}
  \[
	\phi: A_4 \to C_3 \qquad \red{(\kernel \phi = \set{1, x, y, z})}
  \]
\end{frame}
%%%%%%%%%%%%%%%%%%%%

%%%%%%%%%%%%%%%%%%%%
\begin{frame}
  \begin{theorem}[The First Isomorphism Theorem]
	\[
	  \psi: G \to H \red{\implies} \frac{G}{\text{Ker } \psi} \cong \psi(G)
	\]
  \end{theorem}

  \pause
  \[
	\text{To show } \frac{G_1}{N} \cong G_2.
  \]
\end{frame}
%%%%%%%%%%%%%%%%%%%%

%%%%%%%%%%%%%%%%%%%%
\begin{frame}
  \[
	\frac{\z \times \z}{\langle (1, 1) \rangle} \cong \z
  \]

  \pause
  \[
	f: \z \times \z \to \z
  \]

  \pause
  \[
	f(m, n) = m - n
  \]

  \pause
  \[
	\kernel f = \langle (1, 1) \rangle
  \]
  % \pause
  % \vspace{0.80cm}
  % \[
  %   \mathbb{R}/\z \cong \mathbb{T} \qquad (\mathbb{T} = \set{z \in \mathbb{C}: |z| = 1})
  % \]
\end{frame}
%%%%%%%%%%%%%%%%%%%%

%%%%%%%%%%%%%%%%%%%%
\begin{frame}
  \begin{theorem}[The Second Isomorphism Theorem]
	\[
	  H \le G, N \triangleleft G \red{\implies} \frac{H}{H \cap N} \cong \frac{HN}{N} 
	\]
  \end{theorem}

  \pause
  \fig{width = 0.40\textwidth}{figs/2nd-iso-texse}
\end{frame}
%%%%%%%%%%%%%%%%%%%%

%%%%%%%%%%%%%%%%%%%%
\begin{frame}
  \begin{theorem}[The Second Isomorphism Theorem]
	\[
	  H \le G, N \triangleleft G \red{\implies} \frac{H}{H \cap N} \cong \frac{HN}{N} 
	\]
  \end{theorem}

  \pause
  \[
	\red{\it \text{What if } H \cap N = \set{e}?}
  \]

  \pause
  \[
	H \cong \frac{HN}{N}
  \]
  \pause
  \[
	h \in H \leftrightarrow hN \;\teal{\subseteq}\; HN 
  \]

  \pause
  \[
	\red{\it \text{What if } h \in H \cap N\; (h \neq e)?}
  \]
  
  \pause
  \[
	h \in H \cap N \implies hN = N
  \]
\end{frame}
%%%%%%%%%%%%%%%%%%%%

%%%%%%%%%%%%%%%%%%%%
\begin{frame}
  \begin{theorem}[The Second Isomorphism Theorem]
	\[
	  H \le G, N \triangleleft G \red{\implies} \frac{H}{H \cap N} \cong \frac{HN}{N} 
	\]
  \end{theorem}

  \vspace{0.20cm}
  \begin{exampleblock}{Problem $11.4$-$7$}
	\[
	  G = \z_{24},\quad H = \langle 4 \rangle,\quad N = \langle 6 \rangle
	\]
  \end{exampleblock}

  \pause
  \[
	H \cap N = \pause \langle 12 \rangle
  \]

  \pause
  \vspace{-0.80cm}
  \[
	HN = \pause \langle 2 \rangle \pause \;\red{= \bigcup_{h \in H} hN}
  \]

  \pause
  \[
	\frac{H}{H \cap N} \cong \frac{HN}{N} \blue{\implies} 
	\frac{\langle 4 \rangle}{\langle 12 \rangle} \cong \frac{\langle 2 \rangle}{\langle 6 \rangle}
  \]

  \pause
  \[
	ab = \text{gcd}(a, b) \cdot \text{lcm}(a,b)
  \]
\end{frame}
%%%%%%%%%%%%%%%%%%%%

%%%%%%%%%%%%%%%%%%%%
\begin{frame}
  \begin{theorem}[The Third Isomorphism Theorem]
	\[
	  N \triangleleft H \triangleleft G \red{\implies} \frac{G}{H} \cong \frac{G/N}{H/N}
	\]
  \end{theorem}

  \begin{columns}
	\column{0.50\textwidth}
	  \pause
	  \fig{width = 0.60\textwidth}{figs/cancel}
	\column{0.50\textwidth}
	  \pause
	  \fig{width = 0.70\textwidth}{figs/no-not-really}
  \end{columns}
\end{frame}
%%%%%%%%%%%%%%%%%%%%

%%%%%%%%%%%%%%%%%%%%
\begin{frame}
  \begin{theorem}[The Third Isomorphism Theorem]
	\[
	  N \triangleleft H \triangleleft G \red{\implies} \frac{G}{H} \cong \frac{G/N}{H/N}
	\]
  \end{theorem}

  \begin{table}[]
	\begin{tabular}{|c|c|c|c|c|c|}
	\hline
	 &  &  &  &  &  \\ \hline
	 &  &  &  &  &  \\ \hline
	 &  &  &  &  &  \\ \hline
	 &  &  &  &  &  \\ \hline
	 &  &  &  &  &  \\ \hline
	\end{tabular}
  \end{table}

  \begin{center}
	\blue{\large \it View $G$ and $H$ from the point of view of $N$}
  \end{center}
\end{frame}
%%%%%%%%%%%%%%%%%%%%

%%%%%%%%%%%%%%%%%%%%
\begin{frame}
  \begin{theorem}[The Third Isomorphism Theorem]
	\[
	  N \triangleleft H \triangleleft G \red{\implies} \frac{G}{H} \cong \frac{G/N}{H/N}
	\]
  \end{theorem}
  
  \begin{center}
	\red{\it $Q:$ What do the elements in $\frac{G}{H}$ look like?} \pause
	\[
	  gH \in \frac{G}{H}
	\]
	\pause
	\red{\it $Q:$ What do the elements in $\frac{G/N}{H/N}$ look like?} \pause
	\[
	  gN \cdot (H/N)
	\]
  \end{center}

  \pause
  \vspace{-0.60cm}
  \[
	\blue{\boxed{gN \cdot (H/N) \mapsto gH}}
  \]

  \pause
  \begin{center}
	\red{\large Absorption!!!}
  \end{center}
\end{frame}
%%%%%%%%%%%%%%%%%%%%

%%%%%%%%%%%%%%%%%%%%
\begin{frame}
  \begin{theorem}[The Third Isomorphism Theorem]
	\[
	  N \triangleleft H \triangleleft G \red{\implies} \frac{G}{H} \cong \frac{G/N}{H/N}
	\]
  \end{theorem}

  \vspace{0.60cm}
  \begin{columns}
	\column{0.50\textwidth}
	  \[
		n \mid m
	  \]

	  \[
		m\z \triangleleft n\z \triangleleft \z
	  \]

	  \pause
	  \[
		\frac{\z}{n\z} \cong \frac{\z/m\z}{n\z/m\z}
	  \]
	\column{0.50\textwidth}
	  \pause
	  \[
		10\z \triangleleft 2\z \triangleleft \z
	  \]

	  \[
		\frac{\z}{2\z} \cong \frac{\z/10\z}{2\z/10\z}
	  \]

	  \pause
	  \[
		\set{0,1} \cong \frac{\set{0, 1, 2, \cdots, 9}}{\uncover<5->{\red{\set{0, 2, 4, 6, 8}}}}
	  \]
  \end{columns}
\end{frame}
%%%%%%%%%%%%%%%%%%%%

%%%%%%%%%%%%%%%%%%%%
% \begin{frame}
%   \begin{theorem}[The Fourth Isomorphism Theorem \purple{(Correspondence)}]
% 	\vspace{-0.50cm}
% 	\begin{gather*}
% 	  N \triangleleft G \red{\implies} \\
% 	  \big\{\text{\blue{(normal)} subgroups of $G$ containing $N$}\big\} 
% 	  \leftrightarrow \big\{\text{\blue{(normal)} subgroups of $G/N$}\big\}
% 	\end{gather*}
%   \end{theorem}
% \end{frame}
%%%%%%%%%%%%%%%%%%%%
