%%%%%%%%%%%%%%%
\begin{frame}{}
  \centerline{\Large Set Operations}

  \[
    \text{\Large $\cap \qquad \cup \qquad \setminus$}
  \]
\end{frame}
%%%%%%%%%%%%%%%

%%%%%%%%%%%%%%%
\begin{frame}{}
  \begin{exampleblock}{UD $7.1 \;(b)$: Distributive Property}
    \[
      A \cap (B \cup C) = (A \cap B) \cup (A \cap C)
    \]
  \end{exampleblock}

  \pause
  \vspace{0.50cm}
  \begin{theorem}[Distributive Property (Theorem $7.1$)]
    \[
      A \cup (B \cap C) = (A \cup B) \cap (A \cup C)
    \]
  \end{theorem}

  \begin{proof}
    % \fignocaption{width = 0.30\textwidth}{figs/distributive-proof}
  \end{proof}
\end{frame}
%%%%%%%%%%%%%%%

%%%%%%%%%%%%%%%
\begin{frame}{}
  \begin{exampleblock}{UD $7.1 \;(c)$: DeMorgan's Law}
    Let $X$ denote a set, and $A, B \subseteq X$.

    \[
      X \setminus (A \cap B) = (X \setminus A) \cup (X \setminus B)
    \]
  \end{exampleblock}

  \pause
  \vspace{0.50cm}
  \[
    \red{Q: A, B \subseteq X?}
  \]
\end{frame}
%%%%%%%%%%%%%%%

%%%%%%%%%%%%%%%
\begin{frame}{}
  \begin{exampleblock}{UD $7.1 (d)$}
    Let $X$ denote a set, and $A, B \subseteq X$.

    \[
      A \subseteq B \iff (X \setminus B) \subseteq (X \setminus A)
    \]
  \end{exampleblock}

  \pause
  \vspace{0.50cm}
  \fignocaption{width = 0.35\textwidth}{figs/known-unknown}

  \pause
  \[
    \red{Q: A, B \subseteq X?} \pause \qquad \blue{(``\Leftarrow: X = \emptyset")}
  \]
\end{frame}
%%%%%%%%%%%%%%%

%%%%%%%%%%%%%%%
\begin{frame}{}
  \begin{exampleblock}{UD $7.8$}
    Consider the following sets:
    \begin{columns}
      \column{0.50\textwidth}
	\begin{enumerate}[(i)]
	  \item $(A \cap B) \setminus (A \cap B \cap C)$
	  \item \textcolor<2->{red}{$A \cap B \setminus (A \cap B \cap C)$}
	  \item $A \cap B \cap C^{c}$
	  \item $(A \cap B) \setminus C$
	  \item $(A \setminus C) \cap (B \setminus C)$
	\end{enumerate}
      \column{0.50\textwidth}
	\uncover<3->{
	  \fignocaption{width = 0.70\textwidth}{figs/A-B-Cc}
	}
    \end{columns}

    \vspace{0.30cm}
    \begin{enumerate}[(a)]
      \item Which of the sets above are written ambiguously, if any?
      % \item Of the sets above that make sense, which ones equal the set sketched in Figure 7.2?
      \setcounter{enumi}{2}
      \item Prove that $(A \cap B) \setminus C = (A \setminus C) \cap (B \setminus C)$.
    \end{enumerate}
  \end{exampleblock}

  \uncover<4->{
    \[
      A \setminus C = A \cap \red{C^c}
    \]
  }

  \vspace{-0.60cm}
  \uncover<5->{
    \[
      A \setminus C = \set{x \mid x \in A \land x \notin C}
    \]
  }
\end{frame}
%%%%%%%%%%%%%%%

%%%%%%%%%%%%%%%
\begin{frame}{}
  \begin{exampleblock}{UD $7.9$}
    Prove that the union of two sets can be rewritten as the union of two \red{disjoint} sets.
    \begin{enumerate}[(a)]
      \item Prove that $(A \setminus B) \cap B = \emptyset$
      \item Prove that $A \cup B = (A \setminus B) \cup B$
    \end{enumerate}
  \end{exampleblock}

  \vspace{0.30cm}
  \begin{columns}
    \column{0.35\textwidth}
      \uncover<3->{
	\centerline{By contradiction.}
      }
    \column{0.35\textwidth}
      \pause
      \fignocaption{width = 0.80\textwidth}{figs/leng}{\centerline{``太容易了,一时没反应过来''}}
    \column{0.35\textwidth}
      \uncover<4->{
	\[
	  (A \setminus B) \cup B = \cdots
	\]
      }
  \end{columns}
\end{frame}
%%%%%%%%%%%%%%%

%%%%%%%%%%%%%%%
% \begin{frame}{}
%   \begin{exampleblock}{UD $7.11$}
%     Let $X$ be the\only<2->{\purple{\;temporary}} universe and $A, B \subseteq X$.
% 
%     \[
%       (\forall Y \subseteq X: A \cap Y = B \cap Y) \implies A = B.
%     \]
%   \end{exampleblock}
% 
%   \pause
%   \vspace{0.50cm}
%   \begin{proof}
%     \[
%       Y = A  \qquad Y = B
%     \]
% 
%     \pause
%     \[
%       Y = X
%     \]
%   \end{proof}
% \end{frame}
%%%%%%%%%%%%%%%
