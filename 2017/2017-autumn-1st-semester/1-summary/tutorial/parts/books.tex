%%%%%%%%%%%%%%%
\begin{frame}{为什么列书单?}
  \begin{enumerate}
    \setlength{\itemsep}{8pt}
    \item 我感受到了大家强烈的求知欲
    \item 希望大家在本科阶段多读一些书
    \item 有书读多快乐啊
  \end{enumerate}

  \vspace{0.80cm}
  \begin{quote}
    \centering
    ``列书单就是把自己也没读过的书\\[3pt]
    推荐给同样不会去读的其他人,\\[3pt]
    以达到大家都已读过这些书的假象。''
  \end{quote}
\end{frame}
%%%%%%%%%%%%%%%

%%%%%%%%%%%%%%%
\begin{frame}{读什么书?}
  \begin{enumerate}
    \setlength{\itemsep}{6pt}
    \centering
    \item 经典性
    \item 专业性
    \item 趣味性
  \end{enumerate}
\end{frame}
%%%%%%%%%%%%%%%

%%%%%%%%%%%%%%%
\begin{frame}{如何读?}
  \begin{center}
    不要指望一遍就能读懂\\[10pt]

    交叉阅读, 多角度思考\\[10pt]

    三五人同读一本书\\
  \end{center}
\end{frame}
%%%%%%%%%%%%%%%

%%%%%%%%%%%%%%%
\begin{frame}{逻辑}
  \begin{columns}
    \column{0.45\textwidth}
      \fignocaption{width = 0.55\textwidth}{figs/fudan-logic}
    \column{0.55\textwidth}
      \begin{description}
	\setlength{\itemsep}{6pt}
	\item[第一章] 预备知识 \\
	  快速复习集合论知识
	\item[第二章] 命题逻辑 \\
	  \begin{itemize}
	    \item 语法\emph{vs.}语义
	    \item 公理系统
	    \item 可靠性\emph{vs.}完全性
	  \end{itemize}
      \end{description}
  \end{columns}
\end{frame}
%%%%%%%%%%%%%%%

%%%%%%%%%%%%%%%
\begin{frame}{集合论}
  \begin{columns}
    \column{0.45\textwidth}
      \fignocaption{width = 0.55\textwidth}{figs/fudan-set-theory-book}
    \column{0.45\textwidth}
      \begin{center}
	前三章\\[3pt]
	重在基本概念\\[3pt]
	理清概念之间的逻辑关系
      \end{center}
  \end{columns}
\end{frame}
%%%%%%%%%%%%%%%

%%%%%%%%%%%%%%%
\begin{frame}{逻辑与集合论科普书籍}
  \begin{center}
    读故事, 了解历史发展; 不求甚解
  \end{center}

  \begin{columns}
    \column{0.45\textwidth}
      \fignocaption{width = 0.60\textwidth}{figs/engines-of-logic}
    \column{0.45\textwidth}
      \fignocaption{width = 0.60\textwidth}{figs/book-paradox-crisis}
  \end{columns}
\end{frame}
%%%%%%%%%%%%%%%

%%%%%%%%%%%%%%%
\begin{frame}{程序设计}
  \begin{columns}
    \column{0.45\textwidth}
      \fignocaption{width = 0.60\textwidth}{figs/cplusplus-bjarne-book}
    \column{0.45\textwidth}
      \begin{center}
	C++ 语言设计者之作\\[3pt]
	大块头, 有难度\\[8pt]
	宜: 慢读深思\\[3pt]
	忌: 只读不练\\[15pt]

	初读可跳过第2--5章\\
      \end{center}
  \end{columns}
\end{frame}
%%%%%%%%%%%%%%%

%%%%%%%%%%%%%%%
\begin{frame}{数据结构与算法}
  \begin{columns}
    \column{0.45\textwidth}
      \fignocaption{width = 0.90\textwidth}{figs/leiserson}
    \column{0.45\textwidth}
      \fignocaption{width = 0.45\textwidth}{figs/erik-demaine}
  \end{columns}

  \begin{center}
    ``Algorithms''\\[4pt]
    \href{http://open.163.com/special/opencourse/algorithms.html}{MIT 6.046/18.410, Fall 2005}\\[3pt]
    \href{https://ocw.mit.edu/courses/electrical-engineering-and-computer-science/6-006-introduction-to-algorithms-fall-2011/lecture-videos/}{MIT 6.006, Fall 2011}\\[10pt]

    讲得太好, Enjoy it!
  \end{center}
\end{frame}
%%%%%%%%%%%%%%%

%%%%%%%%%%%%%%%
\begin{frame}{数学}
  \begin{columns}
    \column{0.45\textwidth}
      \fignocaption{width = 0.60\textwidth}{figs/book-concrete-mathematics}
    \column{0.60\textwidth}
      \begin{center}
	Knuth 眼中的``计算机科学中的数学''\\[10pt]
	切勿望而却步\\[8pt]
	读进去\\
	就(\teal{才})不(\teal{知})会(\teal{道})觉(\teal{有})得(\teal{多})难\\[3pt]
      \end{center}
  \end{columns}
\end{frame}
%%%%%%%%%%%%%%%