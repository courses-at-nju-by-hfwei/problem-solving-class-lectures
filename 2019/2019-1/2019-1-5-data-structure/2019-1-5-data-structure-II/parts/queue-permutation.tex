%%%%%%%%%%%%%%%
\begin{frame}{}
  \centerline{\LARGE Queueable Permutations}

  \vspace{0.30cm}
  \fig{width = 0.35\textwidth}{figs/queue-child}
\end{frame}
%%%%%%%%%%%%%%%

%%%%%%%%%%%%%%%
\begin{frame}{}
  \begin{exampleblock}{DH 2.14: Queueable Permutations}
    \uncover<2->{
      \[
	\fbox{$\texttt{out} = (a_1, \cdots, a_n) \;\blue{\xleftarrow[X \;=\; \bot]{Q \;=\; \emptyset}}\; \texttt{in} = (1, \cdots, n)$}
      \]
    }

    \fig{width = 0.80\textwidth}{figs/queue-perm-x}
  \end{exampleblock}
\end{frame}
%%%%%%%%%%%%%%%

%%%%%%%%%%%%%%%
\begin{frame}{}
  \begin{exampleblock}{DH 2.14: Queueable Permutations}
    \begin{enumerate}[(a)]
      \setcounter{enumi}{1}
      \item Prove that every permutation are \blue{queueable}.
    \end{enumerate}
  \end{exampleblock}

  \begin{columns}
    \column{0.50\textwidth}
      \fig{width = 0.90\textwidth}{figs/queue-perm-x}
    \column{0.50\textwidth}
      \pause
      % queue-perm-alg.tex

\begin{algorithm}[H]
  \begin{algorithmic}[1]
    \Procedure{Queueable}{$out$}
      \ForAll{$a \in in$}
	\State $\textsl{read}(X)$
	\State $\textsl{add}(X, Q)$
      \EndFor

      \hStatex
      \ForAll{$a \in out$}
	\While{$\teal{\textsl{Head}(Q) \neq a}$}
	  \State $\textsl{remove}(X, Q)$
	  \State $\textsl{add}(X, Q)$
	\EndWhile
        
	\hStatex
	\State $\textsl{remove}(X, Q)$
	\State $\textsl{print}(X)$
      \EndFor
    \EndProcedure
  \end{algorithmic}
\end{algorithm}

  \end{columns}
\end{frame}
%%%%%%%%%%%%%%%

%%%%%%%%%%%%%%%
\begin{frame}{}
  \begin{exampleblock}{DH 2.14: Queueable Permutations}
    \begin{enumerate}[(a)]
      \setcounter{enumi}{2}
      \item Prove that every permutation can be obtained by \blue{two stacks}.
    \end{enumerate}
  \end{exampleblock}

  \begin{columns}
    \column{0.45\textwidth}
      \fig{width = 1.00\textwidth}{figs/two-stacks-perm-x}
    \column{0.60\textwidth}
      \pause
      % two-stack-perm-alg.tex

\begin{algorithm}[H]
  \begin{algorithmic}[1]
    \Procedure{DoubleStackable}{$out$}
      \ForAll{$a \in in$}
	\State $\textsl{read}(X)$
	\State $\textsl{push}(X, S)$
      \EndFor

      \hStatex
      \ForAll{$a \in out$}
	\While{$\teal{\textsl{top}(S) \neq a}$}
	  \State $\textsl{pop}(X, S)$
	  \State $\textsl{push}(X, S')$
	\EndWhile
        
	\hStatex
	\State $\textsl{pop}(X, S)$
	\State $\textsl{print}(X)$

	\hStatex
	\While{$S' \neq \emptyset$}
	  \State $\textsl{pop}(X, S')$
	  \State $\textsl{push}(X, S)$
	\EndWhile
      \EndFor
    \EndProcedure
  \end{algorithmic}
\end{algorithm}

  \end{columns}
\end{frame}
%%%%%%%%%%%%%%%

%%%%%%%%%%%%%%%
\begin{frame}{}
  \begin{columns}
    \column{0.60\textwidth}
      \fig{width = 0.90\textwidth}{figs/queue-perm-x}
      \begin{center}
	\blue{\it All are queueable.}
      \end{center}
    \column{0.40\textwidth}
      \fig{width = 0.45\textwidth}{figs/not-interesting}
  \end{columns}

  \vspace{0.80cm}
  \begin{columns}
    \pause
    \column{0.60\textwidth}
      \fig{width = 0.90\textwidth}{figs/queue-perm}
      \begin{center}
	\blue{\it Only one is queueable.}
      \end{center}
    \pause
    \column{0.40\textwidth}
      \fig{width = 0.60\textwidth}{figs/boring}
  \end{columns}
\end{frame}
%%%%%%%%%%%%%%%

%%%%%%%%%%%%%%%
\begin{frame}{}
  \begin{columns}
    \column{0.60\textwidth}
      \fig{width = 0.90\textwidth}{figs/queue-perm-x-no-circle}

      \pause
      \vspace{0.50cm}
      \fig{width = 0.90\textwidth}{figs/queue-perm-direct}

      \pause
      \[
	\mbox{\Large \red{$3\; 2\; 1$ \text{\it\; is not queueable}}}
      \]
    \column{0.40\textwidth}
      \pause
      \fig{width = 0.60\textwidth}{figs/yisi}
  \end{columns}
\end{frame}
%%%%%%%%%%%%%%%

%%%%%%%%%%%%%%%
\begin{frame}
  \begin{theorem}[Queueable Permutations]
     A permutation $(a_1, \cdots, a_n)$ is \red{queueable} $\iff$ it is not the case that
     \[
       321\text{\it -Pattern}: \fbox{$\texttt{\it out} = \cdots a_i \cdots a_j \cdots a_k: i < j < k \land a_i > a_j > a_k$}
     \]
  \end{theorem}

  \pause
  \vspace{0.30cm}
  \begin{proof}
    \fig{width = 0.40\textwidth}{figs/your-turn}
  \end{proof}
\end{frame}
%%%%%%%%%%%%%%%

%%%%%%%%%%%%%%%
\begin{frame}{}
  \begin{theorem}[\# of Queueable Permutations]
    The number of queueable permutations of $[1 \cdots n]$ is ${2n \choose n} - {2n \choose n-1}$.
  \end{theorem}

  \pause
  \vspace{0.30cm}
  \begin{proof}
    \fig{width = 0.40\textwidth}{figs/your-turn}
  \end{proof}
\end{frame}
%%%%%%%%%%%%%%%
