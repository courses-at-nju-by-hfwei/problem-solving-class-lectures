% file: parts/polygon-triangulation.tex

%%%%%%%%%%%%%%%
\begin{frame}{}
  \centerline{\teal{\Large Triangulating Polygons}}

  \fignocaption{width = 0.55\textwidth}{figs/polygon-triangulation}
\end{frame}
%%%%%%%%%%%%%%%

%%%%%%%%%%%%%%%
\begin{frame}{}
  \centerline{\teal{\large The Art Gallery Problem}}

  \begin{columns}
    \column{0.50\textwidth}
      \only<1-2>{\fignocaption{width = 0.90\textwidth}{figs/art-gallery}}
      \only<3->{\fignocaption{width = 0.75\textwidth}{figs/art-gallery-camera}}
    \column{0.50\textwidth}
      \fignocaption{width = 0.70\textwidth}{figs/watching}
  \end{columns}

  \vspace{0.80cm}
  \centerline{\large \red{$Q:$ How many ``BIG BROs'' to hire?}}
\end{frame}
%%%%%%%%%%%%%%%

%%%%%%%%%%%%%%%
\begin{frame}{}
  \begin{exampleblock}{Another Version of the Ear Lemma (Problem $4.1-16$)}
    A triangulated polygon is either a triangle with three ears or has at least two ears.
  \end{exampleblock}{}
\end{frame}
%%%%%%%%%%%%%%%

%%%%%%%%%%%%%%%
\begin{frame}{}
\end{frame}
%%%%%%%%%%%%%%%

%%%%%%%%%%%%%%%
\begin{frame}{}
  \begin{exampleblock}{\# of triangles (Problem $4.1-17$)}
    \fignocaption{width = 0.30\textwidth}{figs/polygon-triangulation}
    \[
      T(n) = n - 2
    \]
  \end{exampleblock}{}

  \begin{columns}
    \pause
    \column{0.50\textwidth}
      \fignocaption{width = 0.70\textwidth}{figs/polygon-ear}
    \pause
    \column{0.50\textwidth}
      \fignocaption{width = 0.70\textwidth}{figs/polygon-diagonal}
  \end{columns}
\end{frame}
%%%%%%%%%%%%%%%

%%%%%%%%%%%%%%%
\begin{frame}{}
  \begin{lemma}[Ear Lemma]
    A triangle has $3$ ears, and a larger \red{\large triangulated} polygon has at least $2$ non-adjacent ears.
  \end{lemma}

  \pause
  \vspace{0.80cm}
  \centerline{\red{\large $Q:$ Can every polygon be triangulated?}}
\end{frame}
%%%%%%%%%%%%%%%

%%%%%%%%%%%%%%%
\begin{frame}{}
  \begin{theorem}[Existence of Triangulation]
    Any polygon can be triangulated.
  \end{theorem}

  \pause
  \begin{proof}
    \begin{quote}
      {\large
	``\blue{To triangulate a polygon} one keeps \textcolor<6->{purple}{adding diagonals} connecting pairs of vertices 
      until no more diagonals can be added. \\[8pt] \pause

      These \textcolor<6->{purple}{diagonals} must lie entirely interior to the polygon and are not allowed to intersect. \\[8pt] \pause

      They break the interior of the polygon into a number of triangles,
      because any larger polygon can be split by adding a \textcolor<6->{purple}{diagonal}.''
    }
    \end{quote}
  \end{proof}

  \pause
  \begin{center}
    ``(\red{This fact is perhaps not obvious}, \\ but we won't get sidetracked by proving it here.)''
  \end{center}
\end{frame}
%%%%%%%%%%%%%%%

%%%%%%%%%%%%%%%
% \begin{frame}{}
%   \begin{definition}[Diagonal]
%     Given a simple polygon $P$, a \red{\emph{diagonal}} is a line segment between two non-adjacent vertices
%     that lies entirely within the interior of the polygon.
%   \end{definition}
% \end{frame}
%%%%%%%%%%%%%%%

%%%%%%%%%%%%%%%
\begin{frame}{}
  \begin{theorem}[Existence of Diagonal]
    Every polygon with $n > 3$ has a diagonal.
  \end{theorem}

  \pause
  \vspace{0.30cm}
  \begin{definition}[Convex Vertex]
    A vertex $v$ is \red{\it convex} if the \purple{\it interior} angle at $v$ is less than $180^\circ$.
  \end{definition}

  \pause
  \vspace{0.30cm}
  \fignocaption{width = 0.40\textwidth}{figs/convex-vertex} % existence-diagonal
\end{frame}
%%%%%%%%%%%%%%%

%%%%%%%%%%%%%%%
\begin{frame}{}
  \begin{theorem}[Coloring]
    Any triangulated polygon polygon is $3$-colorable.
  \end{theorem}

  \only<1>{\fignocaption{width = 0.40\textwidth}{figs/polygon-triangulation}}

  \only<2->{
    \vspace{0.30cm}
    \fignocaption{width = 0.80\textwidth}{figs/triangulation-3colorable}
  }
\end{frame}
%%%%%%%%%%%%%%%

%%%%%%%%%%%%%%%
\begin{frame}{}
  \centerline{\teal{\large The Art Gallery Problem}}

  \begin{columns}
    \column{0.50\textwidth}
      \only<1-2>{\fignocaption{width = 0.90\textwidth}{figs/art-gallery}}
      \only<3->{\fignocaption{width = 0.75\textwidth}{figs/art-gallery-camera}}
    \column{0.50\textwidth}
      \fignocaption{width = 0.70\textwidth}{figs/watching}
  \end{columns}

  \pause
  \vspace{0.80cm}
  \centerline{\large \red{$Q:$ How many ``BIG BROs'' to hire?}}
\end{frame}
%%%%%%%%%%%%%%%

%%%%%%%%%%%%%%%
\begin{frame}{}
  \fignocaption{width = 0.40\textwidth}{figs/triangulation-3coloring}

  \begin{theorem}[The Art Gallery Theorem ($O$)]
    For any art gallery with $n$ walls, $\lfloor \frac{n}{3} \rfloor$ ``BIG BROs'' suffice.
  \end{theorem}
\end{frame}
%%%%%%%%%%%%%%%

%%%%%%%%%%%%%%%
\begin{frame}{}
  \begin{theorem}[The Art Gallery Theorem ($\Omega$)]
    For any art gallery with $n$ walls, $\lfloor \frac{n}{3} \rfloor$ ``BIG BROs'' suffice.
  \end{theorem}

  \pause
  \vspace{0.80cm}
  \fignocaption{width = 0.50\textwidth}{figs/art-gallery-lowerbound}

  \[
    n = 3m
  \]
\end{frame}
%%%%%%%%%%%%%%%
