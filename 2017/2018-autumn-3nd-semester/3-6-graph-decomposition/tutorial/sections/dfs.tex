% file: sections/dfs-bfs.tex

%%%%%%%%%%%%%%%%%%%%
\begin{frame}{} 
  \centerline{\large Graph \red{\it structure} induced by DFS:}

  \vspace{0.50cm}
  \begin{columns}
    \column{0.50\textwidth}
      \begin{center}
	\purple{states} of \tikz{\node[draw, circle]{$v$}} \\[15pt]

	\purple{types} of \tikz{\node (u) [draw, circle]{$u$}; \node (v) [draw, circle, right = of u] {$v$}; \draw (u) -- (v);}
      \end{center}
    \pause
    \column{0.50\textwidth}
      \begin{center}
	\purple{life time} of \tikz{\node[draw, circle]{$v$}}: \\[15pt]

	$\brown{v}: \text{d}[v], \text{f}[v]$ \\[10pt]

	\brown{$\text{d}[v]$:} \textsc{Bicomp} \\[10pt]

	\brown{$\text{f}[v]$:} \textsc{Toposort}, \textsc{SCC}
      \end{center}
  \end{columns}
\end{frame}
%%%%%%%%%%%%%%%%%%%%

%%%%%%%%%%%%%%%%%%%%
\begin{frame}{}
  \begin{definition}[Classification of Edges]
    We can define four edge edges in terms of the depth-first forest $G_{\pi}$ produced by a DFS on $G$:
    \begin{description}[Forward edge:]
      \setlength{\itemsep}{5pt}
      \item[Tree edge:] edge in $G_{\pi}$
      \item[Back edge:] $\to$ ancestor
      \item[Forward edge:] $\to$ descendant (nontree edge)
      \item[Cross edge:] $\to$ $(\lnot \text{ancestor}) \land (\lnot \text{descendant})$
    \end{description}
  \end{definition}

  \fignocaption{width = 0.40\textwidth}{figs/dfs-digraph}
\end{frame}
%%%%%%%%%%%%%%%%%%%%

%%%%%%%%%%%%%%%%%%%%
\begin{frame}{}
  \begin{exampleblock}{DFS on \red{Undirected} Graphs (Problem $22.3$-$6$)}
    \begin{center}
      Classifying an edge $(u,v)$ as a tree edge or a back edge 
      according to \purple{whether $(u,v)$ or $(v,u)$ is encountered first} during the depth-first search \\[6pt] 

      \red{\large is equivalent to} \\[6pt]
      
      classifying it according to \purple{the ordering of the four types in the classification scheme}.
    \end{center}
  \end{exampleblock}

  \only<2>{
    \fignocaption{width = 0.40\textwidth}{figs/what-is-this}
  }
  \only<3>{
    \fignocaption{width = 0.38\textwidth}{figs/dfs-undirected}
  }
\end{frame}
%%%%%%%%%%%%%%%%%%%%

%%%%%%%%%%%%%%%%%%%%
\begin{frame}{}
  \begin{theorem}[Theorem $22.10$]
    In a depth-first search of an undirected graph $G$, 
    every edge of $G$ is either a tree edge or a back edge.
  \end{theorem}

  \begin{proof}
    Let $(u,v)$ be an arbitrary edge of $G$, 
    and suppose without loss of generality that $u.d < v.d$. 
    Then the search must discover and finish $v$ before it finishes $u$
    (while $u$ is gray), since $v$ is on $u$’s adjacency list. \\[6pt]

    If \textcolor<2->{red}{the first time} that the search explores edge $(u,v)$, 
    it is in the direction from $u$ to $v$, 
    then $v$ is undiscovered (white) until that time, 
    for otherwise the search would have explored this edge
    already in the direction from $v$ to $u$. 
    Thus, $(u,v)$ becomes a tree edge. \\[6pt]
    
    If the search explores $(u,v)$ \textcolor<2->{red}{first} in the direction from $v$ to $u$, 
    then $(u,v)$ is a back edge,
    since $u$ is still gray at the time the edge is first explored.
  \end{proof}
\end{frame}
%%%%%%%%%%%%%%%%%%%%

%%%%%%%%%%%%%%%%%%%%
\begin{frame}{}
  \fignocaption{width = 0.30\textwidth}{figs/keep-calm-prove-it}
\end{frame}
%%%%%%%%%%%%%%%%%%%%

%%%%%%%%%%%%%%%%%%%%
\begin{frame}{}
  \begin{exampleblock}{DFS on \red{Undirected} Graphs (Problem $22.3$-$6$)}
    \begin{center}
      Classifying an edge $(u,v)$ as a tree edge or a back edge 
      according to \purple{whether $(u,v)$ or $(v,u)$ is encountered first} during the depth-first search \\[6pt] 

      \red{\large is equivalent to} \\[6pt]
      
      classifying it according to \purple{the ordering of the four types in the classification scheme}.
    \end{center}
  \end{exampleblock}

  \vspace{0.60cm}
  \begin{table}[]
    \begin{tabular}{ccc}
      \teal{``First Types''} & {\it vs.} & \teal{``First Time''} \\[8pt]
      tree edge     & $\iff$    & tree edge    \\[6pt]
      back edge     & $\iff$    & back edge   
    \end{tabular}
  \end{table}
\end{frame}
%%%%%%%%%%%%%%%%%%%%

%%%%%%%%%%%%%%%%%%%%
\begin{frame}{}
  \begin{table}[]
    \begin{tabular}{ccc}
      \teal{``First Types''} & \purple{$\Longleftarrow$} & \teal{``First Time''} \\[8pt]
      tree edge     & \purple{$\Longleftarrow$}    & tree edge    \\[6pt]
      back edge     & \purple{$\Longleftarrow$}    & back edge   
    \end{tabular}
  \end{table}

  \begin{columns}
    \column{0.50\textwidth}
      \pause
      \fignocaption{width = 0.50\textwidth}{figs/dfs-undirected-firsttime-tree}
    \column{0.50\textwidth}
      \pause
      \fignocaption{width = 0.50\textwidth}{figs/dfs-undirected-firsttime-back}
  \end{columns}
\end{frame}
%%%%%%%%%%%%%%%%%%%%

%%%%%%%%%%%%%%%%%%%%
\begin{frame}{}
  \begin{table}[]
    \begin{tabular}{ccc}
      \teal{``First Types''} & \purple{$\implies$}    & \teal{``First Time''} \\[8pt]
      tree edge     	     & \purple{$\implies$}    & tree edge    \\[6pt]
      back edge              & \purple{$\implies$}    & back edge   
    \end{tabular}
  \end{table}
\end{frame}
%%%%%%%%%%%%%%%%%%%%

%%%%%%%%%%%%%%%%%%%%
\begin{frame}{}
  \begin{theorem}[Disjoint or Contained Lifttime of Vertices in DFS]
    \[
      \forall u,v: [_{u} \; ]_{u} \cap [_{v} \; ]_{v} = \emptyset \bigvee 
      \Big([_{u} \; ]_{u} \subset [_{v} \; ]_{v} \lor [_{v} \; ]_{v} \subset [_{u} \; ]_{u}\Big)
    \]
  \end{theorem}

  \pause

  \begin{proof}
	\fignocaption{width = 0.30\textwidth}{figs/stack.png}
  \end{proof}
\end{frame}
%%%%%%%%%%%%%%%%%%%%

%%%%%%%%%%%%%%%%%%%%
\begin{frame}{}
  \begin{exampleblock}{Edge Types and Lifetime of Vertices in DFS}
    \[
      \forall u \to v:
    \]
    \vspace{-0.30cm}
    \begin{itemize}
      \setlength{\itemsep}{5pt}
      \item tree/forward edge: $\red{[_{u}}\; \teal{[_{v}\; ]_{v}}\; \red{]_{u}}$
      \item back edge: $\teal{[_{v}}\; \red{[_{u}\; ]_{u}}\; \teal{]_{v}}$
      \item cross edge: $\teal{[_{v}\; ]_{v}}\; \red{[_{u}\; ]_{u}}$
    \end{itemize}
  \end{exampleblock}

  \pause
  \[
    \text{f}[v] < \text{d}[u] \iff \text{ \uncover<3->{\brown{cross}} edge}
  \]


  \[
    \uncover<4->{\text{f}[u] < \text{f}[v] \iff } \uncover<5->{\text{ \red{back} edge }}
  \]
  \[
    \uncover<6->{\nexists \text{ cycle} \implies \red{\boxed{u \to v \iff \text{f}[v] < \text{f}[u]}}}
  \]
\end{frame}
%%%%%%%%%%%%%%%%%%%%
