%%%%%%%%%%%%%%%
\begin{frame}{``命题''是什么?}
  \begin{definition}[Statement/Proposition]
    A \textbf{statement} is a \red{sentence} that is either true or false
    (but not both).
  \end{definition}

  \vspace{0.30cm}
  \pause
  \begin{exampleblock}{Exercise 2.1: 以下哪些是命题?}
    \begin{enumerate}
      \item $X + 6 = 0$
      \item $X = X$
      \item 哥德巴赫猜想。
      \item 今天是雨天。
      \item 明天是晴天。
      \item 明天是周二。
      \item 这句话是假话。
    \end{enumerate}
  \end{exampleblock}
\end{frame}
%%%%%%%%%%%%%%%

%%%%%%%%%%%%%%%
\begin{frame}{关于``命题'', 我知道些什么?}
  \begin{itemize}
    \setlength{\itemsep}{6pt}
    \item 命题是一个语句 (sentence),不能含有变量。
    \item 目前不知其真假,但本身必可分辨真假的语句也是命题。
    \item 悖论不是命题。
  \end{itemize}

  \vspace{0.40cm}
  \begin{quote}
    \centerline{\red{\large ``真 (truth)''}在日常语言中不可定义。} \hfill --- Alfred Tarski
  \end{quote}
\end{frame}
%%%%%%%%%%%%%%%

%%%%%%%%%%%%%%%
\begin{frame}{关于``命题究竟是什么'', 我的建议是:}
  \fignocaption{width = 0.50\textwidth}{figs/ignore.jpg}
\end{frame}
%%%%%%%%%%%%%%%

%%%%%%%%%%%%%%%
\begin{frame}{暂时忘掉``命题''与``悖论''吧}
  命题逻辑与一阶谓词逻辑:
  \begin{itemize}
    \item 引入命题符号:将命题视为原子
    \item 关注复合命题:研究命题之间的关系 
      \[
	\land \qquad \lor \qquad \lnot \qquad \to \qquad \leftrightarrow
      \]
    \item 形式语言:``真''是\red{\large ``元语言''}中的概念。 不含悖论
  \end{itemize}
\end{frame}
%%%%%%%%%%%%%%%
