% file: parts/polygon-diameter.tex

%%%%%%%%%%%%%%%
\begin{frame}{}
  \begin{center}
    \teal{\Large Convex Polygon Diameter}
  \end{center}

  \fignocaption{width = 0.60\textwidth}{figs/convex-diameter-alg}
\end{frame}
%%%%%%%%%%%%%%%

%%%%%%%%%%%%%%%
\begin{frame}{}
  \begin{exampleblock}{Convex Polygon Diameter (DH $6.8$)}
    Show that the ``Convex Polygon Diameter'' algorithm is of \red{\bf linear-time} complexity.
  \end{exampleblock}

  \begin{center}
    \red{$Q:$} Linear-time of \red{\bf WHAT}? \\[5pt] \pause
    \blue{$A:$} Linear-time of \blue{\bf the size of input} \\[12pt] \pause

    \red{$Q:$} What is the input? \\[5pt] \pause
    \blue{$A:$} A convex polygon \uncover<5->{\teal{represented by $n$ vertices}} \\[12pt]

    \uncover<6->{
      \red{$Q:$} What are the critical operations? \\[5pt] \pause
    }
    \uncover<7->{
      \blue{$A:$} $d(p_1, p_2) = \sqrt{(x_1 - x_2)^2 + (y_1 - y_2)^2}$
    }
  \end{center}

  \uncover<8->{
    \[
      \Theta(c \cdot n) = \Theta(n)
    \]
  }
\end{frame}
%%%%%%%%%%%%%%%

%%%%%%%%%%%%%%%
\begin{frame}{}
  \centerline{\teal{\Huge Correctness}}
  \vspace{0.30cm}
  \fignocaption{width = 0.60\textwidth}{figs/wait-why}
\end{frame}
%%%%%%%%%%%%%%%

%%%%%%%%%%%%%%%
\begin{frame}{}
  \begin{theorem}
    For a convex polygon, a pair of vertices determine the diameter.
  \end{theorem}

  \pause
  \fignocaption{width = 0.45\textwidth}{figs/convex-polygon-vertex-diameter}

  \pause
  \vspace{0.30cm}
  \begin{center}
    \red{\large BUT, we have {\it not} enumerated {\it all} pairs of vertices.} \\[10pt] \pause
    \teal{We have enumerated {\it all} pairs of vertices \\
    \uncover<5->{\purple{that {\it admits parallel supporting lines}.}}}
  \end{center}
\end{frame}
%%%%%%%%%%%%%%%

%%%%%%%%%%%%%%%
\begin{frame}{}
  \begin{definition}[Line of Support]
    A line $L$ is a \red{\it line of support} of a convex polygon $P$ if
    \[
      L \cap P = \text{ a vertex/an edge of } P.
    \]
  \end{definition}
  \pause
  \vspace{0.20cm}
  \centerline{$L \cap P \neq \emptyset$ and $P$ lies entirely on one side of $L$.}

  \pause
  \vspace{0.50cm}
  \begin{definition}[Antipodal]
    An \red{\it antipodal} is a pair of points that admits parallel supporting lines.
  \end{definition}

  \pause
  \vspace{0.60cm}
  \centerline{\teal{\large We have enumerated {\it all} antipodals}.}
\end{frame}
%%%%%%%%%%%%%%%

%%%%%%%%%%%%%%%
\begin{frame}{}
  \begin{theorem}[]
    If $AB$ is a diameter of a convex polygon $P$, then $AB$ is an antipodal.
  \end{theorem}

  \pause
  \vspace{0.50cm}
  \begin{proof}
    \pause
    \fignocaption{width = 0.40\textwidth}{figs/convex-polygon-antipodal-diameter}
  \end{proof}
\end{frame}
%%%%%%%%%%%%%%%

%%%%%%%%%%%%%%%
\begin{frame}{}
  \centerline{\teal{\Large Rotating Caliper}}
  \vspace{-0.30cm}
  \fignocaption{width = 0.50\textwidth}{figs/caliper}

  \vspace{-0.30cm}
  \begin{columns}
    \pause
    \column{0.50\textwidth}
      \fignocaption{width = 0.40\textwidth}{figs/shamos}
      \begin{center}
	\teal{``Computational Geometry''} \\
	Ph.D Thesis, Michael Shamos, 1978
      \end{center}
    \pause
    \column{0.50\textwidth}
      \fignocaption{width = 0.40\textwidth}{figs/toussaint}
      \begin{center}
	\teal{``Solving Geometric Problems with the Rotating Calipers''}, 1983
      \end{center}
  \end{columns}
\end{frame}
%%%%%%%%%%%%%%%

%%%%%%%%%%%%%%%
\begin{frame}{}
  \begin{exampleblock}{Finding the Closest Pair of Points}
    \fignocaption{width = 0.30\textwidth}{figs/closest-pair-plane}
  \end{exampleblock}

  \pause
  \begin{proof}[A Classic and Beautiful Divide-Conquer\only<4->{\red{-Combine}} Algorithm:]
    \pause
    \fignocaption{width = 0.35\textwidth}{figs/try}
    \vspace{-0.40cm}
    \uncover<5->{
      \centerline{\teal{Section $33.4$, CLRS}}
    }
  \end{proof}
\end{frame}
%%%%%%%%%%%%%%%