%%%%%%%%%%%%%%%
\begin{frame}{}
  \begin{center}
    {\LARGE Special Functions (\Large \emph{-jectivity})}
  \end{center}
\end{frame}
%%%%%%%%%%%%%%%

%%%%%%%%%%%%%%%
\begin{frame}{}
  \begin{definition}[Injective (one-to-one; 1-1) 单射函数]
    \[
      f: A \to B \qquad \uncover<2->{f: A \red{\;\rightarrowtail\;} B}
    \]

    \vspace{-0.60cm}
    \[
      \forall a_1, a_2 \in A: a_1 \neq a_2 \implies f(a_1) \neq f(a_2)
    \]
  \end{definition}

  \vspace{0.50cm}
  \uncover<3->{
    \begin{alertblock}{For Proof:}
      \begin{itemize}
	\item To prove that $f$ \blue{\it is} 1-1:
	  \[
	    \forall a_1, a_2 \in A: f(a_1) = f(a_2) \implies a_1 = a_2
	  \]
	\item<4-> To show that $f$ \blue{\it is not} 1-1:
	  \[
	    \red{\exists} a_1, a_2 \in A: a_1 \neq a_2 \land f(a_1) = f(a_2)
	  \]
      \end{itemize}
    \end{alertblock}
  }
\end{frame}
%%%%%%%%%%%%%%%

%%%%%%%%%%%%%%%
\begin{frame}{}
  \begin{definition}[Surjective (onto) 满射函数]
    \[
      f: A \to B \qquad \uncover<2->{f: A \red{\;\twoheadrightarrow\;} B}
    \]

    \vspace{-0.60cm}
    \[
      \ran{f} = B
    \]
  \end{definition}

  \vspace{0.50cm}
  \uncover<3->{
    \begin{alertblock}{For Proof:}
      \begin{itemize}
	\item To prove that $f$ \blue{\it is} onto:
	  \[
	    \forall b \in B \; \Big(\red{\exists} a \in A: f(a) = b \Big)
	  \]
	\item<4-> To show that $f$ \blue{\it is not} onto:
	  \[
	    \red{\exists} b \in B\; \Big(\red{\forall} a \in A: f(a) \neq b \Big)
	  \]
	\end{itemize}
    \end{alertblock}
  }
\end{frame}
%%%%%%%%%%%%%%%

%%%%%%%%%%%%%%%
% %%%%%%%%%%%%%%%
% \begin{frame}{}
%   \begin{columns}
%     \column{0.45\textwidth}
%       \fignocaption{width = 0.45\textwidth}{figs/cantor-infinity}
%       \vspace{-0.20cm}
%       {\centerline{\footnotesize Georg Cantor (1845 -- 1918)}}
%     \column{0.45\textwidth}
%       \only<5->{
% 	\fignocaption{width = 0.40\textwidth}{figs/hilbert}
% 	\vspace{-0.20cm}
% 	{\centerline{\footnotesize David Hilbert (1862 -- 1943)}}
%       }
%   \end{columns}
% 
%   \begin{columns}
%     \pause
%     \column{0.30\textwidth}
%       \fignocaption{width = 0.50\textwidth}{figs/kronecker}
%       \vspace{-0.50cm}
%       \begin{center}
% 	{\footnotesize Leopold Kronecker}
%       
%         {\footnotesize (1823 -- 1891)}
%       \end{center}
%     \pause
%     \column{0.30\textwidth}
%       \fignocaption{width = 0.50\textwidth}{figs/poincare}
%       \vspace{-0.50cm}
%       \begin{center}
% 	{\footnotesize Henri Poincar\'{e}}
%       
%         {\footnotesize (1854 -- 1912)}
%       \end{center}
%     \pause
%     \column{0.30\textwidth}
%       \fignocaption{width = 0.50\textwidth}{figs/wittgenstein}
%       \vspace{-0.50cm}
%       \begin{center}
% 	{\footnotesize Ludwig Wittgenstein}
%       
%         {\footnotesize (1889 -- 1951)}
%       \end{center}
%   \end{columns}
% \end{frame}
% %%%%%%%%%%%%%%%
% 
% %%%%%%%%%%%%%%%
% \begin{frame}{}
%   \begin{quote}
%     From his paradise that Cantor with us unfolded, 
%     we hold our breath in awe; knowing, we shall not be expelled.
% 
%     \hfill --- David Hilbert
%   \end{quote}
% 
%   \vspace{0.80cm}
%   \begin{quote}
%     \centerline{没有人能把我们从 Cantor 创造的乐园中驱逐出去。}
%   \end{quote}
% 
%   \pause
%   \fignocaption{width = 0.35\textwidth}{figs/cat-door}
% \end{frame}
% %%%%%%%%%%%%%%%

%%%%%%%%%%%%%%%
\begin{frame}{}
  \begin{theorem}[Cantor Theorem (ES Theorem 24.4)]
    Let $A$ be a set. 

    If $f: A \to 2^{A}$, then $f$ is not onto.
  \end{theorem}

  \begin{proof}
    \fignocaption{width = 0.65\textwidth}{figs/cantor-theorem-proof}
  \end{proof}
\end{frame}
%%%%%%%%%%%%%%%

%%%%%%%%%%%%%%%
\begin{frame}{}
  \begin{theorem}[Cantor Theorem]
    Let $A$ be a set. 

    If $f: A \to 2^{A}$, then $f$ is not onto.
  \end{theorem}

  \vspace{0.60cm}
  \begin{columns}
    \pause
    \column{0.28\textwidth}
      \fignocaption{width = 1.00\textwidth}{figs/talking-about}
    \pause
    \column{0.25\textwidth}
      \fignocaption{width = 0.90\textwidth}{figs/interesting}
    \pause
    \column{0.25\textwidth}
      \fignocaption{width = 0.80\textwidth}{figs/genius}
    \pause
    \column{0.25\textwidth}
      \fignocaption{width = 1.00\textwidth}{figs/stupid}
  \end{columns}
\end{frame}
%%%%%%%%%%%%%%%

%%%%%%%%%%%%%%%
\begin{frame}{}
  \begin{theorem}[Cantor Theorem]
    Let $A$ be a set. 

    If $f: A \to 2^{A}$, then $f$ is not onto.
  \end{theorem}

  \vspace{0.30cm}
  Understanding this problem:
  \[
    A = \set{1,2,3}
  \]

  \begin{description}
    \pause
    \item[$2^{A}$]
      \[
	2^{A} = \Big\{\emptyset, \set{1}, \set{2}, \set{3}, \set{1,2}, \set{1,3}, \set{2,3}, \set{1,2,3}\Big\}
      \]
    \pause
    \item[Onto]
      \[
	\forall B \in 2^{A}\; \Big(\exists a \in A\; f(a) = B\Big).
      \]
    \pause
    \item[Not Onto]
      \[
	\red{\exists} B \in 2^{A}\; \Big(\red{\forall} a \in A\; f(a) \neq B\Big).
      \]
  \end{description}
\end{frame}
%%%%%%%%%%%%%%%

%%%%%%%%%%%%%%%
\begin{frame}{}
  \begin{theorem}[Cantor Theorem]
    Let $A$ be a set. 

    If $f: A \to 2^{A}$, then $f$ is not onto.
  \end{theorem}

  \begin{proof}
    \begin{columns}[t]
      \column{0.50\textwidth}
	\begin{itemize}
	  \item<2-> Constructive proof (\red{$\exists$}):
	    \[
	      B = \set{x \in A \mid x \notin f(x)}.
	    \]
	  \item<4-> By contradiction (\red{$\forall$}):
	    \[
	      \exists a \in A: f(a) = B.
	    \]
	\end{itemize}
      \column{0.40\textwidth}
	\uncover<3->{\fignocaption{width = 0.80\textwidth}{figs/what-is-this}}
    \end{columns}

    \uncover<5->{
      \[
	\red{Q: a \in B\emph{?}}
      \]
    }
  \end{proof}
\end{frame}
%%%%%%%%%%%%%%%

%%%%%%%%%%%%%%%
\begin{frame}{}
  \begin{theorem}[Cantor Theorem]
    Let $A$ be a set. 

    If $f: A \to 2^{A}$, then $f$ is not onto.
  \end{theorem}

  \begin{proof}[对角线论证 (Cantor's diagonal argument) \only<5->{\footnotesize (以下仅适用于可数集合 $A$)}]
    \pause
    \begin{table}[]
      \centering
      $\begin{tabu}{|c||c|c|c|c|c|c|}
	\hline
	a      & \multicolumn{6}{c|}{f(a)} \\ \hline
	       & 1      & 2      & 3      & 4      & 5      & \cdots \\ \hline \hline
	1      & \redoverlay{1}{3-}      & 1      & 0      & 0      & 1      & \cdots \\ \hline
	2      & 0      & \redoverlay{0}{3-}      & 0      & 0      & 0      & \cdots \\ \hline
	3      & 1      & 0      & \redoverlay{0}{3-}      & 1      & 0      & \cdots \\ \hline
	4      & 1      & 1      & 1      & \redoverlay{1}{3-}      & 1      & \cdots \\ \hline
	5      & 0      & 1      & 0      & 1      & \redoverlay{0}{3-}      & \cdots \\ \hline
	\vdots & \vdots & \vdots & \vdots & \vdots & \vdots & \cdots \\ \hline
      \end{tabu}$
    \end{table}

    \uncover<4->{
      \[
	B = \blue{\set{0, 1, 1, 0, 1}}
      \]
    }
  \end{proof}
\end{frame}
%%%%%%%%%%%%%%%

%%%%%%%%%%%%%%%

%%%%%%%%%%%%%%%
\begin{frame}{}
  \begin{definition}[Bijective (one-to-one correspondence) 一一对应]
    \[
      f: A \to B \qquad \uncover<2->{f: A \red{\;\xleftrightarrow[onto]{1-1}\;} B}
    \]

    \vspace{0.50cm}
    \centerline{1-1 \& onto}
  \end{definition}
\end{frame}
%%%%%%%%%%%%%%%

%%%%%%%%%%%%%%%
\begin{frame}{}
  \begin{exampleblock}{UD Problem $14.12$}
    \[
      a, b, c, d \in \real{},\; a < b,\; c < d
    \]

    \vspace{0.30cm}
    \centerline{Define a bijective function:}
    \vspace{0.10cm}

    \[
      f: [a,b] \xleftrightarrow[onto]{1-1} [c,d]
    \]
    \[
      f: (a,b) \xleftrightarrow[onto]{1-1} (c,d)
    \]
  \end{exampleblock}

  \pause
  \begin{proof}[Answer]
    \[
      f(x) = c + \frac{d-c}{b-a}(x-a)
    \]
  \end{proof}
\end{frame}
%%%%%%%%%%%%%%%
