% key-exchange.tex

%%%%%%%%%%%%%%%%%%%%
\begin{frame}
  \begin{center}
	\red{\fbox{\large $Q:$ How to share this \textsc{Key}?}}
  \end{center}

  \begin{columns}
	\column{0.50\textwidth}
	  \pause
	  \fig{width = 0.60\textwidth}{figs/mission-impossible}
	\column{0.50\textwidth}
	  \pause
	  \fig{width = 0.80\textwidth}{figs/diffie}
	  \begin{center}
		\violet{\large Whitfield Diffie \teal{($1944 \sim$)}}
	  \end{center}
  \end{columns}
\end{frame}
%%%%%%%%%%%%%%%%%%%%

%%%%%%%%%%%%%%%%%%%%
\begin{frame}
  \begin{columns}
	\column{0.50\textwidth}
	  \fig{width = 0.60\textwidth}{figs/ibm-watson}
	  \begin{center}
		\violet{\large Whitfield Diffie@IBM Watson'1974}
	  \end{center}
	\column{0.50\textwidth}
	  \pause
	  \fig{width = 0.70\textwidth}{figs/Martin-Hellman}
	  \begin{center}
		\violet{\large Martin Hellman \teal{($1945 \sim$)}}
	  \end{center}
  \end{columns}
\end{frame}
%%%%%%%%%%%%%%%%%%%%

%%%%%%%%%%%%%%%%%%%%
\begin{frame}
  \uncover<3->{
  \begin{alertblock}{两千多年来的密码学``公理'':}
	\centering
	\red{``不管采用什么方法, 密钥就是一定得发送, \\ 这个动作无论如何避免不了''}
  \end{alertblock}
  }

  \begin{columns}
	\column{0.50\textwidth}
	  \fig{width = 0.55\textwidth}{figs/diffie-hellman}
	  \begin{center}
		\violet{Martin Hellman, Whitfield Diffie}
	  \end{center}
	\column{0.50\textwidth}
	  \pause
	  \fig{width = 0.70\textwidth}{figs/diffie-hellman-merkle}
	  \begin{center}
		\violet{$+$ Ralph Merkle}
	  \end{center}
  \end{columns}
\end{frame}
%%%%%%%%%%%%%%%%%%%%

%%%%%%%%%%%%%%%%%%%%
\begin{frame}
  \begin{definition}[One-way Function]
	\fig{width = 0.50\textwidth}{figs/one-way-function}
  \end{definition}

  \pause
  \fig{width = 0.15\textwidth}{figs/easy-hard}

  \pause
  \begin{center}
    \red{$Q:$ Hard in {\it worst case} or in {\it average case}?} \\[6pt] \pause
    \red{$Q:$ Do one-way functions exist?}
  \end{center}
\end{frame}
%%%%%%%%%%%%%%%%%%%%

%%%%%%%%%%%%%%%%%%%%
\begin{frame}
  \begin{columns}
	\column{0.50\textwidth}
	  \fig{width = 0.70\textwidth}{figs/qrcode-one-way-function-wiki}
	  \begin{center}
		\href{https://en.wikipedia.org/wiki/One-way\_function}{One-way Function (wiki)}
	  \end{center}
	\column{0.50\textwidth}
	  \fig{width = 0.90\textwidth}{figs/one-way-functions-exist}
  \end{columns}
\end{frame}
%%%%%%%%%%%%%%%%%%%%

%%%%%%%%%%%%%%%%%%%%
\begin{frame}
  \fig{width = 0.50\textwidth}{figs/one-way-function-color}

  \pause
  \vspace{0.50cm}
  \begin{center}
	\href{https://youtu.be/3QnD2c4Xovk}{\large $Q:$ How to share a \textsc{Color}?}
  \end{center}
\end{frame}
%%%%%%%%%%%%%%%%%%%%

%%%%%%%%%%%%%%%%%%%%
\begin{frame}
  \begin{center}
	\purple{\Large Diffie-Hellman-Merkle Key Exchange} \\
	\teal{(Martin Hellman; Spring, $1976$)}
  \end{center}

  \pause
  \begin{definition}[Discrete Logarithm]
	\[
	  g^{\red{x}} \equiv a \;(\text{mod } p)
	\]
  \end{definition}

\end{frame}
%%%%%%%%%%%%%%%%%%%%

%%%%%%%%%%%%%%%%%%%%
\begin{frame}
  \[
	\red{p} \qquad \red{g} \;\text{(generator for $\mathbb{Z}_{p}$)} 
  \]

  \pause
  \begin{columns}
	\column{0.50\textwidth}
	  \begin{center}
		\blue{\large Alice}
	  \end{center}

	  \begin{enumerate}
		\setlength{\itemsep}{6pt}
		\item Randomly choose $a$
		\item Compute $A = g^a \text{ mod } p$
		\item Send $A$ to Bob
		\item Compute $K = B^a \text{ mod } p$
	  \end{enumerate}
	\column{0.50\textwidth}
	  \begin{center}
		\blue{\large Bob}
	  \end{center}

	  \begin{enumerate}
		\setlength{\itemsep}{6pt}
		\item Randomly choose $b$
		\item Compute $B = g^b \text{ mod } p$
		\item Send $B$ to Alice 
		\item Compute $K = A^b \text{ mod } p$
	  \end{enumerate}
  \end{columns}

  \pause
  \vspace{0.60cm}
  \[
	\red{K = g^{ab} \text{ mod } p}
  \]

  \pause
  \begin{center}
	\blue{$K$ used for DES}
  \end{center}
\end{frame}
%%%%%%%%%%%%%%%%%%%%

%%%%%%%%%%%%%%%%%%%%
\begin{frame}
  \fig{width = 0.50\textwidth}{figs/middleman-attack}

  \begin{center}
	\violet{\large Man-in-the-Middle Attack}
  \end{center}
\end{frame}
%%%%%%%%%%%%%%%%%%%%

%%%%%%%%%%%%%%%%%%%%
\begin{frame}
  \fig{width = 0.50\textwidth}{figs/timing}

  \begin{center}
	\red{\large Alice and Bob need to be online simultaneously.}
  \end{center}
\end{frame}
%%%%%%%%%%%%%%%%%%%%

%%%%%%%%%%%%%%%%%%%%
\begin{frame}
  \begin{alertblock}{两千多年来的密码学``公理'':}
	\centering
	\red{``不管采用什么方法, 密钥就是一定得发送, \\ 这个动作无论如何避免不了''}
  \end{alertblock}

  \pause
  \begin{proof}
	\fig{width = 0.40\textwidth}{figs/wrong}
  \end{proof}
\end{frame}
%%%%%%%%%%%%%%%%%%%%
