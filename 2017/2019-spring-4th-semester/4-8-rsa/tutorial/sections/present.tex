% present.tex

%%%%%%%%%%%%%%%%%%%%
\begin{frame}
  \fig{width = 0.65\textwidth}{figs/rsa-people}

  \pause
  \begin{quote}
	\centering
	\violet{\large ``也许、大概、或许并不存在这样的 \\ Trapdoor One-Way Functions''}
  \end{quote}
\end{frame}
%%%%%%%%%%%%%%%%%%%%

%%%%%%%%%%%%%%%%%%%%
\begin{frame}
  \begin{center}
	\purple{\Large RSA} \\
	\teal{(Ron Rivest; April $1977$)}
  \end{center}

  \begin{columns}
	\column{0.50\textwidth}
	  \pause
	  \fig{width = 0.55\textwidth}{figs/wine}
	\column{0.50\textwidth}
	  \pause
	  \fig{width = 0.90\textwidth}{figs/rsa-cacm}
  \end{columns}
\end{frame}
%%%%%%%%%%%%%%%%%%%%

%%%%%%%%%%%%%%%%%%%%
\begin{frame}
  \[
	\red{p,\quad q}
  \]

  \pause
  \[
	\blue{n} = p q
  \]

  \pause
  \[
	\red{\phi(n)} = (p - 1) (q - 1)
  \]

  \pause
  \[
	\blue{e} : (e, \phi(n)) = 1
  \]

  \pause
  \[
	\red{d} : ed \equiv 1 \;\text{mod}\; \phi(n)
  \]

  \begin{columns}
	\column{0.50\textwidth}
	  \pause
	  \[
		\violet{P = (e, n)}
	  \]
	\column{0.50\textwidth}
	  \pause
	  \[
		\violet{S = (d, n)}
	  \]
  \end{columns}

  \vspace{0.40cm}
  \begin{columns}
	\column{0.50\textwidth}
	  \pause
	  \[
		\blue{P(M) = M^{e} \;\text{mod}\; n}
	  \]
	\column{0.50\textwidth}
	  \pause
	  \[
		\red{S(C) = C^{d} \;\text{mod}\; n}
	  \]
  \end{columns}
\end{frame}
%%%%%%%%%%%%%%%%%%%%

%%%%%%%%%%%%%%%%%%%%
\begin{frame}
  \begin{center}
	\red{\large $Q:$ What is the trapdoor one-way function in RSA?}
  \end{center}

  \fig{width = 0.40\textwidth}{figs/trapdoor}

  \pause
  \[
	f(x) = x^{e} \;\text{mod}\; n
  \]

  \pause
  \[
	\red{d}
  \]
\end{frame}
%%%%%%%%%%%%%%%%%%%%

%%%%%%%%%%%%%%%%%%%%
\begin{frame}
  \fig{width = 0.70\textwidth}{figs/one-way-functions-exist}
\end{frame}
%%%%%%%%%%%%%%%%%%%%

%%%%%%%%%%%%%%%%%%%%
\begin{frame}
  \[
	\red{\text{RSA-number}: n = p q}
  \]

  \fig{width = 0.60\textwidth}{figs/rsa-number}
\end{frame}
%%%%%%%%%%%%%%%%%%%%

%%%%%%%%%%%%%%%%%%%%
\begin{frame}
  \begin{exampleblock}{``Small $e$ Attack'' (CLRS $31.7$-$2$)}
	\[
	  e = 3
	\]
	\[
	  0 < d < \phi(n)
	\]

	\pause
	\vspace{-0.30cm}
	\[
	  \purple{n = pq}
	\]
  \end{exampleblock}

  \pause
  \begin{center}
	\red{\large Why do I factor $n$ if I have obtained $d$?}
  \end{center}

  \pause
  \fig{width = 0.30\textwidth}{figs/org-common-n}
  \begin{center}
	\violet{Common $n$; Different $e$'s and $d$'s}
  \end{center}
  \end{frame}
%%%%%%%%%%%%%%%%%%%%

%%%%%%%%%%%%%%%%%%%%
\begin{frame}
  \[
	ed \equiv 1 \;(\text{mod}\; \phi(n))
  \]

  \pause
  \[
	ed = 1 + k \phi(n) (k \in \mathbb{Z})
  \]

  \pause
  \[
	\boxed{ed = 1 + k \phi(n) \;(\red{k \in \mathbb{N}, k < \min\set{e, d}})}
  \]

  \pause
  \[
	3d = 1 + k \phi(n) \;(\red{k \in {1, 2}})
  \]

  \pause
  \[
	\phi(n) = (p-1)(q-1) = n - (p + q) + 1
  \]
  \[
	n = pq
  \]
\end{frame}
%%%%%%%%%%%%%%%%%%%%

%%%%%%%%%%%%%%%%%%%%
% \begin{frame}
%   \begin{exampleblock}{TJ $7$-$12$}
% 	Find integers $n$, $e$, $M$ such that
% 	\[
% 	  M^{e} = M \;(\text{mod}\; n)
% 	\]
% 	Is this a potential problem in RSA?
%   \end{exampleblock}
% \end{frame}
%%%%%%%%%%%%%%%%%%%%

%%%%%%%%%%%%%%%%%%%%
% \begin{frame}
%   \begin{center}
% 	\purple{\large Iterated Encryption Attack}
%   \end{center}
% 
%   \begin{align*}
% 	\onslide<2->{E_1 &= M^e \;\text{mod}\; n \\[4pt]}
% 	\onslide<3->{E_2 &= E_1^e \;\text{mod}\; n \\[4pt]}
% 	\onslide<4->{E_3 &= E_2^e \;\text{mod}\; n \\[4pt]}
% 	\onslide<5->{&\cdots \\[4pt]}
% 	\onslide<6->{E_k &= E_{k-1}^e \;\text{mod}\; n \\[4pt]}
%   \end{align*}
% 
%   \vspace{-0.60cm}
%   \uncover<7->{
% 	\[
% 	  \red{E_k = M^{(e^k)} \;\text{mod}\; n}
% 	\]
%   }
% \end{frame}
%%%%%%%%%%%%%%%%%%%%

%%%%%%%%%%%%%%%%%%%%
\begin{frame}
  \begin{exampleblock}{Common Modulus Attack}
	\fig{width = 0.30\textwidth}{figs/org-common-n}
	\begin{center}
	  \violet{Common $n$; Different $e$'s and $d$'s}
	\end{center}

	\pause
	\vspace{-0.50cm}
	\[
	  (e_i, e_j) = 1
	\]
  \end{exampleblock}

  \pause
  \[
	E = M^{e_1} \;(\text{mod}\; n) \qquad F = M^{e_2} \;(\text{mod}\; n)
  \]

  \pause
  \vspace{-0.30cm}
  \[
	(e_1, e_2) = 1 \pause\red{\implies}\; e_1 x + e_2 y = 1
  \]

  \pause
  \vspace{-0.60cm}
  \[
	\red{M = E^x \cdot F^y \;(\text{mod}\; n)}
  \]
\end{frame}
%%%%%%%%%%%%%%%%%%%%

%%%%%%%%%%%%%%%%%%%%
\begin{frame}
  \begin{columns}
	\column{0.50\textwidth}
	  \fig{width = 0.80\textwidth}{figs/rsa-3people}
	\column{0.50\textwidth}
	  \fig{width = 0.70\textwidth}{figs/turing-award}
  \end{columns}

  \vspace{0.50cm}
  \begin{quote}
	\center
	\violet{``For their ingenious contribution for \\ making public-key cryptography useful in practice.''} \\[5pt]

	\hfill \teal{--- ACM Turing Award, $2002$}
  \end{quote}
\end{frame}
%%%%%%%%%%%%%%%%%%%%

%%%%%%%%%%%%%%%%%%%%
\begin{frame}
  \begin{columns}
	\column{0.50\textwidth}
	  \fig{width = 0.50\textwidth}{figs/diffie-hellman}
	\column{0.50\textwidth}
	  \fig{width = 0.70\textwidth}{figs/turing-award}
  \end{columns}

  \vspace{0.50cm}
  \begin{quote}
	\center
	\violet{``For fundamental contributions to modern cryptography.''} \\[5pt]

	\hfill \teal{--- ACM Turing Award, $2015$}
  \end{quote}
\end{frame}
%%%%%%%%%%%%%%%%%%%%
