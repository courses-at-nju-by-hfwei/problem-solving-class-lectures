% file: parts/binary-search.tex

%%%%%%%%%%%%%%%
\begin{frame}{}
  \begin{exampleblock}{Binary Search (CLRS $4.5-3$)}
    \[
      T(n) = T(n/2) + \Theta(1)
    \]
  \end{exampleblock}

  \begin{columns}
    \column{0.70\textwidth}
      % file: algs/binary-search-recursive.tex

\begin{algorithm}[H]
  % \caption{Recursive binary search.}
  \begin{algorithmic}[1]
    \Procedure{BinarySearch}{$A, L, R, x$}
      \If{$R < L$}
	\State \Return $-1$
      \EndIf

      \hStatex
      \State $m \gets L + (R - L) / 2$

      \hStatex
      \If{$A[m] = x$}
	\State \Return $m$
      \ElsIf{$A[m] > x$}
	\State \Return $\Call{BinarySearch}{A, L, m-1, x}$
      \Else
	\State \Return $\Call{BinarySearch}{A, m+1, R, x}$
      \EndIf
    \EndProcedure
  \end{algorithmic}
\end{algorithm}

    \column{0.30\textwidth}
      \[
	\boxed{\teal{T(n) = \Theta(\log n)}}
      \]
  \end{columns}
\end{frame}
%%%%%%%%%%%%%%%

%%%%%%%%%%%%%%%
% \begin{frame}{}
% 
%   \begin{quote}
%     {\large People who analyze algorithms have \red{double happiness}. \\[8pt]
% 
%     First of all they experience the sheer \teal{beauty of elegant mathematical patterns}
%     that surround elegant computational procedures. \\[6pt]
% 
%     Then they receive a \teal{practical payoff} when their theories 
%     make it possible to get other jobs done more quickly and more economically.} \\[4pt]
% 
%     \begin{columns}[t]
%       \column{0.50\textwidth}
% 	\fignocaption{width = 0.35\textwidth}{figs/knuth}
%       \column{0.50\textwidth}
% 	--- Donald E. Knuth (1995)
%     \end{columns}
%   \end{quote}
% \end{frame}
%%%%%%%%%%%%%%%

%%%%%%%%%%%%%%%
\begin{frame}{}
  \fignocaption{width = 0.40\textwidth}{figs/binary-stride}

  \[
    T(n) = \left\{\begin{array}{lr}
      \max\Big\{T(\lfloor \frac{n-1}{2} \rfloor), T(\lceil \frac{n-1}{2} \rceil)\Big\} + 1, & n > 2 \\
      1, & n = 1
  \end{array}\right.
  \]

  \pause
  \vspace{0.30cm}
  \[
    T(n) = \left\{\begin{array}{lr}
      T(\lfloor \frac{n}{2} \rfloor) + 1, & n > 2 \\
      1, & n = 1
    \end{array}\right.
  \]
\end{frame}
%%%%%%%%%%%%%%%

%%%%%%%%%%%%%%%
\begin{frame}{}
  \[
    T(n) = \left\{\begin{array}{lr}
      T(\lfloor \frac{n}{2} \rfloor) + 1, & n > 2 \\
      1, & n = 1
    \end{array}\right.
  \]

  % \pause
  % \vspace{0.30cm}
  % \[
  %   n = 2^k \implies T(n) = k + 1
  % \]

  \pause
  \vspace{0.30cm}
  \[
    2^k \le n < 2^{k+1} \implies T(n) = k + 1
  \]

  \pause
  \vspace{0.30cm}
  \[
    \boxed{\teal{T(n) = \lfloor \lg n \rfloor + 1}}
  \]

  \pause
  \begin{theorem}
    \begin{center}
      The worst case time complexity of \textsc{BinarySearch} on an input size of $n$ \\
      \red{\bf $=$} \\
      \# of bits in the binary representation of $n$.
    \end{center}
  \end{theorem}
\end{frame}
%%%%%%%%%%%%%%%
