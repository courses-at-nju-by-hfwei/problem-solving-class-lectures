% prop.tex

%%%%%%%%%%%%%%%
\begin{frame}{}
  \fig{width = 0.60\textwidth}{figs/yan-proposition}
\end{frame}
%%%%%%%%%%%%%%%

%%%%%%%%%%%%%%%
\begin{frame}{}
  \begin{definition}[命题 (Statement/Proposition)]
    A \textbf{statement} is a \red{sentence} that is either true or false
    (but not both).
  \end{definition}

  \vspace{0.30cm}
  \pause
  \begin{exampleblock}{Exercise 2.1: 以下哪些是命题?}
    \begin{enumerate}
      \item $X + 6 = 0$
      \item $X = X$
      \item 哥德巴赫猜想。
      \item 今天是雨天。
      \item 明天是晴天。
      \item 明天是周二。
      \item 这句话是假话。
    \end{enumerate}
  \end{exampleblock}
\end{frame}
%%%%%%%%%%%%%%%

%%%%%%%%%%%%%%%
\begin{frame}{来自一位数理逻辑学家的意见与建议}
  \begin{itemize}
    \setlength{\itemsep}{8pt}
    \item<1-> (1)、(2) \red{不是句子} (sentence), 所以也不是命题。
    \item<2-> (4)、(5)、(6) \red{在数学中没有意义}。
    \item<3-> ``真(Truth)''是\red{不能定义}的。所以, (7) 不是命题。
      \uncover<4->{\fig{width = 0.15\textwidth}{figs/meng2.png}}
      \uncover<5->{
	\begin{quote}
	  \centerline{``真 (truth)''在日常语言(或算术)中不可定义。} \hfill --- \blue{Alfred Tarski}
	\end{quote}
      }
  \end{itemize}

  \uncover<6->{\centerline{\red{\large ``我觉得你还是找一本正经的数理逻辑教材看看''}}}
\end{frame}
%%%%%%%%%%%%%%%

%%%%%%%%%%%%%%%
\begin{frame}{关于``命题'', 我们现在知道些什么?}
  \begin{itemize}
    \setlength{\itemsep}{6pt}
    \item 命题是一个语句 (sentence),不能含有变量。
    \item 目前不知其真假,但本身必可分辨真假的语句也是命题。
    \item 悖论不是命题。
  \end{itemize}
\end{frame}
%%%%%%%%%%%%%%%

%%%%%%%%%%%%%%%
\begin{frame}{暂时忘掉``命题''与``悖论''吧}
  命题逻辑与一阶谓词逻辑:
  \begin{itemize}
    \item 引入命题符号:将命题视为原子
    \item 关注复合命题:研究命题之间的关系 
      \[
	\land \qquad \lor \qquad \lnot \qquad \to \qquad \leftrightarrow
      \]
  \end{itemize}
\end{frame}
%%%%%%%%%%%%%%%
