%%%%%%%%%%%%%%%
\begin{frame}{}
  \centerline{\Large Set Family $\set{A_{\alpha}: \alpha \in I}$}

  \[
    \bigcap \qquad \bigcup
  \]
\end{frame}
%%%%%%%%%%%%%%%

%%%%%%%%%%%%%%%
\begin{frame}{}
    \[
      \bigcup_{j = 1}^{n} A_j = A_1 \cup A_2 \cup \cdots \cup A_n \qquad 
      \bigcap_{j = 1}^{n} A_j = A_1 \cap A_2 \cap \cdots \cap A_n
    \]

    \pause
    \vspace{1.00cm}
    \[
      \bigcup_{j = 1}^{\infty} A_j = A_1 \cup A_2 \cup \cdots \qquad 
      \bigcap_{j = 1}^{\infty} A_j = A_1 \cap A_2 \cap \cdots 
    \]
\end{frame}
%%%%%%%%%%%%%%%

%%%%%%%%%%%%%%%
\begin{frame}{}
  \[
    \bigcup_{\alpha \in I} A_{\alpha} = \set{x \mid \exists \alpha \in I: x \in A_{\alpha}}
  \]

  \[
    \bigcap_{\alpha \in I} A_{\alpha} = \set{x \mid \forall \alpha \in I: x \in A_{\alpha}}
  \]

  \pause
  \vspace{0.60cm}
  \[
    \red{Q: \; I \neq \emptyset \text{ for } \bigcap_{\alpha \in I} A_{\alpha}} \quad \blue{(\text{UD } P_{91})}
  \]

  \pause
  \vspace{0.60cm}
  \[
    \purple{Q: \; I \neq \emptyset \text{ for } \bigcup_{\alpha \in I} A_{\alpha}}
  \]
\end{frame}
%%%%%%%%%%%%%%%

%%%%%%%%%%%%%%%
\begin{frame}{}
  \begin{exampleblock}{``$\bigcap_{n = 1}^{\infty}$'': UD $8.1$}
    \[
      A_n = [0, 1/n) \quad B_n = [0, 1/n] \quad C_n = (0, 1/n)
    \]

    \begin{enumerate}[(a)]
      \setcounter{enumi}{1}
      \item Find $\bigcap_{n=1}^{\infty} A_n \uncover<2->{\blue{\;= \set{0}}}
	\quad \bigcap_{n=1}^{\infty} B_n \uncover<2->{\red{\;= \set{0}}} 
	\quad \bigcap_{n=1}^{\infty} C_n \uncover<2->{\blue{\;= \emptyset}}$
    \end{enumerate}
  \end{exampleblock}

  \vspace{0.50cm}
  \only<3>{
    \begin{columns}
      \column{0.45\textwidth}
	\fignocaption{width = 0.85\textwidth}{figs/math-help}
      \column{0.45\textwidth}
	\fignocaption{width = 0.70\textwidth}{figs/wunai}
    \end{columns}
  }
  
  \vspace{0.30cm}
  \only<4->{
    \begin{theorem}[The Nested Interval Theorem (Cantor)]
      设 $\set{[a_n, b_n]}$ 为递降闭区间套序列, 即
      \[
	[a_1, b_1] \supset [a_2, b_2] \supset \cdots \supset [a_n, b_n] \supset \cdots.
      \]
      如果 $\lim\limits_{n\to \infty} (b_n - a_n) = 0$, 则存在唯一的点 $c$, 
      使得 $c \in [a_n, b_n], \forall n \ge 1$.
    \end{theorem}
  }
\end{frame}
%%%%%%%%%%%%%%%

%%%%%%%%%%%%%%%
\begin{frame}{}
  \begin{exampleblock}{``$\bigcap_{n = 1}^{\infty}$'': UD $8.4$}
    \[
      \forall n \in \mathbb{Z}^{+}: A_n \subset B_n \red{\;\nRightarrow} \bigcap_{n=1}^{\infty} A_n \subset \bigcap_{n=1}^{\infty} B_n
    \]
  \end{exampleblock}

  \[
    A_n = [0, 1/n) \quad B_n = [0, 1/n]
  \]

  \pause
  \vspace{0.40cm}
  \fignocaption{width = 0.35\textwidth}{figs/think-be-careful}
\end{frame}
%%%%%%%%%%%%%%%

%%%%%%%%%%%%%%%
% \begin{frame}{}
%   \begin{definition}[Topology]
%     A \blue{topology} on a set $X$ is \fbox{a collection $\mathcal{T}$ of subsets of $X$} such that:
%     \begin{enumerate}[(1)]
%       \item $\emptyset, X \in \mathcal{T}$.
%       \item Any \purple{union} of elements of $\mathcal{T}$ is in $\mathcal{T}$.
%       \item Any \purple{intersection} of \red{finitely many} elements of $\mathcal{T}$ is in $\mathcal{T}$.
%     \end{enumerate}
%   \end{definition}
% 
%   \[
%     X = \set{a, b, c}
%   \]
% 
%   \[
%     \mathcal{T} = \set{X, \emptyset, \set{a,b}, \set{b}, \set{b,c}}
%   \]
% \end{frame}
%%%%%%%%%%%%%%%

%%%%%%%%%%%%%%%
\begin{frame}{}
  \begin{exampleblock}{DeMorgan's Law: UD Exercise $8.9$}
    \[
      X \setminus \bigcup_{\alpha \in I} A_{\alpha} = \bigcap_{\alpha \in I} (X \setminus A_{\alpha})
    \]

    \[
      X \setminus \bigcap_{\alpha \in I} A_{\alpha} = \bigcup_{\alpha \in I} (X \setminus A_{\alpha})
    \]
  \end{exampleblock}
\end{frame}
%%%%%%%%%%%%%%%

%%%%%%%%%%%%%%%
\begin{frame}{}
  \begin{exampleblock}{DeMorgan's Law: UD $8.8$}
    \[
      A = \mathbb{R} \setminus \bigcap_{n \in \mathbb{Z}^{+}} (\mathbb{R} \setminus \set{-n, -n+1, \cdots, 0, \cdots, n-1, n})
    \]
  \end{exampleblock}

  \pause
  \[
    X_n = \set{-n, -n+1, \cdots, 0, \cdots, n-1, n}
  \]

  \pause
  \begin{align*}
     A &= \mathbb{R} \setminus \bigcap_{n \in \mathbb{Z}^{+}} (\mathbb{R} \setminus X_n)\\
       &= \mathbb{R} \setminus \Big(\mathbb{R} \setminus \bigcup_{n \in \mathbb{Z}^{+}} X_n \Big) \\
       % &= \bigcup_{n \in \mathbb{Z}^{+}} X_n \\
       % &= \mathbb{Z}
  \end{align*}
\end{frame}
%%%%%%%%%%%%%%%

%%%%%%%%%%%%%%%
\begin{frame}{}
  \begin{exampleblock}{DeMorgan's Law: UD $8.9$}
    \[
      A = \mathbb{Q} \setminus \bigcap_{n \in \mathbb{Z}} (\mathbb{R} \setminus \set{2n})
    \]
  \end{exampleblock}

  \pause
  \vspace{0.50cm}
  \centerline{\red{$Q:$} What is the \purple{temporary} universe?}

  \pause
  \begin{align*}
     A &= \mathbb{Q} \setminus \bigcap_{n \in \mathbb{Z}} (\mathbb{R} \setminus \set{2n}) \\
     &= \mathbb{Q} \setminus \Big(\mathbb{R} \setminus \bigcup_{n \in \mathbb{Z}} \set{2n} \Big) \\
     % &= \mathbb{Q} \setminus \Big(\bigcup_{n \in \mathbb{Z}} \set{2n}\Big)^{c} \\
     % &= \mathbb{Q} \cap \bigcup_{n \in \mathbb{Z}} \set{2n}\\
     % &= \set{2n: n \in \mathbb{Z}}
  \end{align*}
\end{frame}
%%%%%%%%%%%%%%%

%%%%%%%%%%%%%%%
\begin{frame}{}
  \centerline{\Large \textcolor{teal}{Video:}}
  \vspace{0.20cm}
  \centerline{\large \textcolor{teal}{\href{https://www.youtube.com/watch?v=HXuvEeecZtU}{Message To Future Generations --- Bertrand Russell}}}
\end{frame}
%%%%%%%%%%%%%%%