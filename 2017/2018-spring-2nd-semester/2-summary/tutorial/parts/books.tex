%%%%%%%%%%%%%%%
\begin{frame}{}
  \centerline{\Large \teal{学: (基础) 算法书籍推荐}}
  \begin{quote}
    一个人倘若需要从思想中得到快乐, 那么他的第一个欲望就是学习。

    \hfill --- 王小波《思维的乐趣》
  \end{quote}
\end{frame}
%%%%%%%%%%%%%%%

%%%%%%%%%%%%%%%
\begin{frame}{}
  \centerline{\Large \teal{教: 公开课推荐}}
  \begin{quote}
    我在大学里遇到了把知识当作幸福来传播的数学老师, 他使学习数学变成了一种乐趣。

    \hfill --- 王小波《思维的乐趣》
  \end{quote}
\end{frame}
%%%%%%%%%%%%%%%

%%%%%%%%%%%%%%%
\begin{frame}{}
  \centerline{\href{}{这种颜色的字体是可以点击的链接。你不试一下吗?}}
\end{frame}
%%%%%%%%%%%%%%%

  % \vspace{0.30cm}
  % \centerline{\Large \teal{练: 程序设计书籍推荐}}
  % \begin{quote}
  %   去年托人带来的中文软件, 我用着尚好, 而且又用 C 语言仿编了一个, 程序是我的, 拼音字典是人家的, 执此招摇撞骗, 骗了一些钱。
  %   干这个事, 熟悉了C语言, 都是拜小阳所赐。 

  %   \hfill --- 王小波书信
  % \end{quote}
%%%%%%%%%%%%%%%

%%%%%%%%%%%%%%%
\begin{frame}{}
  \begin{columns}
    \column{0.45\textwidth}
      \fignocaption{width = 0.85\textwidth}{figs/papadimitrous}
    \column{0.45\textwidth}
      \fignocaption{width = 0.65\textwidth}{figs/algorithms-papa}
  \end{columns}

  \vspace{0.80cm}
  \centerline{言简意赅、观点独到}
\end{frame}
%%%%%%%%%%%%%%%

%%%%%%%%%%%%%%%
\begin{frame}{}
  \begin{columns}
    \column{0.45\textwidth}
      \fignocaption{width = 0.85\textwidth}{figs/jon-kleinberg}
    \column{0.45\textwidth}
      \fignocaption{width = 0.75\textwidth}{figs/algorithm-design-china}
  \end{columns}

  \vspace{0.10cm}
  \begin{center}
    侧重算法设计思路、详略得当 \\[6pt]
    高质量的例题与习题 \\[10pt]

    \href{http://www.cs.princeton.edu/~wayne/kleinberg-tardos/}{Pretty slides}
  \end{center}
\end{frame}
%%%%%%%%%%%%%%%

%%%%%%%%%%%%%%%
\begin{frame}{}
  \begin{columns}
    \column{0.45\textwidth}
      \fignocaption{width = 0.80\textwidth}{figs/steven-skiena}
    \column{0.45\textwidth}
      \fignocaption{width = 0.70\textwidth}{figs/algorithm-design-manual-china}
  \end{columns}

  \vspace{0.50cm}
  \begin{center}
    我最喜欢读每一章的 War Story,尤其是第九章的那两个。\\[10pt]

    \href{https://www.youtube.com/playlist?list=PLOtl7M3yp-DX32N0fVIyvn7ipWKNGmwpp}{Skiena's Algorithms Lectures @ YouTube}
  \end{center}
\end{frame}
%%%%%%%%%%%%%%%

%%%%%%%%%%%%%%%
\begin{frame}{}
  \begin{columns}
    \column{0.50\textwidth}
      \fignocaption{width = 0.60\textwidth}{figs/jeffe}
    \column{0.50\textwidth}
      \fignocaption{width = 0.70\textwidth}{figs/algorithms-jeffe}
      \vspace{-0.30cm}
      {\centerline{\href{http://jeffe.cs.illinois.edu/teaching/algorithms/all-algorithms.pdf}{讲义}。自由、亲切、严肃、活泼。}}
  \end{columns}

  \vspace{0.30cm}
  \begin{center}
    \href{https://recordings.engineering.illinois.edu:8443/ess/portal/section/d6d800b7-eb5c-4420-8175-10becaf2d25d}{CS 473: Algorithms (Spring 2017)} by Jeff Erickson
  \end{center}
\end{frame}
%%%%%%%%%%%%%%%

%%%%%%%%%%%%%%%
\begin{frame}{}
  \begin{columns}
    \column{0.45\textwidth}
      \fignocaption{width = 0.85\textwidth}{figs/sedgewick-youtube}
    \column{0.45\textwidth}
      \fignocaption{width = 0.70\textwidth}{figs/algorithms-4th}
  \end{columns}

  \vspace{0.50cm}
  \begin{center}
    包含大量图例, 提供可运行代码 (Java) 及实验数据 \\[8pt]
    在这里,算法是一门``实验科学''。\\[10pt]

    \href{https://www.youtube.com/playlist?list=PLxc4gS-_A5VDXUIOPkJkwQKYiT2T1t0I8}{Algorithms and Data Structures (I) @ YouTube}
  \end{center}
\end{frame}
%%%%%%%%%%%%%%%

%%%%%%%%%%%%%%%
\begin{frame}{}
  \begin{columns}
    \column{0.45\textwidth}
      \fignocaption{width = 0.90\textwidth}{figs/leiserson}
    \column{0.45\textwidth}
      \fignocaption{width = 0.70\textwidth}{figs/erik-demaine-teaching.jpeg}
  \end{columns}

  \vspace{0.50cm}
  \begin{center}
    ``Algorithms''\\[4pt]
    \href{http://open.163.com/special/opencourse/algorithms.html}{MIT 6.046/18.410, Fall 2005}\\[3pt]
    \href{https://ocw.mit.edu/courses/electrical-engineering-and-computer-science/6-006-introduction-to-algorithms-fall-2011/lecture-videos/}{MIT 6.006, Fall 2011}\\[10pt]
    
    感受课堂魅力,教与学都成为一种享受
  \end{center}
\end{frame}
%%%%%%%%%%%%%%%

%%%%%%%%%%%%%%%
\begin{frame}{}
  \fignocaption{width = 0.25\textwidth}{figs/concrete-mathematics}

  \begin{center}
    教材:Concrete Mathematics

    \href{http://www3.cs.stonybrook.edu/~algorith/math-video/}{Skiena's CSE 547 Discrete Mathematics Lectures} \\[8pt]

    自如、热情

    {\small (Pity: Low video quality)}
  \end{center}
\end{frame}
%%%%%%%%%%%%%%%

%%%%%%%%%%%%%%%
\begin{frame}{}
  \begin{columns}
    \column{0.45\textwidth}
      \fignocaption{width = 0.85\textwidth}{figs/sedgewick-teaching}
    \column{0.45\textwidth}
      \fignocaption{width = 0.70\textwidth}{figs/AoA}
  \end{columns}

  \vspace{0.50cm}
  \begin{center}
    关注算法分析技术; 有一定难度
  \end{center}
\end{frame}
%%%%%%%%%%%%%%%

%%%%%%%%%%%%%%%
\begin{frame}{}
  \fignocaption{width = 0.45\textwidth}{figs/knuth-teaching}

  \centerline{\href{https://youtu.be/jmcSzzN1gkc}{Stanford Lecture - Don Knuth: The Analysis of Algorithms @ YouTube}}

  \vspace{0.30cm}
  \centerline{幽默、深入浅出、余味无穷}
\end{frame}
%%%%%%%%%%%%%%%