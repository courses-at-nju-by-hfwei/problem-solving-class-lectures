% infinite.tex

%%%%%%%%%%%%%%%
\begin{frame}{}
  \begin{exampleblock}{Set Union (UD Problem $23.1$)}
    Give an example, if possible, of
    \begin{enumerate}[(a)]
      \item A countably infinite collection of \purple{pairwise disjoint finite sets} whose union is countably infinite.
	\pause
	\[
	  \forall n \in \N: A_n = \set{n} \qquad \bigcup_{n \in \N} A_n = \N
	\]
	\pause
      \item A countably infinite collection of nonempty sets whose union is finite.
	\pause
	\[
	  \forall n \in \N: A_n = \set{1} \qquad \bigcup_{n \in \N} A_n = \set{1}
	\]
	\pause
      \item A countably infinite collection of \teal{pairwise disjoint nonempty sets} whose union is finite.
	\pause
	\[
	  |A| = n \implies \pause |\ps{A}| = 2^n
	\]
    \end{enumerate}
  \end{exampleblock}
\end{frame}
%%%%%%%%%%%%%%%

%%%%%%%%%%%%%%%
\begin{frame}{}
  \begin{exampleblock}{UD Problem 23.3 (d)}
    Is it countable or uncountable?
    \[
      A = \set{(x, y) \in \R \times \R: x + y = 1}
    \]
  \end{exampleblock}

  \pause
  \[
    f: \R \xleftrightarrow[onto]{1-1} A
  \]

  \pause
  \[
    f(x) = (x, 1-x)
  \]
\end{frame}
%%%%%%%%%%%%%%%

%%%%%%%%%%%%%%%
\begin{frame}{}
  \begin{exampleblock}{Infinite Sequences of $0$'s and $1$'s (UD Problem $23.4$)}
    Is the set of all infinite sequences of $0$'s and $1$'s finite,
    countably infinite, or uncountable?
  \end{exampleblock}

  \pause
  \fig{width = 0.35\textwidth}{figs/diagonal-argument-01}
  \begin{center}
    \red{By Diagonal Argument.}
  \end{center}
\end{frame}
%%%%%%%%%%%%%%%

%%%%%%%%%%%%%%%
\begin{frame}{}
  \begin{exampleblock}{Infinite Sequences of $0$'s and $1$'s (UD Problem $23.4$)}
    Is the set of all infinite sequences of $0$'s and $1$'s finite,
    countably infinite, or uncountable?
  \end{exampleblock}

  \[
    f: \set{\set{0,1}^{\ast}} \to \N
  \]

  \pause
  \[
    \purple{f(x_0 x_1 \cdots) = \sum_{i=0}^{\infty} x_i 2^{i}}
  \]

  \pause
  \fig{width = 0.20\textwidth}{figs/wrong}
\end{frame}
%%%%%%%%%%%%%%%

%%%%%%%%%%%%%%%
\begin{frame}{}
  \begin{exampleblock}{Complex Numbers (UD Problem $24.16$)}
    Prove that
    \[
      |\R| = |\CP|, \quad \CP = \set{a + bi: a, b \in \R}
    \]
  \end{exampleblock}

  \pause
  \[
    |\CP| = |\R \times \R| \pause = |\R|
  \]
\end{frame}
%%%%%%%%%%%%%%%

%%%%%%%%%%%%%%%
\begin{frame}{}
  \[
    \R \times \R \approx \R
  \]

  \pause
  \[
    (x = 0.\blue{a_1 a_2 a_3 \cdots}, y = 0.\red{b_1 b_2 b_3 \cdots}) \pause \mapsto 0.\blue{a_1}\red{b_1}\blue{a_2}\red{b_2}\blue{a_3}\red{b_3}\cdots
  \]

  \pause
  \fig{width = 0.25\textwidth}{figs/flawed}

  \begin{center}
    \href{https://www.maa.org/sites/default/files/pdf/pubs/AMM-March11\_Cantor.pdf}{\teal{Was Cantor Surprised?}}
  \end{center}
\end{frame}
%%%%%%%%%%%%%%%

%%%%%%%%%%%%%%%
\begin{frame}{}
  \begin{exampleblock}{UD Problem $24.15$}
    \[
      (0, 1) \approx (0, 1) \times (0, 1)
    \]
  \end{exampleblock}

  \pause
  \begin{theorem}[Cantor-Schr\"{o}der–Bernstein (1887)]
    \[
      |X| \le |Y| \land |Y| \le |X| \implies |X| = |Y|
    \]
    \pause
    \[
      \exists\; \text{\blue{one-to-one} } f: X \to Y \land g: Y \to X \implies \exists\; \text{\blue{bijection} } h: X \to Y
    \]
  \end{theorem}

  \pause
  \[
    f: (0, 1) \to (0, 1) \times (0, 1)
  \]

  \pause
  \[
    \teal{f(x) = (x, 0.5)}
  \]

  \pause
  \[
    g: (0, 1) \times (0, 1) \times (0, 1)
  \]

  \pause
  \[
    \teal{(x = 0.\blue{a_1 a_2 a_3 \cdots}, y = 0.\red{b_1 b_2 b_3 \cdots}) \mapsto 0.\blue{a_1}\red{b_1}\blue{a_2}\red{b_2}\blue{a_3}\red{b_3}\cdots}
  \]
\end{frame}
%%%%%%%%%%%%%%%

%%%%%%%%%%%%%%%
\begin{frame}{}
  \begin{exampleblock}{Bijections (UD Problem $21.21$)}
    \[
      [0, 1] \approx (0, 1)
    \]
  \end{exampleblock}

  \pause
  \[
    \red{0}, \red{1}, \quad \frac{1}{2}, \frac{1}{3}, \quad \frac{1}{4}, \frac{1}{5} \cdots
  \]

  \pause
  \[
    f(0) = \frac{1}{2} \quad f(1) = \frac{1}{3}
  \]

  \pause
  \[
    \forall n \ge 4: f(\frac{1}{n-2}) = \frac{1}{n}
  \]

  \pause
  \[
    f(x) = x, \text{ otherwise}
  \]
\end{frame}
%%%%%%%%%%%%%%%

%%%%%%%%%%%%%%%
\begin{frame}{}
  \[
    (-\infty, \infty) \approx (0, \infty)
  \]

  \pause
  \[
    \teal{f(x) = e^{x}}
  \]

  \pause
  \[
    (0, \infty) \approx (0, 1)
  \]

  \pause
  \[
    \teal{f(x) = \frac{x}{x+1}}
  \]

  \pause
  \[
    [0, 1] \approx (0, 1]
  \]

  \pause
  \[
    \teal{f(0) = \frac{1}{2} \quad f(\frac{1}{2}) = \frac{2}{3} \quad f(\frac{2}{3}) = \frac{3}{4} \quad \cdots \quad f(x) = x}
  \]
\end{frame}
%%%%%%%%%%%%%%%