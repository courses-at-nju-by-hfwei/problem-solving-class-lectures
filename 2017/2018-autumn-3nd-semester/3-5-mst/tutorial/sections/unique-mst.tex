% file: sections/unique-mst.tex

%%%%%%%%%%%%
\begin{frame}{}
  \centerline{\teal{\Large Uniqueness of MST}}
\end{frame}
%%%%%%%%%%%%%

%%%%%%%%%%%%
\begin{frame}{}
  \begin{exampleblock}{Uniqueness of MST (Problem $4.29$)}
    \centerline{Distinct weights $\implies$ Unique MST}
  \end{exampleblock}

  \pause
  \vspace{0.50cm}
  \centerline{\red{By Contradiction.}}

  \pause
  \[
    \exists \text{ MSTs } T_1 \neq T_2
  \]

  \pause
  \[
    \Delta E = \set{e \mid e \in T_1 \setminus T_2 \lor e \in T_2 \setminus T_1}
  \]

  \pause
  \[
    e = \min \Delta E
  \]

  \pause
  \[
    e \in T_1 \setminus T_2 \;\text{\small \it (w.l.o.g)}
  \]
\end{frame}
%%%%%%%%%%%%%

%%%%%%%%%%%%
\begin{frame}{}
  \fignocaption{width = 0.50\textwidth}{figs/mst-unique.pdf}

  \pause
  \vspace{-0.30cm}
  \[
    T_2 + \set{e} \implies C
  \]

  \pause
  \vspace{-0.30cm}
  \[
    \exists (e' \in C) \red{\;\notin\; T_1} \pause \implies e' \in T_{2} \setminus T_{1} \implies e' \in \Delta E \pause \implies w(e') > w(e)
  \]

  \pause
  \vspace{-0.50cm}
  \[
    T' = T_{2} + \set{e} - \set{e'} \implies w(T') < w(T_{2})
  \]
\end{frame}
%%%%%%%%%%%%%

%%%%%%%%%%%%%
\begin{frame}{}
  \begin{exampleblock}{Condition for Uniqueness of MST}
    \centerline{Unique MST $\centernot\implies$ Distinct weights}
  \end{exampleblock}

  \pause
  \fignocaption{width = 0.30\textwidth}{figs/unique-mst-partition.pdf}
\end{frame}
%%%%%%%%%%%%%

%%%%%%%%%%%%%
\begin{frame}{}
  \begin{exampleblock}{Unique MST}
    \centerline{Unique MST $\centernot\implies$ Minimum-weight edge across any cut is unique}
  \end{exampleblock}

  \pause
  \fignocaption{width = 0.30\textwidth}{figs/unique-mst-cut-counterexample.pdf}

  \pause
  \begin{theorem}[After-class Exercise]
    \centerline{Minimum-weight edge across any cut is unique $\implies$ Unique MST}
  \end{theorem}
\end{frame}
%%%%%%%%%%%%%

%%%%%%%%%%%%%
\begin{frame}{}
  \begin{exampleblock}{Unique MST}
    Unique MST $\centernot\implies$ Maximum-weight edge in any cycle is unique
  \end{exampleblock}

  \pause
  \fignocaption{width = 0.30\textwidth}{figs/unique-mst-cycle-counterexample.pdf}

  \pause
  \begin{theorem}[After-class Exercise]
    \centerline{Maximum-weight edge in any cycle is unique $\implies$ Unique MST}
  \end{theorem}

  % \pause
  % \fignocaption{width = 0.20\textwidth}{figs/unique-mst-cycle-noncounterexample.pdf}
\end{frame}
%%%%%%%%%%%%

%%%%%%%%%%%%
\begin{frame}{}
  \fignocaption{width = 0.20\textwidth}{figs/qrcode-jeffe-unique-mst}

  \begin{columns}
    \column{0.50\textwidth}
      \fignocaption{width = 0.40\textwidth}{figs/qrcode-cut-unique-mst}
    \column{0.50\textwidth}
      \fignocaption{width = 0.40\textwidth}{figs/qrcode-cycle-unique-mst}
  \end{columns}
\end{frame}
%%%%%%%%%%%%

%%%%%%%%%%%%
% \begin{frame}{}
%   \begin{exampleblock}{Unique MST}
%     To decide whether a graph has a unique MST.
%   \end{exampleblock}
% 
%   \pause
%   \vspace{0.80cm}
%   \centerline{\large \red{Ties} in Prim's and Kruskal's algorithms}
% 
%   \pause
%   \fignocaption{width = 0.25\textwidth}{figs/unique-mst-cut-counterexample.pdf}
% 
%   \pause
%   \vspace{-0.30cm}
%   \[
%     \underbrace{\red{T}}_{\teal{\text{Any MST}}} +\; \underbrace{\red{\set{e}}, \forall e \notin T}_{\teal{\text{Cycle}}}
%   \]
% 
%   \pause
%   \centerline{\purple{By Kruskal Algorithm.}}
% \end{frame}
%%%%%%%%%%%%

%%%%%%%%%%%%
% \begin{frame}{}
%   \begin{exampleblock}{Critical Edges}
%     \[
%       G \to T, \qquad G' \triangleq G \setminus \red{\set{e}} \to T'
%     \]
%     \[
%       w(T') > w(T)
%     \]
% 
%     \pause
%     \vspace{0.30cm}
%     \centerline{To find all critical edges in $O(m \log m)$ time.}
%   \end{exampleblock}
% 
%   \pause
%   \begin{columns}
%     \column{0.50\textwidth}
%       \fignocaption{width = 0.50\textwidth}{figs/critical-edges}
%       \[
% 	w(e) = 3 \quad w(e) = 2
%       \]
%     \column{0.50\textwidth}
%       \pause
%       \centerline{\teal{By Kruskal Algorithm.}}
% 
%       \pause
%       \vspace{0.50cm}
%       \centerline{\purple{No missing: Check all cycles.}}
% 
%       \pause
%       \[
% 	\red{O(m \log m)}
%       \]
%   \end{columns}
% \end{frame}
%%%%%%%%%%%%
