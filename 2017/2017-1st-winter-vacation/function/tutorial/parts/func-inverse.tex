%%%%%%%%%%%%%%%
\begin{frame}{}
  \begin{definition}[Inverse]
    Let $f: A \to B$ be a \red{bijective} function.

    The \blue{inverse} of $f$ is the function $f^{-1}: B \to A$ defined by
    \[
      f^{-1}(b) = a \iff f(a) = b.
    \]
  \end{definition}

  \begin{alertblock}{``Bijective'' Requirement of $f^{-1}$:}
    \[
      f: A \to B \quad f \subseteq A \times B
    \]

    \pause
    \[
      f^{-1} \subseteq B \times A \quad \text{(as a relation)}
    \]
    \pause
    \[
      \blue{f^{-1}: X \; (\subseteq B) \to A \quad \text{(as a function to $A$)}}
    \]
    \pause
    \[
      \red{f^{-1}: B \to A \quad \text{(as a function from $B$ to $A$)}}
    \]
  \end{alertblock}
\end{frame}
%%%%%%%%%%%%%%%

%%%%%%%%%%%%%%%
\begin{frame}{}
  \begin{theorem}[(UD Theorem $15.4$ (ii))]
    \[
      f: A \to B \text{ is bijective } \implies f^{-1} \text{ is bijective}.
    \]
  \end{theorem}
\end{frame}
%%%%%%%%%%%%%%%

%%%%%%%%%%%%%%%
\begin{frame}{}
  \begin{theorem}[Solving Equations (UD Theorem 15.4)]
    \[
      f: A \to B \text{ is bijective}
    \]

    \begin{enumerate}[(i)]
      \item $f \circ f^{-1} = i_B$
      \item $g: B \to A \land f \circ g = i_B \implies g = f^{-1}$
      \vspace{0.30cm}
      \item $f^{-1} \circ f = i_A$
      \item $g: B \to A \land g \circ f = i_A \implies g = f^{-1}$
    \end{enumerate}
  \end{theorem}

  \pause
  \begin{alertblock}{Solving the equations:}
    \[
      f \circ g = i_B \qquad g \circ f = i_A
    \]
  \end{alertblock}
\end{frame}
%%%%%%%%%%%%%%%

%%%%%%%%%%%%%%%
\begin{frame}{}
  \begin{alertblock}{Bijective $\implies$ Inverse:}
    \[
      f: A \to B \text{ is bijective }
    \]
    \[ 
      \implies 
    \]
    \[
      \exists g: B \to A\; \Big( f \circ g = i_B \land g \circ f = i_A \Big) \pause \red{\;\land\; g = f^{-1}}
    \]
  \end{alertblock}

  \pause
  \vspace{0.50cm}
  \begin{theorem}[Inverse $\implies$ Bijective (UD Theorem $15.8$ (iii))]
    \[
      \exists g: B \to A\; \Big( g \circ f = i_A \land f \circ g = i_B \Big) 
    \]
    \[ 
      \implies 
    \]
    \[
      f: A \to B \text{ is bijective} \pause \red{\;\land\; g = f^{-1}}
    \]
  \end{theorem}
\end{frame}
%%%%%%%%%%%%%%%

%%%%%%%%%%%%%%%
\begin{frame}{}
  \begin{theorem}[Inverse of Composition (UD Theorem $15.6$)]
    \[
      f: A \to B, g: B \to C \text{ are bijective}
    \]

    \begin{enumerate}[(i)]
      \item $g \circ f \text{ is bijective}$
      \item $(g \circ f)^{-1} = f^{-1} \circ g^{-1}$
    \end{enumerate}
  \end{theorem}

  \begin{proof}[Proof for (ii)]
    \[
      (f^{-1} \circ g^{-1}) \circ (g \circ f) = i_A
    \]

    \[
      (g \circ f) \circ (f^{-1} \circ g^{-1}) = i_B
    \]
  \end{proof}
\end{frame}
%%%%%%%%%%%%%%%

%%%%%%%%%%%%%%%
\begin{frame}{}
  \begin{definition}[Symmetric Group]
    Let $A$ be a set.

    Consider all bijective functions on $A$ and the composition ($\circ$) operator.

    \begin{enumerate}[(i)]
      \item $f \circ g$ is a bijective function on $A$
      \item $h \circ (\circ g \circ f) = (h \circ g) \circ f$
      \item $f \circ id_A = f = id_A \circ f$
      \item $f \circ f^{-1} = id_A = f^{-1} \circ f$
    \end{enumerate}
  \end{definition}

  \pause
  \vspace{0.50cm}
  \fignocaption{width = 0.25\textwidth}{figs/stay-tuned}
\end{frame}
%%%%%%%%%%%%%%%
