%%%%%%%%%%%%%%%
\begin{frame}{}
  \begin{center}
    {\LARGE Function}

    \pause
    \fignocaption{width = 0.25\textwidth}{figs/aims}

    \pause
    \vspace{0.20cm}
    \centerline{\Large \red{PROOF! PROOF! PROOF!}}
  \end{center}
\end{frame}
%%%%%%%%%%%%%%%

%%%%%%%%%%%%%%%
\begin{frame}{}
  \begin{center}
    {\LARGE Definition of Function}
  \end{center}
\end{frame}
%%%%%%%%%%%%%%%

%%%%%%%%%%%%%%%
\begin{frame}{}
  \begin{definition}[Relation]
    \centerline{Let $A$ and $B$ be sets.}

    $R$ is a (binary) relation if

    \[
      R \subseteq A \times B = \set{(a, b) \mid a \in A \land b \in B}
    \]
  \end{definition}

  \pause
  \vspace{0.30cm}
  \[
    (a,b) = \set{\set{a}, \set{a,b}} \quad \text{(UD Problem $9.16$)}
  \]
\end{frame}
%%%%%%%%%%%%%%%

%%%%%%%%%%%%%%%
\begin{frame}{}
  \begin{definition}[Function]
    A \blue{\textbf{function}} $f$ from $A$ to $B$ is a \red{\it relation} $f$ from $A$ to $B$ such that
    \[
      \red{\forall} a \in A\; \red{\exists!} b \in B\; (a, b) \in f.
    \]
  \end{definition}

  \pause
  \begin{alertblock}{For Proof:}
    \[
      \red{\forall}
    \]
    \[
      \red{\exists!}: \forall b, b' \in B, (a, b) \in f \land (a, b') \in f \implies b = b'.
    \]
  \end{alertblock}

  \pause
  \begin{exampleblock}{Notations:}
    \[
      f: A \to B, \quad a \mapsto b = f(a)
    \]

    \centerline{$A: \dom{f}$ \qquad $B: \cod{f}$}

    \[
      \ran{f} = f(A) = \set{f(a) \mid a \in A} \subseteq B
    \]
  \end{exampleblock}
\end{frame}
%%%%%%%%%%%%%%%

%%%%%%%%%%%%%%%
\begin{frame}{}
  \[
    D(x) = \left\{\begin{array}{ll}
      1 & \text{if } x \in \mathbb{Q} \\
      0 & \text{if } x \in \real \setminus \mathbb{Q} 
    \end{array}\right.
  \]

  \vspace{0.60cm}
  \centerline{Dirichlet Function}
\end{frame}
%%%%%%%%%%%%%%%

%%%%%%%%%%%%%%%
\begin{frame}{}
  \fignocaption{width = 0.50\textwidth}{figs/weierstrass-function}{\centerline{Weierstrass Function (1872)}}

  \[
    f(x)=\sum_{n=0} ^\infty a^n \cos(b^n \pi x) 
  \]

  \[
    0 < a < 1,\; b \in 2\nat + 1,\; ab > 1+\frac{3}{2} \pi
  \]
\end{frame}
%%%%%%%%%%%%%%%

%%%%%%%%%%%%%%%
\begin{frame}{}
  \begin{exampleblock}{UD Problem $13.3\; (g)$}
    \[
      f: \mathbb{Q} \to \real{}
    \]

    \[
      f(x) = \left\{\begin{array}{ll}
	x + 1 & \text{if } x \in 2\integer{} \\
	x - 1 & \text{if } x \in 3\integer{} \\
	2     & \text{otherwise}
      \end{array}\right.
    \]
  \end{exampleblock}
\end{frame}
%%%%%%%%%%%%%%%

%%%%%%%%%%%%%%%
\begin{frame}{}
  \begin{exampleblock}{UD Problem $13.4$}
    \[
      f: \mathcal{P}(\real{}) \to \integer{}
    \]

    \[
      f(A) = \left\{\begin{array}{ll}
	\min(A \cap \nat{}) & \text{if } A \cap \nat{} \neq \emptyset \\
	-1 & \text{if } A \cap \nat{} = \emptyset
      \end{array}\right.
    \]
  \end{exampleblock}
\end{frame}
%%%%%%%%%%%%%%%

%%%%%%%%%%%%%%%
\begin{frame}{}
  \begin{definition}[Axiom of Extensionality (集合的外延公理)]
    \[
      \forall A \forall B \forall \blue{x} (\blue{x} \in A \iff \blue{x} \in B) \iff A = B.
    \]
  \end{definition}

  % \vspace{0.80cm}
  % \centerline{Intensionality (内涵) \emph{vs.} Extensionality (外延)}
  % \vspace{0.80cm}

  \pause
  \vspace{0.80cm}
  \begin{definition}[函数的外延性原则]
    \[
      f = g \iff \dom{f} = \dom{g} \land \Big(\forall x \in \dom{f}: f(x) = g(x)\Big)
    \]
  \end{definition}
\end{frame}
%%%%%%%%%%%%%%%
