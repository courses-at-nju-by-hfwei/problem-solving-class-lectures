%%%%%%%%%%%%%%%
\begin{frame}{}
  \centerline{\LARGE Bounded Iterations \emph{vs.} Unbounded Iterations}

  \vspace{0.80cm}
  \fignocaption{width = 0.40\textwidth}{figs/sisyphus}

  \vspace{0.40cm}
  \pause
  \centerline{\Large \red{$Q:$} Why unbounded iterations?}
\end{frame}
%%%%%%%%%%%%%%%

%%%%%%%%%%%%%%%
\begin{frame}{}
  \fignocaption{width = 0.15\textwidth}{figs/mu}{\vspace{-0.50cm} \centerline{$\mu$-Recursive Functions}}

  \vspace{0.20cm}
  \[
    \red{\mu} y \big(g(x, y)\big) = \Big(\argmin_{y} g(x,y) = 0\Big)
  \]

  \pause
  \centerline{\red{Unbounded} iterations: ``\texttt{while-do}''}

  \vspace{0.40cm}
  \pause
  \begin{theorem}[Ackermann Function]
    The Ackermann function is \red{$\mu$}-recursive but not \blue{primitive} recursive (which contains \blue{bounded} iterations.).
  \end{theorem}
\end{frame}
%%%%%%%%%%%%%%%

%%%%%%%%%%%%%%%
\begin{frame}[fragile]{}
  \begin{exampleblock}{DH 2.4: Bounded Iteration}
    Given a list $L$ of $N$ integers,
    to produce in $S$ and $P$ the sum of the even numbers in $L$ and the product of the odd ones, respectively.
  \end{exampleblock}

  \pause
  \begin{lstlisting}[style = Cstyle]
    int S = 0, P = 1;
    for (int i = 0; i < N; ++i) {
      if (L(i) % 2 == 0)
        S += L(i);
      else
        P *= L(i);
    }
  \end{lstlisting}

  \begin{columns}
    \pause
    \column{0.40\textwidth}
      \begin{exampleblock}{DH 2.1: Salary Summation}
	$N-1$ vs. $N$ iterations
      \end{exampleblock}
    \pause
    \column{0.55\textwidth}
      \fignocaption{width = 0.30\textwidth}{figs/forget-it}
  \end{columns}
\end{frame}
%%%%%%%%%%%%%%%

%%%%%%%%%%%%%%%
\begin{frame}[fragile]{}
  \begin{exampleblock}{DH 2.7: Compute $n!$}
    Write algorithms that compute $n!$, given a non-negative integer $n$.
    \begin{enumerate}[(a)]
      \item Using iteration statements.
      \item Using recursion.
    \end{enumerate}
  \end{exampleblock}

  \pause
  \begin{lstlisting}[style = Cstyle]
  int P = 1;
  for (int i = 2; i <= n; ++i) {
    P *= i;
  }
  \end{lstlisting}

  \pause
  \begin{lstlisting}[style = Cstyle]
  int recursive-factorial(int n) { |\uncover<4->{\textcolor{cyan}{// define function}}|
    if (n == 0)
      return 1;
      |\only<5->{\textcolor{cyan}{// \red{NOT:} return $n * (n-1)!$}}|
      else return n * recursive-factorial(n-1); 
  }
  \end{lstlisting}
\end{frame}
%%%%%%%%%%%%%%%
