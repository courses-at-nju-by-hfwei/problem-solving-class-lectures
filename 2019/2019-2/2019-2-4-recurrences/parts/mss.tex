% mss.tex

%%%%%%%%%%%%%%%
\begin{frame}{}
  \begin{exampleblock}{Maximum-\red{sum} Subarray (\mss; Problem 4.1-5)}
    \[
      A[0 \ldots n-1] \qquad \forall\; 0 \le i \le n-1: \red{A[i] \in \Z}
    \]

    \pause
    \begin{center}
      To find (the sum of) a maximum-sum \red{subarray} of $A$
    \end{center}
  \end{exampleblock}

  \pause
  \[
    A = [-2, 11, -4, 13, -5, -2]
  \]

  \pause
  \vspace{-0.30cm}
  \[
    \mss = 11 + (-4) + 13 = 20
  \]
  
  % \pause
  % \[
  %   A = [-2, 1, -3, 4, -1, 2, 1, -5, 4]
  % \]

  % \pause
  % \vspace{-0.30cm}
  % \[
  %   \mss = 4 + (-1) + 2 + 1 = 6
  % \]

  \pause
  \[
    \forall\; 0 \le i \le n - 1: A[i] < 0 
  \]
  \[
    \textcolor{gray}{\mss = 0} \;\text{\it vs. } \blue{\mss = \max\limits_{0 \le i \le n-1} A[i]}
  \]
\end{frame}
%%%%%%%%%%%%%%%

%%%%%%%%%%%%%%%
% \begin{frame}{}
%   \[
%     \mssprefix[i]: \text{(the sum of) a maximum-sum subarray in } A[1 \cdots i]
%   \]
% 
%   \pause
%   \[
%     \mss = \mssprefix[n - 1]
%   \]
% 
%   \pause
%   \[
%     \red{Q: \text{Is } a_{i} \in \mssprefix[i]?}
%   \]
%   
%   \pause
%   \[ 
%     \mssprefix[i] = \max \set{\mssprefix[i-1], \red{???}}
%   \]
% \end{frame}
%%%%%%%%%%%%%%%

%%%%%%%%%%%%%%%
\begin{frame}{}
  \[
    \mssat[i]: \text{(the sum of) a maximum-sum subarray \red{ending with }} A[i]
  \]

  \pause
  \[
    \mss = \max_{0 \le i \le n-1} \mssat[i]
  \]

  \pause
  \[
    \red{Q: \text{ Where does } \mssat[i] \text{ start}?}
  \]

  \pause
  \[
    \blue{\boxed{\mssat[i] = \max \set{\mssat[i-1] + A[i], A[i]}}}
  \]

  \pause
  \[
    \mssat[0] = A[0]
  \]

  % \begin{displaymath}
  %   \text{MSS}[i] = \left\{ \begin{array}{ll}
  %     0 & i = 0 \\
  %     \max \set{\text{MSS}[i-1] + a_{i}, 0} & i > 0
  %   \end{array} \right.
  % \end{displaymath}
\end{frame}
%%%%%%%%%%%%%%%

%%%%%%%%%%%%%%%
\begin{frame}
  \begin{columns}
    \column{0.10\textwidth}
    \column{0.80\textwidth}
      % mss-n-1.tex

\begin{algorithm}[H]
  \begin{algorithmic}[1]
    \Procedure{mss}{$A, n$}
      \State $\mssat[0] \gets A[0]$

      \hStatex
      \For{$i \gets 1 \ldots n-1$}
        \State \red{$\mssat[i] \gets \max\set{\mssat[i-1] + A[i], A[i]}$}
      \EndFor

      \hStatex
      \State \Return \blue{$\max\limits_{0 \le i \le n-1} \mssat[i]$}
    \EndProcedure
  \end{algorithmic}
\end{algorithm}
    \column{0.10\textwidth}
  \end{columns}

  \pause
  \vspace{0.80cm}
  \begin{table}[H]
    \centering
    \begin{tabular}{c|c}
      {\it time} & {\it space} \\ \hline
      $O(n)$     & $O(n)$      
    \end{tabular}
  \end{table}
\end{frame}
%%%%%%%%%%%%%%%

%%%%%%%%%%%%%%%
\begin{frame}
  \begin{columns}
    \column{0.10\textwidth}
    \column{0.70\textwidth}
      % mss-n-2.tex

\begin{algorithm}[H]
  \begin{algorithmic}[1]
    \Procedure{mss}{$A, n$}
      \State \blue{$\mss \gets -\infty$}
      \State $\mssat \gets A[0]$

      \hStatex
      \For{$i \gets 1 \ldots n-1$}
          \State \red{$\mssat \gets \max\set{\mssat + A[i], A[i]}$}
          \State \blue{$\mss \gets \max\set{\mss, \mssat}$}
      \EndFor

      \hStatex
      \State \Return $\mss$
    \EndProcedure
  \end{algorithmic}
\end{algorithm}
    \column{0.10\textwidth}
  \end{columns}

  \pause
  \vspace{0.80cm}
  \begin{table}[H]
    \centering
    \begin{tabular}{c|c}
      {\it time} & {\it space} \\ \hline
      $O(n)$     & $O(1)$      
    \end{tabular}
  \end{table}
\end{frame}
%%%%%%%%%%%%%%%

%%%%%%%%%%%%%%%
\begin{frame}{}
  \begin{columns}
    \column{0.40\textwidth}
      \fig{width = 0.60\textwidth}{figs/programming-pearls.jpg}
    \column{0.60\textwidth}
      \begin{description}[<+->][Michael Shamos]
        \setlength{\itemsep}{8pt}
        \item[Ulf Grenander] $O(n^3) \implies O(n^2)$
        \item[Michael Shamos] $O(n \log n)$, one night
        \item[Jon Bentley] \purple{Conjecture}: $\Omega(n \log n)$
        \item[Michael Shamos] Carnegie Mellon seminar
        \item[Jay Kadane] $O(n)$, \uncover<6->{\textcolor{red}{$\le 1$ minute}}
      \end{description}
  \end{columns}
\end{frame}
%%%%%%%%%%%%%%%

%%%%%%%%%%%%%%%
% \begin{frame}{}
%   \begin{exampleblock}{Maximum-\red{product} Subarray (\mps)}
%     \[
%       A[0 \ldots n-1] \qquad \forall 0 \le i \le n-1: \red{A[i] \in \Z}
%     \]
% 
%     \pause
%     \begin{center}
%       To find (the \red{product} of) a maximum-\red{product} subarray of $A$
%     \end{center}
%   \end{exampleblock}
% 
%   \pause
%   \vspace{0.30cm}
%   \[
%     A = [\frac{1}{2}, 4, -2, 5, -\frac{1}{5}, 8]
%   \]
% 
%   \pause
%   \[
%     \mps = 4 \times (-2) \times 5 \times (-\frac{1}{5}) \times 8 = 64
%   \]
% \end{frame}
%%%%%%%%%%%%%%%

%%%%%%%%%%%%%%%
% \begin{frame}{}
%   \[
%     \mpsat[i]: \text{(the product of) a maximum-product subarray \red{ending with }} A[i]
%   \]
% 
%   \pause
%   \[
%     \mps = \max_{0 \le i \le n-1} \mpsat[i]
%   \]
% 
%   \pause
%   \[
%     \red{Q: \text{ Where does } \mpsat[i] \text{ start}?}
%   \]
% 
%   \pause
%   \[
%     \blue{\mpsat[i] = \max \set{\mpsat[i-1] \cdot A[i], A[i]}}
%   \]
% 
%   \pause
%   \begin{table}
%     \renewcommand{\arraystretch}{1.8}
%     \centering
%     \begin{tabular}{|C||C|C|C|C|C|C|C|}
%       \hline
%       $A$ &	& \frac{1}{2} & 4 & -2 & \textcolor{red}{5} & \textcolor{red}{-\frac{1}{5}} & 8 \\ \hline
%       \mpsat[i] & 1	& \frac{1}{2} & 4 & -2 & 5 & 8 & 64 \\ \hline
%     \end{tabular}
%   \end{table}
% \end{frame}
%%%%%%%%%%%%%%%

%%%%%%%%%%%%%%%
% \begin{frame}{}
%   \begin{table}
%     \renewcommand{\arraystretch}{1.8}
%     \centering
%     \begin{tabular}{|C||C|C|C|C|C|C|C|}
%       \hline
%       $A$ &	& \frac{1}{2} & 4 & -2 & \textcolor{red}{5} & \textcolor{red}{-\frac{1}{5}} & 8 \\ \hline
%       \red{\text{MaxP}[i]} & 1	& \frac{1}{2} & 4 & -2 & 5 & 8 & 64 \\ \hline
%       \blue{\text{MinP}[i]} & 1	& \frac{1}{2} & 2 & -8 & \red{-40} & -1 & -8  \\ \hline
%     \end{tabular}
%   \end{table}
% 
%   \pause
%   \begin{align*}
%     \red{\text{MaxP}[i]} &= \max\set{\text{MaxP}[i-1] \cdot A[i],\;
%          \text{MinP}[i-1] \cdot A[i],\; A[i]} \\[6pt]
%     \blue{\text{MinP}[i]} &= \min\set{\text{MaxP}[i-1] \cdot A[i],\;
%           \text{MinP}[i-1] \cdot A[i],\; A[i]}
%   \end{align*}
% \end{frame}
%%%%%%%%%%%%%%%

%%%%%%%%%%%%%%%
% \begin{frame}{}
%   2d version
% \end{frame}
%%%%%%%%%%%%%%%