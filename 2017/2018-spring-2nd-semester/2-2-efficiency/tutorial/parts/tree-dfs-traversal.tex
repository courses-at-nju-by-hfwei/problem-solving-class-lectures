% file: tree-dfs-traversal.tex
% DH 4.2 (a), (b), (c)

%%%%%%%%%%%%%%%
\begin{frame}{}
  \begin{exampleblock}{DH 4.2 (a): Sum of Depths}
    \fignocaption{width = 0.45\textwidth}{figs/tree-traversal-depth}
  \end{exampleblock}

  \pause
  \vspace{-0.80cm}
  \[
    \textsf{sum-of-depths}(r) = \left\{\begin{array}{lr}
      \uncover<3->{\textsf{depth of } r, & r \textsf{ is a leaf}} \\[6pt]
      \sum\limits_{v: \textsf{child of } r} \textsf{sum-of-depths}(v) + \textsf{depth of } r, & \uncover<3->{\textsf{o.w.}}
    \end{array}\right.
  \]
\end{frame}
%%%%%%%%%%%%%%%

%%%%%%%%%%%%%%%
\begin{frame}{}
  \begin{exampleblock}{DH 4.2 (a): Sum of Depths}
    \fignocaption{width = 0.45\textwidth}{figs/tree-traversal-depth}
  \end{exampleblock}

  \vspace{-0.80cm}
  \[
    \textsf{sum-of-depths}(r, \red{d}) = \left\{\begin{array}{lr}
      d, & r \textsf{ is a leaf} \\[6pt]
      \sum\limits_{v: \textsf{child of } r} \textsf{sum-of-depths}(v, \red{d + 1}) + d, & \textsf{o.w.}
    \end{array}\right.
  \]
\end{frame}
%%%%%%%%%%%%%%%

%%%%%%%%%%%%%%%
\begin{frame}{}
  % file: algs/sum-of-depths.tex
% for UD Problem 4.2 (a)

\begin{algorithm}[H]
  \caption{Calculate the sum of depths of all nodes of a tree $T$.}
  \label{alg:sum-of-depths}
  \begin{algorithmic}[1]
    \Procedure{Sum-of-Depths}{\null}
      \State \Return $\Call{Sum-of-Depths}{T, 0}$ 
    \EndProcedure

    \Statex
    \Procedure{Sum-of-Depths}{$r, depth$} \Comment{$r$: root of a tree}
      \If{$r$ is a leaf}
        \State \Return $depth$
      \EndIf

      \hStatex
      \State $sum \gets depth$
      \ForAll{child vertex $v$ of $r$}
	\State $sum \gets sum + \Call{Sum-of-Depths}{v, depth + 1}$
      \EndFor
      \State \Return $sum$
    \EndProcedure
  \end{algorithmic}
\end{algorithm}
\end{frame}
%%%%%%%%%%%%%%%

%%%%%%%%%%%%%%%
\begin{frame}{}
  \[
    \textsf{sum-of-depths}(r, \red{d}) = \left\{\begin{array}{lr}
      d, & r \textsf{ is a leaf} \\[6pt]
      \sum\limits_{v: \textsf{child of } r} \textsf{sum-of-depths}(v, \red{d + 1}) + d, & \textsf{o.w.}
    \end{array}\right.
  \]

  \pause
  \centerline{\red{\large Master Theorem?}}

  \pause
  \[
    \Theta(m + n) = \Theta(n)
  \]
\end{frame}
%%%%%%%%%%%%%%%

%%%%%%%%%%%%%%%
\begin{frame}{}
  \begin{exampleblock}{DH 4.2 (b): Number of Nodes at Depth $K$}
    \fignocaption{width = 0.45\textwidth}{figs/tree-traversal-depth}
  \end{exampleblock}

  \pause
  \vspace{-0.60cm}
  \[
    \textsf{nodes-at-depth}(r, \red{k}) = \left\{\begin{array}{ll}
      \uncover<3->{1, \hspace{4.00cm} k = 0} & \\[5pt]
      \uncover<3->{0, \hspace{4.00cm} k > 0 \land r \textsf{ is a leaf}} & \\[5pt]
      \uncover<2->{\sum\limits_{v: \textsf{child of } r} \textsf{nodes-at-depth}(v, \red{k - 1}),} \qquad \uncover<3->{\textsf{o.w.}} &
      \end{array}\right.
  \]
\end{frame}
%%%%%%%%%%%%%%%

%%%%%%%%%%%%%%%
\begin{frame}{}
  % file: algs/nodes-at-depth-K.tex
% for UD Problem 4.2 (b)

\begin{algorithm}[H]
  \caption{Count the number of nodes in $T$ at depth $K$.}
  \label{alg:nodes-at-depth-K}
  \begin{algorithmic}[1]
    \Procedure{Nodes-at-Depth}{\null}
      \State \Return $\Call{Nodes-at-Depth}{T, K}$ 
    \EndProcedure

    \Statex
    \Procedure{Nodes-at-Depth}{$r, k$} \Comment{$r$: root of a tree}
      \If{$k = 0$}
        \State \Return 1
      \EndIf

      \Statex
      \If{$r$ is a leaf}
	\State \Return 0
      \EndIf

      \Statex
      \State $num \gets 0$
      \ForAll{child vertex $v$ of $r$}
	\State $num \gets num + \Call{Nodes-at-Depth}{v, k - 1}$
      \EndFor
      \State \Return $num$
    \EndProcedure
  \end{algorithmic}
\end{algorithm}

\end{frame}
%%%%%%%%%%%%%%%

%%%%%%%%%%%%%%%
\begin{frame}{}
  \begin{exampleblock}{DH 4.2 (c): Any Leaf at an Even Depth?}
    \fignocaption{width = 0.45\textwidth}{figs/tree-traversal-depth}
  \end{exampleblock}

  \pause
  \vspace{-0.60cm}
  \[
    \textsf{leaf-at-depth}(r, \red{parity}) = \left\{\begin{array}{lr}
      \uncover<3->{1 - parity, & r \textsf{ is a leaf}} \\[5pt]
      \sum\limits_{v: \textsf{child of } r} \textsf{}(v, \red{1 - parity}), & \qquad \uncover<3->{\textsf{o.w.}}
      \end{array}\right.
  \]
\end{frame}
%%%%%%%%%%%%%%%

%%%%%%%%%%%%%%%
\begin{frame}{}
  % file: algs/leaf-at-even-depth.tex
% for UD Problem 4.2 (c)

\begin{algorithm}[H]
  \caption{Check whether a tree $T$ has any leaf at an even depth.}
  \label{alg:leaf-at-even-depth}
  \begin{algorithmic}[1]
    \Procedure{Leaf-at-Even-Depth}{\null}
      \State \Return $\Call{Leaf-at-Depth}{T, even = 0}$
    \EndProcedure

    \Statex
    \Procedure{Leaf-at-Depth}{$r, parity$} \Comment{$r$: root of a tree}
      \If{$r$ is a leaf}
        \State \Return $1-parity$
      \EndIf

      \hStatex
      \State $result \gets 0$
      \ForAll{child vertex $v$ of $r$}
	\State $result \gets result \lor \Call{Leaf-at-Depth}{v, 1-parity}$
      \EndFor
      \State \Return $result$
    \EndProcedure
  \end{algorithmic}
\end{algorithm}


\end{frame}
%%%%%%%%%%%%%%%
