% file: stack-queue.tex

%%%%%%%%%%%%%%%
\begin{frame}{}
  \begin{columns}
    \column{0.50\textwidth}
      \fignocaption{width = 0.90\textwidth}{figs/circular-queue-array}
    \column{0.50\textwidth}
      \pause
      \fignocaption{width = 0.80\textwidth}{figs/circular-queue}
  \end{columns}

  \pause
  \vspace{0.80cm}
  \centerline{\red{\purple{head \& teal}: following the same direction}}
\end{frame}
%%%%%%%%%%%%%%%

%%%%%%%%%%%%%%%
\begin{frame}{}
  \begin{exampleblock}{Underflow and Overflow of a Circular Queue (Problem 10.1-4)}
    \fignocaption{width = 0.40\textwidth}{figs/circular-queue}
  \end{exampleblock}

  \begin{columns}
    \pause
    \column{0.50\textwidth}
      % file: algs/circular-queue-dequeue.tex

\begin{algorithm}[H]
  \begin{algorithmic}[]
    \Procedure{Dequeue}{$Q$}
      \If{\red{$Q.head = Q.tail$}}
	\State \Return ``UNDERFLOW''
      \EndIf

      \State $\cdots$
    \EndProcedure
  \end{algorithmic}
\end{algorithm}

    \pause
    \column{0.50\textwidth}
      % file: algs/circular-queue-enqueue.tex

\begin{algorithm}[H]
  \begin{algorithmic}[]
    \Procedure{Enqueue}{$Q, x$}
      \If{\red{$Q.head = Q.tail + 1$}}
	\State \Return ``OVERFLOW''
      \EndIf
      
      \State $\cdots$
    \EndProcedure
  \end{algorithmic}
\end{algorithm}

  \end{columns}
\end{frame}
%%%%%%%%%%%%%%%

%%%%%%%%%%%%%%%
\begin{frame}{}
  \fignocaption{width = 0.40\textwidth}{figs/circular-queue}

  \begin{CJK*}{UTF8}{gbsn}
    \centerline{\red{反馈:} \purple{tail} 为什么指向最后一个元素的后面? 这个太难受了。}
  \end{CJK*}

  \pause
  \[
    \teal{\textsc{Queue-Empty}}
  \]
\end{frame}
%%%%%%%%%%%%%%%

%%%%%%%%%%%%%%%
\begin{frame}{}
  \[
    \red{[l, r)}\quad (l, r]\quad [l, r]\quad (l, r)
  \]

  \pause
  \begin{columns}
    \column{0.60\textwidth}
      \fignocaption{width = 0.95\textwidth}{figs/EWD-numbering}
      \centerline{\teal{Why Numbering Should Start at Zero}}
    \column{0.40\textwidth}
      \fignocaption{width = 0.50\textwidth}{figs/dijkstra}
  \end{columns}
\end{frame}
%%%%%%%%%%%%%%%

%%%%%%%%%%%%%%%
% \begin{frame}{}
%   \begin{exampleblock}{Deque; Double-ended Queue (Problem 10.1-5)}
%   \end{exampleblock}
% \end{frame}
%%%%%%%%%%%%%%%

%%%%%%%%%%%%%%%
\begin{frame}{}
  \begin{exampleblock}{A Queue, Two Stacks (Problem 10.1-6)}
    Show how to implement \purple{a queue} using \teal{two stacks}.

    \uncover<5->{Analyze the \red{running time} of the queue operations.}
  \end{exampleblock}

  \begin{columns}
    \column{0.50\textwidth}
      \pause
      % file: algs/stack-queue/queue-2stack.tex

\begin{algorithm}[H]
  \begin{algorithmic}[]
    \Procedure{\teal{Enqueue}}{$x$}
      \State \textsl{Push}($S_1, x$)
    \EndProcedure

    \Statex
    \Procedure{\teal{Dequeue}}{\null}
      \If{$S_2 = \emptyset$}
	\While{$S_1 \neq \emptyset$}
	  \State $\textsl{Push}\Big(S_2, \textsl{Pop}(S_1)\Big)$
	\EndWhile
      \EndIf

      \State \textsl{Pop}($S_2$)
    \EndProcedure
  \end{algorithmic}
\end{algorithm}

    \column{0.50\textwidth}
      \pause
      \centerline{\red{\large Correctness?}}

      \pause
      \begin{gather*}
	\textsc{Enq}(x, t_1), \textsc{Enq}(y, t_2) \land \purple{t_1 < t_2} \\
	\implies \\
	\textsc{Deq}(x, t_3), \textsc{Deq}(y, t_4) \land \purple{t_3 < t_4}
      \end{gather*}
  \end{columns}
\end{frame}
%%%%%%%%%%%%%%%

%%%%%%%%%%%%%%%
\begin{frame}{}
  \begin{table}
    \begin{tabular}{ccccc}
      {\it \red{item}} & Push into $S_1$ & Pop from $S_1$ & Push into $S_2$ & Pop from $S_2$ \\
      \red{$x$} & $1$ & $1$ & $1$ & $1$
    \end{tabular}
  \end{table}

  \begin{columns}
    \column{0.50\textwidth}
      \pause
      \begin{align*}
	\hat{c}_{\textsc{Enq}} &= 4 \\
	\hat{c}_{\textsc{Deq}} &= 0
      \end{align*}
    \column{0.50\textwidth}
      \pause
      \begin{align*}
	\hat{c}_{\textsc{Enq}} &= 3 \\
	\hat{c}_{\textsc{Deq}} &= 1 
      \end{align*}
  \end{columns}

  % \pause
  % \vspace{-0.60cm}
  % \[ 
  %   \sum_{i=1}^{n} a_i \ge 0 \pause \Longleftarrow \sum_{i=1}^{n} a_i = \#S_1 \times 2
  % \]
\end{frame}
%%%%%%%%%%%%%%%