%%%%%%%%%%%%%%%
\begin{frame}{}
  \begin{exampleblock}{$4.$ 关系与序 (Order)}
    一个自反(\blue{reflexive})且传递(\blue{transitive})的二元关系 $R \subseteq X \times X$ 
    称为 $X$ 上的拟序 (\red{preorder/\textcolor{gray}{quasiorder}})。\\[8pt]

    令 $\le \;\subseteq X \times X$ 为拟序, 请证明:
    \begin{enumerate}[(1)]
      \item 如果定义 $X$ 上的关系 $\sim$ 为
	\[
	  x \sim y \triangleq x \le y \land y \le x,
	\]
	则 $\sim$ 是 $X$ 上的等价关系 (equivalence relation)。
    \end{enumerate}
  \end{exampleblock}

  \pause
  \vspace{0.50cm}
  \centerline{\blue{reflexive + symmetric + transitive}}
\end{frame}
%%%%%%%%%%%%%%%

%%%%%%%%%%%%%%%
\begin{frame}{}
  \begin{exampleblock}{$4.$ 关系与序 (Order)}
    一个自反(\blue{reflexive})且传递(\blue{transitive})的二元关系 $R \subseteq X \times X$ 
    称为 $X$ 上的拟序 (\red{preorder/\textcolor{gray}{quasiorder}})。\\[8pt]

    令 $\le \;\subseteq X \times X$ 为拟序, 请证明:
    \begin{enumerate}[(1)]
      \setcounter{enumi}{1}
      \item 如果定义商集 (quotient set) $X/\sim$ 上的关系 $\preceq$ 为
	\[
	  [x]_{\sim} \preceq [y]_{\sim} \triangleq x \le y,
	\]
	则 $\preceq$ 是偏序关系 (partial order)。
    \end{enumerate}
  \end{exampleblock}

  \pause
  \vspace{0.30cm}
  \centerline{\blue{reflexive + antisymmetric + transitive}}

  \pause
  \vspace{0.60cm}
  \centerline{\red{\Large Well-definedness!!!}}
\end{frame}
%%%%%%%%%%%%%%%

%%%%%%%%%%%%%%%
\begin{frame}{}
  \centerline{\large Well-definedness: \blue{Independence of Representative}}

  \pause
  \vspace{0.50cm}
  \[
    [x_1] = [x_2] \land [y_1] = [y_2]
  \]
  \[
    \teal{\implies}
  \]
  \[
    [x_1] \preceq [y_1] \iff [x_2] \preceq [y_2]
  \]
\end{frame}
%%%%%%%%%%%%%%%

%%%%%%%%%%%%%%%
\begin{frame}{}
  \[
    [a]_n + [b]_n = [a + b]_n
  \]

  \[
    [a]_n \times [b]_n = [ab]_n
  \]

  % \pause
  % \[
  %   \red{Q: [a]_n^{[b]_n} = [a^b]_n}
  % \]
\end{frame}
%%%%%%%%%%%%%%%

%%%%%%%%%%%%%%%
\begin{frame}{}
  \fig{width = 0.40\textwidth}{figs/example}

  \vspace{0.20cm}
  \centerline{\red{\large $\teal{\le:}$ \only<1>{$\teal{???}$}\only<2->{Reachability} relationship in a directed graph}}

  \uncover<1->{
    \vspace{0.20cm}
    \fig{width = 0.40\textwidth}{figs/scc}
  }

  \uncover<3->{\centerline{\teal{$\sim, [x]_{\sim}:$} \uncover<4->{Strongly Connected Component (SCC)}}}

  \uncover<5->{
    \vspace{0.40cm}
    \centerline{\red{\large $\teal{\preceq:}$ \uncover<6->{Reachability relationship in a condensed directed acyclic graph}}}
  }
\end{frame}
%%%%%%%%%%%%%%%
