\section{Chinese Remainder Theorem}

%%%%%%%%%%%%%%%%%%%%
\begin{frame}{Pairwise relatively prime (Problem 31.2-9)}
  \begin{exampleblock}{}
	\begin{gather*}
	  n_1, n_2, n_3, n_4 \text{ are pairwise relatively prime} \\
	  \iff \\
	  \text{gcd}(n_1n_2, n_3n_4) = \text{gcd}(n_1n_3, n_2n_4) = 1
	\end{gather*}
  \end{exampleblock}
\end{frame}
%%%%%%%%%%%%%%%%%%%%
\begin{frame}{}
  \begin{exampleblock}{}
	\begin{gather*}
	  n_1, n_2, \dots, n_k \text{ are pairwise relatively prime} \\
	  \iff \\
	  \text{a set of } \lceil \lg k \rceil \text{ pairs of numbers derived from the } n_i \text{ are relatively prime}.
	\end{gather*}
  \end{exampleblock}

  \pause
  \[
	\textcolor{red}{\text{gcd}(\fbox{$1_L$}, \fbox{$1_R$}) 
	= \text{gcd}(\fbox{$2_L$}, \fbox{$2_R$}) 
	= \cdots 
	= \text{gcd}(\fbox{$\lceil \lg k \rceil_L$}, \fbox{$\lceil \lg k \rceil_R$}) = 1}
  \]

  \pause
  \begin{gather*}
    k = 4: \quad \text{gcd}(n_1n_2, n_3n_4) = \text{gcd}(n_1n_3, n_2n_4) = 1 \\
	k = 3: \quad \text{gcd}(n_1, n_2n_3) = \text{gcd}(n_2, n_3) = 1 \\
	k = 2: \quad \text{gcd}(n_1, n_2) = 1
  \end{gather*}
\end{frame}
%%%%%%%%%%%%%%%%%%%%
\begin{frame}{}
  % \begin{exampleblock}{}
  %   \begin{gather*}
  %     n_1, n_2, \dots, n_k \text{ are pairwise relatively prime} \\
  %     \iff \\
  %     \text{a set of } \lceil \lg k \rceil \text{ pairs of numbers derived from the } n_i \text{ are relatively prime}.
  %   \end{gather*}
  % \end{exampleblock}

  \[
	k = 7: \quad n_1, n_2, n_3, n_4, n_5, n_6, n_7
  \]

  \pause
  \[
    \text{gcd}(n_1n_2n_3, n_4n_5n_6n_7) = 1
  \]

  \centerline{TODO: figure here.}

  \begin{equation*}
	\begin{cases}
	  T(1) = 0 \\
	  T(2) = 1 \\
	  T(k) = 2T(\frac{k}{2}) + 1
	\end{cases}\pause \implies T(k) = k - 1 = \Theta(k)
  \end{equation*}
  % \[
  %   k = 8: \quad n_1, n_2, n_3, n_4, n_5, n_6, n_7, n_8
  % \]

  % \pause
  % \[
  %   \text{gcd}(n_1n_2n_3n_4, n_5n_6n_7n_8) = 1
  % \]

  % \pause
  % \begin{gather*}
  %   k' = 4: \quad \text{gcd}(n_1n_2, n_3n_4) = \text{gcd}(n_1n_3, n_2n_4) = 1 \\
  %   k' = 4: \quad \text{gcd}(n_4n_5, n_6n_7) = \text{gcd}(n_4n_6, n_5n_7) = 1
  % \end{gather*}
\end{frame}
%%%%%%%%%%%%%%%%%%%%
\begin{frame}{} % {Being a little smarter with divide-and-conquer}
  \[
	k = 7: \quad n_1, n_2, n_3, n_4, n_5, n_6, n_7
  \]

  \pause
  \[
    \text{gcd}(n_1n_2n_3, n_4n_5n_6n_7) = 1
  \]

  \centerline{TODO: figure here.}

  \begin{equation*}
	\begin{cases}
	  T(1) = 0 \\
	  T(2) = 1 \\
	  T(k) = T(\frac{k}{2}) + 1
	\end{cases}\pause \implies T(k) = \lceil \lg k \rceil
  \end{equation*}
\end{frame}
%%%%%%%%%%%%%%%%%%%%
\begin{frame}{Looking into the divide steps}
\end{frame}
%%%%%%%%%%%%%%%%%%%%
\begin{frame}{Not exactly the same}
\end{frame}
%%%%%%%%%%%%%%%%%%%%
\begin{frame}{Can we do even better?}
\end{frame}
%%%%%%%%%%%%%%%%%%%%
\begin{frame}{Biclique covering}
\end{frame}
%%%%%%%%%%%%%%%%%%%%
\begin{frame}{Biclique covering}
\end{frame}
%%%%%%%%%%%%%%%%%%%%
%%%%%%%%%%%%%%%%%%%%
%%%%%%%%%%%%%%%%%%%%
%%%%%%%%%%%%%%%%%%%%
\begin{frame}{Chinese Remainder Theorem (CRT)}
  Where do $m_i$, $c_i$, and $a$ come from?
\end{frame}
%%%%%%%%%%%%%%%%%%%%
\begin{frame}{History of CRT}
\end{frame}
%%%%%%%%%%%%%%%%%%%%
\begin{frame}{Proof of CRT (1)}
\end{frame}
%%%%%%%%%%%%%%%%%%%%
\begin{frame}{Proof of CRT (2)}
\end{frame}
%%%%%%%%%%%%%%%%%%%%
\begin{frame}{Proof of CRT (3)}
\end{frame}
%%%%%%%%%%%%%%%%%%%%
\begin{frame}{CRT}
  Meaning of Figure 31.3

  $\equiv 1$ and $\equiv 0$ elsewhere
\end{frame}
%%%%%%%%%%%%%%%%%%%%
\begin{frame}{$\phi$ function}
\end{frame}
%%%%%%%%%%%%%%%%%%%%
\begin{frame}{CRT with non-pairwise coprime moduli}
\end{frame}
%%%%%%%%%%%%%%%%%%%%
\begin{frame}{}
\end{frame}
%%%%%%%%%%%%%%%%%%%%
\begin{frame}{Application?}
\end{frame}
%%%%%%%%%%%%%%%%%%%%
\begin{frame}{Application?}
\end{frame}
%%%%%%%%%%%%%%%%%%%%
