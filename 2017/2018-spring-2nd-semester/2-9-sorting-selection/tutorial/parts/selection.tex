% file: parts/selection.tex

%%%%%%%%%%%%%%%
\begin{frame}{}
  \begin{exampleblock}{Finding the $2$nd Smallest Element (Problem $9.1-1$)}
    Show that the $2$nd smallest of $n$ elements can be found with $n + \lceil \log n \rceil - 2$
    comparisons in the worst case.
  \end{exampleblock}

  \pause
  \[
    (n-1) + (n-1-1) = 2n-3
  \]
\end{frame}
%%%%%%%%%%%%%%%

%%%%%%%%%%%%%%%
\begin{frame}{}
  \begin{adjustbox}{max totalsize={.7\textwidth}{.6\textheight},center}
    % file: tournament.tex

\begin{tikzpicture}[level distance = 50pt, sibling distance = 18pt,
  edge from parent/.style= { % added code
      draw, edge from parent path = {(\tikzparentnode) -- (\tikzchildnode)}}]
  \tikzset{every tree node/.style = 
    {align = center, circle, draw, fill = blue!20, font = \Large}}

    \Tree [.{$x_6$}	
	    [.\textcolor<2->{red}{{$x_4$}}
	       [.{$x_2$}
		 [.{$x_1$} ]
		 [.{$x_2$} ]
	       ] 
	       [.{$x_4$} 
		 [.{$x_3$} ]
		 [.{$x_4$} ]
	       ]
            ]
	    [.{$x_6$} 
	      [.{$x_6$} 
		[.\textcolor<2->{red}{{$x_5$}} ] 
		[.{$x_6$} ] 
	      ] 
	      [.{\textcolor<3->{red}{$x_7$}} 
		[.{$x_7$} ] 
		[.{$x_8$} ] 
	      ]
	   ] 
        ]
\end{tikzpicture}

  \end{adjustbox}

  \uncover<3->{
    \[
      \#\; \text{\red{Potential} $2$nd smallest elements} \;\red{\le}\; \lceil \log n \rceil
    \]
  }

  \uncover<4->{
    \vspace{-0.50cm}
    \[
      n + \lceil \log n \rceil - 2 = (n-1) + (\red{\lceil \log n \rceil} - 1)
    \]
  }

  \uncover<5->{
    \centerline{\red{\large $Q:$ Can we do even better?}}
  }
\end{frame}
%%%%%%%%%%%%%%%

%%%%%%%%%%%%%%%
\begin{frame}{}
  \centerline{\teal{\large Adversary Argument}}

  \pause
  \[
    \Omega = n + \lceil \log n \rceil - 2 = (n-1) + (\red{\lceil \log n \rceil} - 1)
  \]

  \pause
  \fignocaption{width = 0.25\textwidth}{figs/knuth-vol3}
  \centerline{\teal{TAOCP, Vol. 3 {\small (Page 209, Section 5.3.3)}}}
\end{frame}
%%%%%%%%%%%%%%%

%%%%%%%%%%%%%%%
% \begin{frame}{}
%   \begin{enumerate}[(I)]
%     \setlength{\itemsep}{10pt}
%     \item Any algorithm that finds \textsl{secondLargest} must also find \textsl{max}.
%       \begin{itemize}
% 	\item The \textsl{secondLargest} must have lost in one comparison.
%       \end{itemize}
%     \item $n-1$ comparisons with distinct losers are needed to find \textsl{max}.
%       \begin{itemize}
% 	\item Every other key than \textsl{max} must have lost directly or indirectly to \textsl{max}.
%       \end{itemize}
%     \item There is an \red{adversary strategy} that can force that \\
%       $\lceil \log n \rceil$ distinct losers of these $n-1$ ones have lost directly to \textsl{max}.
%     \item All but one of the $\lceil \log n \rceil$ distinct (direct) losers must lose again.
%       \begin{itemize}
% 	\item One of the $\lceil \log n \rceil$ losers is the \textsl{secondLargest}.
%       \end{itemize}
%   \end{enumerate}
% \end{frame}
%%%%%%%%%%%%%%%

%%%%%%%%%%%%%%%
\begin{frame}{}
  \begin{theorem}
    Any comparison-based algorithm to find \textsl{secondLargest} in $n$ keys
    must do at least $n + \lceil \log n \rceil - 2$ comparisons in the worst case.
  \end{theorem}
\end{frame}
%%%%%%%%%%%%%%%

%%%%%%%%%%%%%%%
\begin{frame}{}
  \[
    n + \lceil \log n \rceil - 2 = (n-1) + (\lceil \log n \rceil - 1)
  \]

  \fignocaption{width = 0.55\textwidth}{figs/second-largest-roadmap}
\end{frame}
%%%%%%%%%%%%%%%

%%%%%%%%%%%%%%%
\begin{frame}{}
  \[
    p \ge \lceil \log n \rceil
  \]

  \begin{columns}
    \column{0.30\textwidth}
      \centerline{\red{Adversary $\mathcal{A}$:}}
      \[
	x >=< y
      \]
    \column{0.40\textwidth}
      \fignocaption{width = 0.80\textwidth}{figs/adversary-alg}
    \column{0.30\textwidth}
      \centerline{\teal{Algorithm $\mathbb{A}$:}}
      \[
	\textsc{Compare}(x,y)
      \]
  \end{columns}
  
  \pause
  \vspace{0.50cm}
  \[
    \red{\exists \mathcal{A}}\; \teal{\forall \mathbb{A}}\; \Big(\ge \lceil \log n \rceil \text{ distinct keys lost directly to \textsl{max} in } \mathbb{A}\Big) 
  \]
\end{frame}
%%%%%%%%%%%%%%%

%%%%%%%%%%%%%%%
\begin{frame}{}
  \fignocaption{width = 0.40\textwidth}{figs/adversary}
\end{frame}
%%%%%%%%%%%%%%%

%%%%%%%%%%%%%%%
\begin{frame}{}
  \[
    \forall x: w(x) = 1
  \]

  % \vspace{0.60cm}
  \[
    \red{\textsc{Compare}(x,y)}
  \]
  \renewcommand{\arraystretch}{1.5}
  \begin{center}
    \begin{tabular}{l l c}
      \textsc{Case} & \textsc{Reply} & \textsc{(Local) Action} \\ \hline
      $w(x) > w(y)$ 		& $x > y$	& $w'(x) \gets w(x) + w(y); w'(y) \gets 0$ \\ \hline
      $w(x) = w(y) > 0$ 	& $x > y$	& $w'(x) \gets w(x) + w(y); w'(y) \gets 0$ \\ \hline
      $w(x) = w(y) = 0$ 	& \textsf{No Conflict} & \textsc{Nop} \\ \hline
      $w(y) > w(x)$ 		& $y > x$	& $w'(y) \gets w(x) + w(y); w'(x) \gets 0$ \\ \hline
    \end{tabular}
  \end{center}

  \pause
  \vspace{0.30cm}
  \begin{CJK*}{UTF8}{gbsn}
    \centerline{\red{\Large 赢者通吃; 败者永不得翻身}}
  \end{CJK*}
\end{frame}
%%%%%%%%%%%%%%%

%%%%%%%%%%%%%%%
\begin{frame}{}
  \[
    \sum_x w(x) = n
  \]

  \pause
  \[
    w(x) = 0 \iff x \text{ has lost}
  \]

  \pause
  \[
    w(x) > 0 \iff x \text{ has not yet lost}
  \]

  \pause
  \begin{theorem}
    For any algorithm $\mathbb{A}$ that finds \textsl{secondLargest}, when it stops,
    \[
      \exists! x: w(x) > 0.
    \]
  \end{theorem}

  \pause
  \[
    \red{\boxed{t = \textsl{max}, \quad w(t) = n}}
  \]
\end{frame}
%%%%%%%%%%%%%%%

%%%%%%%%%%%%%%%
\begin{frame}{}
  \begin{theorem}
    $t$ has directly won against $\ge \lceil \log n \rceil$ distinct keys.
  \end{theorem}

  \begin{proof}
    \[
      K: \# \text{ of comparisons } x \text{ wins against \purple{previously undefeated} keys}
    \]
    \pause
    \vspace{-0.30cm}
    \[
      \teal{\boxed{n = w_K}}
    \]

    \pause
    \vspace{-0.60cm}
    \begin{align*}
      w_k: w(t) \text{ after the } k\text{th comparison won by } t \\
      \text{ against a previously undefeated key}
    \end{align*}
    \pause
    \vspace{-0.30cm}
    \[
      \teal{\boxed{w_k \le 2 w_{k-1}}}
    \]

    \pause
    \[
      \red{\boxed{n = w_K \le 2^K w_0 = 2^K \pause \implies K \ge \log n \pause \implies K \ge \lceil \log n \rceil}}
    \]
  \end{proof}
\end{frame}
%%%%%%%%%%%%%%%

%%%%%%%%%%%%%%%
\begin{frame}{}
  \begin{exampleblock}{\textsl{max}\&\textsl{min} (Problem 9.1-2)}
    Prove the lower bound of $\lceil 3n/2 \rceil - 2$ comparisons in the worst case \\
    to find both the \textsl{max} and the \textsl{min} of $n$ numbers.
  \end{exampleblock}

  \pause
  \[
    \mathbb{A}: \lceil 3n/2 \rceil - 2
  \]

  \pause
  \begin{columns}
    \column{0.35\textwidth}
      \centerline{\red{Adversary $\mathcal{A}$:}}
      \[
	n/2 + (n-2)
      \]
    \column{0.30\textwidth}
      \fignocaption{width = 0.80\textwidth}{figs/adversary-alg}
    \column{0.35\textwidth}
      \centerline{\teal{Algorithm $\mathbb{A}$:}}
      \[
	2n - 2 
      \]
      \centerline{units of \textsc{Win/Lose} info.}
  \end{columns}

  \pause
  \vspace{0.30cm}
  \[
    2n - 2 = 2 \times (n/2) + 1 \times (n-2)
  \]
\end{frame}
%%%%%%%%%%%%%%%

%%%%%%%%%%%%%%%
\begin{frame}{}
  \begin{exampleblock}{``$01$'' Pattern}
    To check whether an array $A[1 \cdots n]$ of $n$ bits contains the ``$01$'' pattern.

    Do we have to check every bit?
  \end{exampleblock}

  \pause
  \vspace{0.50cm}
  \begin{center}
    No, if $n$ is odd; \\[8pt]
    Yes, if $n$ is even.
  \end{center}
\end{frame}
%%%%%%%%%%%%%%%

%%%%%%%%%%%%%%%
\begin{frame}{}
  \fignocaption{width = 0.40\textwidth}{figs/01-pattern-odd}

  \centerline{\red{First checking $A[2, 4,\ldots, n-1]$}}
  
  \pause
  \fignocaption{width = 0.40\textwidth}{figs/01-pattern-odd-cases}
\end{frame}
%%%%%%%%%%%%%%%

%%%%%%%%%%%%%%%
\begin{frame}{}
  \begin{columns}
    \column{0.30\textwidth}
      \centerline{\red{Adversary $\mathcal{A}$:}}
      \[
	0/1
      \]
    \column{0.40\textwidth}
      \fignocaption{width = 0.80\textwidth}{figs/adversary-alg}
    \column{0.30\textwidth}
      \centerline{\teal{Algorithm $\mathbb{A}$:}}
      \[
	\textsc{Check}(i)
      \]
  \end{columns}

  \pause
  \vspace{0.60cm}
  \fignocaption{width = 0.95\textwidth}{figs/01-pattern-even-cases}
\end{frame}
%%%%%%%%%%%%%%%

%%%%%%%%%%%%%%%
\begin{frame}{}
  \begin{exampleblock}{$k$ Numbers Closest to the Median (Problem $9.3-7$)}
    \[
      S: n \text{ distinct numbers} \qquad k \le n
    \]
  \end{exampleblock}

  \pause
  \[
    \frac{n}{2} - \frac{k}{2} \sim \frac{n}{2} + \frac{k}{2} \pause \qquad \scalebox{2}{\red{$\times$}}
  \]

  \pause
  \[
    S = \set{800, 6, 900, \red{50}, 7}, \quad k = 2 \implies \set{6,7}
  \]

  \pause
  \[
    S - 50 = \set{750, -44, 850, \red{0}, -43}
  \]

  \pause
  \vspace{0.50cm}
  \centerline{\teal{median + subtraction} + \red{$(k+1)$-th smallest + partition} + \teal{add back}}
\end{frame}
%%%%%%%%%%%%%%%
