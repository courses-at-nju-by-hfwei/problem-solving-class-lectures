%%%%%%%%%%%%%%%
\begin{frame}{}
  \centerline{\LARGE Flowcharts}

  \vspace{1.00cm}
  \centerline{\textcolor{violet}{How to Argue with Your Girlfriend?}}
  \fignocaption{width = 0.30\textwidth}{figs/argue-flowchart}
\end{frame}
%%%%%%%%%%%%%%%

%%%%%%%%%%%%%%%
\begin{frame}{}
  \begin{columns}
    \column{0.45\textwidth}
      \fignocaption{width = 0.60\textwidth}{figs/von}
    \column{0.45\textwidth}
      \fignocaption{width = 0.60\textwidth}{figs/flowchart-eci}
  \end{columns}
  
  \vspace{0.50cm}
  \pause
  \begin{quote}
    We feel certain that a moderate amount of experience with this stage of \blue{coding}
    suffices to remove from it all difficulties,
    and to make it a perfectly \blue{routine operation}.

    \hfill --- John von Neumann and Herman Goldstine, late 1940s
  \end{quote}
\end{frame}
%%%%%%%%%%%%%%%

%%%%%%%%%%%%%%%
\begin{frame}{}
  \begin{columns}
    \column{0.40\textwidth}
      \fignocaption{width = 0.80\textwidth}{figs/bengkui}
    \pause
    \column{0.40\textwidth}
      \fignocaption{width = 0.90\textwidth}{figs/flowchart-wrong}
  \end{columns}

  % \vspace{0.40cm}
  % \pause
  % \begin{quote}
  %   Any resemblance between our flow charts and the present program is purely coincidental.

  %   \hfill --- Donald Knuth, 1963
  % \end{quote}
\end{frame}
%%%%%%%%%%%%%%%

%%%%%%%%%%%%%%%
\begin{frame}{}
  \fignocaption{width = 0.40\textwidth}{figs/flowchart-eci}

  \centerline{Flowcharts Considered Harmful.}
\end{frame}
%%%%%%%%%%%%%%%

% \fignocaption{width = 0.30\textwidth}{figs/taocp-flowchart}
%%%%%%%%%%%%%%%
\begin{frame}{}
  \fignocaption{width = 0.60\textwidth}{figs/my-opinion}

  \vspace{0.50cm}
  \pause
  \centerline{\Large Draw it when it does help} 

  \vspace{0.20cm}
  \pause
  \centerline{\Large OR you have to.}
\end{frame}
%%%%%%%%%%%%%%%