%%%%%%%%%%%%%%%
\begin{frame}{}
  \centerline{\Large A Little Axiomatic Set Theory \blue{(ZFC)}}

  \vspace{0.80cm}
  \begin{columns}
    \column{0.45\textwidth}
      \fignocaption{width = 0.50\textwidth}{figs/Zermelo}{\centerline{Ernst Zermelo (1871--1953)}}
    \column{0.45\textwidth}
      \fignocaption{width = 0.48\textwidth}{figs/Fraenkel}{\centerline{Abraham Fraenkel (1891--1965)}}
  \end{columns}
\end{frame}
%%%%%%%%%%%%%%%

%%%%%%%%%%%%%%%
% \begin{frame}{}
%   First-order language
% \end{frame}
%%%%%%%%%%%%%%%

%%%%%%%%%%%%%%%
% \begin{frame}{}
%   公理系统
% \end{frame}
%%%%%%%%%%%%%%%

%%%%%%%%%%%%%%%
\begin{frame}{}
  \begin{definition}[Axiom Schema of Separation]
    \[
      \forall \psi(x): \Big(\forall X \exists Y: Y = \set{\red{x \in X} \mid \psi(x)}\Big).
    \]
  \end{definition}
\end{frame}
%%%%%%%%%%%%%%%

%%%%%%%%%%%%%%%
\begin{frame}{}
  \[
    \psi(x) = x \notin x
  \]

  \pause
  \[
    R = \set{x \mid x \notin x}
  \]

  \pause
  \begin{theorem}
    \[
      \set{x \mid x \notin x} \red{\text{\bf \;is not a set.}}
    \]
  \end{theorem}
\end{frame}
%%%%%%%%%%%%%%%

%%%%%%%%%%%%%%%
\begin{frame}{}
  \[
    \forall X: R_X = \set{x \in X \mid x \notin x}
  \]

  \pause
  \[
    \red{Q: R_X \in R_X ?}
  \]

  \pause
  \begin{theorem}
    There is no universe set. \uncover<8->{\it \textcolor{teal}{(It is too ``big'' to be a set!)}}

    \pause
    \[
      \forall C \exists x: x \notin C.
    \]
  \end{theorem}

  \pause
  \begin{proof}
    \centerline{By contradiction.}

    \pause
    \[
      \set{x \in C \mid x \notin x} \pause = \set{x \mid x \notin x}
    \]
  \end{proof}
\end{frame}
%%%%%%%%%%%%%%%

%%%%%%%%%%%%%%%
\begin{frame}{}
  \begin{definition}[``$\cap$'']
    \vspace{-0.50cm}
    \begin{align*}
      A \cap B &= \set{x \in A \mid x \in B} \\
	       &= \set{x \mid x \in A \land x \in B}
    \end{align*}
  \end{definition}

  \vspace{0.80cm}
  \begin{definition}[``$\setminus$'']
    \vspace{-0.50cm}
    \begin{align*}
      A \setminus B &= \set{x \in A \mid x \notin B} \\
	       &= \set{x \mid x \in A \land x \notin B}
    \end{align*}
  \end{definition}

  \pause
  \vspace{0.30cm}
  \begin{quote}
    We can never look for objects ``not in $B$'' \red{unless we know where to start looking}.
    So we use $A$ to tell us where to look for elements not in $B$.
    \hfill -- UD (Chapter 6)
  \end{quote}
\end{frame}
%%%%%%%%%%%%%%%

%%%%%%%%%%%%%%%
\begin{frame}{}
  \begin{definition}[Axiom of Extensionality]
    \[
      \forall A \forall B \forall \blue{x} (\blue{x} \in A \iff \blue{x} \in B) \iff A = B.
    \]
  \end{definition}

  \pause
  \[
    \set{a, a} = \set{a}
  \]

  % \[
  %   A \subseteq B \triangleq \forall x: x \in A \implies x \in B
  % \]

  % \[
  %   A \subset B \triangleq A \subseteq B \land A \neq B
  % \]
\end{frame}
%%%%%%%%%%%%%%%

%%%%%%%%%%%%%%%
% \begin{frame}{}
%   \begin{definition}[Axiom of Paring]
%     \[
%       \forall a \forall b \exists c (a \in c \land b \in c)
%     \]
% 
%     \[
%       a, b \implies \set{a, b}
%     \]
%   \end{definition}
% 
%   \[
%     \set{a, a} \pause = \set{a}
%   \]
% 
%   \pause
%   \begin{definition}[Axiom of Union]
%   \end{definition}
% 
%   \[
%     X \cup Y \triangleq \bigcup \set{X, Y}
%   \]
% \end{frame}
%%%%%%%%%%%%%%%
