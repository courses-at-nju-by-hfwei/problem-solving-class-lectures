% b-tree-bst.tex

%%%%%%%%%%%%%%%
\begin{frame}{}
  \begin{exampleblock}{Minimum (TC 18.2-3)}
    Explain how to find the \red{minimum} key stored in a B-tree. 
  \end{exampleblock}

  \fig{width = 0.75\textwidth}{figs/b-tree-example}

  \pause
  \vspace{0.30cm}
  \begin{center}
    \blue{the leftmost key in the leftmost node}
  \end{center}
\end{frame}
%%%%%%%%%%%%%%%

%%%%%%%%%%%%%%%
\begin{frame}{}
  \begin{exampleblock}{Predecessor (TC 18.2-3)}
    Explain how to find the \red{predecessor of a given key} stored in a B-tree.
  \end{exampleblock}

  \fig{width = 0.75\textwidth}{figs/b-tree-example}
\end{frame}
%%%%%%%%%%%%%%%

%%%%%%%%%%%%%%%
\begin{frame}{}
\end{frame}
%%%%%%%%%%%%%%%

%%%%%%%%%%%%%%%
\begin{frame}{}
\end{frame}
%%%%%%%%%%%%%%%