% proof-continued.tex

%%%%%%%%%%%%%%%
\begin{frame}{}
  \begin{exampleblock}{UD 4.20: 一阶谓词逻辑的推理规则与语义}
    Decide whether (3) is true \red{\large if} (1) and (2) are both true.
  \end{exampleblock}

  \pause
  \vspace{0.30cm}
  \red{$Q$:} 该如何理解这道题? 依据什么 ``decide'' 真假?
  
  \pause
  \begin{description}
    \item[逻辑知识] 
      \[
	(1) \land (2) \to (3)
      \]
    \pause
    \item[数学知识] ``True'' 是语义概念 
      \begin{itemize}
	\item 与选定的``结构''中的知识有关
      \end{itemize}
  \end{description}

  \pause
  \fig{width = 0.30\textwidth}{figs/mass.png}
\end{frame}
%%%%%%%%%%%%%%%

%%%%%%%%%%%%%%%
\begin{frame}{}
  \begin{exampleblock}{UD 4.20 (a): 一阶谓词逻辑的推理规则与语义}
    Decide whether (3) is true \red{\large if} (1) and (2) are both true.

    \begin{enumerate}[(a)]
      \item 
	\begin{enumerate}[(1)]
	  \item Everyone who loves Bill loves Sam.
	  \item I don't love Sam.
	  \item I don't love Bill.
	\end{enumerate}
    \end{enumerate}
  \end{exampleblock}

  \vspace{0.50cm}
  \pause
  \begin{center}
    {\red{$Q$:} 如何在一阶谓词逻辑框架中``\red{\large 算出来}''?}
  \end{center}
\end{frame}
%%%%%%%%%%%%%%%

%%%%%%%%%%%%%%%
\begin{frame}{}
  \begin{exampleblock}{UD 4.20 (b): 一阶谓词逻辑的推理规则与语义}
    Decide whether (3) is true \red{\large if} (1) and (2) are both true.

    \begin{enumerate}[(a)]
      \setcounter{enumi}{1}
      \item 
	\begin{enumerate}[(1)]
	  \item If Susie goes to the ball in the red dress, I will stay home.
	  \item Susie went to the ball in the green dress.
	  \item I did not stay home.
	\end{enumerate}
    \end{enumerate}
  \end{exampleblock}

  \begin{center}
    {\red{$Q$}: 这是真的吗?}
  \end{center}

  \pause
  \vspace{0.30cm}
  到底是真是假?
  \vspace{0.30cm}
  \begin{columns}
    \column{0.45\textwidth}
      \begin{itemize}
	\item (3) is true:

	Whether I stay at home or not, (3) is always true.
      \end{itemize}
    \column{0.45\textwidth}
      \begin{itemize}
        \item (3) is false:

	No matter what I do, the implication is always true.
      \end{itemize}
  \end{columns}

  \pause
  \vspace{0.50cm}
  \centerline{实际上,仅根据 (1)、(2), 我们无法判断 (3) 的真假。}
\end{frame}
%%%%%%%%%%%%%%%

%%%%%%%%%%%%%%%
\begin{frame}{}
  \begin{exampleblock}{UD 4.20 (c): 一阶谓词逻辑的推理规则与语义}
    Decide whether (3) is true \red{\large if} (1) and (2) are both true.

    \begin{enumerate}[(a)]
      \setcounter{enumi}{2}
      \item 
	\begin{enumerate}[(1)]
	  \item If $l$ is a positive real number, then there exists a real number $m$ such that $m > l$.
	  \item Every real number $m$ is less than $t$.
	  \item The real number $t$ is not positive.
	\end{enumerate}
    \end{enumerate}
  \end{exampleblock}

  \vspace{0.30cm}
  \pause
  \red{$Q:$ 如何符号化 (1)、(2)、(3)?}

  \begin{columns}
    \column{0.60\textwidth}
      \begin{enumerate}[(1)]
	\pause
	\item $\forall l$ 还是仅是 $l$?
	\pause
	\item $t$ 究竟是不是实数?
	\pause
	\item $R(t) \land \lnot P(t)$ 还是 $R(t) \to \lnot P(t)$?
      \end{enumerate}
    \column{0.40\textwidth}
      \pause
      \begin{enumerate}[(1)]
	\item $\forall l$
	\item $R(t)$
	\item $R(t) \to \lnot P(t)$
      \end{enumerate}
  \end{columns}

  \pause
  \vspace{0.30cm}
  \begin{center}
    {现在,让我们来``\red{\large 算}''一下吧。}
  \end{center}
\end{frame}
%%%%%%%%%%%%%%%

%%%%%%%%%%%%%%%
\begin{frame}{}
  \begin{exampleblock}{UD 4.20 (d): 一阶谓词逻辑的推理规则与语义}
    Decide whether (3) is true \red{\large if} (1) and (2) are both true.

    \begin{enumerate}[(a)]
      \setcounter{enumi}{3}
      \item 
	\begin{enumerate}[(1)]
	  \item Every little breeze seems to whisper Louise or my name is Igor.
	  \item My name is Stewart.
	  \item Every little breeze seems to whisper Louise.
	\end{enumerate}
    \end{enumerate}
  \end{exampleblock}

  \pause
  \begin{center}
    \red{$Q:$ 命题逻辑公式还是一阶谓词逻辑公式?}

    \pause
    \[
      \set{p \lor q, \lnot q} \vdash p
    \]
  \end{center}
\end{frame}
%%%%%%%%%%%%%%%

%%%%%%%%%%%%%%%
\begin{frame}{}
  \begin{exampleblock}{UD 4.20 (e): 一阶谓词逻辑的推理规则与语义}
    Decide whether (3) is true \red{\large if} (1) and (2) are both true.

    \begin{enumerate}[(a)]
      \setcounter{enumi}{4}
      \item 
	\begin{enumerate}[(1)]
	  \item There is a house on every street such that if that house is blue,
	    the one next to it is black.
	  \item There is no blue house on my street.
	  \item There is no black house on my street.
	\end{enumerate}
    \end{enumerate}
  \end{exampleblock}

  \vspace{0.30cm}
  \only<1-2>{
    \begin{center}
      {(1) 在说什么? 翻译成汉语是什么意思?}
    \end{center}
  }

  \only<2>{\fig{width = 0.40\textwidth}{figs/what.jpg}}

  \uncover<3->{
    \[
      \forall s\in S\, \exists h \in H 
      \Big(\text{On}(h,s) \land \big(\text{Blue}(h) \to \text{Black}\big(\text{next-to}(h)\big) \big)\Big)
    \]
  }

  \uncover<4->{
    \[
      (1) \land (2) \to (3)?
    \]
  }
  \uncover<5->{
    \[
      (1) \land (3) \to (2)?
    \]
  }
\end{frame}
%%%%%%%%%%%%%%%

%%%%%%%%%%%%%%%
\begin{frame}{}
  \begin{exampleblock}{UD 4.20 (f): 一阶谓词逻辑的推理规则与语义}
    Decide whether (3) is true \red{\large if} (1) and (2) are both true.

    \begin{enumerate}[(a)]
      \setcounter{enumi}{5}
      \item Let $x$ and $y$ be real numbers.
	\begin{enumerate}[(1)]
	  \item If $x > 5$, then $y < 1/5$.
	  \item We know $y = 1$.
	  \item So $x \le 5$.
	\end{enumerate}
    \end{enumerate}
  \end{exampleblock}

  \pause
  \vspace{0.30cm}
  \begin{center}
    {\red{先``算一算''}} \\[15pt] \pause
    {\red{$Q$}: 在推理过程中, 我们用到了哪些数学知识 (\blue{非逻辑}知识)?}
  \end{center}
\end{frame}
%%%%%%%%%%%%%%%

%%%%%%%%%%%%%%%
\begin{frame}{}
  \begin{exampleblock}{UD 4.20 (g): 一阶谓词逻辑的推理规则与语义}
    Decide whether (3) is true \red{\large if} (1) and (2) are both true.

    \begin{enumerate}[(a)]
      \setcounter{enumi}{6}
      \item Let $M$ and $n$ be real numbers.
	\begin{enumerate}[(1)]
	  \item If $n > M$, then $n^2 > M^2$.
	  \item We know $n < M$.
	  \item So $n^2 \le M^2$.
	\end{enumerate}
    \end{enumerate}
  \end{exampleblock}

  \begin{columns}[t]
    \pause
    \column{0.30\textwidth}
      \begin{itemize}
	\item (3) is false: 
	  \[
	    n = -2, \; M = -1
	  \]
      \end{itemize}
    \pause
    \column{0.30\textwidth}
      \begin{itemize}
        \item (3) is true:
	\begin{align*}
	  (1)\; & n > 0 \\
	  (2)\; & 0 < n < M
	\end{align*}
      \end{itemize}
    \pause
    \column{0.30\textwidth}
      \begin{itemize}
        \item 无法判断
	\[
	  (1) \land (2) \to (3)
	\]
      \end{itemize}
  \end{columns}
  \pause
  \begin{columns}
    \column{0.45\textwidth}
      \fig{width = 0.45\textwidth}{figs/yun.jpg}
    \column{0.50\textwidth}
      \fig{width = 0.30\textwidth}{figs/qrcode-math-stackexchange.png}
      \vspace{-0.80cm}
      \centerline{\scriptsize \url{https://math.stackexchange.com/q/2471687/51434}}
  \end{columns}
\end{frame}
%%%%%%%%%%%%%%%

%%%%%%%%%%%%%%%
\begin{frame}{}
  \begin{exampleblock}{UD 4.20 (h): 一阶谓词逻辑的推理规则与语义}
    Decide whether (3) is true \red{\large if} (1) and (2) are both true.

    \begin{enumerate}[(a)]
      \setcounter{enumi}{7}
      \item Let $x,y$, and $z$ be real numbers.
	\begin{enumerate}[(1)]
	  \item If $y > x$ and $y > 0$, then $y > z$.
	  \item We know that $y \le z$.
	  \item Then $y \le x$ or $y \le 0$.
	\end{enumerate}
    \end{enumerate}
  \end{exampleblock}

  \pause
  \vspace{0.30cm}
  \begin{center}
    {\red{先``算一算''}} \\[15pt] \pause
    {\red{$Q$}: 在推理过程中, 我们用到了哪些数学知识 (\blue{非逻辑}知识)?}
  \end{center}
\end{frame}
%%%%%%%%%%%%%%%
