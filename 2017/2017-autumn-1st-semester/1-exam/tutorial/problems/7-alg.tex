%%%%%%%%%%%%%%%
\begin{frame}{}
  \begin{exampleblock}{$7.$ 算法设计与正确性证明}
    在``掼蛋''游戏中, $5$ 张大小连续的扑克牌构成一个顺子 

    (如 $A\; 2\; 3\; 4\; 5$ 和 $10\; J\; Q\; K\; A$ 都是顺子; 不考虑花色)。

    任给$13$张从小到大的牌 (允许不同花色重复, 如 $A\; 3\; 3\; 4\; 5\; 7\; 8\; 9\; 10\; J\; J\; Q\; K$):
    \begin{enumerate}[(1)]
      \item 请设计算法,找到所有的顺子。 
      \item 请使用``循环不变式''\;(loop invariants)证明你设计的算法的正确性。
      \item 数学归纳法的正确性也是需要证明的。请证明第一数学归纳法的正确性。 
	(不允许使用第二数学归纳法证明。)
    \end{enumerate}
  \end{exampleblock}
\end{frame}
%%%%%%%%%%%%%%%

%%%%%%%%%%%%%%%
\begin{frame}{}
  \begin{exampleblock}{$7.$ 算法设计与正确性证明}
    \begin{enumerate}[(1)]
      \item 任给$13$张从小到大的牌, 请设计算法,找到所有的顺子。 
    \end{enumerate}
  \end{exampleblock}

  \begin{alertblock}{Preprocessing:}
    \pause
    % A\; A\; 2\; 2\; 3\; 3\; 4\; 4\; 5\; 5\; 6\; 6\; 7
    \[
      A\; 3\; 3\; 4\; 5\; 7\; 8\; 9\; 10\; J\; J\; Q\; K
    \]
    \pause
    % A\; 2\; 3\; 4\; 5\; 6\; 7
    \[
      A\; 3\; 4\; 5\; 7\; 8\; 9\; 10\; J\; Q\; K
    \]
    % A\; 2\; 3\; 4\; 5\; 6\; 7\; \red{A}
    \pause
    \[
      A\; 3\; 4\; 5\; 7\; 8\; 9\; 10\; J\; Q\; K\; \red{A}
    \]
  \end{alertblock}
\end{frame}
%%%%%%%%%%%%%%%

%%%%%%%%%%%%%%%
\begin{frame}[fragile]{}
  \begin{proof}[``Sliding Window'' Algorithm:]
    \begin{lstlisting}[style = CStyle]
    i $\gets$ 1  // starting index of the sliding window
    cnt $\gets$ 1
    while ($i \le n - 4$)  // $n$: # of cards
      while (cnt $\neq 5$)
        if (P[i + cnt] == P[i + cnt - 1] + 1)
          |\teal{cnt++}|
        else  // fail: skip cnt
          |\teal{i $\gets$ i + cnt}|
          |\teal{cnt $\gets$ 1}|
          break
      if (cnt == 5)  // succeed: slid by one
        print P[i] $\cdots$ P[i + cnt - 1]
        |\teal{i++}|
        |\teal{cnt $\gets$ 4}|  // only need to check the new card
    \end{lstlisting}
  \end{proof}
\end{frame}
%%%%%%%%%%%%%%%

%%%%%%%%%%%%%%%
\begin{frame}{}
  \begin{exampleblock}{$7.$ 算法设计与正确性证明}
    \begin{enumerate}[(1)]
      \setcounter{enumi}{1}
      \item 请使用``循环不变式''\;(loop invariants)证明你设计的算法的正确性。
    \end{enumerate}
  \end{exampleblock}

  \pause
  \fignocaption{width = 0.45\textwidth}{figs/stay-tuned-soon}

  \pause
  \vspace{-0.80cm}
  \[
    \text{\blue{while} } (i \le n - 4): \pause \text{ the \teal{straight} starting at } P[k]\;(k < i) \text{ has been found}
  \]
  \pause
  \centerline{\red{OR} does not exist}
\end{frame}
%%%%%%%%%%%%%%%

%%%%%%%%%%%%%%%
\begin{frame}{}
  \begin{exampleblock}{$7.$ 算法设计与正确性证明}
    \begin{enumerate}[(1)]
      \setcounter{enumi}{2}
      \item 数学归纳法的正确性也是需要证明的。请证明第一数学归纳法的正确性。 
	(不允许使用第二数学归纳法证明。)
    \end{enumerate}
  \end{exampleblock}

  \pause
  \vspace{0.60cm}
  \centerline{Well-ordering Principle $\implies$ Principle of Mathematical Induction}
\end{frame}
%%%%%%%%%%%%%%%

%%%%%%%%%%%%%%%
\begin{frame}{}
  \begin{definition}[Well-ordering Principle]
    Every non-empty subset of $\mathbb{N}$ contains a least element.
  \end{definition}

  \pause
  \vspace{0.20cm}
  \begin{definition}[Principle of Mathematical Induction]
  \[
    \Big[ P(1) \land \forall n \in \mathbb{N}^{+} \big(P(n) \to P(n+1) \big) \Big]
      \to \forall n \in \mathbb{N}^{+} P(n).
  \]
  \end{definition}

  \pause
  \begin{proof}[WOP $\implies$ PMI]
    \centerline{\large By contradiction.}

    \pause
    \[
      Q = \set{k \in \N \mid \lnot P(k)} \neq \emptyset
    \]
    \pause
    \vspace{-0.50cm}
    \[
      k = \min\; Q
    \]
    \pause
    \vspace{-0.50cm}
    \[
      k \neq 1 \implies k > 1 
    \]
    \pause
    \vspace{-0.50cm}
    \[
      k-1 \notin Q \implies P(k-1) \implies P(k)
    \]
  \end{proof}
\end{frame}
%%%%%%%%%%%%%%%
