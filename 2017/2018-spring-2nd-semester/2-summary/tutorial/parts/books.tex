%%%%%%%%%%%%%%%
\begin{frame}{为什么列书单?}
  \begin{enumerate}
    \setlength{\itemsep}{8pt}
    \item 我感受到了大家强烈的求知欲
    \item 希望大家在本科阶段多读一些书
    \item 有书读多快乐啊
  \end{enumerate}

  \vspace{0.80cm}
  \begin{quote}
    \centering
    ``列书单就是把自己也没读过的书\\[3pt]
    推荐给同样不会去读的其他人,\\[3pt]
    以达到大家都已读过这些书的假象。''
  \end{quote}
\end{frame}
%%%%%%%%%%%%%%%

%%%%%%%%%%%%%%%
\begin{frame}{读什么书?}
  \begin{enumerate}
    \setlength{\itemsep}{6pt}
    \centering
    \item 经典性
    \item 专业性
    \item 趣味性
  \end{enumerate}
\end{frame}
%%%%%%%%%%%%%%%

%%%%%%%%%%%%%%%
\begin{frame}{如何读?}
  \begin{center}
    不要指望一遍就能读懂\\[10pt]

    交叉阅读, 多角度思考\\[10pt]

    三五人同读一本书\\
  \end{center}
\end{frame}
%%%%%%%%%%%%%%%

%%%%%%%%%%%%%%%
\begin{frame}{}
  \centerline{\Large 书籍}
\end{frame}
%%%%%%%%%%%%%%%

%%%%%%%%%%%%%%%
\begin{frame}{}
  \begin{columns}
    \column{0.45\textwidth}
      \fignocaption{width = 0.85\textwidth}{figs/papadimitrous}
    \column{0.45\textwidth}
      \fignocaption{width = 0.65\textwidth}{figs/algorithms-papa}
  \end{columns}

  \vspace{0.80cm}
  \centerline{言简意赅、观点独到}
\end{frame}
%%%%%%%%%%%%%%%

%%%%%%%%%%%%%%%
\begin{frame}{}
  \begin{columns}
    \column{0.45\textwidth}
      \fignocaption{width = 0.85\textwidth}{figs/jon-kleinberg}
    \column{0.45\textwidth}
      \fignocaption{width = 0.80\textwidth}{figs/algorithm-design-china}
  \end{columns}

  \vspace{0.10cm}
  \begin{center}
    侧重算法设计思路、详略得当 \\[6pt]
    有别的书籍里找不到的算法题目
  \end{center}
\end{frame}
%%%%%%%%%%%%%%%

%%%%%%%%%%%%%%%
\begin{frame}{}
  \begin{columns}
    \column{0.45\textwidth}
      \fignocaption{width = 0.80\textwidth}{figs/steven-skiena}
    \column{0.45\textwidth}
      \fignocaption{width = 0.70\textwidth}{figs/algorithm-design-manual-china}
  \end{columns}

  \vspace{0.50cm}
  \begin{center}
    我最喜欢读每一章的 War Story,尤其是第九章的那两个。
  \end{center}
\end{frame}
%%%%%%%%%%%%%%%

%%%%%%%%%%%%%%%
\begin{frame}{}
  \begin{columns}
    \column{0.45\textwidth}
      \fignocaption{width = 0.85\textwidth}{figs/sedgewick-teaching}
    \column{0.45\textwidth}
      \fignocaption{width = 0.70\textwidth}{figs/AoA}
  \end{columns}

  \vspace{0.50cm}
  \begin{center}
    关注算法分析技术; 有一定难度
  \end{center}
\end{frame}
%%%%%%%%%%%%%%%

%%%%%%%%%%%%%%%
\begin{frame}{}
  \begin{columns}
    \column{0.45\textwidth}
      \fignocaption{width = 0.85\textwidth}{figs/sedgewick-youtube}
    \column{0.45\textwidth}
      \fignocaption{width = 0.70\textwidth}{figs/algorithms-4th}
  \end{columns}

  \vspace{0.50cm}
  \begin{center}
    包含大量图例, 提供可运行代码 (Java) 及实验数据 \\[8pt]
    在这里,算法是一门``实验科学''。
  \end{center}
\end{frame}
%%%%%%%%%%%%%%%

%%%%%%%%%%%%%%%
\begin{frame}{}
  \centerline{\Huge 公开课 {\small (内附链接)}}
\end{frame}
%%%%%%%%%%%%%%%

%%%%%%%%%%%%%%%
\begin{frame}{}
  \begin{columns}
    \column{0.45\textwidth}
      \fignocaption{width = 0.90\textwidth}{figs/leiserson}
    \column{0.45\textwidth}
      \fignocaption{width = 0.90\textwidth}{figs/erik-demaine-teaching}
  \end{columns}

  \begin{center}
    ``Algorithms''\\[4pt]
    \href{http://open.163.com/special/opencourse/algorithms.html}{MIT 6.046/18.410, Fall 2005}\\[3pt]
    \href{https://ocw.mit.edu/courses/electrical-engineering-and-computer-science/6-006-introduction-to-algorithms-fall-2011/lecture-videos/}{MIT 6.006, Fall 2011}\\[10pt]

    讲得太好, 再次安利。
  \end{center}
\end{frame}
%%%%%%%%%%%%%%%

%%%%%%%%%%%%%%%
\begin{frame}{数学}
  \begin{columns}
    \column{0.45\textwidth}
    \column{0.60\textwidth}
  \end{columns}
\end{frame}
%%%%%%%%%%%%%%%