% file: sections/hamiltonian-graph.tex

%%%%%%%%%%%%%%%%%%%%
\begin{frame}{}
  \begin{theorem}[Necessary Condition; Theorem $6.5$]
    If $G$ is a Hamiltonian graph, then for each nonempty set $S \subset V(G)$,
    \[
      k(G - S) \le \big\lvert S \big\rvert.
    \]
  \end{theorem}

  \pause
  \[
    \implies \text{ Hamiltonian graphs are } 2\text{-connected.} \quad \red{(\text{Why?})}
  \]

  \pause
  \vspace{0.30cm}
  \begin{center}
    Hamiltonian graphs has ``good'' connectedness. \\[10pt] \pause
    \teal{Graphs with ``bad'' connectedness are \red{not} Hamiltonian.}
  \end{center}
\end{frame}
%%%%%%%%%%%%%%%%%%%%

%%%%%%%%%%%%%%%%%%%%
\begin{frame}{}
  \begin{theorem}[Ore's Theorem, $1960$; Theorem $6.6$]
    Let $G$ be a graph of order $n \ge 3$. If
    \[
      \text{deg}(u) + \text{deg}(v) \ge n
    \]
    for each pair $u,v$ of nonadjacent vertices of $G$,
    then $G$ is Hamiltonian.
  \end{theorem}

  \pause
  \begin{proof}
    \begin{center}
      By \purple{Contradiction} and \red{Extremality}. \\[10pt] \pause
      By \purple{Contradiction}: \\ [5pt]
      \teal{$\exists G$ satisfying Ore's Condition, but $G$ is \red{not} Hamiltonian.} \\[8pt]
      By \red{Extremality}: \\[5pt]
      \teal{Consider a \red{critical} $G$: $G$ is not Hamiltonian but $G + uv$ is Hamiltonian.} \\[8pt]
      \blue{Contradiction: This critical $G$ \red{is} actually Hamiltonian.}
    \end{center}

  \end{proof}
\end{frame}
%%%%%%%%%%%%%%%%%%%%

%%%%%%%%%%%%%%%%%%%%
\begin{frame}{}
  \begin{theorem}[Dirac's Theorem, $1952$; Corollary $6.7$]
    Let $G$ be a graph of order $n \ge 3$. If
    \[
      \forall v \in V(G): \text{deg}(v) \ge n/2,
    \]
    then $G$ is Hamiltonian.
  \end{theorem}
\end{frame}
%%%%%%%%%%%%%%%%%%%%

%%%%%%%%%%%%%%%%%%%%
\begin{frame}{}
  \begin{theorem}[Ore's Theorem, $1960$; Theorem $6.8$]
    Let $u$ and $v$ be nonadjacent vertices in a graph $G$ of order $n$ such that
    \[
      \text{deg}(u) + \text{deg}(v) \ge n.
    \]
    Then $G + uv$ is Hamiltonian $\iff$ $G$ is Hamiltonian.
  \end{theorem}

  \pause
  \vspace{0.50cm}
  \begin{definition}[Closure $C(G)$]
    The closure $C(G)$ of a graph $G$ is the graph obtained from $G$
    by iteratively adding edges joining pairs of nonadjacent vertices
    $u$ and $v$ such that $\text{deg}(u) + \text{deg}(v) \ge n$,
    until no such pair remains.
  \end{definition}
\end{frame}
%%%%%%%%%%%%%%%%%%%%

%%%%%%%%%%%%%%%%%%%%
\begin{frame}{}
  \begin{theorem}[Bondy-Chav\'atal Theorem, $1976$; Theorem $6.9$]
    $G$ is Hamiltonian \textcolor<2->{red}{$\iff$} $C(G)$ is Hamiltonian.
  \end{theorem}
  
  \vspace{0.50cm}
  \uncover<3->{
    \begin{corollary}[Corollary $6.10$]
      If $G$ is a graph of order $n \ge 3$ such that $C(G) = K_{n}$, \\
      then $G$ is Hamiltonian.
    \end{corollary}
  }

  \vspace{0.50cm}
  \uncover<4->{
    \begin{theorem}[Lajos P\'osa]
      Let $G$ be a graph of order $n \ge 3$.
      \red{If} for each integer $j$ with $1 \le j \le \frac{n}{2}$,
      the number of vertices of $G$ with degree at most $j$ is less than $j$, \\
      then $G$ is Hamiltonian.
    \end{theorem}
  }
\end{frame}
%%%%%%%%%%%%%%%%%%%%

%%%%%%%%%%%%%%%%%%%%
\begin{frame}{}
  \begin{theorem}[Well-definedness of $C(G)$]
    \begin{center}
      $C(G)$ is well-defined. \\[6pt] \pause
      $C(G)$ does not depend on the order in which we choose to add edges. 
      \pause
      \[
	\red{\boxed{\Big(G + \langle e_1, \cdots, e_r \rangle = G_1\Big) \land 
	\Big(G + \langle f_1, \cdots, f_s \rangle = G_2\Big) \implies G_1 = G_2}}
      \]
    \end{center}
  \end{theorem}

  \pause
  \[
    G_1 \subseteq G_2 \land G_2 \subseteq G_1
  \]
  \pause
  \[
    \forall e_i \in G_1: e_i \in G_2
  \]
  \pause
  \begin{center}
    \red{By induction on the order $e_i$ is added to $G_1$.}
  \end{center}
\end{frame}
%%%%%%%%%%%%%%%%%%%%

%%%%%%%%%%%%%%%%%%%%
\begin{frame}{}
  \begin{exampleblock}{Hamiltonian Graphs and $2$-Connectedness (Problem $6.20$)}
    Let $G$ be a graph of order $n \ge 3$ having the property that
    for each $v \in V(G)$, there is a Hamiltonian path with initial vertex $v$. \\
    Show that \red{$G$ is $2$-connected} but not necessarily Hamiltonian.
  \end{exampleblock}

  \pause
  \begin{center}
    \teal{$2$-connected: Connected + No cut-vertex}
  \end{center}

  \pause
  \begin{center}
    Suppose, \red{by contradiction}, $v$ is a cut-vertex of $G$. \pause
    \fig{width = 0.25\textwidth}{figs/cut-vertex-structure}
    \pause
    \red{Contradiction:} No Hamiltonian path with initial vertex $v$.
  \end{center}
\end{frame}
%%%%%%%%%%%%%%%%%%%%

%%%%%%%%%%%%%%%%%%%%
\begin{frame}{}
  \begin{exampleblock}{Hamiltonian Graphs and $2$-Connectedness (Problem $6.20$)}
    Let $G$ be a graph of order $n \ge 3$ having the property that
    for each $v \in V(G)$, there is a Hamiltonian path with initial vertex $v$. \\
    Show that $G$ is $2$-connected but \red{not necessarily Hamiltonian}.
  \end{exampleblock}

  \pause
  \fig{width = 0.30\textwidth}{figs/petersen-graph}
\end{frame}
%%%%%%%%%%%%%%%%%%%%
