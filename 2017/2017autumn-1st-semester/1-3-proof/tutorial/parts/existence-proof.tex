%%%%%%%%%%%%%%%
\begin{frame}{}
  \begin{exampleblock}{存在性证明 (Existence Proof)}
    \begin{enumerate}
      \item 构造性证明 (Constructive proof)
      \item 反证法 (By contradiction)
      \item 概率法 (Probabilistic Method)
    \end{enumerate}
  \end{exampleblock}
\end{frame}
%%%%%%%%%%%%%%%

%%%%%%%%%%%%%%%
\begin{frame}{}
  \begin{theorem}[Dov Jarden (1953)]
    \[
      \exists a, b \in \mathbb{R} \setminus \mathbb{Q}: a^{b} \in \mathbb{Q}.
    \]
  \end{theorem}

  \pause
  \[
    \sqrt{2} \in \mathbb{R} \setminus \mathbb{Q} \qquad \text{Theorem 5.2}
  \]

  \pause
  \begin{proof}
    \[
      \sqrt{2}^{\sqrt{2}}
    \]

    \pause
    \[
      (\sqrt{2}^{\sqrt{2}})^{\sqrt{2}}
    \]
  \end{proof}

  \vspace{0.80cm}
  \pause
  \centerline{\red{Q:} 这是构造性证明吗? 这是反证法吗?}
\end{frame}
%%%%%%%%%%%%%%%
