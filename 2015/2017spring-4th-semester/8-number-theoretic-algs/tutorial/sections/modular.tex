\section{Modular Arithmetic}

\begin{frame}{``Mod''}
  \begin{exampleblock}{(TC 31.4.2)}
	\[
	  ad \equiv bd \pmod{n}, \textcolor{red}{a \bot n} \implies a \equiv b \pmod{n} 
	\]
  \end{exampleblock}

  \[
	3 \cdot 2 \equiv 5 \cdot 2 \pmod{4} \quad \textcolor{red}{3 \not\equiv 5 \pmod{4}} \quad \pause \textcolor{blue}{3 \equiv 5 \pmod{2}}
  \]
\end{frame}
%%%%%%%%%%%%%%%%%%%%
\begin{frame}{Changing the modulus}
  \[
	ad \equiv bd \pmod{nd} \iff a \equiv b \pmod{n} \quad (d \neq 0)
  \]

  \[
	ad \equiv bd \pmod{n} \iff a \equiv b \pmod{\frac{n}{\gcd(d,n)}}
  \]
\end{frame}
%%%%%%%%%%%%%%%%%%%%
\begin{frame}{Changing the modulus}
  \[
	a \equiv b \pmod{100} \implies a \equiv b \pmod{20} \implies a \equiv b \pmod{5}
  \]

  \[
	a \equiv b \pmod{nd} \implies a \equiv b \pmod{n}, d \in \mathbb{Z}
  \]

  \centerline{\rule{0.80\paperwidth}{0.4pt}}

  \[
	a \equiv b \pmod{n_1}, a \equiv b \pmod{n_2} \iff a \equiv b \pmod{\lcm(n_1, n_2)}
  \]

  % \[
  %   a \equiv b \pmod{12}, a \equiv b \pmod{18} \iff a \equiv b \pmod{36}
  % \]

  \[
	a \equiv b \pmod{n_1}, a \equiv b \pmod{n_2} \iff a \equiv b \pmod{n_1n_2}, \text{ if } n_1 \bot n_2
  \]

  \[
	a \equiv b \pmod{n} \iff a \equiv b \pmod{p^{n_p}}, \quad n = \prod_{p} p^{n_p}
  \]
\end{frame}
%%%%%%%%%%%%%%%%%%%%
\begin{frame}{Changing the modulus}
\end{frame}
%%%%%%%%%%%%%%%%%%%%
%%%%%%%%%%%%%%%%%%%%

