%%%%%%%%%%%%%%%
\begin{frame}{}
  \begin{center}
    {\LARGE Partition}
  \end{center}

  \fig{width = 0.60\textwidth}{figs/partition}
\end{frame}
%%%%%%%%%%%%%%%

%%%%%%%%%%%%%%%
\begin{frame}{}
  \begin{definition}[Partition]
    A family of sets \red{$\set{A_{\alpha}: \alpha \in I}$} is a \blue{\it partition} of \purple{$X$} if

    \begin{enumerate}[(i)]
      \item 
	\[
	  \forall \alpha \in I: A_{\alpha} \neq \emptyset
	\]
	\[
	  \textcolor{teal}{\forall \alpha \in I \; \exists x \in X: x \in A_{\alpha}}
	\]
      \item 
	\[
	  \bigcup_{\alpha \in I} A_{\alpha} = X
	\]
	\[
	  \textcolor{teal}{\forall x \in X \; \exists \alpha \in I: x \in A_{\alpha}}
	\]
      \item 
	\[
	  \forall \alpha, \beta \in I: A_{\alpha} \cap A_{\beta} = \emptyset \lor A_{\alpha} = A_{\beta}
	\]
	\[
	  \textcolor{teal}{\forall \alpha, \beta \in I: A_{\alpha} \cap A_{\beta} \neq \emptyset \implies A_{\alpha} = A_{\beta}}
	\]
    \end{enumerate}
  \end{definition}
\end{frame}
%%%%%%%%%%%%%%%

%%%%%%%%%%%%%%%
\begin{frame}{}
  \begin{exampleblock}{UD Problem 11.8: Partitions of the Set of Polynomials}
    \[
      p(x) = a_n x^n + a_{n-1} x^{n-1} + \cdots + a_1 x^1 + a_0 \quad (a_j \in \mathbb{R}, n \in \mathbb{N})
    \]
    \[
      \text{deg}(p = 0) = -\infty
    \]

    \begin{enumerate}[(a)]
      \item
	\[
	  A_m = \set{p: \text{deg}(p) = m} \quad m \in \mathbb{N}
	\]

	\pause
	\[
	  \red{(p = 0) \notin \bigcup\limits_{m \in \mathbb{N}} A_m}
	\]
      \setcounter{enumi}{2}
      \item 
	\pause
	\[
	  A_q = \set{p: \exists r (p = qr)} \quad q \in P
	\]
	\pause
	\vspace{-0.30cm}
	\[
	  \textcolor{teal}{q \in A_q}
	\]
	\pause
	\vspace{-0.50cm}
	\[
	  \textcolor{cyan}{p \in A_p}
	\]
	\pause
	\vspace{-0.50cm}
	\[
	  \textcolor{purple}{p \neq q \land r = pq \implies (r \in A_p \cap A_q) \land (A_p \neq A_q)}
	\]
    \end{enumerate}
  \end{exampleblock}
\end{frame}
%%%%%%%%%%%%%%%

%%%%%%%%%%%%%%%
\begin{frame}{}
  \begin{exampleblock}{UD Problem 11.8: Partitions of the Set of Polynomials}
    \[
      p(x) = a_n x^n + a_{n-1} x^{n-1} + \cdots + a_1 x^1 + a_0 \quad (a_j \in \mathbb{R}, n \in \mathbb{N})
    \]
    \[
      \text{deg}(p = 0) = -\infty
    \]

    \begin{enumerate}[(a)]
      \setcounter{enumi}{1}
      \item 
	\[
	  A_c = \set{p: p(0) = c} \quad c \in \mathbb{R}
	\]
      \setcounter{enumi}{3}
      \item 
	\pause
	\[
	  A_c = \set{p: p(c) = 0} \quad c \in \mathbb{R}
	\]
	\pause
	\vspace{-0.20cm}
	\[
	  \red{(p(x) = x^2 + 1) \notin \bigcup\limits_{c \in \mathbb{R}} A_c}
	\]
    \end{enumerate}
  \end{exampleblock}
\end{frame}
%%%%%%%%%%%%%%%

%%%%%%%%%%%%%%%
\begin{frame}{}
  \begin{exampleblock}{UD Problem 11.4: Partitions of $\mathbb{R}^{3}$}
    Is $\set{A_r \mid r \in \mathbb{R}}$ a partition of $\mathbb{R}^{3}$?

    \[
      A_{r} = \set{(x,y,z) \in \mathbb{R}^{3}: x + y + z = r}
    \]
  \end{exampleblock}

  \pause
  \fig{width = 0.32\textwidth}{figs/xyzr}
\end{frame}
%%%%%%%%%%%%%%%

%%%%%%%%%%%%%%%
\begin{frame}{}
  \begin{exampleblock}{UD Problem 11.4: Partitions of $\mathbb{R}^{3}$}
    Is $\set{A_r \mid r \in \mathbb{R}}$ a partition of $\mathbb{R}^{3}$?

    \[
      A_{r} = \set{(x,y,z) \in \mathbb{R}^{3}: x^2 + y^2 + z^2 = r^2}
    \]
  \end{exampleblock}

  \pause
  \fig{width = 0.35\textwidth}{figs/x2y2z2r2}
\end{frame}
%%%%%%%%%%%%%%%

%%%%%%%%%%%%%%%
\begin{frame}{}
  \begin{exampleblock}{UD Problem 11.9: Subset and Partition}
    \begin{center}
      Let $\set{A_{\alpha}: \alpha \in I}$ be a partition of $X \neq \emptyset$.
    \end{center}

    \begin{enumerate}[(a)]
      \item 
	\[
	  B \subseteq X, \quad \forall \alpha \in I: A_{\alpha} \cap B \neq \emptyset
	\]

	To prove that
	\[
	  \set{A_{\alpha} \cap B: \alpha \in I} \text{ \red{\it is} a partition of } B.
	\]
    \end{enumerate}
  \end{exampleblock}

  \fig{width = 0.50\textwidth}{figs/subset-partition}
\end{frame}
%%%%%%%%%%%%%%%

%%%%%%%%%%%%%%%
\begin{frame}{}
  \[
    \blue{\bigcup_{i \in I} (A \cap X_i) = A \cap \bigcup_{i \in I} X_i}
  \]

  \begin{align*}
    \onslide<2->{&\textcolor{white}{\iff}\; x \in \bigcup_{i \in I} (A \cap X_i) \\}
    \onslide<3->{&\iff \exists i \in I: x \in A \cap X_i \\}
    \onslide<4->{&\iff \exists i \in I: x \in A \land x \in X_i \\}
    \onslide<5->{&\red{\;\iff x \in A \land \exists i \in I: x \in X_i} \\}
    \onslide<6->{&\iff x \in A \land x \in \bigcup_{i \in I} X_i \\}
    \onslide<7->{&\iff x \in A \cap \bigcup_{i \in I} X_i }
  \end{align*}
\end{frame}
%%%%%%%%%%%%%%%

%%%%%%%%%%%%%%%
\begin{frame}{}
  \[
    \bigcup_{i \in I} (A \cap X_i) = A \cap \bigcup_{i \in I} X_i
  \]

  \vspace{0.30cm}
  \[
    \bigcap_{i \in I} (A \cup X_i) = A \cup \bigcap_{i \in I} X_i
  \]

  \pause
  \vspace{0.60cm}
  \[
    A \setminus \bigcup_{i \in I} X_i = \bigcap_{i \in I} (A \setminus X_i)
  \]

  \vspace{0.30cm}
  \[
    A \setminus \bigcap_{i \in I} X_i = \bigcup_{i \in I} (A \setminus X_i)
  \]
\end{frame}
%%%%%%%%%%%%%%%

%%%%%%%%%%%%%%%
\begin{frame}
  \[
    \bigcup_{i \in I} X_i \cap \bigcup_{i \in I} Y_i \supseteq \bigcup_{i \in I} (X_i \cap Y_i)
  \]

  \[
    \bigcap_{i \in I} X_i \cup \bigcap_{i \in I} Y_i \subseteq \bigcap_{i \in I} (X_i \cup Y_i)
  \]

  \begin{align*}
    \onslide<2->{&\textcolor{white}{\implies}\; x \in \bigcap_{i \in I} X_i \cup \bigcap_{i \in I} Y_i \\}
    \onslide<3->{&\iff x \in \bigcap_{i \in I} X_i \lor x \in \bigcap_{i \in I} Y_i  \\}
    \onslide<4->{&\iff \forall i \in I: x \in X_i \lor \forall i \in I: x \in Y_i \\}
    \onslide<6->{&\red{\;\implies \forall i \in I: (x \in X_i \lor x \in Y_i)} \\}
    \onslide<7->{&\iff x \in \bigcap_{i \in I} (X_i \cup Y_i)}
  \end{align*}

  \uncover<8->{
    \[
      X_1 = \set{1}, X_2 = \set{2}, \qquad Y_1 = \set{2}, Y_2 = \set{1}
    \]
  }
\end{frame}
%%%%%%%%%%%%%%%
