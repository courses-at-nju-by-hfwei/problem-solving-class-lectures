%%%%%%%%%%%%%%%
\begin{frame}{为什么列书单?}
  \begin{itemize}
    \item 我感受到了大家强烈的求知欲。
    \item 希望大家在本科阶段多读一些书。
    \item 读书多快乐啊。
  \end{itemize}
\end{frame}
%%%%%%%%%%%%%%%

%%%%%%%%%%%%%%%
\begin{frame}{读什么书?}
  \begin{enumerate}
    \centering
    \item 经典性
    \item 专业性
    \item 趣味性
  \end{enumerate}
\end{frame}
%%%%%%%%%%%%%%%

%%%%%%%%%%%%%%%
\begin{frame}{如何读?}
  放慢速度

  交叉阅读

  组队
\end{frame}
%%%%%%%%%%%%%%%

%%%%%%%%%%%%%%%
\begin{frame}{逻辑}
  \begin{columns}
    \column{0.45\textwidth}
      \fignocaption{width = 0.55\textwidth}{figs/fudan-logic}
      \begin{center}
	第一章: 语法\emph{vs.}语义 \\
	第二章: 正确性\emph{vs.}完全性
      \end{center}
    \column{0.45\textwidth}
      \fignocaption{width = 0.60\textwidth}{figs/engines-of-logic}
      \begin{center}
	读故事, 了解历史发展 \\
	不求甚解
      \end{center}
  \end{columns}
\end{frame}
%%%%%%%%%%%%%%%

%%%%%%%%%%%%%%%
\begin{frame}{集合论}
  \begin{columns}
    \column{0.45\textwidth}
      \fignocaption{width = 0.55\textwidth}{figs/fudan-set-theory-book}
      \begin{center}
	第一章: \\
	第二章: 
      \end{center}
    \column{0.45\textwidth}
      \fignocaption{width = 0.60\textwidth}{figs/book-paradox-crisis}
      \begin{center}
	知其然, 更知其所以然。
      \end{center}
  \end{columns}
\end{frame}
%%%%%%%%%%%%%%%

%%%%%%%%%%%%%%%
\begin{frame}{程序设计}
  \begin{columns}
    \column{0.45\textwidth}
    the c 
    \column{0.45\textwidth}
    the c++
  \end{columns}
\end{frame}
%%%%%%%%%%%%%%%

%%%%%%%%%%%%%%%
\begin{frame}{数据结构与算法}
  \begin{columns}
    \column{0.45\textwidth}
      \fignocaption{width = 0.60\textwidth}{figs/erik-demaine}
      \centerline{\href{Click: http://open.163.com/special/opencourse/algorithms.html}{MIT OCW: Introduction to Algorithms}}
    \column{0.45\textwidth}
  \end{columns}
\end{frame}
%%%%%%%%%%%%%%%

%%%%%%%%%%%%%%%
\begin{frame}{数学 (一)}
  \begin{columns}
    \column{0.50\textwidth}
      \fignocaption{width = 0.60\textwidth}{figs/book-concrete-mathematics}
      \begin{center}
	Knuth 眼中的计算机科学中的数学
      \end{center}
    \column{0.45\textwidth}
      % \fignocaption{width = 0.60\textwidth}{figs/}
      \begin{center}
      \end{center}
  \end{columns}
\end{frame}
%%%%%%%%%%%%%%%

%%%%%%%%%%%%%%%
\begin{frame}{数学 (二)}
  \begin{columns}
    \column{0.50\textwidth}
    torrence tao
    \column{0.50\textwidth}
    数学分析八讲
  \end{columns}
\end{frame}
%%%%%%%%%%%%%%%