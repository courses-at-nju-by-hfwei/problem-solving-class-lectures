\documentclass{beamer}
\usepackage{lmodern}

\usepackage{xeCJK}

\usepackage[safe]{silence}
\WarningFilter*{latexfont}{Font}

\usetheme{CambridgeUS} % try Madrid, Pittsburgh
\usecolortheme{beaver}
\usefonttheme[onlymath]{serif} % try "professionalfonts"

\setbeamertemplate{itemize items}[default]
\setbeamertemplate{enumerate items}[default]

\usepackage{amsmath, amsfonts, latexsym, mathtools}
% \usepackage{kbordermatrix}
% \usepackage{ntheorem}

\usepackage{centernot}

\DeclareMathOperator*{\argmin}{arg\,min}
\DeclareMathOperator*{\argmax}{arg\,max}

% colors
\newcommand{\red}[1]{\textcolor{red}{#1}}
\newcommand{\green}[1]{\textcolor{green}{#1}}
\newcommand{\blue}[1]{\textcolor{blue}{#1}}
\newcommand{\purple}[1]{\textcolor{purple}{#1}}

% color box
\newcommand{\rbox}[1]{\red{\boxed{#1}}}
\newcommand{\gbox}[1]{\green{\boxed{#1}}}
\newcommand{\bbox}[1]{\blue{\boxed{#1}}}
\newcommand{\pbox}[1]{\purple{\boxed{#1}}}

\usepackage{tikz}
% see http://tex.stackexchange.com/a/7045/23098
\newcommand*\circled[1]{\tikz[baseline=(char.base)]{
            \node[shape=circle,draw,inner sep=2pt] (char) {#1};}}
\usepackage{tikz-qtree}
\usetikzlibrary{backgrounds, fit}

\usepackage{pifont}
\usepackage{wasysym}

\usepackage[normalem]{ulem}
\newcommand{\middlewave}[1]{\raisebox{0.5em}{\uwave{\hspace{#1}}}}

\usepackage{graphicx, subcaption}

\usepackage{algorithm}
\usepackage[noend]{algpseudocode}

\newcommand{\pno}[1]{\textcolor{blue}{\scriptsize [Problem: #1]}}
\newcommand{\set}[1]{\{#1\}}
\DeclareMathOperator{\lcm}{lcm}

\newcommand{\cmark}{\green{\ding{51}}}
\newcommand{\xmark}{\red{\ding{55}}}
%%%%%%%%%%%%%%%%%%%%%%%%%%%%%%%%%%%%%%%%%%%%%%%%%%%%%%%%%%%%%%
% for fig without caption: #1: width/size; #2: fig file
\newcommand{\fignocaption}[2]{
  \begin{figure}[htp]
    \centering
      \includegraphics[#1]{#2}
  \end{figure}
}

% for fig with caption: #1: width/size; #2: fig file; #3: fig caption
\newcommand{\fig}[3]{
  \begin{figure}[htp]
    \centering
      \includegraphics[#1]{#2}
      \caption[labelInTOC]{#3}
  \end{figure}
}

\newcommand{\p}{\text{P}}
\newcommand{\np}{\text{NP}}
\newcommand{\nph}{\text{NP}-hard}
\newcommand{\npc}{\text{NP}-complete}

\newcommand{\titletext}{P, NP, NP-hard, and NP-complete}

\newcommand{\tx}{%
  \begin{frame}[noframenumbering]
	\fignocaption{width = 0.50\textwidth}{figs/thankyou.jpg}
  \end{frame}
}
%%%%%%%%%%%%%%%%%%%%
\title[\titletext]{\titletext}
% \subtitle{}

\author[Hengfeng Wei]{Hengfeng Wei}
% \titlegraphic{\includegraphics[height = 1.5cm]{figs/qrcode-coding-theory-20170420.png}}
\institute{hfwei@nju.edu.cn}
\date{May 01 $\sim$ May 04, 2017}

\AtBeginSection[]{
  \begin{frame}[noframenumbering, plain]
    \frametitle{\titletext}
    \tableofcontents[currentsection, sectionstyle=show/shaded, subsectionstyle=show/show/hide]
  \end{frame}
}
%%%%%%%%%%
\begin{document}
\maketitle

\section{Complexity Class}

%%%%%%%%%%%%%%%%%%%%
\begin{frame}{\p}
  \[
	\p = \bigcup_{c \ge 1} \dtime(n^c)
  \]

  \begin{exampleblock}{TC 34.1--5}
	\begin{align*}
	  f(n) &= O(n^c) \\
	  t(n) &= O(n^d)
	\end{align*}
  \end{exampleblock}

  \[
	T(n) = kf(n) + t(n)
  \]

  \[
	T_{k}(n) = \sum_{i=0}^{k}f^{(i)}(n) + t(n) 
  \]

  \[
	k = \Theta(n^{O(1)})
  \]
\end{frame}
%%%%%%%%%%%%%%%%%%%%
\begin{frame}{\np}
  \begin{definition}[\np]
	$(L \subseteq \set{0,1}^{\ast}) \in \np$ if there exists a polynomial-time \emph{verifier} $V(x,y)$ such that $\forall x \in \set{0,1}^{\ast}$,
	\[
	  x \in L \iff \exists y \in \set{0,1}^{\ast}, V(x,y) = 1.
	\]
  \end{definition}
\end{frame}
%%%%%%%%%%%%%%%%%%%%
\begin{frame}{\np}
  \begin{exampleblock}{TC 34.2--4}
	\np{} is closed under $\cup, \cap, \cdot, \ast$.
  \end{exampleblock}

  \begin{alertblock}{Remark}
	\[
	  O \emph{\text{ vs. }} \Omega
	\]
  \end{alertblock}

  \begin{alertblock}{Question}
	\centerline{Is \npc{} closed under $\cup, \cap, \cdot, \ast$?}
  \end{alertblock}
\end{frame}
%%%%%%%%%%%%%%%%%%%%
\begin{frame}{\conp{}}
  \[
	\conp = \set{L: \bar{L} \in \np}
  \]

  \[
	\overline{\text{SAT}} = \set{\phi: \phi \text{ is not satisfiable}}
  \]

  \[
	\text{TAUT} = \set{\phi: \phi \text{ is a tautology}}
  \]

  \begin{definition}[\conp]
	$(L \subseteq \set{0,1}^{\ast}) \in \conp$ if there exists a polynomial-time \emph{verifier} $V(x,y)$ such that $\forall x \in \set{0,1}^{\ast}$,
	\[
	  x \in L \iff \red{\forall} y \in \set{0,1}^{\ast}, V(x,y) = 1.
	\]
  \end{definition}
\end{frame}
%%%%%%%%%%%%%%%%%%%%
\begin{frame}{\conp{}}
  \[
	\conp{} \red{\neq} \set{0,1}^{\ast} \setminus \np
  \]

  \[
	\p \subseteq \np \cap \conp
  \]

  \[
	\p = \np \implies \np = \conp = \p
  \]

  \[
	\np \neq \conp \implies \p \neq \np
  \]
\end{frame}
%%%%%%%%%%%%%%%%%%%%
\begin{frame}{\nph{} and \npc}
  \[
	\forall L \in \np, L \pr L' \implies L' \text{ is } \nph
  \]

  \[
	\npc = \np \cap \nph
  \]
\end{frame}
%%%%%%%%%%%%%%%%%%%%
\begin{frame}{\nph{} and \npc}
  \begin{exampleblock}{TC 34.5--6}
	\centerline{$\text{HAM-PATH}$ is $\npc$.}
  \end{exampleblock}

  \[
	\text{HAM-CYCLE} \pr \text{HAM-PATH}
  \]

  \centerline{$\pr$: split $v$ into $v_1, v_2$; add $s, t, (s,v_1), (v_2, t)$}

  \vspace{0.50cm}
  \begin{alertblock}{Question:}
	\[
	  \text{HAM-PATH} \pr \text{HAM-CYCLE}
	\]

	\centerline{$\pr$: add $v'; (v', v), \forall v \in V $}
  \end{alertblock}
\end{frame}
%%%%%%%%%%%%%%%%%%%%
\begin{frame}{\expcls}
  \[
	\expcls = \bigcup_{c \ge 1} \dtime(2^{n^c})
  \]

  \[
    \p \subseteq \np \subseteq \expcls
  \]
\end{frame}
%%%%%%%%%%%%%%%%%%%%
\begin{frame}{Time Hierarchy Theorem}
  \[
	\p \subsetneqq \expcls
  \]

  \begin{theorem}[Time hierarchy theorem]
	\[
	  f(n) \log f(n) = o(g(n)) \implies \emph{\dtime}(f(n)) \subsetneqq \emph{\dtime}(g(n))
	\]
  \end{theorem}
\end{frame}
%%%%%%%%%%%%%%%%%%%%
\begin{frame}{\rcls}
  \[
	\rcls = \dtime(<\infty)
  \]

  \vspace{0.80cm}
  \centerline{\#decidable \emph{vs.} \#undecidable}
\end{frame}
%%%%%%%%%%%%%%%%%%%%
\begin{frame}{\pspace}
  \[
	\pspace = \bigcup_{c > 0} \dspace(n^c)
  \]

  \[
	\p \subseteq \pspace
  \]
  
  \[
	\np \subseteq \pspace \subseteq \expcls
  \]
\end{frame}
%%%%%%%%%%%%%%%%%%%%
\begin{frame}{\pspacec}
  \begin{definition}[QBF: Quantified Boolean Formula]
	\[
	  Q_1 x_1 Q_2 x_2 \cdots Q_n x_n \varphi(x_1, x_2, \ldots, x_n)
	\]

	\[
	  Q_i: \forall, \exists
	\]
  \end{definition}

  \[
	\text{TQBF} = \set{\text{True QBF}} \in \pspacec
  \]

  \[
	\text{SAT}: \phi = \exists x_1,\ldots,x_n \varphi(x_1, x_2, \ldots, x_n) \in \npc
  \]

  \[
	\text{TAUT}: \phi = \forall x_1,\ldots,x_n \varphi(x_1, x_2, \ldots, x_n) \in \conpc
  \]
\end{frame}
%%%%%%%%%%%%%%%%%%%%
\begin{frame}{\pspacec}
  \begin{exampleblock}{The QBF game}
	\[
	  \varphi(x_1,x_2,\dots,x_{2n})
	\]

	\begin{center}
	  Player 1 wins $\iff$ $\varphi(x_1,x_2,\dots,x_{2n})$ is true.\\[6pt]
	  Does player 1 has a \emph{winning strategy}?
	\end{center}

	\[
	  \exists x_1 \forall x_2 \exists x_3 \forall x_4 \cdots \forall x_{2n} \varphi(x_1,x_2,\dots,x_{2n})
	\]
  \end{exampleblock}
\end{frame}
%%%%%%%%%%%%%%%%%%%%
\begin{frame}{PH}

\end{frame}
%%%%%%%%%%%%%%%%%%%%

\section{Reductions: Tetris is \npc{}}

%%%%%%%%%%%%%%%%%%%%
\begin{frame}{Tetris is \npc{}}
  \begin{alertblock}{References}
	\begin{itemize}
	  \item ``6.890 Algorithmic Lower Bounds: Fun with Hardness Proofs'',
	  	by Prof. Erik Demaine, Fall 2014 (Lecture 03, from 00:51:00)
	  \item ``Tetris is Hard, Made Easy'' by Ron Breukelaar, Hendrik Jan Hoogeboom, and Walter A. Kosters, 2003
	\end{itemize}
  \end{alertblock}
\end{frame}
%%%%%%%%%%%%%%%%%%%%
\begin{frame}{Tetris}
  \fignocaption{width = 0.45\textwidth}{figs/tetris.jpg}
\end{frame}
%%%%%%%%%%%%%%%%%%%%
\begin{frame}{\tetris}
  \begin{definition}[\tetris: The Tetris Problem]
  \end{definition}

  \[
	\tetris \in \np
  \]
\end{frame}
%%%%%%%%%%%%%%%%%%%%
\begin{frame}{\partition}
  \begin{definition}[\partition]
  \end{definition}
\end{frame}
%%%%%%%%%%%%%%%%%%%%
\begin{frame}{$\partition \pr \tetris$: the initial board}
  \fignocaption{width = 0.45\textwidth}{figs/tetris-init.png}
\end{frame}
%%%%%%%%%%%%%%%%%%%%
\begin{frame}{$\partition \pr \tetris$: the piece sequence}
  \fignocaption{width = 0.45\textwidth}{figs/tetris-seq.png}
\end{frame}
%%%%%%%%%%%%%%%%%%%%
\begin{frame}{$\partition \pr \tetris$: ``$\implies$''}
  \begin{columns}
	\column{0.60\textwidth}
	  \fignocaption{width = 0.70\textwidth}{figs/tetris-ai.png}
	\column{0.30\textwidth}
	  \fignocaption{width = 0.75\textwidth}{figs/tetris-subset-filler.png}
  \end{columns}
\end{frame}
%%%%%%%%%%%%%%%%%%%%
\begin{frame}{$\partition \pr \tetris$: ``$\;\Longleftarrow\;$''}
\end{frame}
%%%%%%%%%%%%%%%%%%%%
%%%%%%%%%%%%%%%%%%%%
%%%%%%%%%%%%%%%%%%%%
%%%%%%%%%%%%%%%%%%%%
%%%%%%%%%%%%%%%%%%%%
%%%%%%%%%%%%%%%%%%%%


\end{document}
%%%%%%%%%%
