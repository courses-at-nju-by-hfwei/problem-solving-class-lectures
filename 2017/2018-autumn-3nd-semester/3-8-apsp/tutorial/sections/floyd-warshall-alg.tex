% file: sections/floyd-warshall-alg.tex

%%%%%%%%%%%%%%%%%%%%
\begin{frame}{}
  \fig{width = 0.20\textwidth}{figs/floyd}
  {\centerline{\teal{Robert W. Floyd ($1936$--$2001$)}}}

  \begin{quote}
    {\small
      For having a clear influence on \red{methodologies} for the creation of efficient and reliable software,
      and for helping to \red{found} the following important subfields of computer science: \\[3pt] \pause
      \violet{the theory of parsing, the semantics of programming languages, \\ automatic program verification, 
      automatic program synthesis, \\ and analysis of algorithms}
    }

    \hfill --- \teal{Turing Award}, $1978$
  \end{quote}
\end{frame}
%%%%%%%%%%%%%%%%%%%%

%%%%%%%%%%%%%%%%%%%%
\begin{frame}{}
  \[
    \red{\boxed{d_{ij}^{(k)}:}}
  \]
  \begin{center}
    the weight of a shortest path from $i$ to $j$ \\[3pt]
    for which all \purple{\it intermediate} vertices are in $\set{1, 2, \cdots, k}$
  \end{center}

  \pause
  \[
    \blue{\boxed{D^{(n)} \triangleq \Big(d_{ij}^{(n)}\Big)}}
  \]
\end{frame}
%%%%%%%%%%%%%%%%%%%%

%%%%%%%%%%%%%%%%%%%%
\begin{frame}{}
  \[
    \red{k \in \text{SP}_{ij}^{(k)}?}
  \]

  \vspace{-0.20cm}
  \fig{width = 0.50\textwidth}{figs/floyd-warshall-dp}

  \vspace{-0.20cm}
  \[
    d_{ij}^{(k)} = \begin{cases}
      w_{ij}	& k = 0 \\
      \min\Big\{ d_{ij}^{(k-1)}, \underbrace{d_{ik}^{(k-1)} + d_{kj}^{(k-1)}}_{\onslide<2->{\red{\text{why?}}}} \Big\} & k \ge 1
    \end{cases}
  \]

  \uncover<3->{
    \begin{quote}
      $\cdots$, but we assume that there are \red{no} negative-weight cycles. \\
      \hfill --- Section $25.2$ of CLRS
    \end{quote}
  }
\end{frame}
%%%%%%%%%%%%%%%%%%%%

%%%%%%%%%%%%%%%%%%%%
\begin{frame}{}
  % file: sections/floyd-warshall-alg.tex

%%%%%%%%%%%%%%%%%%%%
\begin{frame}{}
  \fig{width = 0.20\textwidth}{figs/floyd}
  {\centerline{\teal{Robert W. Floyd ($1936$--$2001$)}}}

  \begin{quote}
    {\small
      For having a clear influence on \red{methodologies} for the creation of efficient and reliable software,
      and for helping to \red{found} the following important subfields of computer science: \\[3pt] \pause
      \violet{the theory of parsing, the semantics of programming languages, \\ automatic program verification, 
      automatic program synthesis, \\ and analysis of algorithms}
    }

    \hfill --- \teal{Turing Award}, $1978$
  \end{quote}
\end{frame}
%%%%%%%%%%%%%%%%%%%%

%%%%%%%%%%%%%%%%%%%%
\begin{frame}{}
  \[
    \red{\boxed{d_{ij}^{(k)}:}}
  \]
  \begin{center}
    the weight of a shortest path from $i$ to $j$ \\[3pt]
    for which all \purple{\it intermediate} vertices are in $\set{1, 2, \cdots, k}$
  \end{center}

  \pause
  \[
    \blue{\boxed{D^{(n)} \triangleq \Big(d_{ij}^{(n)}\Big)}}
  \]
\end{frame}
%%%%%%%%%%%%%%%%%%%%

%%%%%%%%%%%%%%%%%%%%
\begin{frame}{}
  \[
    \red{k \in \text{SP}_{ij}^{(k)}?}
  \]

  \vspace{-0.20cm}
  \fig{width = 0.50\textwidth}{figs/floyd-warshall-dp}

  \vspace{-0.20cm}
  \[
    d_{ij}^{(k)} = \begin{cases}
      w_{ij}	& k = 0 \\
      \min\Big\{ d_{ij}^{(k-1)}, \underbrace{d_{ik}^{(k-1)} + d_{kj}^{(k-1)}}_{\onslide<2->{\red{\text{why?}}}} \Big\} & k \ge 1
    \end{cases}
  \]

  \uncover<3->{
    \begin{quote}
      $\cdots$, but we assume that there are \red{no} negative-weight cycles. \\
      \hfill --- Section $25.2$ of CLRS
    \end{quote}
  }
\end{frame}
%%%%%%%%%%%%%%%%%%%%

%%%%%%%%%%%%%%%%%%%%
\begin{frame}{}
  % file: sections/floyd-warshall-alg.tex

%%%%%%%%%%%%%%%%%%%%
\begin{frame}{}
  \fig{width = 0.20\textwidth}{figs/floyd}
  {\centerline{\teal{Robert W. Floyd ($1936$--$2001$)}}}

  \begin{quote}
    {\small
      For having a clear influence on \red{methodologies} for the creation of efficient and reliable software,
      and for helping to \red{found} the following important subfields of computer science: \\[3pt] \pause
      \violet{the theory of parsing, the semantics of programming languages, \\ automatic program verification, 
      automatic program synthesis, \\ and analysis of algorithms}
    }

    \hfill --- \teal{Turing Award}, $1978$
  \end{quote}
\end{frame}
%%%%%%%%%%%%%%%%%%%%

%%%%%%%%%%%%%%%%%%%%
\begin{frame}{}
  \[
    \red{\boxed{d_{ij}^{(k)}:}}
  \]
  \begin{center}
    the weight of a shortest path from $i$ to $j$ \\[3pt]
    for which all \purple{\it intermediate} vertices are in $\set{1, 2, \cdots, k}$
  \end{center}

  \pause
  \[
    \blue{\boxed{D^{(n)} \triangleq \Big(d_{ij}^{(n)}\Big)}}
  \]
\end{frame}
%%%%%%%%%%%%%%%%%%%%

%%%%%%%%%%%%%%%%%%%%
\begin{frame}{}
  \[
    \red{k \in \text{SP}_{ij}^{(k)}?}
  \]

  \vspace{-0.20cm}
  \fig{width = 0.50\textwidth}{figs/floyd-warshall-dp}

  \vspace{-0.20cm}
  \[
    d_{ij}^{(k)} = \begin{cases}
      w_{ij}	& k = 0 \\
      \min\Big\{ d_{ij}^{(k-1)}, \underbrace{d_{ik}^{(k-1)} + d_{kj}^{(k-1)}}_{\onslide<2->{\red{\text{why?}}}} \Big\} & k \ge 1
    \end{cases}
  \]

  \uncover<3->{
    \begin{quote}
      $\cdots$, but we assume that there are \red{no} negative-weight cycles. \\
      \hfill --- Section $25.2$ of CLRS
    \end{quote}
  }
\end{frame}
%%%%%%%%%%%%%%%%%%%%

%%%%%%%%%%%%%%%%%%%%
\begin{frame}{}
  % file: sections/floyd-warshall-alg.tex

%%%%%%%%%%%%%%%%%%%%
\begin{frame}{}
  \fig{width = 0.20\textwidth}{figs/floyd}
  {\centerline{\teal{Robert W. Floyd ($1936$--$2001$)}}}

  \begin{quote}
    {\small
      For having a clear influence on \red{methodologies} for the creation of efficient and reliable software,
      and for helping to \red{found} the following important subfields of computer science: \\[3pt] \pause
      \violet{the theory of parsing, the semantics of programming languages, \\ automatic program verification, 
      automatic program synthesis, \\ and analysis of algorithms}
    }

    \hfill --- \teal{Turing Award}, $1978$
  \end{quote}
\end{frame}
%%%%%%%%%%%%%%%%%%%%

%%%%%%%%%%%%%%%%%%%%
\begin{frame}{}
  \[
    \red{\boxed{d_{ij}^{(k)}:}}
  \]
  \begin{center}
    the weight of a shortest path from $i$ to $j$ \\[3pt]
    for which all \purple{\it intermediate} vertices are in $\set{1, 2, \cdots, k}$
  \end{center}

  \pause
  \[
    \blue{\boxed{D^{(n)} \triangleq \Big(d_{ij}^{(n)}\Big)}}
  \]
\end{frame}
%%%%%%%%%%%%%%%%%%%%

%%%%%%%%%%%%%%%%%%%%
\begin{frame}{}
  \[
    \red{k \in \text{SP}_{ij}^{(k)}?}
  \]

  \vspace{-0.20cm}
  \fig{width = 0.50\textwidth}{figs/floyd-warshall-dp}

  \vspace{-0.20cm}
  \[
    d_{ij}^{(k)} = \begin{cases}
      w_{ij}	& k = 0 \\
      \min\Big\{ d_{ij}^{(k-1)}, \underbrace{d_{ik}^{(k-1)} + d_{kj}^{(k-1)}}_{\onslide<2->{\red{\text{why?}}}} \Big\} & k \ge 1
    \end{cases}
  \]

  \uncover<3->{
    \begin{quote}
      $\cdots$, but we assume that there are \red{no} negative-weight cycles. \\
      \hfill --- Section $25.2$ of CLRS
    \end{quote}
  }
\end{frame}
%%%%%%%%%%%%%%%%%%%%

%%%%%%%%%%%%%%%%%%%%
\begin{frame}{}
  \input{algs/floyd-warshall-alg}

  \pause
  \[
    \text{Space}: \Theta(n^3) \pause \implies \Theta(n^2)
  \]
\end{frame}
%%%%%%%%%%%%%%%%%%%%

%%%%%%%%%%%%%%%%%%%%
\begin{frame}{}
  \begin{exampleblock}{\textsc{Floyd-Warshall} Made Simple (Problem $25.2$-$4$)}
    \input{algs/floyd-warshall-alg-simplified}
  \end{exampleblock}

  \pause
  \vspace{0.60cm}
  \[
    d_{ij}^{(k-1)}, d_{ik}^{(k-1)}, d_{kj}^{(k-1)} \text{ does not change.}
  \]
\end{frame}
%%%%%%%%%%%%%%%%%%%%

%%%%%%%%%%%%%%%%%%%%
\begin{frame}{}
  \begin{exampleblock}{Negative-weight Cycle Detection (Problem $25.2$-$6$)}
    To detect negative-weight cycle (\textsf{NC}) using \textsc{Floyd-Warshall}.
  \end{exampleblock}

  \pause
  \[
    \red{\exists i: d_{ii}^{(n)} < 0}
  \]

  \pause
  \begin{proof}
    \begin{columns}
      \column{0.50\textwidth}
	\pause
	\fig{width = 0.40\textwidth}{figs/trivial-proof}
      \column{0.50\textwidth}
	\pause
	\fig{width = 0.60\textwidth}{figs/show-me-your-proof}
    \end{columns}

    \pause
    \vspace{0.50cm}
    \[
      \exists i: d_{ii}^{(n)} < 0 \iff \exists\; \textsf{NC} \subseteq G
    \]
  \end{proof}
\end{frame}
%%%%%%%%%%%%%%%%%%%%

%%%%%%%%%%%%%%%%%%%%
\begin{frame}{}
  \[
    \exists i: d_{ii}^{(n)} < 0 \iff \exists\; \textsf{NC} \subseteq G
  \]

  \begin{proof}
    \begin{columns}
      \column{0.50\textwidth}
        \pause
	\[
	  \teal{``\implies"}
	\]

	\pause
	\fig{width = 0.80\textwidth}{figs/negative-weight-cycle}
	{\vspace{-0.30cm}\centerline{\teal{A Negative Cycle}}}
      \column{0.50\textwidth}
        \pause
	\[
	  \teal{``\Longleftarrow"}
	\]

	\pause
	\fig{width = 0.30\textwidth}{figs/negative-weight-cycle-k}
	{\vspace{-0.30cm}\centerline{\teal{A Simple Negative Cycle}}}
	\[
	  k: \max \#
	\]

	\pause
	\[
	  \red{d_{ii}^{(k)} < 0}
	\]
    \end{columns}
  \end{proof}
\end{frame}
%%%%%%%%%%%%%%%%%%%%

% problem 25.2-8


  \pause
  \[
    \text{Space}: \Theta(n^3) \pause \implies \Theta(n^2)
  \]
\end{frame}
%%%%%%%%%%%%%%%%%%%%

%%%%%%%%%%%%%%%%%%%%
\begin{frame}{}
  \begin{exampleblock}{\textsc{Floyd-Warshall} Made Simple (Problem $25.2$-$4$)}
    % file: algs/floyd-warshall-alg-simplified.tex

\begin{algorithm}[H]
  % \caption{Floyd-Warshall Algorithm Made Simple.}
  % \label{alg:floyd-warshall-simplified}
  \begin{algorithmic}[1]
    \Procedure{Floyd-Warshall-Simplified}{$W$}
      \State $D = W$

      \hStatex
      \For{$k \gets 1 \;\text{\bf to}\; n$}
	\For{$i \gets 1 \;\text{\bf to}\; n$}
	  \For{$j \gets 1 \;\text{\bf to}\; n$}
	    \State $d_{ij} = \min \Big\{ d_{ij}, d_{ik} + d_{kj} \Big\}$
	  \EndFor
	\EndFor
      \EndFor

      \hStatex
      \State \Return $D$
    \EndProcedure
  \end{algorithmic}
\end{algorithm}

  \end{exampleblock}

  \pause
  \vspace{0.60cm}
  \[
    d_{ij}^{(k-1)}, d_{ik}^{(k-1)}, d_{kj}^{(k-1)} \text{ does not change.}
  \]
\end{frame}
%%%%%%%%%%%%%%%%%%%%

%%%%%%%%%%%%%%%%%%%%
\begin{frame}{}
  \begin{exampleblock}{Negative-weight Cycle Detection (Problem $25.2$-$6$)}
    To detect negative-weight cycle (\textsf{NC}) using \textsc{Floyd-Warshall}.
  \end{exampleblock}

  \pause
  \[
    \red{\exists i: d_{ii}^{(n)} < 0}
  \]

  \pause
  \begin{proof}
    \begin{columns}
      \column{0.50\textwidth}
	\pause
	\fig{width = 0.40\textwidth}{figs/trivial-proof}
      \column{0.50\textwidth}
	\pause
	\fig{width = 0.60\textwidth}{figs/show-me-your-proof}
    \end{columns}

    \pause
    \vspace{0.50cm}
    \[
      \exists i: d_{ii}^{(n)} < 0 \iff \exists\; \textsf{NC} \subseteq G
    \]
  \end{proof}
\end{frame}
%%%%%%%%%%%%%%%%%%%%

%%%%%%%%%%%%%%%%%%%%
\begin{frame}{}
  \[
    \exists i: d_{ii}^{(n)} < 0 \iff \exists\; \textsf{NC} \subseteq G
  \]

  \begin{proof}
    \begin{columns}
      \column{0.50\textwidth}
        \pause
	\[
	  \teal{``\implies"}
	\]

	\pause
	\fig{width = 0.80\textwidth}{figs/negative-weight-cycle}
	{\vspace{-0.30cm}\centerline{\teal{A Negative Cycle}}}
      \column{0.50\textwidth}
        \pause
	\[
	  \teal{``\Longleftarrow"}
	\]

	\pause
	\fig{width = 0.30\textwidth}{figs/negative-weight-cycle-k}
	{\vspace{-0.30cm}\centerline{\teal{A Simple Negative Cycle}}}
	\[
	  k: \max \#
	\]

	\pause
	\[
	  \red{d_{ii}^{(k)} < 0}
	\]
    \end{columns}
  \end{proof}
\end{frame}
%%%%%%%%%%%%%%%%%%%%

% problem 25.2-8


  \pause
  \[
    \text{Space}: \Theta(n^3) \pause \implies \Theta(n^2)
  \]
\end{frame}
%%%%%%%%%%%%%%%%%%%%

%%%%%%%%%%%%%%%%%%%%
\begin{frame}{}
  \begin{exampleblock}{\textsc{Floyd-Warshall} Made Simple (Problem $25.2$-$4$)}
    % file: algs/floyd-warshall-alg-simplified.tex

\begin{algorithm}[H]
  % \caption{Floyd-Warshall Algorithm Made Simple.}
  % \label{alg:floyd-warshall-simplified}
  \begin{algorithmic}[1]
    \Procedure{Floyd-Warshall-Simplified}{$W$}
      \State $D = W$

      \hStatex
      \For{$k \gets 1 \;\text{\bf to}\; n$}
	\For{$i \gets 1 \;\text{\bf to}\; n$}
	  \For{$j \gets 1 \;\text{\bf to}\; n$}
	    \State $d_{ij} = \min \Big\{ d_{ij}, d_{ik} + d_{kj} \Big\}$
	  \EndFor
	\EndFor
      \EndFor

      \hStatex
      \State \Return $D$
    \EndProcedure
  \end{algorithmic}
\end{algorithm}

  \end{exampleblock}

  \pause
  \vspace{0.60cm}
  \[
    d_{ij}^{(k-1)}, d_{ik}^{(k-1)}, d_{kj}^{(k-1)} \text{ does not change.}
  \]
\end{frame}
%%%%%%%%%%%%%%%%%%%%

%%%%%%%%%%%%%%%%%%%%
\begin{frame}{}
  \begin{exampleblock}{Negative-weight Cycle Detection (Problem $25.2$-$6$)}
    To detect negative-weight cycle (\textsf{NC}) using \textsc{Floyd-Warshall}.
  \end{exampleblock}

  \pause
  \[
    \red{\exists i: d_{ii}^{(n)} < 0}
  \]

  \pause
  \begin{proof}
    \begin{columns}
      \column{0.50\textwidth}
	\pause
	\fig{width = 0.40\textwidth}{figs/trivial-proof}
      \column{0.50\textwidth}
	\pause
	\fig{width = 0.60\textwidth}{figs/show-me-your-proof}
    \end{columns}

    \pause
    \vspace{0.50cm}
    \[
      \exists i: d_{ii}^{(n)} < 0 \iff \exists\; \textsf{NC} \subseteq G
    \]
  \end{proof}
\end{frame}
%%%%%%%%%%%%%%%%%%%%

%%%%%%%%%%%%%%%%%%%%
\begin{frame}{}
  \[
    \exists i: d_{ii}^{(n)} < 0 \iff \exists\; \textsf{NC} \subseteq G
  \]

  \begin{proof}
    \begin{columns}
      \column{0.50\textwidth}
        \pause
	\[
	  \teal{``\implies"}
	\]

	\pause
	\fig{width = 0.80\textwidth}{figs/negative-weight-cycle}
	{\vspace{-0.30cm}\centerline{\teal{A Negative Cycle}}}
      \column{0.50\textwidth}
        \pause
	\[
	  \teal{``\Longleftarrow"}
	\]

	\pause
	\fig{width = 0.30\textwidth}{figs/negative-weight-cycle-k}
	{\vspace{-0.30cm}\centerline{\teal{A Simple Negative Cycle}}}
	\[
	  k: \max \#
	\]

	\pause
	\[
	  \red{d_{ii}^{(k)} < 0}
	\]
    \end{columns}
  \end{proof}
\end{frame}
%%%%%%%%%%%%%%%%%%%%

% problem 25.2-8


  \pause
  \[
    \text{Space}: \Theta(n^3) \pause \implies \Theta(n^2)
  \]
\end{frame}
%%%%%%%%%%%%%%%%%%%%

%%%%%%%%%%%%%%%%%%%%
\begin{frame}{}
  \begin{exampleblock}{\textsc{Floyd-Warshall} Made Simple (Problem $25.2$-$4$)}
    % file: algs/floyd-warshall-alg-simplified.tex

\begin{algorithm}[H]
  % \caption{Floyd-Warshall Algorithm Made Simple.}
  % \label{alg:floyd-warshall-simplified}
  \begin{algorithmic}[1]
    \Procedure{Floyd-Warshall-Simplified}{$W$}
      \State $D = W$

      \hStatex
      \For{$k \gets 1 \;\text{\bf to}\; n$}
	\For{$i \gets 1 \;\text{\bf to}\; n$}
	  \For{$j \gets 1 \;\text{\bf to}\; n$}
	    \State $d_{ij} = \min \Big\{ d_{ij}, d_{ik} + d_{kj} \Big\}$
	  \EndFor
	\EndFor
      \EndFor

      \hStatex
      \State \Return $D$
    \EndProcedure
  \end{algorithmic}
\end{algorithm}

  \end{exampleblock}

  \pause
  \vspace{0.60cm}
  \[
    d_{ij}^{(k-1)}, d_{ik}^{(k-1)}, d_{kj}^{(k-1)} \text{ does not change.}
  \]
\end{frame}
%%%%%%%%%%%%%%%%%%%%

%%%%%%%%%%%%%%%%%%%%
\begin{frame}{}
  \begin{exampleblock}{Negative-weight Cycle Detection (Problem $25.2$-$6$)}
    To detect negative-weight cycle (\textsf{NC}) using \textsc{Floyd-Warshall}.
  \end{exampleblock}

  \pause
  \[
    \red{\exists i: d_{ii}^{(n)} < 0}
  \]

  \pause
  \begin{proof}
    \begin{columns}
      \column{0.50\textwidth}
	\pause
	\fig{width = 0.40\textwidth}{figs/trivial-proof}
      \column{0.50\textwidth}
	\pause
	\fig{width = 0.60\textwidth}{figs/show-me-your-proof}
    \end{columns}

    \pause
    \vspace{0.50cm}
    \[
      \exists i: d_{ii}^{(n)} < 0 \iff \exists\; \textsf{NC} \subseteq G
    \]
  \end{proof}
\end{frame}
%%%%%%%%%%%%%%%%%%%%

%%%%%%%%%%%%%%%%%%%%
\begin{frame}{}
  \[
    \exists i: d_{ii}^{(n)} < 0 \iff \exists\; \textsf{NC} \subseteq G
  \]

  \begin{proof}
    \begin{columns}
      \column{0.50\textwidth}
        \pause
	\[
	  \teal{``\implies"}
	\]

	\pause
	\fig{width = 0.80\textwidth}{figs/negative-weight-cycle}
	{\vspace{-0.30cm}\centerline{\teal{A Negative Cycle}}}
      \column{0.50\textwidth}
        \pause
	\[
	  \teal{``\Longleftarrow"}
	\]

	\pause
	\fig{width = 0.30\textwidth}{figs/negative-weight-cycle-k}
	{\vspace{-0.30cm}\centerline{\teal{A Simple Negative Cycle}}}
	\[
	  k: \max \#
	\]

	\pause
	\[
	  \red{d_{ii}^{(k)} < 0}
	\]
    \end{columns}
  \end{proof}
\end{frame}
%%%%%%%%%%%%%%%%%%%%

% problem 25.2-8
