%%%%%%%%%%%%%%%
\begin{frame}{}
  \begin{columns}
    \column{0.45\textwidth}
      \fignocaption{width = 0.45\textwidth}{figs/cantor-infinity}
      \vspace{-0.20cm}
      {\centerline{\footnotesize Georg Cantor (1845 -- 1918)}}
    \column{0.45\textwidth}
      \only<5->{
	\fignocaption{width = 0.40\textwidth}{figs/hilbert}
	\vspace{-0.20cm}
	{\centerline{\footnotesize David Hilbert (1862 -- 1943)}}
      }
  \end{columns}

  \begin{columns}
    \pause
    \column{0.30\textwidth}
      \fignocaption{width = 0.50\textwidth}{figs/kronecker}
      \vspace{-0.50cm}
      \begin{center}
	{\footnotesize Leopold Kronecker}
      
        {\footnotesize (1823 -- 1891)}
      \end{center}
    \pause
    \column{0.30\textwidth}
      \fignocaption{width = 0.50\textwidth}{figs/poincare}
      \vspace{-0.50cm}
      \begin{center}
	{\footnotesize Henri Poincar\'{e}}
      
        {\footnotesize (1854 -- 1912)}
      \end{center}
    \pause
    \column{0.30\textwidth}
      \fignocaption{width = 0.48\textwidth}{figs/wittgenstein}
      \vspace{-0.50cm}
      \begin{center}
	{\footnotesize Ludwig Wittgenstein}
      
        {\footnotesize (1889 -- 1951)}
      \end{center}
  \end{columns}
\end{frame}
%%%%%%%%%%%%%%%

%%%%%%%%%%%%%%%
\begin{frame}{}
  \begin{quote}
    From his paradise that Cantor with us unfolded, 
    we hold our breath in awe; knowing, we shall not be expelled.

    \hfill --- David Hilbert
  \end{quote}

  \vspace{0.80cm}
  \begin{quote}
    \centerline{``没有人能把我们从 Cantor 创造的乐园中驱逐出去''}
  \end{quote}

  \pause
  \fignocaption{width = 0.35\textwidth}{figs/cat-door}
\end{frame}
%%%%%%%%%%%%%%%

%%%%%%%%%%%%%%%
\begin{frame}{}
  \fignocaption{width = 0.55\textwidth}{figs/cantor-monumento}

  \pause
  \begin{quote}
    \begin{center}
    ``das wesen der mathematik liegt in ihrer freiheit'' \\[8pt]
    \pause
    ``The essence of mathematics lies in its freedom''
    \end{center}
  \end{quote}
\end{frame}
%%%%%%%%%%%%%%%

%%%%%%%%%%%%%%%
\begin{frame}{}
  \begin{columns}
    \column{0.50\textwidth}
      \fignocaption{width = 0.50\textwidth}{figs/galileo}{\centerline{Galileo Galilei (1564 -- 1642)}}
    \column{0.50\textwidth}
      \fignocaption{width = 0.45\textwidth}{figs/galileo-book}{\centerline{《关于两门新科学的对话》(1638)}}
  \end{columns}
\end{frame}
%%%%%%%%%%%%%%%

%%%%%%%%%%%%%%%
\begin{frame}{}
  \centerline{``用我们有限的心智来讨论无限 $\cdots$''}

  \[
    S_1 = \set{1, 2, 3, \cdots, n, \cdots}
  \]
  \[
    S_2 = \set{1, 4, 9, \cdots, n^2, \cdots}
  \]

  \begin{columns}
    \column{0.50\textwidth}
      \pause
      \[
	\blue{|S_1| = |S_2|, S_1 \subsetneqq S_2}
      \]

      \pause
      \centerline{\red{``部分等于全体''}}
    \column{0.50\textwidth}
      \pause
      \fignocaption{width = 0.35\textwidth}{figs/chijing}
  \end{columns}

  \pause
  \vspace{0.20cm}
  \begin{quote}
    说到底,``等于''、``大于''和``小于''诸性质不能用于无限,而只能用于有限的数量。 \hfill --- Galileo Galilei
  \end{quote}

  \pause
  \vspace{0.20cm}
  \begin{quote}
    无穷数是不可能的。 \hfill --- Gottfried Wilhelm Leibniz
  \end{quote}
\end{frame}
%%%%%%%%%%%%%%%

%%%%%%%%%%%%%%%
\begin{frame}{}
  \begin{quote}
    这些证明一开始就期望那些数要具有有穷数的一切性质,
    或者甚至于\blue{把有穷数的性质强加于无穷}。\\[6pt]

    相反,这些无穷数,如果它们能够以任何形式被理解的话,
    倒是由于它们与有穷数的对应,\red{它们必须具有完全新的数量特征}。\\[6pt]

    \teal{这些性质完全依赖于事物的本性},$\cdots$而并非来自我们的主观任意性
    或我们的偏见。

    \hfill --- Georg Cantor (1885)
  \end{quote}

  \pause
  \vspace{0.40cm}
  \begin{definition}[Dedekind-infinite \& Dedekind-finite (Dedekind, 1888)]
    A set $A$ is \blue{Dedekind-infinite} if there is a bijective function from $A$ onto some proper subset $B$ of $A$. \\[3pt]
    
    A set is \blue{Dedekind-finite} if it is not Dedekind-infinite.
  \end{definition}
\end{frame}
%%%%%%%%%%%%%%%

%%%%%%%%%%%%%%%
\begin{frame}{}
  \fignocaption{width = 0.35\textwidth}{figs/infinity-logo}

  \centerline{\LARGE Comparing Sets}

  \pause
  \vspace{-0.30cm}
  \begin{columns}
    \column{0.50\textwidth}
      \fignocaption{width = 0.50\textwidth}{figs/equal-logo}
    \column{0.50\textwidth}
      \fignocaption{width = 0.50\textwidth}{figs/less-logo}
  \end{columns}

  \pause
  \centerline{\Large \red{Function}}
\end{frame}
%%%%%%%%%%%%%%%
