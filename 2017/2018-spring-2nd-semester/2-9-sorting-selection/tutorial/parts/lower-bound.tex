% file: parts/lower-bound.tex

%%%%%%%%%%%%%%%
\begin{frame}{}
  \begin{exampleblock}{Sorts a $\frac{n}{k}$-sorted Array (Problem $8.1-4$)}
    \fignocaption{width = 0.40\textwidth}{figs/k-sorted}

    \pause
    \vspace{-0.50cm}
    \[
      \Omega(n \log k) \pause \qquad O(n \log k)
    \]
  \end{exampleblock}

  \pause
  \vspace{-0.30cm}
  \[
    \Omega: \frac{n}{k} (k \log k)
  \]

  \pause
  \[
    \teal{(k!)^{\frac{n}{k}}} \le \red{L} \le \teal{2^{H}}
  \]
\end{frame}
%%%%%%%%%%%%%%%

%%%%%%%%%%%%%%%
\begin{frame}{}
  \centerline{\teal{$\frac{n}{k}$-sorts an arbitrary array}}
  \fignocaption{width = 0.40\textwidth}{figs/k-sorted}

  \pause
  \vspace{-0.30cm}
  \[
    \red{O(?)} \pause \qquad \red{\Omega(?)}
  \]

  \pause
  \vspace{-0.30cm}
  \[
    L \ge \red{\binom{n}{\underbrace{k, \ldots, k}_{\frac{n}{k}}}} = \frac{n!}{(k!)^{\frac{n}{k}}} \pause \implies \Omega(n \log(n/k))
  \]
\end{frame}
%%%%%%%%%%%%%%%

%%%%%%%%%%%%%%%
\begin{frame}{}
  \[
    O(n \log k)
  \]

  \pause
  \begin{CJK*}{UTF8}{gbsn}
    \centerline{\red{反馈:} 这是什么意思? 我们并没学过多变量的渐近符号的定义。} 
  \end{CJK*}

  \pause
  \vspace{0.30cm}
  \begin{exampleblock}{Problem $3.1-8$}
    When $n$ and $m$ go to $\infty$ \red{independently at different rates}:
    \begin{align*}
      O(g(n,m)) = \{f(n,m) \mid \exists & c > 0, \exists n_0 > 0, \exists m_0 > 0: \\
	& \forall n \ge n_0 \red{\lor} m \ge m_0, 0 \le f(n,m) \le cg(n,m)\}
    \end{align*}
  \end{exampleblock}

  \pause
  \vspace{0.30cm}
  \[
    k = C_1 \qquad k = \Theta(n); \qquad n \to \infty
  \]
\end{frame}
%%%%%%%%%%%%%%%

%%%%%%%%%%%%%%%
\begin{frame}{}
  \[
    \Theta(1 + \alpha), \quad \alpha = \frac{n}{m}
  \]

  \pause
  \vspace{0.30cm}
  \centerline{\red{\large Two ways of understanding}}

  \pause
  \begin{columns}
    \column{0.50\textwidth}
      \[
	\teal{1 + \alpha}
      \]
      \fignocaption{width = 0.45\textwidth}{figs/knuth-vol3}
    \pause
    \column{0.50\textwidth}
      \[
	\teal{\Theta(1 + \alpha)}
      \]

      \[
	n \to \infty, m \to \infty
      \]

      \[
	n = f(m), m \to \infty
      \]
  \end{columns}
\end{frame}
%%%%%%%%%%%%%%%
