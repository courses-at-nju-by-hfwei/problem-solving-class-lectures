\section{Concepts: Computational Complexity Classes}

%%%%%%%%%%%%%%%%%%%%
\begin{frame}{\p}
  \[
	\p = \bigcup_{c > 0} \dtime(n^c)
  \]

  \begin{exampleblock}{TC 34.1--5}
	\[ 
	  f(n) = O(n^c) \qquad t(n) = O(n^d)
	\]
  \end{exampleblock}

  \[
	\hbox{\sout{$T(n) = kf(n) + t(n)$}}
  \]

  \[
	T_{k}(n) = \sum_{i=1}^{k}f^{(i)}(n) + t(n) 
  \]

  \[
	k = O(1) \text{\emph{ vs. }} k = \Theta(n^{O(1)}) \qquad O \emph{\text{ vs. }} \Theta, \Omega
  \]
\end{frame}
%%%%%%%%%%%%%%%%%%%%
\begin{frame}{\np}
  \begin{definition}[\np]
	$L \in \np$ if $\exists$ polynomial-time \emph{verifier} $V(x,c)$ such that $\forall x \in \set{0,1}^{\ast}$,
	\[
	  x \in L \iff \exists c \in \set{0,1}^{\ast}, V(x,c) = 1.
	\]
  \end{definition}

  \vspace{0.50cm}
  \centerline{\np-problems has short certificates that are easy to verify.}

  \vspace{0.60cm}
  \begin{exampleblock}{TC 34.2--6}
	\[
	  \text{HAM-PATH} \in \np
	\]
  \end{exampleblock}
\end{frame}
%%%%%%%%%%%%%%%%%%%%
\begin{frame}{\np}
  \begin{exampleblock}{TC 34.2--4}
	\np{} is closed under $\cup, \cap, \cdot, \ast$.

	\[
	  L_1 \in \np, L_2 \in \np \implies L = L_1 \circ L_2 \in \np
	\]
  \end{exampleblock}

  \vspace{0.60cm}
  \begin{alertblock}{Question:}
	\centerline{Is \npc{} closed under $\cup, \cap, \cdot, \ast$?}
  \end{alertblock}
\end{frame}
%%%%%%%%%%%%%%%%%%%%
\begin{frame}{\np}
  \begin{theorem}{}
	\np{} is closed under ``$\ast$''.
  \end{theorem}

  \[
	c = c_1\#c_2\#\dots\#c_k\#m_1\&m_2\&\dots\&m_{k-1}
  \]

  $A^{\ast}(x,y): \forall 1 \le k \le |x|$
  \begin{align*}
	c = &c_1\#c_2\#\dots\#c_k\#m_1\&m_2\&\dots\&m_{k-1} \\
	  &\bigwedge \land_{i=1}^{i=k} A(x_i, c_i)
  \end{align*}

  % \fignocaption{width = 0.35\textwidth}{figs/np-closure-star.png}

  \[
	x \in L^{\ast} \iff \exists c, A(x,c) = 1
  \]

  \begin{alertblock}{Reference}
	\url{http://www.dei.unipd.it/~geppo/AA/DOCS/NPC.pdf}
  \end{alertblock}
\end{frame}
%%%%%%%%%%%%%%%%%%%%
\begin{frame}{\conp{}}
  \[
	L \in \np \xLongrightarrow{?} \overline{L} \in \np
  \]

  \[
	\overline{\text{SAT}} = \set{\phi: \phi \text{ is not satisfiable}}
  \]

  \[
	\text{TAUT} = \set{\phi: \phi \text{ is a tautology}}
  \]

  \[
	\conp = \set{L: \bar{L} \in \np}
  \]

  \begin{definition}[\conp]
	$L \in \conp$ if $\exists$ polynomial-time \emph{verifier} $V(x,c)$ such that $\forall x \in \set{0,1}^{\ast}$,
	\[
	  x \in L \iff \red{\forall} c \in \set{0,1}^{\ast}, V(x,c) = 1.
	\]
  \end{definition}
\end{frame}
%%%%%%%%%%%%%%%%%%%%
\begin{frame}{\np{} \emph{vs.} \conp{}}
  \[
	\conp{} \red{\neq} \set{0,1}^{\ast} \setminus \np
  \]

  \[
	\p \subseteq \np \cap \conp
  \]

  \[
	\p = \np \implies \np = \conp
  \]

  \[
	\np \neq \conp \implies \p \neq \np
  \]
\end{frame}
%%%%%%%%%%%%%%%%%%%%
\begin{frame}{\nph{} and \npc}
  \[
	\forall L \in \np, L \pr L' \implies L' \text{ is } \nph
  \]

  \[
	\npc = \np \cap \nph
  \]
\end{frame}
%%%%%%%%%%%%%%%%%%%%
\begin{frame}{\nph{} and \npc}
  \begin{exampleblock}{TC 34.5--6}
	\centerline{$\text{HAM-PATH}$ is $\npc$.}
  \end{exampleblock}

  \[
	\text{HAM-CYCLE} \pr \text{HAM-PATH}
  \]

  \centerline{$\pr$: split $v$ into $v_1, v_2$; add $s, t, (s,v_1), (v_2, t)$}

  \vspace{0.50cm}
  \begin{alertblock}{Question:}
	\[
	  \text{HAM-PATH} \pr \text{HAM-CYCLE}
	\]

	\centerline{$\pr$: add $v'; (v', v), \forall v \in V $}
  \end{alertblock}
\end{frame}
%%%%%%%%%%%%%%%%%%%%
\begin{frame}{\p{} \emph{vs.} \np{}}
  \begin{center}
	solve \emph{vs.} verify \\[10pt]
	exhaustive search avoidable?
  \end{center}

  \[
	\p \neq \np \implies \p \neq \npc
  \]

  \begin{theorem}[\npi: Ladner's theorem, 1975]
	\[
	  \p \neq \np \implies \exists L \in \np \setminus \p \land L \notin \npc
	\]
  \end{theorem}

  \centerline{Factoring, Graph (group) isomorphism {\footnotesize (\emph{vs. Subgraph isomorphism})}}
\end{frame}
%%%%%%%%%%%%%%%%%%%%
\begin{frame}{\expcls}
  \[
	\expcls = \bigcup_{c > 0} \dtime(2^{n^c})
  \]

  \[
    \p \subseteq \np \subseteq \expcls
  \]
\end{frame}
%%%%%%%%%%%%%%%%%%%%
\begin{frame}{Time Hierarchy Theorem}
  \[
	\p \subsetneqq \expcls
  \]

  \begin{theorem}[Time Hierarchy Theorem, 1965]
	\[
	  f(n) \log f(n) = o(g(n)) \implies \dtime(f(n)) \subsetneqq \dtime(g(n))
	\]
  \end{theorem}
\end{frame}
%%%%%%%%%%%%%%%%%%%%
\begin{frame}{\rcls}
  \[
	\rcls = \dtime(<\infty)
  \]

  \[
	\#\text{undecidable} \gg \#\text{decidable}
  \]

  \begin{align*}
	\#\text{algs} &= \mathbb{N} \\
	\#\text{problems} &= 2^{\mathbb{N}} = \mathbb{R}
  \end{align*}
\end{frame}
%%%%%%%%%%%%%%%%%%%%
\begin{frame}{\pspace}
  \[
	\pspace = \bigcup_{c > 0} \dspace(n^c)
  \]

  \[
	\p \subseteq \pspace
  \]
  
  \[
	\np \subseteq \pspace \subseteq \expcls
  \]
\end{frame}
%%%%%%%%%%%%%%%%%%%%
\begin{frame}{\pspacec}
  \begin{definition}[QBF: Quantified Boolean Formula]
	\[
	  Q_1 x_1 Q_2 x_2 \cdots Q_n x_n \varphi(x_1, x_2, \ldots, x_n)
	\]

	\[
	  Q_i: \forall, \exists
	\]
  \end{definition}

  \[
	\text{TQBF} = \set{\text{True QBF}} \in \pspacec
  \]

  \[
	\text{SAT}: \phi = \exists x_1,\ldots,x_n \varphi(x_1, x_2, \ldots, x_n) \in \npc
  \]

  \[
	\text{TAUT}: \phi = \forall x_1,\ldots,x_n \varphi(x_1, x_2, \ldots, x_n) \in \conpc
  \]
\end{frame}
%%%%%%%%%%%%%%%%%%%%
\begin{frame}{\pspacec}
  \begin{exampleblock}{The QBF game}
	\[
	  \varphi(x_1,x_2,\dots,x_{2n})
	\]

	\begin{center}
	  Player 1 wins $\iff$ $\varphi(x_1,x_2,\dots,x_{2n})$ is true.\\[6pt]
	  Does player 1 has a \emph{winning strategy}?
	\end{center}

	\[
	  \exists x_1 \forall x_2 \exists x_3 \forall x_4 \cdots \forall x_{2n} \varphi(x_1,x_2,\dots,x_{2n})
	\]
  \end{exampleblock}
\end{frame}
%%%%%%%%%%%%%%%%%%%%
\begin{frame}{\np{} \emph{vs.} \pspace}
  \[
	\np \stackrel{?}{=} \pspace
  \]
  
  \centerline{Short certificate for winning strategy?}
\end{frame}
%%%%%%%%%%%%%%%%%%%%
\begin{frame}{PH}
  \begin{definition}[Polynomial Hierarchy]
	$L \in \sum_{i}^{p}$ if $\exists$ polynomial-time decidable \emph{relation} $R(x,u_1,u_2,\dots,u_i)$ such that $\forall x \in \set{0,1}^{\ast}$,
	\begin{align*}
	  x \in L \iff \exists u_1 &\in \set{0,1} \forall u_2 \in \set{0,1} \cdots Q_i u_i \in \set{0,1} \\
	 	 &R(x,u_1,u_2,\dots,u_i) = 1
	\end{align*}
  \end{definition}

  \begin{center}
	$\prod_{1}^{p} = \text{co}\sum_{1}^{p}$ \\[10pt]
	$\sum_{1}^{p} = \np \qquad \prod_{1}^{p} = \conp$ \\[10pt]
	$\ph = \bigcup_{i} \sum_{i}^{p}$ \\[15pt]
	$\text{Unique-SAT} \in \sum_{2}^{p}$
  \end{center}
\end{frame}
%%%%%%%%%%%%%%%%%%%%
\begin{frame}{Summary}
  \[
	\p \subseteq \np \subseteq \ph \subseteq \pspace \subseteq \expcls
  \]

  \[
	\p \subsetneqq \expcls
  \]

  \begin{alertblock}{References}
	\begin{itemize}
	  \item ``Computational Complexity --- A Modern Approach'' by Arora and Barak {\footnotesize (the first 5 chapters)}
	  \item ``Computer and Intractability --- A Guide to the Theory of NP-Completeness'' by Garey and Johnson
	\end{itemize}
  \end{alertblock}
\end{frame}  
%%%%%%%%%%%%%%%%%%%%
\begin{frame}{If $\hc \in \p$}
  \begin{exampleblock}{TC 34.2--3}
    \[
	  \hc \in \p \implies \hc\text{-LIST} \in \p
	\]
  \end{exampleblock}

  \begin{enumerate}
	\item starting from $v$
	\item removing each edge $e$ on $v$
	\item checking $G \setminus e$
	\item restoring and marking the critical edge $e = (v, u)$
	\item $v = u$
  \end{enumerate}

  \begin{alertblock}{Reference}
	\url{http://www.cs.wustl.edu/~pless/441/hw3soln.pdf}
  \end{alertblock}

  \begin{alertblock}{Question}
	\centerline{remove $e \in E$ in arbitrary order if $(G \setminus e) \in \hc$?}
  \end{alertblock}
\end{frame}
%%%%%%%%%%%%%%%%%%%%
\begin{frame}{$G^3 \in \hc$}
  \begin{exampleblock}{TC 34.2--11 (Karaganis, 1968)}
	\[
	  G^3 \in \hc
	\]
  \end{exampleblock}

  \begin{alertblock}{References}
	\begin{itemize}
	  \item ``On the Cube of a Graph'' by Jerome J. Karaganis, 1968
	  \item ``The Cube of Every Connected Graph is 1-Hamiltonian'' by Gary Chartrand and S. F. Kapoor, 1968
	  \item \url{http://www.aco.gatech.edu/sites/default/files/documents/comp-fa14sol.pdf}
	  \end{itemize}
  \end{alertblock}
\end{frame}
%%%%%%%%%%%%%%%%%%%%
\begin{frame}{$G^3 \in \hc$}
  \begin{theorem}[$T^3 \in \hc$]
	Let $T = (V, E)$ be a tree. For any edge $e \in E$, there is a Hamilton cycle on $T^3$ that contains $e$.
  \end{theorem}

  \begin{proof}
	\centerline{By induction on subtrees obtained by removing any edge $e = (u,v)$.}
  \end{proof}

  \begin{alertblock}{Question}
	In I.S., choose edge $e = (u,v)$ with $u$ or $v$ being a leaf?
  \end{alertblock}
\end{frame}
%%%%%%%%%%%%%%%%%%%%
