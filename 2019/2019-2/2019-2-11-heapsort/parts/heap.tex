% heap.tex

%%%%%%%%%%%%%%%
\begin{frame}{}
  \begin{exampleblock}{Heap Identity (Additional)}
    \[
      \forall h \ge 1: \lceil \log(\lfloor \frac{1}{2}h \rfloor + 1)\rceil + 1 = \lceil \log (h+1) \rceil
    \]
  \end{exampleblock}

  \pause
  \[
    \blue{\boxed{\lceil \log (h + 1) \rceil = \lfloor \log h \rfloor + 1, \forall h \ge 1}}
  \]

  \pause
  \[
    \red{\lfloor \log \lfloor \frac{1}{2} h \rfloor \rfloor + 1}
    = \lceil \log(\lfloor \frac{1}{2}h \rfloor + 1)\rceil 
    = \lceil \log (h+1) \rceil - 1
    = \red{\lfloor \log h \rfloor}
  \]

  \pause
  \[
    \teal{\text{Depth }} \text{of } h = (\teal{\text{Depth }} \text{of the parent of } h) + 1 
  \]
\end{frame}
%%%%%%%%%%%%%%%

%%%%%%%%%%%%%%%
% \begin{frame}{}
%   \begin{exampleblock}{\# of Nodes at Height $h$ (TC $6.3-3$)}
%     There are at most $\lceil \frac{n}{2^{h+1}} \rceil$ nodes of height $h$ in any $n$-element heap.
%   \end{exampleblock}
% \end{frame}
%%%%%%%%%%%%%%%

%%%%%%%%%%%%%%%
% \begin{frame}{}
%   \begin{exampleblock}{Sum of Heights of Nodes}
%     In an $n$-element heap, we have
%     \[
%       \sum_{\text{node } x} H(x) \le n-1
%     \]
%   \end{exampleblock}
% \end{frame}
%%%%%%%%%%%%%%%