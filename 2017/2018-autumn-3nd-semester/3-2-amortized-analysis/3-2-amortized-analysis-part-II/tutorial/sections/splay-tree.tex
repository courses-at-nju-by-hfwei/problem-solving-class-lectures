% file: sections/splay-tree.tex

%%%%%%%%%%%%%%%
\begin{frame}{}
  \begin{quote}
    \centering
    \purple{\Large What work are you proudest of?} \\[12pt]

    \begin{columns}
      \column{0.50\textwidth}
	\fignocaption{width = 0.50\textwidth}{figs/qrcode-tarjan-interview}
      \column{0.50\textwidth}
	\uncover<3->{\fignocaption{width = 0.50\textwidth}{figs/self-adjusting-tree}}
    \end{columns}

    \pause
    \vspace{0.60cm}
    \brown{\large Proudest? It’s hard to choose. \\[5pt] \pause
      I like the \red{\Large self-adjusting search tree} data structure \\[4pt]
      that Daniel Sleator and I developed. \\[5pt] % \pause
      % That’s a nice one.
    }
  \end{quote}
\end{frame}
%%%%%%%%%%%%%%%

%%%%%%%%%%%%%%%
\begin{frame}{}
  \begin{columns}
    \column{0.50\textwidth}
      \fignocaption{width = 0.80\textwidth}{figs/daming.png}
    \column{0.50\textwidth}
      \fignocaption{width = 0.98\textwidth}{figs/self-adjusting-bst-paper}
  \end{columns}

  \vspace{0.80cm}
  \begin{quote}
    \centering
    \teal{``Self-Adjusting Binary Search Trees -- \red{\it Splay Tree}'', JACM, $1985$}
  \end{quote}
\end{frame}
%%%%%%%%%%%%%%%

%%%%%%%%%%%%%%%
\begin{frame}{}
  \begin{center}
    \[
      \red{\textsc{Splay}(x, T):}
    \]

    Moving node $x$ to the \red{\large \it root} of the tree $T$ by $\cdots$
  \end{center}

  \pause
  \vspace{-0.50cm}
  \begin{align*}
    \textsc{Search}(x, T) & \quad \purple{\text{\footnotesize RETURN }} x^{\ast}/\Lambda \\
    \textsc{Insert}(x, T) & \quad \purple{\text{\footnotesize ASSUME }} x \notin T \\
    \textsc{Delete}(x, T) & \quad \purple{\text{\footnotesize ASSUME }} x \in T \\
  \end{align*}

  \pause
  \vspace{-0.80cm}
  \begin{align*}
    T \gets \textsc{Join}(T_1, T_2) & \quad \purple{\text{\footnotesize ASSUME }} x \in T_1 < y \in T_2 \\
    (T_1, T_2) \gets \textsc{Split}(x, T) & \quad \purple{\text{\footnotesize RETURN }} x \in T_1 \le i \land y \in T_2 > i
  \end{align*}
\end{frame}
%%%%%%%%%%%%%%%

%%%%%%%%%%%%%%%
\begin{frame}{}
  \begin{center}
    \[
      \red{\textsc{Splay}(x, T):}
    \]

    Moving node $x$ to the \red{\large \it root} of the tree $T$ by performing a sequence of \purple{\large \it rotations}
    along the path from $x$ to the root.
  \end{center}

  \pause
  \fignocaption{width = 0.75\textwidth}{figs/rotation}
\end{frame}
%%%%%%%%%%%%%%%

%%%%%%%%%%%%%%%
\begin{frame}{}
  \begin{center}
    A chain of length $n$ \\[8pt]
    A sequence of $n$ \splay{}
  \end{center}

  \begin{columns}
    \column{0.50\textwidth}
      \fignocaption{width = 0.60\textwidth}{figs/splay-tree-chain}
    \column{0.50\textwidth}
      \pause
      \[
	\sum_{i = 1}^{n} c_i = \Theta(n^2)
      \]

      \[
	\bar{c_i} = \Theta(n)
      \]
  \end{columns}
\end{frame}
%%%%%%%%%%%%%%%

%%%%%%%%%%%%%%%
\begin{frame}{}
  \fignocaption{width = 0.40\textwidth}{figs/splay-tree-case-zero}

  \vspace{0.30cm}
  \centerline{\textsc{Case $0$}: $x$ is the root}
\end{frame}
%%%%%%%%%%%%%%%

%%%%%%%%%%%%%%%
\begin{frame}{}
  \fignocaption{width = 0.80\textwidth}{figs/splay-tree-case-zig}

  \vspace{0.30cm}
  \centerline{\textsc{Case $1$}: zig (zag)}
  \[
    y = p(x) \text{ is the root}
  \]
\end{frame}
%%%%%%%%%%%%%%%

%%%%%%%%%%%%%%%
\begin{frame}{}
  \fignocaption{width = 0.85\textwidth}{figs/splay-tree-case-zigzig}

  \vspace{0.30cm}
  \centerline{\textsc{Case $2$}: zig-zig (zag-zag)}
  \[
    y = p(x) \qquad z = p(y)
  \]
  \[
    x = lc(y) \qquad y = lc(z)
  \]

  \pause
  \[
    \red{(1): y - z \qquad (2): x - y}
  \]
\end{frame}
%%%%%%%%%%%%%%%

%%%%%%%%%%%%%%%
\begin{frame}{}
  \fignocaption{width = 0.80\textwidth}{figs/splay-tree-case-zigzag}

  \vspace{0.30cm}
  \centerline{\textsc{Case $3$}: zig-zag (zag-zig)}
  \[
    y = p(x) \qquad z = p(y)
  \]
  \[
    x = rc(y) \qquad y = lc(z)
  \]

  \pause
  \[
    \red{(1): x - y \qquad (2): x - z}
  \]
\end{frame}
%%%%%%%%%%%%%%%

%%%%%%%%%%%%%%%
\begin{frame}{}
  \begin{columns}
    \column{0.20\textwidth}
    \column{0.60\textwidth}
      % file: algs/splay.tex

\begin{algorithm}[H]
  % \caption{Splay at node $x$ of tree $T$.}
  % \label{alg:splay}
  \begin{algorithmic}[1]
    \Procedure{Splay}{$x$, $T$}
    \While{$x \neq T.root$}	\Comment{\textbf{Case}\;{$0$}}
	\Switch{$\cdots$}
	  \Case{$1:$ zig}
	    \State $\cdots$ 
	    \State \textcolor{red}{\Return}
	  \EndCase
	  \Case{$2:$ zig-zig}
	    \State $\cdots$ 
	  \EndCase
	  \Case{$3:$ zig-zag}
	    \State $\cdots$ 
	  \EndCase
	\EndSwitch
      \EndWhile
    \EndProcedure
  \end{algorithmic}
\end{algorithm}

    \column{0.20\textwidth}
  \end{columns}
\end{frame}
%%%%%%%%%%%%%%%

%%%%%%%%%%%%%%%
\begin{frame}{}
  \fignocaption{width = 0.95\textwidth}{figs/splay-at-1}{\centerline{$\textsc{Splay}(1)$}}

  \pause
  \fignocaption{width = 0.50\textwidth}{figs/splay-at-2}{\centerline{$\textsc{Splay}(2)$}}
\end{frame}
%%%%%%%%%%%%%%%

%%%%%%%%%%%%%%%
\begin{frame}{}
  \begin{center}
    \purple{\Large Amortized Analysis of \textsc{Splay}} \\[30pt] \pause

    A splay tree $T$ of \red{$n$}-node \\[6pt]
    An arbitrary sequence of \red{$m$} \splay
  \end{center}

  \pause
  \[
    \# \text{ of rotations}
  \]

  \pause
  \vspace{0.50cm}
  \begin{theorem}
    \[
      \hat{c}_{\splay} = O(\log n).
    \]
  \end{theorem}
\end{frame}
%%%%%%%%%%%%%%%

%%%%%%%%%%%%%%%
% file: sections/potential.tex

%%%%%%%%%%%%%%%
\begin{frame}{}
  \centerline{\teal{\Large The Potential Method}}

  \vspace{0.50cm}
  \fignocaption{width = 0.55\textwidth}{figs/potential}
\end{frame}
%%%%%%%%%%%%%%%

%%%%%%%%%%%%%%%
\begin{frame}{}
  \[
    \red{D_0},\; o_1,\; \red{D_1},\; o_2,\; \cdots,\; \underbrace{\red{D_{i-1}},\; o_{i},\; \red{D_{i}}}_{\text{\purple{the $i$-th operation}}},\; 
    \cdots,\; \red{D_{n-1}},\; o_{n},\; \red{D_{n}}
  \]

  \pause
  \vspace{0.50cm}
  \[
    \Phi: \Big\{D_{i} \mid 0 \le i \le n\Big\} \to \mathcal{R}
  \]

  \pause
  \vspace{0.30cm}
  \[
    \boxed{\red{\hat{c_i} = c_i + \Big(\Phi(D_{i}) - \Phi(D_{i-1})\Big)}}
  \]

  \pause
  \vspace{0.50cm}
  \[
    \sum_{1 \le i \le n} c_i = \left( \sum_{1 \le i \le n} \hat{c_i} \right) 
      + \Big(\underbrace{\Phi(D_{0}) - \Phi(D_{n})}_{\text{\purple{net decrease in potential}}} \Big)
  \]
\end{frame}
%%%%%%%%%%%%%%%

%%%%%%%%%%%%%%%
\begin{frame}{}
  \[
    \sum_{1 \le i \le n} c_i = \left( \sum_{1 \le i \le n} \hat{c_i} \right) 
      + \Big(\underbrace{\Phi(D_{0}) - \Phi(D_{n})}_{\text{\purple{net decrease in potential}}} \Big)
  \]

  \pause
  \vspace{0.30cm}
  \[
    \underbrace{\Phi(D_{0}) - \Phi(D_{n})}_{\text{\purple{net decrease in potential}}} \le \red{\Box}
    \implies
    \boxed{\sum_{1 \le i \le n} c_i \le \left( \sum_{1 \le i \le n} \hat{c_i} \right) + \red{\Box}}
  \]

  \pause
  \vspace{0.30cm}
  \[
    \red{\Box} = 0 \;\big(\red{\forall i,\; \Phi(D_i) \ge \Phi(D_0)}\big)
    \implies 
    \forall n,\; \sum_{1 \le i \le n} c_i \le \sum_{1 \le i \le n} \hat{c_i}
  \]

  % \pause
  % \[
  %   \forall m,\; \sum_{i=1}^{m} c_i \red{\;\le\;} \sum_{i=1}^{m} \hat{c_i} 
  %   \pause \quad\Longleftarrow\quad \boxed{\red{\forall i:\; \Phi(D_{i}) \ge \Phi(D_{0})}}
  % \]

  \pause
  \vspace{0.30cm}
  \[
    \Phi(D_{0}) = 0,\quad \forall 1 \le i \le n:\; \Phi(D_{i}) \ge 0 \pause \quad \text{\brown{\footnotesize (Typically)}}
  \]
\end{frame}
%%%%%%%%%%%%%%%

%%%%%%%%%%%%%%%
\begin{frame}{}
  \centerline{\teal{\large The Potential Method for Dynamic Tables}}

  \[
    \boxed{\red{\alpha = \frac{T.num}{T.size}}}
  \]

  \pause
  \vspace{0.30cm}
  \[
    \textsc{Expansion}: \left.
    \begin{cases}
      \text{When to expand?}		& \onslide<3->{\alpha = 1} \\
      \text{How large to expand to?} 	& \onslide<3->{\alpha = 1/2}
    \end{cases}
    \right.
  \]

  \onslide<4->{
    \vspace{0.30cm}
    \[
      \textsc{Contraction}: \left.
      \begin{cases}
	\text{When to contract?}	& \onslide<5->{\alpha = 1/4} \\
	\text{How small to contract to?} 	& \onslide<5->{\alpha = 1/2}
      \end{cases} \right.
    \]
  }

  \onslide<6->{
    \vspace{0.30cm}
    \[
      \boxed{\frac{1}{4} \le \alpha \le 1}
    \]
  }
\end{frame}
%%%%%%%%%%%%%%%

%%%%%%%%%%%%%%%
\begin{frame}{}
  \[
    \Phi(T) = \left.
    \begin{cases}
      2 \cdot T.num - T.size & \text{ if } \alpha(T) \ge 1/2	\\
      T.size/2 - T.num	     & \text{ if } \alpha(T) < 1/2
    \end{cases} \right.
  \]

  \pause
  \[
    \Phi(T_0) = 0,\quad \Phi(T_{i}) \ge 0
  \]

  \pause
  \vspace{0.30cm}
  \[
    \alpha = 1/2 \implies \Phi(T) = 0
  \]

  \vspace{0.30cm}
  \pause
  \[
    \alpha = 1/2 \leadsto \alpha = 1 \implies \Phi(T): 0 \leadsto T.num
  \]

  \pause
  \[
    \alpha = 1/2 \leadsto \alpha = 1/4 \implies \Phi(T): 0 \leadsto T.num
  \]
\end{frame}
%%%%%%%%%%%%%%%

%%%%%%%%%%%%%%%
\begin{frame}{}
  \[
    \Phi(T) = \left.
    \begin{cases}
      2 \cdot T.num - T.size & \text{ if } \alpha(T) \ge 1/2	\\
      T.size/2 - T.num	     & \text{ if } \alpha(T) < 1/2
    \end{cases} \right.
  \]
  \[
    \hat{c}_i = c_i + \Big(\Phi_{i} - \Phi_{i-1}\Big)
  \]

  \pause
  \vspace{0.20cm}
  \centerline{\large \it \red{By Case Analysis.}}

  \vspace{0.60cm}
  \begin{columns}
    \column{0.45\textwidth}
      \pause
      \centerline{\large \textsc{\teal{Table-Insert}}}
      \[
	\left.
	\begin{cases}
	  \alpha_{i-1} < 1/2 \left.
	  \begin{cases}
	    \alpha_{i} < 1/2 \\[6pt]
	    \alpha_{i} \ge 1/2
	  \end{cases} \right.
	  \\[20pt]
	  \alpha_{i-1} \ge 1/2 \left.
	    \begin{cases}
	      \alpha_{i-1} < 1 \\[6pt] 
	      \alpha_{i-1} = 1
	    \end{cases} \right.
	\end{cases} \right.
      \]
    \column{0.55\textwidth}
      \pause
      \centerline{\large \textsc{\teal{Table-Delete}}}
      \[
	\left.
	\begin{cases}
	  \alpha_{i-1} < 1/2 \left.
	    \begin{cases}
	      \red{\frac{num_{i-1} - 1}{size_{i-1}} \ge \frac{1}{4}} \\[6pt]
	      \frac{num_{i-1} - 1}{size_{i-1}} < \frac{1}{4}
	    \end{cases} \right.
	  \\[20pt]
	  \alpha_{i-1} \ge 1/2 \left.
	  \begin{cases}
	    \alpha_{i} < 1/2\; \onslide<5->{(\purple{\frac{num_{i-1} - 1}{size_{i-1}} < \frac{1}{4}?})}  \\[6pt]
	    \alpha_{i} \ge 1/2
	  \end{cases} \right.
	\end{cases} \right.
      \]
  \end{columns}
\end{frame}
%%%%%%%%%%%%%%%

%%%%%%%%%%%%%%%
\begin{frame}{}
  \centerline{\large \textsc{\teal{Table-Delete}}}

  \[
    \alpha_{i-1} < 1/2 \;\land\; \frac{num_{i-1} - 1}{size_{i-1}} \ge \frac{1}{4}
  \]

  \begin{align*}
    \hat{c}_i &= c_i + \Big(\Phi_{i} - \Phi_{i-1}\Big) \\
      \onslide<2->{
	&= 1 + (size_i/2 - num_i) - (size_{i-1}/2 - num_{i-1}) \\
      }
      \onslide<3->{
	&= 1 + (size_i/2 - num_i) - (size_i/2 - (num_i + 1)) \\
	&= \red{2}
      }
  \end{align*}

  \only<4->{
    \vspace{-0.30cm}
    \fignocaption{width = 0.30\textwidth}{figs/why}
  }
\end{frame}
%%%%%%%%%%%%%%%

%%%%%%%%%%%%%%%
\begin{frame}{}
  \centerline{\large \textsc{\teal{Table-Delete}}}

  \[
    \alpha_{i-1} \ge 1/2 \;\land\; \alpha_{i} \ge 1/2
  \]

  \begin{align*}
    \hat{c}_i &= c_i + \Big(\Phi_{i} - \Phi_{i-1}\Big) \\
      \onslide<2->{
	&= 1 + (2 \cdot num_i - size_i) - (2 \cdot num_{i-1} - size_{i-1}) \\
      }
      \onslide<3->{
	&= 1 + (2 \cdot num_i - size_i) - (2 \cdot (num_i + 1) -size_i) \\
	&= \red{-1}
      }
  \end{align*}

  \only<4->{
    \vspace{-0.30cm}
    \fignocaption{width = 0.20\textwidth}{figs/wtf-panda}
  }
\end{frame}
%%%%%%%%%%%%%%%

%%%%%%%%%%%%%%%
\begin{frame}{}
  \begin{columns}
    \column{0.50\textwidth}
      \centerline{\large \textsc{\teal{Table-Insert}}}

      \[
	\left.
	\begin{cases}
	  \alpha_{i-1} < 1/2 \left.
	  \begin{cases}
	    \alpha_{i} < 1/2 \; \red{(0)}\\[6pt]
	    \alpha_{i} \ge 1/2 \; \red{(3)}
	  \end{cases} \right.
	  \\[20pt]
	  \alpha_{i-1} \ge 1/2 \left.
	    \begin{cases}
	      \alpha_{i-1} < 1 \; \red{(3)} \\[6pt] 
	      \alpha_{i-1} = 1 \; \red{(3)}
	    \end{cases} \right.
	\end{cases} \right.
      \]
    \column{0.50\textwidth}
      \centerline{\large \textsc{\teal{Table-Delete}}}

      \[
	\left.
	\begin{cases}
	  \alpha_{i-1} < 1/2 \left.
	    \begin{cases}
	      \frac{num_{i-1} - 1}{size_{i-1}} \ge \frac{1}{4} \; \red{(1)} \\[6pt]
	      \frac{num_{i-1} - 1}{size_{i-1}} < \frac{1}{4} \; \red{(2)}
	    \end{cases} \right.
	  \\[20pt]
	  \alpha_{i-1} \ge 1/2 \left.
	  \begin{cases}
	    \alpha_{i} < 1/2 \; \red{(1/2)} \\[6pt]
	    \alpha_{i} \ge 1/2 \; \red{(-1)}
	  \end{cases} \right.
	\end{cases} \right.
      \]
  \end{columns}

  \only<2->{
    \vspace{0.50cm}
    \fignocaption{width = 0.40\textwidth}{figs/why-why-why}
  }
\end{frame}
%%%%%%%%%%%%%%%

%%%%%%%%%%%%%%%

%%%%%%%%%%%%%%%
\begin{frame}{}
  \[
    \Phi_{0}\;\; \splay_{1} \;\;
    \Phi_{1}\;\; \splay_{2} \;\;
    \Phi_{2}\;
    \cdots\;
    \underbrace{\Phi_{i-1}\;\; \splay_{i}\;\; \Phi_{i}\;}_{\purple{\text{the $i$-th \textsc{Splay}}}}
    \cdots\;
    \splay_{m}\;\; 
    \Phi_{m}
  \]

  \vspace{0.30cm}
  \[
    \hat{c}_{\splay_{i}} =  c_{\splay_{i}} + (\Phi_{\splay_{i}} - \Phi_{\splay_{i-1}})
  \]

  \pause
  \fignocaption{width = 0.25\textwidth}{figs/Phi}
\end{frame}
%%%%%%%%%%%%%%%

%%%%%%%%%%%%%%%
\begin{frame}{}
  \[
    s(x): \# \text{ of nodes in the subtree rooted at } x
  \]

  \pause
  \[
    r(x) = \log s(x)
  \]

  \pause
  \[
    \purple{\Phi = \sum_{x \in T} r(x)}
  \]

  \pause
  \vspace{0.50cm}
  \[
    \hat{c}_{\splay_{i}} =  c_{\splay_{i}} + (\Phi_{\splay_{i}} - \Phi_{\splay_{i-1}})
  \]
\end{frame}
%%%%%%%%%%%%%%%

%%%%%%%%%%%%%%%
\begin{frame}{}
  \[
    \Phi_{0}\;\; \splay_{1} \;\;
    \Phi_{1}\;\; \splay_{2} \;\;
    \Phi_{2}\;
    \cdots\;
    \underbrace{\Phi_{i-1}\;\; \splay_{i}\;\; \Phi_{i}\;}_{\purple{\text{the $i$-th \textsc{Splay}}}}
    \cdots\;
    \splay_{m}\;\; 
    \Phi_{m}
  \]

  \pause
  \begin{gather*}
    \underbrace{\Phi_{i-1}\;\; \splay_{i}\;\; \Phi_{i}}_{\purple{\text{the $i$-th \textsc{Splay}}}}: \\
    \Phi_{i-1} \;\triangleq\; \Phi_{0'}\; \iter_{1}\; \Phi_{1'}\; 
    \cdots\;
    \underbrace{\Phi_{k-1}\; \iter_{k}\; \Phi_{k}\;}_{\purple{\text{the $k$-th \textsc{Iteration}}}}
    \cdots\;
    \iter_{l}\; \Phi_{l} \;\triangleq\; \Phi_{i}\;
  \end{gather*}

  \pause
  \begin{align*}
    \hat{c}_{\splay_{i}} &= \sum_{1 \le j \le l} \hat{c}_{\iter_{j}} \\
      &= \sum_{1 \le j \le l} \Big(c_{\iter_{j}} + (\Phi_{\iter_{j}} - \Phi_{\iter_{j-1}})\Big)
  \end{align*}
\end{frame}
%%%%%%%%%%%%%%%

%%%%%%%%%%%%%%%
\begin{frame}{}
  \[
    \hat{c}_{\iter_{j}} =  c_{\iter_{j}} + (\Phi_{\iter_{j}} - \Phi_{\iter_{j-1}})
  \]

  \pause
  \vspace{0.60cm}
  \center{\large \it \red{By Case Analysis.}}

  \pause
  \[
    \hat{c}_{j} =  c_{j} + (\Phi_{j} - \Phi_{j-1})
  \]

  \vspace{0.30cm}
  \centerline{\blue{Remember:} \iter{}}
\end{frame}
%%%%%%%%%%%%%%%

%%%%%%%%%%%%%%%
\begin{frame}{}
  \fignocaption{width = 0.40\textwidth}{figs/splay-tree-case-zero}

  \centerline{\textsc{Case $0$}}

  \pause
  \begin{align*}
    \hat{c}_{j} &=  c_{j} + (\Phi_{j} - \Phi_{j-1}) \\
	\onslide<3->{
	  &= 0 + 0 \\
	  &= 0
	}
  \end{align*}
\end{frame}
%%%%%%%%%%%%%%%

%%%%%%%%%%%%%%%
\begin{frame}{}
  \fignocaption{width = 0.80\textwidth}{figs/splay-tree-case-zig}

  \centerline{\textsc{Case $1$}: zig}

  \begin{align*}
    \hat{c}_{j} &=  c_{j} + (\Phi_{j} - \Phi_{j-1}) \\
      \onslide<2->{
	&= 1 + r_{j}(x) + r_{j}(y) - r_{j-1}(x) - r_{j-1}(y) \\
      }
      \onslide<3->{
	&\le 1 + r_{j}(x) - r_{j-1}(x) \\
      }
      \onslide<4->{
	&\red{\;\le\;} 1 + 3\big(r_{j}(x) - r_{j-1}(x)\big)
      }
  \end{align*}
\end{frame}
%%%%%%%%%%%%%%%

%%%%%%%%%%%%%%%
\begin{frame}{}
  \fignocaption{width = 0.80\textwidth}{figs/splay-tree-case-zigzig}

  \centerline{\textsc{Case $2$}: zig-zig}

  \begin{align*}
    \hat{c}_{j} &=  c_{j} + (\Phi_{j} - \Phi_{j-1}) \\
      \onslide<2->{
	&= 2 + r_{j}(x) + r_{j}(y) + r_{j}(y)- r_{j-1}(x) - r_{j-1}(y) - r_{j-1}(z) \\
      }
      \onslide<3->{
	&= 2 + r_{j}(y) + r_{j}(y)- r_{j-1}(x) - r_{j-1}(y) \\
      }
      \onslide<4->{
	&\le 2 + r_{j}(x) + r_{j}(z) - 2r_{j-1}(x) \\
      }
      \onslide<5->{
	&\red{\;\le\;} 3\big(r_{j}(x) - r_{j-1}(x)\big)
      }
  \end{align*}
\end{frame}
%%%%%%%%%%%%%%%

%%%%%%%%%%%%%%%
\begin{frame}{}
  \begin{align*}
    r_{j-1}(x) + r_{j}(z) &= \log s_{j-1} (x) + \log s_i(z) \\
      \onslide<2->{&\red{\le}\; 2 \log \left(\frac{s_{j-1}(x) + s_{j}(z)}{2}\right) \\}
      \onslide<3->{&\le 2 \log \left(\frac{s_{j}(x)}{2}\right) \\}
    \onslide<4->{
      &= 2 \log s_{j}(x) - 2 \\
      &= 2 r_{j}(x) - 2
    }
  \end{align*}

  \uncover<5->{
    \[
      r_{j}(z) \le 2 r_{j}(x) - r_{j-1}(x) - 2
    \]
  }
\end{frame}
%%%%%%%%%%%%%%%

%%%%%%%%%%%%%%%
\begin{frame}{}
  \fignocaption{width = 0.80\textwidth}{figs/splay-tree-case-zigzag}

  \centerline{\textsc{Case $3$}: zig-zag}

  \begin{align*}
    \hat{c}_{j} &=  c_{j} + (\Phi_{j} - \Phi_{j-1}) \\
      \onslide<2->{
	&= 2 + r_{j}(x) + r_{j}(y) + r_{j}(y)- r_{j-1}(x) - r_{j-1}(y) - r_{j-1}(z) \\
      }
      \onslide<3->{
	&\le 2 + r_{j}(y) + r_{j}(z) - 2r_{j-1}(x) \\
      }
      \onslide<4->{
	&\red{\;\le\;} 3\big(r_{j}(x) - r_{j-1}(x)\big)
      }
  \end{align*}
\end{frame}
%%%%%%%%%%%%%%%

%%%%%%%%%%%%%%%
\begin{frame}{}
  \[
    \hat{c}_{\iter_{j}} \le \left.
    \begin{cases}
      0, & \text{\textsc{Case} $0$} \\
      1 + 3\big(r_{j}(x) - r_{j-1}(x)\big), & \text{\textsc{Case} $1$} \\
      3\big(r_{j}(x) - r_{j-1}(x)\big), & \text{\textsc{Case} $2$} \\
      3\big(r_{j}(x) - r_{j-1}(x)\big), & \text{\textsc{Case} $3$} \\
    \end{cases} \right.
  \]

  \pause
  \begin{align*}
    \hat{c}_{\splay_{i}} &= \sum_{1 \le j \le l} \hat{c}_{\iter_{j}} \\
      &= \sum_{1 \le j \le l} \Big(c_{\iter_{j}} + (\Phi_{\iter_{j}} - \Phi_{\iter_{j-1}})\Big) \\
    \onslide<3->{
      &\le 3\big(r_{\iter_{l}}(x) - r_{\iter_{0}}(x)\big) \red{\; + 1} \\
    }
    \onslide<4->{
      &= 3\big(\log n - r_{\iter_{0}}(x)\big) + 1 \\
    }
    \onslide<5->{
      &\le 3 \log n + 1 \\
      &= O(\log n)
    }
  \end{align*}
\end{frame}
%%%%%%%%%%%%%%%

%%%%%%%%%%%%%%%
\begin{frame}{}
  \begin{theorem}[\textsc{Balance Theorem}]
    \[
      \sum_{1 \le i \le m} c_{\splay_{i}} = O\Big((m + n) \log n\Big)
    \]
  \end{theorem}

  \vspace{0.50cm}
  \begin{proof}
    \begin{align*}
      \sum_{1 \le i \le m} c_{\splay_{i}} &= \left(\sum_{1 \le i \le m} \hat{c}_{\splay_{i}}\right) + 
      \big(\underbrace{\Phi_{\splay_{0}} - \Phi_{\splay_{m}}}_{\purple{\text{net decrease in potential}}}\big) \\
      \onslide<3->{
        &\le m \log n + \;\red{n \log n} \\
        &= (m + n) \log n
      }
    \end{align*}
  \end{proof}
\end{frame}
%%%%%%%%%%%%%%%

%%%%%%%%%%%%%%%
% \begin{frame}{}
%   \fignocaption{width = 0.80\textwidth}{figs/splay-tree-case-zigzig}
% 
%   \vspace{0.30cm}
%   \begin{center}
%     \teal{MTR (Move To Root) heuristic:} \\[6pt]
%     Keeping rotate the edge joining $x$ to its parent.
%   \end{center}
% 
%   \pause
%   \vspace{0.30cm}
%   \centerline{\large \red{Does this work?}}
% \end{frame}
%%%%%%%%%%%%%%%

%%%%%%%%%%%%%%%
\begin{frame}{}
  \begin{columns}
    \column{0.50\textwidth}
      \[
	\purple{\Phi = \sum_{x \in T} r(x)}
      \]

      \fignocaption{width = 0.80\textwidth}{figs/idea}
    \column{0.50\textwidth}
      \pause
      \fignocaption{width = 0.60\textwidth}{figs/qrcode-cstheory-splay-tree-potential}
  \end{columns}
\end{frame}
%%%%%%%%%%%%%%%

%%%%%%%%%%%%%%%
\begin{frame}{}
  \begin{columns}
    \column{0.40\textwidth}
      \begin{center}
	\[
	  \red{\textsc{Splay}(x)}
	\]

	\[
	  \textsc{Search}(x, t)
	\]
	\[
	  \textsc{Insert}(x, t)
	\]
	\[
	  \textsc{Delete}(x, t)
	\]

	\[
	  \textsc{Join}(t_1, t_2)
	\]
	\[
	  \textsc{Split}(x, t)
	\]
      \end{center}
    \column{0.60\textwidth}
      \pause
      \fignocaption{width = 0.75\textwidth}{figs/iceberg}
      \pause
      \fignocaption{width = 0.85\textwidth}{figs/self-adjusting-bst-paper}
  \end{columns}
\end{frame}
%%%%%%%%%%%%%%%
