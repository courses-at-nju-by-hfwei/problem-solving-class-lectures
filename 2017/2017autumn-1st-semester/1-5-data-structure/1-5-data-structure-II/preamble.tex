% File: preamble.tex
\usepackage{lmodern}

\usepackage{xeCJK}
\usetheme{CambridgeUS} % try Madrid, Pittsburgh
\usecolortheme{beaver}
\usefonttheme[onlymath]{serif} % try "professionalfonts"

\setbeamertemplate{itemize items}[default]
\setbeamertemplate{enumerate items}[default]

\usepackage{comment}
\usepackage{ulem}

\usepackage{amsmath, amsfonts, latexsym, mathtools, tabu}
\newcommand{\set}[1]{\{#1\}}
\usepackage{bm}
\DeclareMathOperator*{\argmin}{\arg\!\min}

\ExplSyntaxOn
\bool_new:N \l__xeCJK_listings_letter_bool
\ExplSyntaxOff
\usepackage{listings}

\definecolor{bgcolor}{rgb}{0.95,0.95,0.92}

\lstdefinestyle{CStyle}{
    language = C,
    basicstyle = \ttfamily\bfseries,
    backgroundcolor = \color{bgcolor},   
    keywordstyle = \color{blue},
    stringstyle = \color{teal},
    commentstyle = \color{cyan},
    breakatwhitespace = false,
    breaklines = true,                 
    mathescape = true,
    escapeinside = ||,
    morekeywords = {procedure, end, foreach, repeat, until},
    showspaces = false,                
    showstringspaces = false,
    showtabs = false,                  
}

% colors
\newcommand{\red}[1]{\textcolor{red}{#1}}
\newcommand{\redoverlay}[2]{\textcolor<#2>{red}{#1}}
\newcommand{\green}[1]{\textcolor{green}{#1}}
\newcommand{\greenoverlay}[2]{\textcolor<#2>{green}{#1}}
\newcommand{\blue}[1]{\textcolor{blue}{#1}}
\newcommand{\blueoverlay}[2]{\textcolor<#2>{blue}{#1}}
\newcommand{\purple}[1]{\textcolor{purple}{#1}}
\newcommand{\cyan}[1]{\textcolor{cyan}{#1}}

% colorded box
\newcommand{\rbox}[1]{\red{\boxed{#1}}}
\newcommand{\gbox}[1]{\green{\boxed{#1}}}
\newcommand{\bbox}[1]{\blue{\boxed{#1}}}
\newcommand{\pbox}[1]{\purple{\boxed{#1}}}

\usepackage{pifont}
\usepackage{wasysym}

\newcommand{\cmark}{\green{\ding{51}}}
\newcommand{\xmark}{\red{\ding{55}}}
%%%%%%%%%%%%%%%%%%%%%%%%%%%%%%%%%%%%%%%%%%%%%%%%%%%%%%%%%%%%%%
% for fig without caption: #1: width/size; #2: fig file
\newcommand{\fignocaption}[2]{
  \begin{figure}[htp]
    \centering
      \includegraphics[#1]{#2}
  \end{figure}
}

\usepackage{tikz}

\newcommand{\push}{\texttt{Push}}
\newcommand{\pop}{\texttt{Pop}}

\newcommand{\titletext}{1-5 数据与数据结构 (II)}

\newcommand{\thankyou}{
  \begin{frame}[noframenumbering]{}
    \fignocaption{width = 0.50\textwidth}{figs/thankyou.png}
  \end{frame}
}