%%%%%%%%%%%%%%%
\begin{frame}{}
  \centerline{\LARGE Simulations}
\end{frame}
%%%%%%%%%%%%%%%

%%%%%%%%%%%%%%%
\begin{frame}[fragile]{}
  \begin{exampleblock}{DH 2.5: Simulations}
    Show how to perform the following simulations of some control constructs by others.
    \begin{enumerate}[(a)]
      \item ``\texttt{for-do}'' by ``\texttt{while-do}''
    \end{enumerate}
  \end{exampleblock}

  \begin{columns}
    \column{0.45\textwidth}
      \begin{lstlisting}[style = Cstyle, backgroundcolor = \color{teal!10!lightgray}]
  for (init; cond; inc)
    statement
      \end{lstlisting}
    \column{0.45\textwidth}
    \pause
      \begin{lstlisting}[style = Cstyle]
  init;
  while (cond)
    statement
    inc
      \end{lstlisting}
  \end{columns}

  \vspace{0.50cm}
  \pause
  \begin{quote}
    Whether to use ``while'' or ``for'' is largely a matter of personal preference.
     
    \hfill --- K\&R C Bible
  \end{quote}
\end{frame}
%%%%%%%%%%%%%%%

%%%%%%%%%%%%%%%
\begin{frame}[fragile]{}
  \begin{exampleblock}{DH 2.5: Simulations}
    Show how to perform the following simulations of some control constructs by others.
    \begin{enumerate}[(a)]
      \setcounter{enumi}{(1)}
      \item ``\texttt{if-then} \& \texttt{if-then-else}'' by ``\texttt{while-do}''
    \end{enumerate}
  \end{exampleblock}

  \begin{columns}
    \column{0.45\textwidth}
      \pause
      \begin{lstlisting}[style = Cstyle, backgroundcolor = \color{teal!10!lightgray}]
  if (A)
    B
      \end{lstlisting}

      \pause
      \begin{lstlisting}[style = Cstyle]
  flag = 1
  while (A && flag)
    B
    flag = 0
      \end{lstlisting}
    \column{0.45\textwidth}
      \pause
      \begin{lstlisting}[style = Cstyle, backgroundcolor = \color{teal!10!lightgray}]
  if (A)
    B
  else
    C
      \end{lstlisting}

      \pause
      \begin{lstlisting}[style = Cstyle]
  flag = 1
  while (A && flag)
    B
    flag = 0
  while ($\lnot$ A && flag)
    B
    flag = 0
      \end{lstlisting}
  \end{columns}
\end{frame}
%%%%%%%%%%%%%%%

%%%%%%%%%%%%%%%
\begin{frame}[fragile]{}
  \begin{exampleblock}{DH 2.5: Simulations}
    Simulate the following control constructs by others.
    \begin{enumerate}[(a)]
      \setcounter{enumi}{(2)}
      \item ``\texttt{while-do}'' by ``\texttt{if-then} \& \texttt{goto}''
      \item ``\texttt{while-do}'' by ``\texttt{repeat-until} \& \texttt{if-then}''
    \end{enumerate}
  \end{exampleblock}

  \begin{lstlisting}[style = Cstyle, backgroundcolor = \color{teal!10!lightgray}]
                      while (A)
                        B
  \end{lstlisting}

  \begin{columns}
    \column{0.45\textwidth}
      \pause
      \begin{lstlisting}[style = Cstyle]
  L: if (A)
       B
       goto loop
      \end{lstlisting}
    \column{0.45\textwidth}
      \pause
      \begin{lstlisting}[style = Cstyle]
  if (A)
    repeat
      B
    until ($\lnot$ A)
      \end{lstlisting}
  \end{columns}
\end{frame}
%%%%%%%%%%%%%%%

%%%%%%%%%%%%%%%
\begin{frame}[fragile]{}
  \begin{exampleblock}{DH 2.8: Simulations}
    Simulate ``\texttt{while-do}'' by ``\texttt{if-then-else} \& \texttt{recursive}''.
  \end{exampleblock}

  \begin{columns}
    \column{0.45\textwidth}
      \begin{lstlisting}[style = Cstyle, backgroundcolor = \color{teal!10!lightgray}]
  while (A)
    B
      \end{lstlisting}
    \column{0.45\textwidth}
      \pause
      \begin{lstlisting}[style = Cstyle]
  simulateWhile() {
    if (A)
      B
      simulateWhile();

    return;
  }
      \end{lstlisting}
  \end{columns}
\end{frame}
%%%%%%%%%%%%%%%

%%%%%%%%%%%%%%%
\begin{frame}[fragile]{}
  \begin{columns}
    \column{0.40\textwidth}
      \begin{enumerate}[(1)]
	\item \texttt{A;B}
	\item \texttt{if-then} 
	\item \texttt{if-then-else} 
	\item \texttt{for-do} 
	\item \red{\texttt{while-do}} 
	\item \blue{\texttt{repeat-until}}
      \end{enumerate}
    \column{0.55\textwidth}
      \pause
      \begin{lstlisting}[style = Cstyle, backgroundcolor = \color{teal!10!lightgray}]
  repeat
    B
  until ($\lnot$ A)
      \end{lstlisting}
      \pause
      \begin{lstlisting}[style = Cstyle]
  B
  while (A)
    B
      \end{lstlisting}
  \end{columns}

  \vspace{0.60cm}
  \pause
  \begin{theorem}[``On Folk Theorems'' (David Harel, 1980)]
     Any \blue{computable function} can be computed by a 
    ``\red{\textsf{while-do}}'' (and ``\red{\textsf{;}}'') program
    (with additional Boolean variables).
  \end{theorem}
\end{frame}
%%%%%%%%%%%%%%%

%%%%%%%%%%%%%%%
\begin{frame}{}
  \fignocaption{width = 0.40\textwidth}{figs/minimalism}

  \begin{columns}
    \column{0.25\textwidth}
      \fignocaption{width = 0.80\textwidth}{figs/turing-machine}
    \column{0.25\textwidth}
      \fignocaption{width = 0.50\textwidth}{figs/lambda}
    \column{0.25\textwidth}
      \fignocaption{width = 0.80\textwidth}{figs/mu}
    \column{0.25\textwidth}
      \fignocaption{width = 0.80\textwidth}{figs/abacus}
  \end{columns}

  Simulations
\end{frame}
%%%%%%%%%%%%%%%

%%%%%%%%%%%%%%%
% \begin{frame}{}
%   \fignocaption{width = 0.40\textwidth}{figs/minimalism}
% 
%   \begin{columns}
%     \column{0.30\textwidth}
%       \fignocaption{width = 0.80\textwidth}{figs/turing-machine}{\centerline{Truing Machine}}
%     \column{0.30\textwidth}
%       
%     \column{0.30\textwidth}
%   \end{columns}
% \end{frame}
%%%%%%%%%%%%%%%

%%%%%%%%%%%%%%%
\begin{frame}{}
  Bounded iteration vs. Unbounded iteration
\end{frame}
%%%%%%%%%%%%%%%

%%%%%%%%%%%%%%%
\begin{frame}[fragile]{}
  \begin{exampleblock}{DH 2.4: Bounded Iteration}
    Given a list $L$ of $N$ integers,
    to produce in $S$ and $P$ the sum of the even numbers in $L$ and the product of the odd ones, respectively.
  \end{exampleblock}

  \pause
  \begin{lstlisting}[style = Cstyle]
    int S = 0, P = 1;
    for (int i = 0; i < N; ++i) {
      if (L(i) % 2 == 0)
        S += L(i);
      else
        P *= L(i);
    }
  \end{lstlisting}

  \begin{columns}
    \pause
    \column{0.40\textwidth}
      \begin{exampleblock}{DH 2.1: Salary Summation}
	$N-1$ vs. $N$ iterations
      \end{exampleblock}
    \pause
    \column{0.55\textwidth}
      \fignocaption{width = 0.30\textwidth}{figs/forget-it}
  \end{columns}
\end{frame}
%%%%%%%%%%%%%%%

%%%%%%%%%%%%%%%
\begin{frame}[fragile]{}
  \begin{exampleblock}{DH 2.7: Compute $n!$}
    Write algorithms that compute $n!$, given a non-negative integer $n$.
    \begin{enumerate}[(a)]
      \item Using iteration statements.
      \item Using recursion.
    \end{enumerate}
  \end{exampleblock}

  \pause
  \begin{lstlisting}[style = Cstyle]
  int P = 1;
  for (int i = 1; i <= n; ++i) {
    P *= i;
  }
  \end{lstlisting}

  \pause
  \begin{lstlisting}[style = Cstyle]
  int recursive-factorial(int n) {
    if (n == 1)
      return 1;
    else n * recursive-factorial(n-1);
  }
  \end{lstlisting}
\end{frame}
%%%%%%%%%%%%%%%
