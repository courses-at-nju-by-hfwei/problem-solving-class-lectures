%%%%%%%%%%%%%%%
\begin{frame}{}
  \centerline{\LARGE Simulations}
\end{frame}
%%%%%%%%%%%%%%%

%%%%%%%%%%%%%%%
\begin{frame}[fragile]{}
  \begin{exampleblock}{DH 2.5: Simulations}
    Perform the following simulations of some control constructs by others.
    \begin{enumerate}[(a)]
      \item ``\texttt{for-do}'' by ``\texttt{while-do}''
    \end{enumerate}
  \end{exampleblock}

  \vspace{0.40cm}
  \pause
  \begin{lstlisting}[style = Cstyle, backgroundcolor = \color{teal!10!lightgray}]
  for (int i = 0; i < N; ++i) |\uncover<3->{\red{// not general!}}|
    statement
  \end{lstlisting}

  \vspace{0.20cm}
  \begin{lstlisting}[style = Cstyle]
  int i = 0;
  while (i < N)
    statement
    ++i
  \end{lstlisting}
\end{frame}
%%%%%%%%%%%%%%%
\begin{frame}[fragile]{}
  \begin{exampleblock}{DH 2.5: Simulations}
    Perform the following simulations of some control constructs by others.
    \begin{enumerate}[(a)]
      \item ``\texttt{for-do}'' by ``\texttt{while-do}''
    \end{enumerate}
  \end{exampleblock}

  \begin{columns}
    \column{0.45\textwidth}
      \begin{lstlisting}[style = Cstyle, backgroundcolor = \color{teal!10!lightgray}]
  for (init; cond; inc)
    statement
      \end{lstlisting}
    \column{0.45\textwidth}
      \begin{lstlisting}[style = Cstyle]
  init;
  while (cond)
    statement
    inc
      \end{lstlisting}
  \end{columns}

  \vspace{0.50cm}
  \pause
  \begin{quote}
    Whether to use ``while'' or ``for'' is largely a matter of personal preference.
     
    \hfill --- K\&R C Bible
  \end{quote}
\end{frame}
%%%%%%%%%%%%%%%

%%%%%%%%%%%%%%%
\begin{frame}[fragile]{}
  \begin{exampleblock}{DH 2.5: Simulations}
    Perform the following simulations of some control constructs by others.
    \begin{enumerate}[(a)]
      \setcounter{enumi}{(1)}
      \item ``\texttt{if-then} \& \texttt{if-then-else}'' by ``\texttt{while-do}''
    \end{enumerate}
  \end{exampleblock}

  \begin{columns}
    \column{0.40\textwidth}
      \begin{lstlisting}[style = Cstyle, backgroundcolor = \color{teal!10!lightgray}]
  if (A)
    B
      \end{lstlisting}

    \column{0.55\textwidth}
      \pause
      \begin{lstlisting}[style = Cstyle]
  while (A)
    B
    $\lnot$ A |\pause \textcolor{cyan}{// \red{Wrong:} side effects?}|
      \end{lstlisting}

      \pause
      \begin{lstlisting}[style = Cstyle]
  flag = 1
  while (A && flag)
    B
    flag = 0
      \end{lstlisting}
  \end{columns}
\end{frame}
%%%%%%%%%%%%%%%

%%%%%%%%%%%%%%%
\begin{frame}[fragile]{}
  \begin{exampleblock}{DH 2.5: Simulations}
    Perform the following simulations of some control constructs by others.
    \begin{enumerate}[(a)]
      \setcounter{enumi}{(1)}
      \item ``\texttt{if-then} \& \texttt{if-then-else}'' by ``\texttt{while-do}''
    \end{enumerate}
  \end{exampleblock}

  \begin{columns}
    \column{0.55\textwidth}
      \begin{lstlisting}[style = Cstyle, backgroundcolor = \color{teal!10!lightgray}]
  if (A)
    B
  else
    C
      \end{lstlisting}

      \pause
      \begin{lstlisting}[style = Cstyle]
  flag_if = 1
  while (A && flag_if)
    B  |\uncover<3->{\textcolor{cyan}{// \red{Wrong:} side effects?}}|
    flag_if = 0
  flag_else = 1
  while ($\lnot$ A && flag_else)
    C
    flag_else = 0
      \end{lstlisting}

    \column{0.45\textwidth}
      \begin{onlyenv}<4->
      \begin{lstlisting}[style = Cstyle]
  flag = 1
  while (A && flag)
    B
    flag = 0
  |\uncover<5->{\textcolor{cyan}{// $\lnot A$ not necessary}}|
  while ($\lnot$ A && flag)
    C
    flag = 0
      \end{lstlisting}
      \end{onlyenv}
  \end{columns}
\end{frame}
%%%%%%%%%%%%%%%
\begin{frame}[fragile]{}
  \begin{exampleblock}{DH 2.5: Simulations}
    Perform the following simulations of some control constructs by others.
    \begin{enumerate}[(a)]
      \setcounter{enumi}{(2)}
      \item ``\texttt{while-do}'' by ``\texttt{if-then} \& \texttt{goto}''
      \item ``\texttt{while-do}'' by ``\texttt{repeat-until} \& \texttt{if-then}''
    \end{enumerate}
  \end{exampleblock}

  \begin{lstlisting}[style = Cstyle, backgroundcolor = \color{teal!10!lightgray}]
                      while (A)
                        B
  \end{lstlisting}

  \begin{columns}
    \column{0.40\textwidth}
      \pause
      \begin{lstlisting}[style = Cstyle]
  L: if (A)
       B
       goto L
      \end{lstlisting}
    \column{0.55\textwidth}
      \pause
      \begin{lstlisting}[style = Cstyle]
  if (A)  |\uncover<4->{\red{// no ''\texttt{if}''?}}|
    repeat
      B
    until ($\lnot$ A)
      \end{lstlisting}
  \end{columns}
\end{frame}
%%%%%%%%%%%%%%%

%%%%%%%%%%%%%%%
\begin{frame}[fragile]{}
  \begin{exampleblock}{DH 2.8: Simulations}
    Simulate ``\texttt{while-do}'' by ``\texttt{if-then-else} \& \texttt{recursive}''.
  \end{exampleblock}

  \begin{columns}
    \column{0.30\textwidth}
      \begin{lstlisting}[style = Cstyle, backgroundcolor = \color{teal!10!lightgray}]
  while (A)
    B
      \end{lstlisting}
    \column{0.70\textwidth}
      \pause
      \begin{lstlisting}[style = Cstyle]
  simulateWhile() { |\uncover<3->{\textcolor{cyan}{// define function}}|
    if (A)
      B
      simulateWhile();

    return;
  }
      \end{lstlisting}
  \end{columns}

  \uncover<4->{\fignocaption{width = 0.30\textwidth}{figs/recursion-draw}}
\end{frame}
%%%%%%%%%%%%%%%

%%%%%%%%%%%%%%%
\begin{frame}[fragile]{}
  \begin{columns}
    \column{0.40\textwidth}
      \begin{enumerate}[(1)]
	\item \texttt{A;B}
	\item \texttt{if-then} 
	\item \texttt{if-then-else} 
	\item \texttt{for-do} 
	\item \red{\texttt{while-do}} 
	\item \blueoverlay{\texttt{repeat-until}}{2-}
      \end{enumerate}
    \column{0.55\textwidth}
      \pause
      \begin{lstlisting}[style = Cstyle, backgroundcolor = \color{teal!10!lightgray}]
  repeat
    B
  until ($\lnot$ A)
      \end{lstlisting}
      \pause
      \begin{lstlisting}[style = Cstyle]
  B
  while (A)
    B
      \end{lstlisting}
  \end{columns}

  \vspace{0.60cm}
  \pause
  \begin{theorem}[``On Folk Theorems'' (David Harel, 1980)]
     Any \blue{computable function} can be computed by a 
    ``\red{\textsf{while-do}}'' (and ``\red{\textsf{;}}'') program
    (with additional Boolean variables).
  \end{theorem}
\end{frame}
%%%%%%%%%%%%%%%

%%%%%%%%%%%%%%%
\begin{frame}{}
  \begin{columns}
    \column{0.45\textwidth}
      \fignocaption{width = 0.80\textwidth}{figs/minimalism}
    \column{0.45\textwidth}
    \uncover<3->{\centerline{\Large \textcolor{cyan}{Simulations} for Equivalence}}
  \end{columns}

  \pause
  \begin{columns}
    \column{0.30\textwidth}
      \fignocaption{width = 0.85\textwidth}{figs/turing-machine}
    \column{0.30\textwidth}
      \fignocaption{width = 0.50\textwidth}{figs/lambda}
    \column{0.30\textwidth}
      \fignocaption{width = 0.80\textwidth}{figs/mu}
    %\column{0.25\textwidth}
    %  \fignocaption{width = 0.80\textwidth}{figs/abacus}
  \end{columns}
\end{frame}
%%%%%%%%%%%%%%%
