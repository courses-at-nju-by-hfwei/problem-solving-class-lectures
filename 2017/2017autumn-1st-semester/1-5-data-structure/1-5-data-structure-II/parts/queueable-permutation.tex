%%%%%%%%%%%%%%%
\begin{frame}{}
  \centerline{\LARGE \textcolor{gray}{Stackable}/Queueable Permutations}
\end{frame}
%%%%%%%%%%%%%%%

%%%%%%%%%%%%%%%
\begin{frame}{}
  \begin{exampleblock}{DH 2.14: Queueable Permutations}
    \[
      \fbox{$\texttt{out} = (a_1, \cdots, a_n) \blue{\xleftarrow[X = 0]{Q = \emptyset}} \texttt{in} = (1, \cdots, n)$}
    \]

    \fignocaption{width = 0.80\textwidth}{figs/queue-perm-x}
  \end{exampleblock}
\end{frame}
%%%%%%%%%%%%%%%

%%%%%%%%%%%%%%%
\begin{frame}{}
  \begin{exampleblock}{DH 2.14: Queueable Permutations}
    \begin{enumerate}[(a)]
      \item Show that the permutations given in Excecise 2.12(b) are queueable.
	\begin{enumerate}[(i)]
	  \item \textcolor{gray}{$(3,1,2)$}
	  \item $(4,5,3,7,2,1,6)$
	\end{enumerate}
    \end{enumerate}

    \begin{columns}
      \column{0.45\textwidth}
	\fignocaption{width = 0.60\textwidth}{figs/impossible-cube}
      \column{0.45\textwidth}
	\fignocaption{width = 0.70\textwidth}{figs/possible}
    \end{columns}
  \end{exampleblock}
\end{frame}
%%%%%%%%%%%%%%%


%%%%%%%%%%%%%%%
\begin{frame}{}
  \begin{exampleblock}{DH 2.14: Queueable Permutations}
    \begin{enumerate}[(a)]
      \setcounter{enumi}{1}
      \item Prove that every permutation can be obtained by \blue{a queue}.
    \end{enumerate}
  \end{exampleblock}

  Alg here.
\end{frame}
%%%%%%%%%%%%%%%

%%%%%%%%%%%%%%%
\begin{frame}{}
  \begin{exampleblock}{DH 2.14: Queueable Permutations}
    \begin{enumerate}[(a)]
      \setcounter{enumi}{2}
      \item Prove that every permutation can be obtained by \blue{two stacks}.
    \end{enumerate}
  \end{exampleblock}
\end{frame}
%%%%%%%%%%%%%%%

%%%%%%%%%%%%%%%
\begin{frame}{}
  \begin{exampleblock}{DH 2.15: Algorithm for Queueable Permutations}
    Extend the algorithm you were asked to design in Exercise 2.13,
    so that \red{if} the given permutation cannot be obtained by \red{a stack},
    the algorithm will print the series of operations on \red{two stacks} that will generate it.
  \end{exampleblock}
\end{frame}
%%%%%%%%%%%%%%%

%%%%%%%%%%%%%%%
\begin{frame}{}
  \begin{columns}
    \column{0.60\textwidth}
      \fignocaption{width = 0.90\textwidth}{figs/queue-perm-x}
    \column{0.40\textwidth}
      \fignocaption{width = 0.45\textwidth}{figs/not-interesting}
  \end{columns}

  \vspace{0.80cm}
  \begin{columns}
    \column{0.60\textwidth}
      \fignocaption{width = 0.90\textwidth}{figs/queue-perm}
    \column{0.40\textwidth}
      \fignocaption{width = 0.60\textwidth}{figs/boring}
  \end{columns}
\end{frame}
%%%%%%%%%%%%%%%

%%%%%%%%%%%%%%%
\begin{frame}{}
  \begin{columns}
    \column{0.60\textwidth}
      \fignocaption{width = 0.90\textwidth}{figs/queue-perm-x-no-circle}

      \vspace{0.80cm}
      \fignocaption{width = 0.90\textwidth}{figs/queue-perm-direct}
    \column{0.40\textwidth}
      \fignocaption{width = 0.60\textwidth}{figs/yisi}
  \end{columns}
\end{frame}
%%%%%%%%%%%%%%%

%%%%%%%%%%%%%%%
\begin{frame}
  \begin{theorem}[Queueable Permutations]
    A permutation $(a_1, \cdots, a_n)$ is queueable $\iff$ it is not the case that
    \[
      321\text{\it -Pattern}: \fbox{$\texttt{\it out} = \cdots a_i \cdots a_j \cdots a_k: i < j < k \land a_i > a_j > a_k$}
    \]
  \end{theorem}

  \vspace{0.30cm}
  \pause
  \begin{proof}
    % \fignocaption{width = 0.40\textwidth}{figs/no-proof}
  \end{proof}
\end{proof}
%%%%%%%%%%%%%%%

%%%%%%%%%%%%%%%
%%%%%%%%%%%%%%%
%%%%%%%%%%%%%%%
%%%%%%%%%%%%%%%
