% well-ordered-isomorphic.tex

%%%%%%%%%%%%%%%
\begin{frame}{}
  \begin{exampleblock}{SM Problem $14.62$: Isomorphic Well-Ordered Sets}
    Suppose $A$ and $B$ are \red{well-ordered} isomorphic sets.
    Show that there is only one isomorphic mapping $f: A \to B$.
  \end{exampleblock}
  
  \pause
  \begin{center}
    Well-ordered $\implies$ Totally-ordered 
    \[
      (\mathbb{N}, <)
    \]
    \pause
    Totally-ordered $\centernot\implies$ Well-ordered
    \pause
    \[
      (\mathbb{Z}, <)
    \]
    \pause
    \red{$Q:$ What about ``totally-ordered'' isomorphic sets?}
  \end{center}
\end{frame}
%%%%%%%%%%%%%%%

%%%%%%%%%%%%%%%
\begin{frame}{}
  \begin{exampleblock}{SM Problem $14.62$: Isomorphic Well-Ordered Sets}
    Suppose $A$ and $B$ are \red{well-ordered} isomorphic sets.
    Show that there is only one isomorphic mapping $f: A \to B$.
  \end{exampleblock}
  
  \[
    f: A \to B
  \]
  \begin{center}
    Make use of the ``well-ordered'' property.
  \end{center}
  
  \pause
  \begin{gather*}
    a \gets \min A \qquad b \gets \min B \\[6pt]
    \onslide<3->{f(a) = b \\[6pt]}
    \onslide<4->{f\left(\min (A \setminus \set{a})\right) = \min (B \setminus \set{b})}
  \end{gather*}
  
  \onslide<5->{
    \[
      \red{\boxed{f(x) = \min \Big(B \setminus f\big(\blue{\set{a \in A: a < x}}\big)\Big)}}
    \]
  }
\end{frame}
%%%%%%%%%%%%%%%

%%%%%%%%%%%%%%%
\begin{frame}{}
  \[
    \red{\boxed{f(x) = \min \Big(B \setminus f\big(\blue{\set{a \in A: a < x}}\big)\Big)}}
  \]
  
  \pause
  \[
    f: A \xrightarrow[onto]{1-1} B
  \]
  
  \pause
  \vspace{-0.50cm}
  \fig{width = 0.40\textwidth}{figs/your-turn}
  \vspace{-0.80cm}
  
  \pause
  \[
    f \text{ is unique}
  \]
  \pause
  \vspace{-0.80cm}
  \begin{center}
    For any isomorphic mapping $g: f \to B$, we show that $g = f$.
  \end{center}
\end{frame}
%%%%%%%%%%%%%%%

%%%%%%%%%%%%%%%
\begin{frame}{}
  \[
    \red{\boxed{f(x) = \min \Big(B \setminus f\big(\blue{\set{a \in A: a < x}}\big)\Big)}}
  \]
  
  \pause
  \fig{width = 0.50\textwidth}{figs/ruler-align}
  
  \pause
  \begin{theorem}[Mathematical Induction for Well-Ordered Sets]
    Let $\mathcal{S} = (S, <)$ be a well-ordered set.
    If $P(x)$ is a predicate such that
    \begin{enumerate}
      \item $P(\min S)$ holds,
      \item $\big(\forall y < x: P(y)\big) \implies P(x)$,
    \end{enumerate}
    then $\forall x \in S: P(x)$.
  \end{theorem}
  
  % \pause
  % \begin{center}
  %   $\mathcal{S}$ is more general than those in ordinary mathematical induction.
  % \end{center}
\end{frame}
%%%%%%%%%%%%%%%

%%%%%%%%%%%%%%%
\begin{frame}{}
  \[
    \red{\boxed{f(x) = \min \Big(B \setminus f\big(\blue{\set{a \in A: a < x}}\big)\Big)}}
  \]
  
  \vspace{0.50cm}
  \begin{center}
    We need to prove $\forall x \in A: g(x) = f(x)$. \\[15pt]
    By induction on the structure of $A$.
  \end{center}
\end{frame}
%%%%%%%%%%%%%%%

%%%%%%%%%%%%%%%
\begin{frame}{}
  \[
    \red{\boxed{f(x) = \min \Big(B \setminus f\big(\blue{\set{a \in A: a < x}}\big)\Big)}}
  \]
  
  \begin{center}
    \teal{Base Case: Consider $a \gets \min A$.} \\[8pt] \pause
    
    We need to show that $g(a) = f(a) = b.$ \\[20pt] \pause
    
    Suppose \red{by contradiction} that $g(a) = b_1 \neq b$.
    \pause
    \[
      \exists a_1 > a: g(a_1) = b \pause \;\red{<}\; b_1 = g(a)
    \]
  \end{center}
\end{frame}
%%%%%%%%%%%%%%%

%%%%%%%%%%%%%%%
\begin{frame}{}
  \[
    \red{\boxed{f(x) = \min \Big(B \setminus f\big(\blue{\set{a \in A: a < x}}\big)\Big)}}
  \]
  
  \begin{center}
    \teal{Induction Hypothesis:}
    $\teal{\forall y < x: g(y) = f(y)}$
    % \pause
    % \[
    %   \forall y < x: g(y) = \min \Big(B \setminus g\big(\set{a \in A: a < y}\big)\Big)
    % \]
    
    \pause
    \vspace{0.80cm}
    \teal{Induction Step: We need to show that $g(x) = f(x)$.} \\[8pt] \pause
    Suppose by contradiction that $g(x) \neq f(x)$. \pause
    \[
      f(x) = \min \Big(\cdot\Big) \triangleq M
    \]
    \pause
    \vspace{-0.30cm}
    \[
      g(x) > f(x) = M
    \]
    \pause
    \vspace{-0.30cm}
    \[
      \exists x_1 > x: g(x_1) = M \pause\; \red{= f(x) < g(x)}
    \]
  \end{center}
\end{frame}
%%%%%%%%%%%%%%%