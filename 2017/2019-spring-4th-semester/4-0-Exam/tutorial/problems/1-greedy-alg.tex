%%%%%%%%%%%%%%%
\begin{frame}{}
  \begin{exampleblock}{$1.$ ``算一算'' (Let us Calculate!)}	
    \begin{enumerate}[(1)]
      \item 某公司要从赵、钱、孙、李、吴5名员工中选派某些人出国考察。\\
	由于某些不可描述的原因, 选派要求如下:

	\begin{columns}
	  \column{0.42\textwidth}
	    \begin{enumerate}[(1)]
	      \item 若赵去, 钱也去;
	      \item 李、吴两人中必有一人去; 
	      \item 钱、孙两人中去且仅去一人; 
	      \item 孙、李两人同去或同不去; 
	      \item 若吴去, 则赵、钱也去;
	      \item 只有孙去, 赵才会去。
	    \end{enumerate}
	  \column{0.40\textwidth}
	    \uncover<3->{
	      \begin{enumerate}[(1)]
		\item $\teal{Z \to Q}$;
		\item $\teal{L \lor W}$; 
		\item $\teal{(Q \land \lnot S) \lor (S \land \lnot Q)}$; 
		\item $\teal{(S \land L) \lor (\lnot S \land \lnot L)}$; 
		\item $\teal{W \to Z \land Q}$;
		\item $\teal{Z \to S}$。
	      \end{enumerate}
	    }
      \end{columns}

      \vspace{0.20cm}
      请使用形式化推理的方法帮该公司判断应选哪些人出国考察。
    \end{enumerate}
  \end{exampleblock}

  \pause
  \vspace{0.30cm}
  \[
    \red{Z,\; Q,\; S,\; L,\; W} \emph{ vs. } \blue{P,\; Q,\; R,\; S,\; T\;}
  \]
\end{frame}
%%%%%%%%%%%%%%%

%%%%%%%%%%%%%%%
% \begin{frame}{}
%   \setlength{\jot}{1.5ex}
%   \begin{gather}
%     Z \to Q \\
%     L \lor W \\
%     (Q \land \lnot S) \lor (S \land \lnot Q) \\
%     (S \land L) \lor (\lnot S \land \lnot L) \\
%     W \to Z \land Q \\
%     Z \to S
%   \end{gather}
% 
%   \pause
%   \vspace{0.60cm}
%   \[
%     \blue{\lnot Z,\; \lnot Q,\; S,\; L,\; \lnot W}
%   \]
% \end{frame}
%%%%%%%%%%%%%%%

%%%%%%%%%%%%%%%
\begin{frame}{}
  \setlength{\jot}{1.5ex}
  \begin{align*}
    & (1) \land (2) \land (3) \land (4) \land (5) \land (6) \\
    =\; & \cdots \\
    =\; & \red{\text{\Large ONE PAGE HERE } \cdots } \\
    =\; & \lnot Z \land \lnot Q \land S \land L \land \lnot W
  \end{align*}

  \pause
  \fignocaption{width = 0.35\textwidth}{figs/you}
\end{frame}
%%%%%%%%%%%%%%%

%%%%%%%%%%%%%%%
\begin{frame}{}
  \begin{columns}
    \column{0.45\textwidth}
      \fignocaption{width = 0.60\textwidth}{figs/hushi-dadan}
    \column{0.45\textwidth}
      \pause
      \fignocaption{width = 0.85\textwidth}{figs/possibilities}
  \end{columns}
\end{frame}
%%%%%%%%%%%%%%%

\begin{frame}{}
  \begin{exampleblock}{$1.$ ``算一算'' (Let us Calculate!)}	
    \begin{enumerate}[(1)]
      \setcounter{enumi}{1}
      \item 给定如下``前提'', 请判断``结论''是否有效, 并说明理由。\\
	前提如下:

	\begin{enumerate}[(1)]
	  \setlength{\itemsep}{6pt}
	  \item 每个人或者喜欢美剧, 或者喜欢韩剧 (可以同时喜欢二者);

	    \only<3->{$\teal{\forall x: A(x) \lor K(x)}$}
	  \item 任何人如果他喜欢抗日神剧, 他就不喜欢美剧;

	    \only<3->{$\teal{\forall x: J(x) \to \lnot A(x)}$}
	  \item 有的人不喜欢韩剧。

	    \only<3->{$\teal{\exists x: \lnot K(x)}$}
	\end{enumerate}
	结论: 有的人不喜欢抗日神剧。 \only<3->{\qquad $\teal{\exists x: \lnot J(x)}$}
    \end{enumerate}
  \end{exampleblock}

  \uncover<2->{
    \vspace{0.40cm}
    \centerline{$x$: Human \qquad \only<4->{\red{$Q: H(x)?$}}}
    \[
      A(x),\quad K(x),\quad J(x)
    \]
  }
\end{frame}
%%%%%%%%%%%%%%%

%%%%%%%%%%%%%%%
% \begin{frame}{}
%   \centerline{$x$: Human}
% 
%   \[
%     A(x),\quad K(x),\quad J(x)
%   \]
% 
%   \pause
%   \setlength{\jot}{1.5ex}
%   \begin{gather}
%     \setcounter{equation}{0}
%     \forall x: A(x) \lor K(x) \\
%     \forall x: J(x) \to \lnot A(x) \\
%     \exists x: \lnot K(x) \\[3ex]
%     \teal{\exists x: \lnot J(x)}
%   \end{gather}
% 
%   \pause
%   \[
%     \red{Q: H(x)?}
%   \]
% \end{frame}
%%%%%%%%%%%%%%%

%%%%%%%%%%%%%%%
\begin{frame}{}
  \[
    A,\quad K,\quad J
  \]

  \begin{gather*}
    \setcounter{equation}{0}
    \forall x: A \lor K \\[1ex]
    \forall x: J \to \lnot A \\[1ex]
    \exists x: \lnot K \\[3ex]
    \red{\exists x: \lnot J}
  \end{gather*}
\end{frame}
%%%%%%%%%%%%%%%