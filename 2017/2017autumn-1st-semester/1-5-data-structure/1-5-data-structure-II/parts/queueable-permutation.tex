%%%%%%%%%%%%%%%
\begin{frame}{}
  \centerline{\LARGE Queueable Permutations}

  \vspace{0.30cm}
  \fignocaption{width = 0.35\textwidth}{figs/queue-child}
\end{frame}
%%%%%%%%%%%%%%%

%%%%%%%%%%%%%%%
\begin{frame}{}
  \begin{exampleblock}{DH 2.14: Queueable Permutations}
    \uncover<2->{
      \[
	\fbox{$\texttt{out} = (a_1, \cdots, a_n) \;\blue{\xleftarrow[X = 0]{Q = \emptyset}}\; \texttt{in} = (1, \cdots, n)$}
      \]
    }

    \fignocaption{width = 0.80\textwidth}{figs/queue-perm-x}
  \end{exampleblock}
\end{frame}
%%%%%%%%%%%%%%%

%%%%%%%%%%%%%%%
\begin{frame}{}
  \begin{exampleblock}{DH 2.14: Queueable Permutations}
    \begin{enumerate}[(a)]
      \item Show that the permutations given in Excecise 2.12(b) are queueable.
	\begin{enumerate}[(i)]
	  \item \textcolor{gray}{$(3,1,2)$}
	  \item $(4,5,3,7,2,1,6)$
	\end{enumerate}
    \end{enumerate}
  \end{exampleblock}

  \vspace{0.30cm}
  \begin{columns}
    \pause
    \column{0.45\textwidth}
      \fignocaption{width = 0.85\textwidth}{figs/mission-impossible-left}
    \pause
    \column{0.45\textwidth}
      \fignocaption{width = 0.75\textwidth}{figs/possible}
  \end{columns}
\end{frame}
%%%%%%%%%%%%%%%

%%%%%%%%%%%%%%%
\begin{frame}[fragile]{}
  \begin{exampleblock}{DH 2.14: Queueable Permutations}
    \begin{enumerate}[(a)]
      \setcounter{enumi}{1}
      \item Prove that every permutation are \blue{queueable}.
    \end{enumerate}
  \end{exampleblock}

  \pause
  \begin{lstlisting}[style = Cstyle]
              X = 0    Q = $\emptyset$    in != EOF
  \end{lstlisting}

  \begin{columns}
    \column{0.45\textwidth}
      \begin{lstlisting}[style = Cstyle]
foreach 'a' $\!\in\!$ out:
  if |\redoverlay{('a' == in)}{5-}|
    read(X)
    print(X)
  else if |\redoverlay{('a' > in)}{5-}|
    |\purple{add-Q-till}|('a')
  else // ('a' < in)
    |\purple{cycle-Q-till}|('a')
      \end{lstlisting}
    \pause
    \column{0.50\textwidth}
      \begin{lstlisting}[style = Cstyle]
|\purple{add-Q-till}|('a'):
  while (('x' $\!\!\in\!\!$ in) != 'a')
    add(X, Q)
  read(X)
  print(X)
      \end{lstlisting}
      \pause
      \begin{lstlisting}[style = Cstyle]
|\purple{cycle-Q-till}|('a'):
  while (('x' $\!\in\!$ $Q$) != 'a')
    remove(X, Q)
    add(X, Q)
  remove(X, Q)
  print(X)
      \end{lstlisting}
  \end{columns}
\end{frame}
%%%%%%%%%%%%%%%

%%%%%%%%%%%%%%%
\begin{frame}[fragile]{}
  \begin{exampleblock}{DH 2.14: Queueable Permutations}
    \begin{enumerate}[(a)]
      \setcounter{enumi}{1}
      \item Prove that every permutation are \blue{queueable}.
    \end{enumerate}
  \end{exampleblock}

  \begin{proof}
    \begin{columns}
      \column{0.45\textwidth}
	\begin{lstlisting}[style = Cstyle]
foreach 'a' $\!\in\!$ out:
  if ('a' >= in)
    |\purple{add-Q-till}|('a')
  else // ('a' < in)
    |\purple{cycle-Q-till}|('a')
        \end{lstlisting}
      \pause
      \column{0.45\textwidth}
	\begin{lstlisting}[style = Cstyle]
foreach 'a' $\!\in\!$ out:
  if ('a' $\in$ in)
    |\purple{add-Q-till}|('a')
  else // ('a' $\in$ Q)
    |\purple{cycle-Q-till}|('a')
      \end{lstlisting}
      \only<3->{\fignocaption{width = 0.65\textwidth}{figs/what-is-wrong}}
    \end{columns}
  \end{proof}
\end{frame}
%%%%%%%%%%%%%%%

%%%%%%%%%%%%%%%
\begin{frame}{}
  \centerline{\LARGE \red{Pseudo}\blue{code}}

  \vspace{0.30cm}
  \pause
  \fignocaption{width = 0.40\textwidth}{figs/simple-einstein}

  % \begin{columns}
  %   \pause
  %   \column{0.50\textwidth}
  %     \fignocaption{width = 0.80\textwidth}{figs/porn-seeit}
  %   \pause
  %   \column{0.50\textwidth}
  %     \fignocaption{width = 0.80\textwidth}{figs/simple-einstein}
  % \end{columns}

  \pause
  \vspace{1.00cm}
  \centerline{\Large \blue{``Executable''} at an \red{abstract} level.}
\end{frame}
%%%%%%%%%%%%%%%

%%%%%%%%%%%%%%%
\begin{frame}[fragile]{}
  \begin{exampleblock}{DH 2.14: Queueable Permutations}
    \begin{enumerate}[(a)]
      \setcounter{enumi}{1}
      \item Prove that every permutation are \blue{queueable}.
    \end{enumerate}
  \end{exampleblock}

  \begin{proof}[An ``AHA!'' Proof]
    \begin{columns}
      \column{0.45\textwidth}
	\begin{lstlisting}[style = Cstyle]
  foreach 'a' $\in$ in:
    add(X, Q)

  foreach 'a' $\in$ out:
    |\purple{cycle-Q-till}|('a')
        \end{lstlisting}
      \pause
      \column{0.45\textwidth}
	\fignocaption{width = 0.40\textwidth}{figs/aha}
    \end{columns}
  \end{proof}
\end{frame}
%%%%%%%%%%%%%%%

%%%%%%%%%%%%%%%
\begin{frame}{}
  \begin{exampleblock}{DH 2.14: Queueable Permutations}
    \begin{enumerate}[(a)]
      \setcounter{enumi}{2}
      \item Prove that every permutation can be obtained by \blue{two stacks}.
    \end{enumerate}
  \end{exampleblock}

  \pause
  \vspace{0.30cm}
  \begin{columns}[b]
    \column{0.50\textwidth}
      \fignocaption{width = 0.95\textwidth}{figs/stack-perm-x}
    \column{0.45\textwidth}
      \fignocaption{width = 0.95\textwidth}{figs/stack-perm}
  \end{columns}

  \pause
  \vspace{0.30cm}
  \begin{quote}
    We can similarly speak of a permutation obtained by \blue{two stacks},
    if we permit the \purple{\textsf{push}} and \purple{\textsf{pop}} operations 
    on two stacks $S$ and $S'$.
  \end{quote}
\end{frame}
%%%%%%%%%%%%%%%

%%%%%%%%%%%%%%%
\begin{frame}[fragile]{}
  \begin{exampleblock}{DH 2.14: Queueable Permutations}
    \begin{enumerate}[(a)]
      \setcounter{enumi}{2}
      \item Prove that every permutation can be obtained by \blue{two stacks}.
    \end{enumerate}
  \end{exampleblock}

  \vspace{0.30cm}
  \begin{columns}
    \column{0.45\textwidth}
      \fignocaption{width = 1.00\textwidth}{figs/two-stacks-perm-x}
    \pause
    \column{0.55\textwidth}
      \begin{lstlisting}[style = Cstyle]
foreach 'a' $\in$ in:
  read(X)
  push(X, S$'$)

foreach 'a' $\in$ out:
  if ('a' <= top(S$'$)) // $\in S'$
    |\purple{transfer-till}|(S$'$, S, 'a')
  else // $\in S$
    |\purple{transfer-till}|(S, S$'$, 'a')
      \end{lstlisting}
  \end{columns}
\end{frame}
%%%%%%%%%%%%%%%

%%%%%%%%%%%%%%%
\begin{frame}[fragile]{}
  \begin{exampleblock}{DH 2.15: Algorithm for Queueable Permutations}
    Extend the algorithm you were asked to design in Exercise 2.13,
    so that \red{\bf if} the given permutation \red{cannot} be obtained by \red{a stack},
    the algorithm will print the series of operations on \blue{two stacks} that will generate it.
  \end{exampleblock}

  \pause
  \fignocaption{width = 0.25\textwidth}{figs/what-is-this-bear}

  \pause
  \vspace{-0.20cm}
  \begin{lstlisting}[style = Cstyle]
        two-stackable-perm(in, X, S, S$'$)
  \end{lstlisting}

  \pause
  \begin{lstlisting}[style = Cstyle]
        if (! stackable-perm(in, X, S))
          two-stackable-perm(in, X, S, S$'$)
  \end{lstlisting}

  \pause
  \centerline{\large \red{Embedding ``\texttt{transfer}'' into ``\texttt{stackable-perm}''.}}
\end{frame}
%%%%%%%%%%%%%%%

%%%%%%%%%%%%%%%
\begin{frame}{}
  \begin{exampleblock}{DH 2.15: Algorithm for Queueable Permutations}
    Extend the algorithm you were asked to design in Exercise 2.13,
    so that \red{\bf if} the given permutation \red{cannot} be obtained by \red{a stack},
    the algorithm will print the series of operations on \blue{two stacks} that will generate it.
  \end{exampleblock}

  \fignocaption{width = 0.60\textwidth}{figs/two-stacks-perm-x}

  \pause
  \centerline{\purple{\texttt{transfer-till(S, S', top(S) == 'a')}}} 
  
  \pause
  \centerline{\purple{\texttt{transfer-till(S', S, S' == $\emptyset$)}}}
\end{frame}
%%%%%%%%%%%%%%%

%%%%%%%%%%%%%%%
\begin{frame}{}
  \fignocaption{width = 0.30\textwidth}{figs/how-to-solve-it-book}{\centerline{\Large Step 4: Looking Back!}}
\end{frame}
%%%%%%%%%%%%%%%

%%%%%%%%%%%%%%%
\begin{frame}{}
  \begin{columns}
    \column{0.60\textwidth}
      \fignocaption{width = 0.90\textwidth}{figs/queue-perm-x}
    \pause
    \column{0.40\textwidth}
      \fignocaption{width = 0.45\textwidth}{figs/not-interesting}
  \end{columns}

  \vspace{0.80cm}
  \begin{columns}
    \pause
    \column{0.60\textwidth}
      \fignocaption{width = 0.90\textwidth}{figs/queue-perm}
    \pause
    \column{0.40\textwidth}
      \fignocaption{width = 0.60\textwidth}{figs/boring}
  \end{columns}
\end{frame}
%%%%%%%%%%%%%%%

%%%%%%%%%%%%%%%
\begin{frame}{}
  \begin{columns}
    \column{0.60\textwidth}
      \fignocaption{width = 0.90\textwidth}{figs/queue-perm-x-no-circle}

      \pause
      \vspace{0.50cm}
      \fignocaption{width = 0.90\textwidth}{figs/queue-perm-direct}

      \pause
      \[
	\mbox{\Large \red{\sout{$3\; 2\; 1$}}}
      \]
    \column{0.40\textwidth}
      \pause
      \fignocaption{width = 0.60\textwidth}{figs/yisi}
  \end{columns}
\end{frame}
%%%%%%%%%%%%%%%

%%%%%%%%%%%%%%%
\begin{frame}
  \begin{theorem}[Queueable Permutations]
     A permutation $(a_1, \cdots, a_n)$ is \red{queueable} $\iff$ it is not the case that
     \[
       321\text{\it -Pattern}: \fbox{$\texttt{\it out} = \cdots a_i \cdots a_j \cdots a_k: i < j < k \land a_i > a_j > a_k$}
     \]
  \end{theorem}

  \pause
  \vspace{0.30cm}
  \begin{proof}
    \centerline{\Large Left as an exercise.}
  \end{proof}
\end{frame}
%%%%%%%%%%%%%%%

%%%%%%%%%%%%%%%
\begin{frame}{}
  \begin{theorem}[\# of Queueable Permutations]
    The number of queueable permutations of $[1 \cdots n]$ is ${2n \choose n} - {2n \choose n-1}$.
  \end{theorem}

  \vspace{0.20cm}
  \begin{columns}
    \pause
    \column{0.65\textwidth}
      \fignocaption{width = 0.80\textwidth}{figs/grid}
    \pause
    \column{0.30\textwidth}
      \fignocaption{width = 0.95\textwidth}{figs/sad}
  \end{columns}
\end{frame}
%%%%%%%%%%%%%%%

%%%%%%%%%%%%%%%
\begin{frame}{}
  \begin{theorem}[\# of Queueable Permutations]
    The number of queueable permutations of $[1 \cdots n]$ is ${2n \choose n} - {2n \choose n-1}$.
  \end{theorem}

  \vspace{0.30cm}
  \begin{proof}
    \begin{center}
      {\Large Left for your research.}
    \end{center}

    \pause
    \fignocaption{width = 0.20\textwidth}{figs/be-brave}
  \end{proof}
\end{frame}
%%%%%%%%%%%%%%%

%%%%%%%%%%%%%%%
% \begin{frame}{}
%   \begin{theorem}
%     321-avoiding $\iff$ two increasing sequences
%   \end{theorem}
% \end{frame}
%%%%%%%%%%%%%%%
