% file: sections/hamiltonian-graph.tex

%%%%%%%%%%%%%%%%%%%%
\begin{frame}{}
  \begin{theorem}[Necessary Condition; Theorem $6.5$]
    If $G$ is a Hamiltonian graph, then for each nonempty set $S \subset V(G)$,
    \[
      k(G - S) \le \big\lvert S \big\rvert.
    \]
  \end{theorem}
\end{frame}
%%%%%%%%%%%%%%%%%%%%

%%%%%%%%%%%%%%%%%%%%
\begin{frame}{}
  \begin{theorem}[Ore's Theorem, $1960$; Theorem $6.6$]
    Let $G$ be a graph of order $n \ge 3$. If
    \[
      \text{deg}(u) + \text{deg}(v) \ge n
    \]
    for each pair $u,v$ of nonadjacent vertices of $G$,
    then $G$ is Hamiltonian.
  \end{theorem}

  \pause
  \begin{proof}
    \begin{center}
      By \red{Extremality} and \purple{Contradiction}.
    \end{center}
  \end{proof}
\end{frame}
%%%%%%%%%%%%%%%%%%%%

%%%%%%%%%%%%%%%%%%%%
\begin{frame}{}
  \begin{theorem}[Dirac's Theorem, $1952$; Corollary $6.7$]
    Let $G$ be a graph of order $n \ge 3$. If
    \[
      \forall v \in V(G): \text{deg}(v) \ge n/2,
    \]
    then $G$ is Hamiltonian.
  \end{theorem}
\end{frame}
%%%%%%%%%%%%%%%%%%%%

%%%%%%%%%%%%%%%%%%%%
\begin{frame}{}
  \begin{theorem}[Ore's Theorem, $1960$; Theorem $6.8$]
    Let $u$ and $v$ be nonadjacent vertices in a graph $G$ of order $n$ such that
    \[
      \text{deg}(u) + \text{deg}(v) \ge n.
    \]
    Then $G + uv$ is Hamiltonian $\iff$ $G$ is Hamiltonian.
  \end{theorem}

  \pause
  \begin{definition}[Closure $C(G)$]
    The closure $C(G)$ of a graph $G$ is the graph obtained from $G$
    by iteratively adding edges joining pairs of nonadjacent vertices
    $u$ and $v$ such that $\text{deg}(u) + \text{deg}(v) \ge n$,
    until no such pair remains.
  \end{definition}
\end{frame}
%%%%%%%%%%%%%%%%%%%%

%%%%%%%%%%%%%%%%%%%%
\begin{frame}{}
  \begin{theorem}[Bondy-Chavatal Theorem, $1976$; Theorem $6.9$]
    $G$ is Hamiltonian \textcolor<2->{red}{$\iff$} $C(G)$ is Hamiltonian.
  \end{theorem}
  
  \uncover<3->{
    \begin{corollary}[Corollary $6.10$]
      If $G$ is a graph of order $n \ge 3$ such that $C(G) = K_{n}$,
      then $G$ is Hamiltonian.
    \end{corollary}
  }

  \uncover<4->{
    \begin{theorem}[Lajos Posa]
      Let $G$ be a graph of order $n \ge 3$.
      If for each integer $j$ with $1 \le j \le \frac{n}{2}$,
      the number of vertices of $G$ with degree at most $j$ is less than $j$,
      then $G$ is Hamiltonian.
    \end{theorem}
  }
\end{frame}
%%%%%%%%%%%%%%%%%%%%

%%%%%%%%%%%%%%%%%%%%
\begin{frame}{}
  \begin{theorem}[]
    \begin{center}
      $C(G)$ is well-defined.
    \end{center}
  \end{theorem}
\end{frame}
%%%%%%%%%%%%%%%%%%%%

%%%%%%%%%%%%%%%%%%%%
\begin{frame}{}
  \begin{exampleblock}{(Problem $6.20$)}
    Let $G$ be a graph of order $n \ge 3$ having the property that
    for each $v \in V(G)$, there is a Hamiltonian path with initial vertex $v$.
    Show that $G$ is $2$-connected but not necessarily Hamiltonian.
  \end{exampleblock}

  \pause
  \begin{center}
    $2$-connected: Connected + No cut-vertex
  \end{center}

  \pause
  \begin{columns}
    \column{0.50\textwidth}
    \column{0.50\textwidth}
  \end{columns}
\end{frame}
%%%%%%%%%%%%%%%%%%%%

%%%%%%%%%%%%%%%%%%%%
\begin{frame}{}
\end{frame}
%%%%%%%%%%%%%%%%%%%%
