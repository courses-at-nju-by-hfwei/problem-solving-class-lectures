% 1-logic.tex

%%%%%%%%%%%%%%%
\begin{frame}{}
  \begin{exampleblock}{Problem 1: Mathematical Logic}	
    请判断以下推理是否正确, 请给出证明或者指出错误之处。
    要求给出严格的\purple{符号化推理}过程。
    
    前提:
    \begin{enumerate}[(1)]
      \item 如果\red{一个人}害怕挑战,那么他就不会获得成功。
      \item \red{每个人}或者获得过成功,或者失败过。
      \item \blue{王二}未曾失败过。
    \end{enumerate}

    结论: 王二不害怕挑战。
  \end{exampleblock}

  \pause
  \[
    \blue{x \in \textsl{Human}} \pause \qquad \purple{w \in \textsl{Human}: \text{王二}}
  \]

  \pause
  \vspace{-0.80cm}
  \begin{align*}
    C(x): &\;x \text{ 害怕挑战} \\[3pt]
    W(x): &\;x \text{ 成功过} \\[3pt]
    L(x): &\;x \text{ 失败过}
  \end{align*}
\end{frame}
%%%%%%%%%%%%%%%

%%%%%%%%%%%%%%%
\begin{frame}{}
  \begin{enumerate}[(1)]
    \item 如果\red{一个人}害怕挑战,那么他就不会获得成功。
    \item \red{每个人}或者获得过成功,或者失败过。
    \item \blue{王二}未曾失败过。
  \end{enumerate}
    结论: 王二不害怕挑战。

  \begin{gather}
    \forall x: C(x) \implies \lnot W(x) \\[5pt]
    \forall x: W(x) \lor L(x) \\[5pt]
    \lnot L(w)
  \end{gather}
  
  \pause
  \hrule
  \[
    \lnot C(w)
  \]
\end{frame}
%%%%%%%%%%%%%%%

%%%%%%%%%%%%%%%
\begin{frame}{}
  \setcounter{equation}{0}
  \begin{gather}
    \forall x: C(x) \implies \lnot W(x) \\[5pt]
    \forall x: W(x) \lor L(x) \\[5pt]
    \lnot L(w)
  \end{gather}
  
  \hrule
  \[
    \lnot C(w)
  \]

  \pause
  \[
    \red{\forall x: C \implies W}
  \]

  \pause
  \[
    \red{C(x) \implies W(x)}
  \]
\end{frame}
%%%%%%%%%%%%%%%

%%%%%%%%%%%%%%%
\begin{frame}{}
  \setcounter{equation}{0}
  \begin{gather}
    \forall x: C(x) \implies \lnot W(x) \\[5pt]
    \forall x: W(x) \lor L(x) \\[5pt]
    \lnot L(w)
  \end{gather}
  
  \hrule
  \[
    \lnot C(w)
  \]

  \begin{columns}
    \column{0.50\textwidth}
      \pause
      \begin{gather}
	C(w) \implies \lnot W(w) \\[5pt]
	W(w) \lor L(w) \\[5pt]
	\lnot L(w)
      \end{gather}
    \column{0.50\textwidth}
      \pause
      \begin{gather}
	W(w) \quad \blue{(5) + (6)} \\[5pt]
	\red{\lnot C(w)} \quad \blue{(4) + (7)} 
      \end{gather}
  \end{columns}
\end{frame}
%%%%%%%%%%%%%%%

%%%%%%%%%%%%%%%
% \begin{frame}
% \end{frame}
%%%%%%%%%%%%%%%