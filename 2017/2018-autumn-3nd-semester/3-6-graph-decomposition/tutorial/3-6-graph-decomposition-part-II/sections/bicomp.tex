% file: sections/biconn.tex

%%%%%%%%%%%%%%%%%%%%
\begin{frame}{}
  \begin{definition}[Biconnected Graph]
    A connected undirected graph is \red{\it biconnected} if it contains no ``cut-nodes''.
  \end{definition}

  \pause
  \vspace{0.80cm}
  \begin{definition}[Biconnected Component \teal{(Bicomponent)}]
    A \red{\it bicomponent} of an undirected graph is a maximal biconnected subgraph.
  \end{definition}
\end{frame}
%%%%%%%%%%%%%%%%%%%%

%%%%%%%%%%%%%%%%%%%%
\begin{frame}{}
  \begin{columns}
    \column{0.50\textwidth}
      \fignocaption{width = 0.80\textwidth}{figs/bicomponent-example-graph}
    \column{0.50\textwidth}
      \pause
      \fignocaption{width = 0.80\textwidth}{figs/bicomponent-example-bicomponents}
  \end{columns}

  \pause
  \vspace{0.30cm}
  \centerline{\red{Paritition of edges} \teal{(not of nodes)}}
\end{frame}
%%%%%%%%%%%%%%%%%%%%

%%%%%%%%%%%%%%%%%%%%
\begin{frame}{}
  \begin{columns}
    \column{0.40\textwidth}
      \fignocaption{width = 0.80\textwidth}{figs/bicomponent-example-bicomponents-common-node}
    \column{0.65\textwidth}
    \begin{theorem}[Cut-nodes and Bicomponents]
	Let $G_i = (V_i, E_i)$ be the bicomponents of a connected undirected graph $G = (V, E)$.
	
	\begin{enumerate}[(a)]
	  \pause
	  \item 
	    \[
	      \forall i \neq j: \Big\lvert V_i \cap V_j \Big\rvert \le 1
	    \]
	  \pause
	  \item
	    \[
	      v \text{ is a cut-node } \red{\iff} \exists i \neq j: v \in V_i \cap V_j
	    \]
	\end{enumerate}
      \end{theorem}
  \end{columns}
\end{frame}
%%%%%%%%%%%%%%%%%%%%

%%%%%%%%%%%%%%%%%%%%
\begin{frame}{}
  \begin{center}
    {\large The Power of the Hammer of DFS on \red{Undirected Graphs}}
  \end{center}

  \fignocaption{width = 0.50\textwidth}{figs/luo-hammer}

  \pause
  \begin{theorem}[Theorem $22.10$]
    In a depth-first search of an undirected graph $G$, 
    every edge of $G$ is either a \red{tree edge} or a \red{back edge}.
  \end{theorem}
\end{frame}
%%%%%%%%%%%%%%%%%%%%

%%%%%%%%%%%%%%%%%%%%
\begin{frame}{}
  \fignocaption{width = 0.50\textwidth}{figs/bicomponent-dfs-tree}

  \pause
  \begin{center}
    \red{\large Cut-nodes? \qquad \pause Bicomponents?}
  \end{center}
\end{frame}
%%%%%%%%%%%%%%%%%%%%

%%%%%%%%%%%%%%%%%%%%
\begin{frame}{}
  \centerline{\Large \textsc{Bicomp}: \red{Back!}}

  \begin{center}
    % file: sections/bicomp-dfs-tree.tex

\begin{tikzpicture} [level distance = 50pt, sibling distance = 20pt,
  edge from parent/.style= { % added code
      draw, thick, edge from parent path = {(\tikzparentnode) -- (\tikzchildnode)}}]
  \tikzset{every tree node/.style = 
    {align = center, circle, draw, fill = blue!20, font = \Large, minimum size = 20pt}}
    \Tree [.\textcolor{red}{$r$}
	    [.\node (x) {$x$};
	       [.\node (ll) {};
	        \node (w) {\textcolor{blue}{$w$}};
		$$
	       ]
	       [.\node (v) {\textcolor{red}{$v$}};
		 \node (u) {\textcolor{teal}{$u$}};
	       ]
            ]
	    [.$$ 
	      $$
	      $$
	    ] 
        ]

  \uncover<2->{\draw[thick, teal, bend left] (u) to (x);}
  \uncover<3->{\draw[thick, red, bend right] (v) to (r);}
  \uncover<4->{\draw[thick, blue, bend left = 45] (w) to (r);}
\end{tikzpicture}

  \end{center}
\end{frame}
%%%%%%%%%%%%%%%%%%%%

%%%%%%%%%%%%%%%%%%%%
\begin{frame}{}
  \fignocaption{width = 0.40\textwidth}{figs/bicomp-cutnode-theorem}
\end{frame}
%%%%%%%%%%%%%%%%%%%%

%%%%%%%%%%%%%%%%%%%%
\begin{frame}{}
  \begin{theorem}[Characterization of Cut-nodes]
    \centerline{In a DFS tree, $v$ is a cut-node}

    \[
      \iff
    \]

    \begin{columns}
      \column{0.10\textwidth}
      \column{0.80\textwidth}
	\begin{enumerate}[(a)]
	  \item $v$ is the \red{root} and $\text{deg}(v) \ge 2$
	  \item $v$ is not the root and \red{some} subtree of $v$ has no back edge to a \teal{proper ancestor} of $v$
	\end{enumerate}
      \column{0.10\textwidth}
    \end{columns}
  \end{theorem}

  \pause
  \vspace{0.80cm}
  \begin{enumerate}[(I)]
    \centering
    \item \red{When and how to \teal{identify} a bicomponent?}
  \end{enumerate}
\end{frame}
%%%%%%%%%%%%%%%%%%%%

%%%%%%%%%%%%%%%%%%%%
\begin{frame}{}
  \fignocaption{width = 0.35\textwidth}{figs/back-to-past}

  \[
    \red{\back{v}:}
  \]

  \begin{center}
    The \teal{earliest ancestor} $v$ can get \\[6pt]
    by following tree edges $\T$ and back edges $\B$. \\[15pt]

    \pause
    \begin{enumerate}[(I)]
      \centering
      \setcounter{enumi}{1}
      \item \red{\large When and how to \cyan{initialize\&update} $\back{v}$?}
    \end{enumerate}
  \end{center}
\end{frame}
%%%%%%%%%%%%%%%%%%%%

%%%%%%%%%%%%%%%%%%%%
\begin{frame}{}
  \[
    \red{\back{v}:}
  \]
  \begin{center}
    The \teal{earliest ancestor} $v$ can get \\[3pt]
    by following tree edges $\T$ and back edges $\B$.
  \end{center}

  \pause
  \[
    \back{v} = \min \begin{cases} 
      \set{v} \\[5pt]
      \set{w \mid (v, w) \in \B} \\[5pt]
      \set{\back{w} \mid (v,w) \in \T}
    \end{cases}
  \]

  \pause
  \begin{columns}
    \column{0.10\textwidth}
    \column{0.80\textwidth}
      \begin{description}[backtracking from $w$:]
	\item[tree edge ($\to v$):]   $\back{v} = \red{\d{v}}$
	\item[back edge ($v \to w$):] $\back{v} = \min \set{\back{v}, \red{\d{w}}}$
	\item[backtracking from $w$:] $\back{v} = \min \set{\back{v}, \purple{\back{w}}}$
      \end{description}
    \column{0.10\textwidth}
  \end{columns}

  \pause
  \vspace{0.30cm}
  \[
    \red{\text{Backtracking from $w$ to $v$}: \back{w} \ge \d{v}}
  \]
\end{frame}
%%%%%%%%%%%%%%%%%%%%

%%%%%%%%%%%%%%%%%%%%
\begin{frame}[fragile]{}
  \begin{exampleblock}{After-class Exercise: \textsc{Bicomp}}
    % file: algs/bicomp-proc.tex

\begin{algorithm}[H]
  % \caption{Bicomponent Algorithm.}
  % \label{alg:bicomponent}
  \begin{algorithmic}[1]
    \Procedure{Bicomp}{$G$}
      \State \red{Here: Your Code Based on the DFS Framework}
    \EndProcedure
  \end{algorithmic}
\end{algorithm}

  \end{exampleblock}

  \vspace{0.60cm}
  \fignocaption{width = 0.30\textwidth}{figs/highly-recommended}
\end{frame}
%%%%%%%%%%%%%%%%%%%%

%%%%%%%%%%%%%%%%%%%%
\begin{frame}{}
  \fignocaption{width = 0.55\textwidth}{figs/cutnode-types.png}
\end{frame}
%%%%%%%%%%%%%%%%%%%%
