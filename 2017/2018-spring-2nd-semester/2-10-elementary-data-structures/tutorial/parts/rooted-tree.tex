% file: parts/rooted-trees.tex

%%%%%%%%%%%%%%%
\begin{frame}{}
  \begin{exampleblock}{Recursive Binary Tree Traversal (Problem $10.4-2$)}
    \begin{columns}
      \column{0.45\textwidth}
	\[
	  O(n)
	\]

	\fignocaption{width = 0.80\textwidth}{figs/binary-tree}
      \column{0.55\textwidth}
        \pause
	% file: algs/tree-traversal/dfs/dfs-print-recursive.tex

\begin{algorithm}[H]
  \begin{algorithmic}[]
    \Procedure{Recursive-DFS}{$t$}
      \State \textcolor<3->{red}{print $t.key$}

      \Statex
      \If{$t.left \neq \textsf{NIL}$}
	\State \Call{Recursive-DFS}{$t.left$}
      \EndIf
      
      \If{$t.right \neq \textsf{NIL}$}
	\State \Call{Recursive-DFS}{$t.right$}
      \EndIf
    \EndProcedure

    \Statex
    \State \teal{\Call{Recursive-DFS}{$T.root$}}
  \end{algorithmic}
\end{algorithm}

    \end{columns}
  \end{exampleblock}
\end{frame}
%%%%%%%%%%%%%%%

%%%%%%%%%%%%%%%
\begin{frame}{}
  \begin{exampleblock}{Non-recursive Binary Tree Traversal (Problem $10.4-2$)}
    \begin{columns}
      \column{0.45\textwidth}
	\[
	  O(n)
	\]

	\fignocaption{width = 0.80\textwidth}{figs/binary-tree}
      \column{0.55\textwidth}
        \pause
	% file: algs/tree-traversal/dfs/dfs-print-nonrecursive.tex

\begin{algorithm}[H]
  \begin{algorithmic}[]
    \Procedure{Iterative-DFS}{$t$}
      \State $S.\Call{Push}{t}$ \Comment{$S:$ stack}

      \Statex
      \While{$S \neq \emptyset$}
	\State $v \gets S.\Call{Pop}{\null}$
	\State print $v.key$

	\Statex
	\If{$v.right \neq \textsf{NIL}$}
	  \State $S.\Call{Push}{v.right}$
	\EndIf

	\If{$v.left \neq \textsf{NIL}$}
	  \State $S.\Call{Push}{v.left}$
	\EndIf
      \EndWhile
    \EndProcedure

    \Statex
    \State \teal{\Call{Iterative-DFS}{$T.root$}}
  \end{algorithmic}
\end{algorithm}

    \end{columns}
  \end{exampleblock}
\end{frame}
%%%%%%%%%%%%%%%

%%%%%%%%%%%%%%%
\begin{frame}{}
  \begin{exampleblock}{``LCRS'' Tree Traversal (Problem $10.4-2$)}
    \begin{columns}
      \column{0.50\textwidth}
	\[
	  O(n)
	\]

	\fignocaption{width = 0.95\textwidth}{figs/LCRS-rooted-tree}
      \column{0.50\textwidth}
        \pause
	% file: algs/tree-traversal/dfs/LCRS-rooted-tree-traversal.tex

\begin{algorithm}[H]
  \begin{algorithmic}[]
    \Procedure{Recursive-DFS}{$t$}
      \State print $t.key$

      \Statex
      \If{$t.lc \neq \textsf{NIL}$}
	\State \Call{Recursive-DFS}{$t.lc$}
      \EndIf
      
      \If{$t.rs \neq \textsf{NIL}$}
	\State \Call{Recursive-DFS}{$t.rs$}
      \EndIf
    \EndProcedure

    \Statex
    \State \Call{Recursive-DFS}{$T.root$}
  \end{algorithmic}
\end{algorithm}

    \end{columns}
  \end{exampleblock}
\end{frame}
%%%%%%%%%%%%%%%
