% file: sections/dijkstra-alg.tex

%%%%%%%%%%%%%%%%%%%%
\begin{frame}{}
  \fignocaption{width = 0.30\textwidth}{figs/dijkstra}

  \begin{quote}
    {\small
      For fundamental contributions to \red{programming} as a high, intellectual challenge; \\[3pt]
      for eloquent insistence and practical demonstration that programs should be composed correctly,
      not just debugged into \red{correctness}; \\[3pt]
      for illuminating \red{perception of problems} at the foundations of program design.
    }

    \hfill --- \teal{Turing Award}, $1972$
  \end{quote}
\end{frame}
%%%%%%%%%%%%%%%%%%%%

%%%%%%%%%%%%%%%%%%%%
\begin{frame}[fragile]
  \begin{columns}
    \column{0.60\textwidth}
      % file: algs/dijkstra-alg-emph.tex

\begin{algorithm}[H]
  % \caption{Dijkstra Algorithm.}
  % \label{alg:dijkstra}
  \begin{algorithmic}[1]
    \Procedure{Dijkstra}{$G, w, s$}
      \State \Call{Init-Single-Source}{$G, s$}

      \hStatex
      \State \textcolor<3->{red}{$S = \emptyset$}
      \State $Q = G.V$
      \While{$Q \neq \emptyset$}
	\State $u \gets \Call{Extract-Min}{Q}$
	\State \textcolor<3->{red}{$S \gets S \cup \set{u}$}

	\hStatex
	\For{$v \in G.Adj[u]$}
	  \State $\Call{Relax}{u,v,w}$
	\EndFor
      \EndWhile
    \EndProcedure
  \end{algorithmic}
\end{algorithm}

  \end{columns}

  \pause
  \vspace{0.50cm}
  \begin{columns}
    \column{0.40\textwidth}
      \begin{description}[Min-heap:]
	\item[Array:] $O(n^2)$
	\item[Min-heap:] \red{$O(E \log V)$}
	\item[Fib-heap:] $O(V \log V + E)$
      \end{description}
  \end{columns}
\end{frame}
%%%%%%%%%%%%%%%%%%%%

%%%%%%%%%%%%%%%%%%%%
\begin{frame}{}
  \fignocaption{width = 0.60\textwidth}{figs/dijkstra-correctness}
\end{frame}
%%%%%%%%%%%%%%%%%%%%

%%%%%%%%%%%%%%%%%%%%
\begin{frame}{}
  \begin{exampleblock}{Negative-weight Edges for Dijkstra's Algorithm (Problem $24.3$-$2$)}
    \fignocaption{width = 0.30\textwidth}{figs/dijkstra-negative-edge}
  \end{exampleblock}
\end{frame}
%%%%%%%%%%%%%%%%%%%%

%%%%%%%%%%%%%%%%%%%%
\begin{frame}{}
  \begin{exampleblock}{Negative-weight Edges for Dijkstra's Algorithm (Additional Problem $24.3$-$10$)}
    \begin{columns}
      \column{0.60\textwidth}
	\begin{itemize}
	  \item All negative-weight egdes are from $s$
	  \item No negative-weight cycles
	\end{itemize}
    \end{columns}
  \end{exampleblock}

  \pause
  \fignocaption{width = 0.35\textwidth}{figs/dijkstra-correctness}
\end{frame}
%%%%%%%%%%%%%%%%%%%%

%%%%%%%%%%%%%%%%%%%%
\begin{frame}{}
  \begin{exampleblock}{Checking Output of Dijkstra's Algorithm (Problem $24.3$-$4$)}
    \[
      \forall v \in V: v.\pi, v.d
    \]

    \begin{center}
      To check whether $\pi$ and $d$ match some shortest-paths tree?
    \end{center}
    \[
      O(V + E)
    \]
  \end{exampleblock}
\end{frame}
%%%%%%%%%%%%%%%%%%%%

%%%%%%%%%%%%%%%%%%%%
\begin{frame}{}
  \[
    (1)\; \pi \text{ forms a tree}
  \]
  
  \pause
  \[
    (2)\; s.d = 0
  \]
  
  \pause
  \[
    \blue{\boxed{u \triangleq v.\pi}}
  \]
  
  \[
    (3)\; \forall v \in V: v.d = u.d + w(u, v)
  \]
  
  \pause
  \[
    (4)\; \forall v \in V: u.d + w(u, v) = \min\limits_{(v', v) \in E} \Big\{ v'.d + w(v', v) \Big\}
  \]
  
  \pause
  \[
    \red{(4)\;   \forall (v', v) \in E: v'.d + w(v', v) \ge v.d}
  \]
\end{frame}
%%%%%%%%%%%%%%%%%%%%

%%%%%%%%%%%%%%%%%%%%
\begin{frame}{}
  \fignocaption{width = 0.45\textwidth}{figs/ever-want}
\end{frame}
%%%%%%%%%%%%%%%%%%%%

%%%%%%%%%%%%%%%%%%%%
\begin{frame}{}
  \[
    \blue{\boxed{\forall v \in V: v.d = \delta(s, v)}}
  \]
  
  \pause
  \[
    \red{\exists v \in V: v.d \neq \delta(s, v)}
  \]
  
  \pause
  \begin{columns}
    \column{0.50\textwidth}
      \[
        \teal{v.d < \delta(s, v)}
      \]
      
      \pause
      \begin{align*}
        v.d &= u.d + w(u,v) \\
            &< \delta(s ,v) \\
            &\le \delta(s, u) + w(u,v)
      \end{align*}
      
      \pause
      \[
        \red{\boxed{u.d < \delta(s, u)}}
      \]
    \column{0.50\textwidth}
      \pause
      \[
        \teal{v.d > \delta(s, v)}
      \]
      
      \pause
      \[
        v.d = u.d + w(u,v) > \delta(s,v)
      \]
      
      \pause
      \[
        \begin{cases}
          u.d = \delta(s, u) & \uncover<9->{\red{\boxed{v.\pi}}} \\[10pt]
          \red{\boxed{u.d > \delta(s, u)}}
        \end{cases}
      \]
  \end{columns}
\end{frame}
%%%%%%%%%%%%%%%%%%%%

%%%%%%%%%%%%%%%%%%%%
\begin{frame}{}
  \begin{center}
    \purple{Lawler's Algorithm} on \teal{DAG} \\[6pt]
    $\Longleftarrow$ \\[30pt]

    \red{Dijkstra's Algorithm} on Digraph with \teal{Nonnegative-weight} Edges \\[30pt]
    
    $\implies$ \\[6pt]
    \purple{Bellman-Ford} Algorithm on Digraph with \teal{Negative-weight} Edges
  \end{center}
\end{frame}
%%%%%%%%%%%%%%%%%%%%
