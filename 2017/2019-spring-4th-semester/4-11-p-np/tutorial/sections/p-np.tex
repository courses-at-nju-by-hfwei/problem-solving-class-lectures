% p-np.tex

%%%%%%%%%%%%%%%%%%%%
\begin{frame}{\p}
  \begin{exampleblock}{TC 34.1--5}
	\[ 
	  f(n) = O(n^c) \qquad t(n) = O(n^d)
	\]
  \end{exampleblock}

  \[
	\hbox{\sout{$T(n) = kf(n) + t(n)$}}
  \]

  \[
	T_{k}(n) = \sum_{i=1}^{k}f^{(i)}(n) + t(n) 
  \]

  \[
	k = O(1) \text{\emph{ vs. }} k = \Theta(n^{O(1)}) \qquad O \emph{\text{ vs. }} \Theta, \Omega
  \]
\end{frame}
%%%%%%%%%%%%%%%%%%%%

%%%%%%%%%%%%%%%%%%%%
\begin{frame}{\conp{}}
  \[
	L \in \np \xLongrightarrow{?} \overline{L} \in \np
  \]

  \[
	\overline{\text{SAT}} = \set{\phi: \phi \text{ is not satisfiable}}
  \]

  \[
	\text{TAUT} = \set{\phi: \phi \text{ is a tautology}}
  \]

  \[
	\conp = \set{L: \bar{L} \in \np}
  \]

  \begin{definition}[\conp]
	$L \in \conp$ if $\exists$ polynomial-time \emph{verifier} $V(x,c)$ such that $\forall x \in \set{0,1}^{\ast}$,
	\[
	  x \in L \iff \red{\forall} c \in \set{0,1}^{\ast}, V(x,c) = 1.
	\]
  \end{definition}
\end{frame}
%%%%%%%%%%%%%%%%%%%%

%%%%%%%%%%%%%%%%%%%%
\begin{frame}{\np{} \emph{vs.} \conp{}}
  \[
	\conp{} \red{\neq} \set{0,1}^{\ast} \setminus \np
  \]

  \[
	\p \subseteq \np \cap \conp
  \]

  \[
	\p = \np \implies \np = \conp
  \]

  \[
	\np \neq \conp \implies \p \neq \np
  \]
\end{frame}
%%%%%%%%%%%%%%%%%%%%

%%%%%%%%%%%%%%%%%%%%
\begin{frame}{\expcls}
  \[
	\expcls = \bigcup_{c > 0} \dtime(2^{n^c})
  \]

  \[
    \p \subseteq \np \subseteq \expcls
  \]
\end{frame}
%%%%%%%%%%%%%%%%%%%%

%%%%%%%%%%%%%%%%%%%%
\begin{frame}{If $\hc \in \p$}
  \begin{exampleblock}{TC 34.2--3}
    \[
	  \hc \in \p \implies \hc\text{-LIST} \in \p
	\]
  \end{exampleblock}

  \begin{enumerate}
	\item starting from $v$
	\item removing each edge $e$ on $v$
	\item checking $G \setminus e$
	\item restoring and marking the critical edge $e = (v, u)$
	\item $v = u$
  \end{enumerate}

  \begin{alertblock}{Question}
	\centerline{remove $e \in E$ in arbitrary order if $(G \setminus e) \in \hc$?}
  \end{alertblock}
\end{frame}
%%%%%%%%%%%%%%%%%%%%

%%%%%%%%%%%%%%%%%%%%
\begin{frame}{$G^3 \in \hc$}
  \begin{exampleblock}{TC 34.2--11 (Karaganis, 1968)}
	\[
	  G^3 \in \hc
	\]
  \end{exampleblock}

  \begin{alertblock}{References}
	\begin{itemize}
	  \item ``On the Cube of a Graph'' by Jerome J. Karaganis, 1968
	  \item ``The Cube of Every Connected Graph is 1-Hamiltonian'' by Gary Chartrand and S. F. Kapoor, 1968
	  \item \url{http://www.aco.gatech.edu/sites/default/files/documents/comp-fa14sol.pdf}
	  \end{itemize}
  \end{alertblock}
\end{frame}
%%%%%%%%%%%%%%%%%%%%

%%%%%%%%%%%%%%%%%%%%
\begin{frame}{$G^3 \in \hc$}
  \begin{theorem}[$T^3 \in \hc$]
	Let $T = (V, E)$ be a tree. For any edge $e \in E$, there is a Hamilton cycle on $T^3$ that contains $e$.
  \end{theorem}

  \begin{proof}
	\centerline{By induction on subtrees obtained by removing any edge $e = (u,v)$.}
  \end{proof}

  \begin{alertblock}{Question}
	In I.S., choose edge $e = (u,v)$ with $u$ or $v$ being a leaf?
  \end{alertblock}
\end{frame}
%%%%%%%%%%%%%%%%%%%%
