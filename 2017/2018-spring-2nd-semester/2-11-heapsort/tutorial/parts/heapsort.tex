% file: parts/heapsort.tex

%%%%%%%%%%%%%%%
\begin{frame}{}
  \begin{center}
    \movie[showcontrols, poster, width = 0.50\textwidth, height = 0.30\textwidth]
    % {\includegraphics[width = 0.60\textwidth]{}}
    {}
    {obama.mp4}
  \end{center}

  \pause
  \[
    \red{O,\; \Omega,\; \Theta}
  \]

  \pause
  \begin{center}
    \purple{``What is most efficient way to sort a million $32$-bit integers?''}  \\[5pt]
    \teal{``The bubblesort would be the wrong way to go.''}
  \end{center}
\end{frame}
%%%%%%%%%%%%%%%

%%%%%%%%%%%%%%%
\begin{frame}{}
  \begin{CJK*}{UTF8}{gbsn}
    \begin{center}
      {\red{\Large 反馈:} \\[10pt]
      $O, \Omega, \Theta$ 傻傻分不清。\\[8pt]
      什么时候用哪个? \\[8pt]
      $6.2-6$ 这道题为什么问的是~$\Omega$, 而不问~$O$~或~$\Theta$?}
    \end{center}
  \end{CJK*}
\end{frame}
%%%%%%%%%%%%%%%

%%%%%%%%%%%%%%%
\begin{frame}{}
  \begin{exampleblock}{Worst-case of \textsc{Max-Heapify} (TC $6.2-6$)}
    Show that the \red{worst-case} running time of \textsc{Max-Heapify} on an $n$-element heap is $\red{\Omega}(\log n)$.
  \end{exampleblock}

  \pause
  \vspace{0.60cm}
  \centerline{\red{\textsc{Move}} {\it vs.} \textsc{Compare}}
\end{frame}
%%%%%%%%%%%%%%%

%%%%%%%%%%%%%%%
\begin{frame}{}
  \[
    \text{Algorithm } \mathcal{A}
  \]

  \[
    \text{Inputs } \mathcal{I} \text{ of size } n
  \]

  \begin{table}
    \centering
    \renewcommand*{\arraystretch}{1.5}
    \begin{tabular}{c||c|c|c}
      \hline
		  	& $O$ 				& $\Omega$ 				& $\Theta$ \\ \hline \hline
      {\it Best-case} 	& \uncover<3->{\teal{by example}}	
			& \uncover<3->{\purple{``weakness'' of $\mathcal{A}$}}	
			& \uncover<2->{\violet{$O = \Omega$}}    \\ \hline
      {\it Worst-case} 	& \uncover<4->{\purple{``power'' of $\mathcal{A}$}}
			& \uncover<4->{\teal{by example}}	
			& \uncover<2->{\violet{$O = \Omega$}}    \\ \hline
      {\it Average-case}& \uncover<5->{\blue{$\le$}}		
			& \uncover<5->{\blue{$\ge$}}			
			& \uncover<2->{\violet{$O = \Omega$}}    \\ \hline
    \end{tabular}
  \end{table}
\end{frame}
%%%%%%%%%%%%%%%

%%%%%%%%%%%%%%%
\begin{frame}{}
  \begin{exampleblock}{Worst-case of \textsc{Max-Heapify} (TC $6.2-6$)}
    Show that the \red{worst-case} running time of \textsc{Max-Heapify} on an $n$-element heap is $\red{\Omega}(\log n)$.
  \end{exampleblock}

  \pause
  \vspace{0.50cm}
  \centerline{\red{\large By Example.}}

  \pause
  \fignocaption{width = 0.40\textwidth}{figs/max-heapify-worst}
\end{frame}
%%%%%%%%%%%%%%%

%%%%%%%%%%%%%%%
\begin{frame}{}
  \begin{exampleblock}{Worst-case of \textsc{Max-Heapify} (TC $6.2-6$)}
    Show that the \red{worst-case} running time of \textsc{Max-Heapify} on an $n$-element heap is $\red{O}(\log n)$.
  \end{exampleblock}

  \pause
  \fignocaption{width = 0.35\textwidth}{figs/heap-structure}

  \vspace{-0.30cm}
  \[
    W(n) \le H(n)
  \]

  \pause
  \vspace{0.20cm}
  \centerline{\red{\Large No Examples Here!}}
\end{frame}
%%%%%%%%%%%%%%%

%%%%%%%%%%%%%%%
\begin{frame}{}
  \fignocaption{width = 0.40\textwidth}{figs/therefore}

  \begin{exampleblock}{Worst-case of \textsc{Max-Heapify} (TC $6.2-6$)}
    Show that the \red{worst-case} running time of \textsc{Max-Heapify} on an $n$-element heap is $\red{\Theta}(\log n)$.
  \end{exampleblock}

  \begin{table}
    \centering
    \renewcommand*{\arraystretch}{1.5}
    \begin{tabular}{c||c|c|c}
      \hline
		  	& $O$ 				& $\Omega$ 				& $\Theta$ \\ \hline \hline
      {\it Worst-case} 	& \purple{``power'' of $\mathcal{A}$}
			& \teal{by example}	
			& \violet{$O = \Omega$}    \\ \hline
    \end{tabular}
  \end{table}
\end{frame}
%%%%%%%%%%%%%%%

%%%%%%%%%%%%%%%
\begin{frame}{}
  \begin{exampleblock}{Worst-case of \textsc{Heapsort} (TC $6.4-4$)}
    Show that the \red{worst-case} running time of \textsc{Heapsort} is $\red{\Omega}(n \log n)$.
  \end{exampleblock}

  \pause
  \vspace{0.50cm}
  \centerline{\red{\large By Example.}}

  \pause
  \begin{proof}[Non-proof.]
    \[
      \underbrace{\Theta(n)}_{\teal{\textsc{Extract-Max}}} \times \underbrace{\Omega(\log n)}_{\teal{\textsc{Max-Heapify}}} = \Omega(n \log n)
    \]
  \end{proof}
\end{frame}
%%%%%%%%%%%%%%%

%%%%%%%%%%%%%%%
\begin{frame}{}
  \begin{exampleblock}{Worst-case of \textsc{Heapsort} (TC $6.4-4$)}
    Show that the \red{worst-case} running time of \textsc{Heapsort} is $\red{\Omega}(n \log n)$.
  \end{exampleblock}

  \vspace{0.30cm}
  \fignocaption{width = 0.50\textwidth}{figs/example}
\end{frame}
%%%%%%%%%%%%%%%

%%%%%%%%%%%%%%%
\begin{frame}{}
  \centerline{\red{\Large Heap in decreasing order?}}

  \pause
  \fignocaption{width = 0.45\textwidth}{figs/heap-decreasing-order-7}

  \pause
  \vspace{-0.30cm}
  \[
    T(7) = 2 + 1 + 1 + 1 + 0 + 0 = 5
  \]

  % \[
  %   T(12) = 3 + 2 + 2 + 2 + 2 + 2 + 2 + 1 + 1 + 0 + 0 = 17
  % \]
\end{frame}
%%%%%%%%%%%%%%%

%%%%%%%%%%%%%%%
\begin{frame}{}
  \begin{columns}
    \column{0.50\textwidth}
      \fignocaption{width = 0.70\textwidth}{figs/heapsort-selection-worst-7}
      \pause
      \[
	T(7) = 2 + 2 + 2 + 1 + 1 + 0 = 8
      \]
      \pause \centerline{\footnotesize \teal{(Ex. $23$, Section $5.2.3$, TAOCP Vol $3$)}}
    \pause
    \column{0.50\textwidth}
      \fignocaption{width = 0.75\textwidth}{figs/heapsort-selection-worst-8}
  \end{columns}

  \pause
  \vspace{0.60cm}
  \[
    \sum_{r=1}^{n-1} \lfloor \log r \rfloor \pause = n \lfloor \log  n \rfloor - 2^{\lfloor \log n \rfloor + 1} + 2 \pause = \Omega(n \log n)
  \]
\end{frame}
%%%%%%%%%%%%%%%

%%%%%%%%%%%%%%%
\begin{frame}{}
  \begin{exampleblock}{Worst-case of \textsc{Heapsort} (TC $6.4-4$)}
    Show that the \red{worst-case} running time of \textsc{Heapsort} is $\red{O}(n \log n)$.
  \end{exampleblock}

  \pause
  \[
    \sum_{r=1}^{n-1} \lfloor \log r \rfloor = n \lfloor \log  n \rfloor - 2^{\lfloor \log n \rfloor + 1} + 2 = O(n \log n)
  \]

  \pause
  \vspace{0.30cm}
  \centerline{\red{\Large No Examples Here!}}

  \pause
  \vspace{0.30cm}
  \[
    \underbrace{\Theta(n)}_{\teal{\textsc{Extract-Max}}} \times \underbrace{\red{O}(\log n)}_{\teal{\textsc{Max-Heapify}}} = O(n \log n)
  \]
\end{frame}
%%%%%%%%%%%%%%%

%%%%%%%%%%%%%%%
\begin{frame}{}
  \fignocaption{width = 0.40\textwidth}{figs/therefore}

  \begin{exampleblock}{Worst-case of \textsc{Heapsort} (TC $6.4-4$)}
    Show that the \red{worst-case} running time of \textsc{Heapsort} is $\red{\Theta}(n \log n)$.
  \end{exampleblock}

  \begin{table}
    \centering
    \renewcommand*{\arraystretch}{1.5}
    \begin{tabular}{c||c|c|c}
      \hline
		  	& $O$ 				& $\Omega$ 				& $\Theta$ \\ \hline \hline
      {\it Worst-case} 	& \purple{``power'' of $\mathcal{A}$}
			& \teal{by example}	
			& \violet{$O = \Omega$}    \\ \hline
    \end{tabular}
  \end{table}
\end{frame}
%%%%%%%%%%%%%%%

%%%%%%%%%%%%%%%
\begin{frame}{}
  \begin{exampleblock}{Best-case of \textsc{Heapsort} (TC $6.4-5$)}
    Show that when \violet{all elements are distinct}, 
    the \red{best-case} running time of \textsc{Heapsort} is $\red{\Omega}(n \log n)$.
  \end{exampleblock}

  \pause
  \vspace{0.30cm}
  \centerline{\red{\large Consider the largest $m = \lceil n/2 \rceil$ elements.}}

  \begin{center}
    The largest $m$ elements form a subtree. \\[8pt] \pause

    \purple{$\ge \lfloor m/2 \rfloor$ of $m$} must be nonleaves of that subtree. \\[8pt] \pause

    \purple{$\ge \lfloor m/2 \rfloor$ of $m$} appear in the first $\lfloor n/2 \rfloor$ positions. \\[8pt] \pause

    \teal{\purple{They} must be \red{promoted} to the root before being \textsc{Extract-Max}.}
  \end{center}

  \pause
  \[
    \sum_{k=1}^{\lfloor m/2 \rfloor} \lfloor \log k \rfloor = \frac{1}{2} m \log m + O(m) 
  \]

  \pause
  \[
    B(n) \ge \frac{1}{4} n \log n + O(n) + B(\lfloor n/2 \rfloor) \pause \implies B(n) \ge \frac{1}{2} n \log n + O(n)
  \]
\end{frame}
%%%%%%%%%%%%%%%

%%%%%%%%%%%%%%%
\begin{frame}{}
  \begin{exampleblock}{Best-case of \textsc{Heapsort} (TC $6.4-5$)}
    Show that when all elements are distinct, 
    the \red{best-case} running time of \textsc{Heapsort} is $\red{O}(n \log n)$.
  \end{exampleblock}

  \pause
  \vspace{0.60cm}
  \centerline{\red{\Large By Example.}}

  \pause
  \fignocaption{width = 0.35\textwidth}{figs/very-hard}
\end{frame}
%%%%%%%%%%%%%%%

%%%%%%%%%%%%%%%
\begin{frame}{}
  \centerline{\red{\large ``On the Best Case of Heapsort''} \teal{(1994)}}

  \begin{columns}
    \column{0.33\textwidth}
      \fignocaption{width = 0.95\textwidth}{figs/best-case-initial-heap}
    \column{0.33\textwidth}
      \fignocaption{width = 0.90\textwidth}{figs/best-case-heap-odd}
    \column{0.33\textwidth}
      \fignocaption{width = 0.90\textwidth}{figs/best-case-heap-even}
  \end{columns}

  \pause
  \vspace{0.30cm}
  \[
    B(n) \le \frac{1}{2} n \log n + O(n \log \log n)
  \]
\end{frame}
%%%%%%%%%%%%%%%

%%%%%%%%%%%%%%%
\begin{frame}{}
  \fignocaption{width = 0.40\textwidth}{figs/therefore}

  \begin{exampleblock}{Best-case of \textsc{Heapsort} (TC $6.4-5$)}
    Show that when all elements are distinct, 
    the \red{best-case} running time of \textsc{Heapsort} is $\red{\Theta}(n \log n)$.
  \end{exampleblock}

  \begin{table}
    \centering
    \renewcommand*{\arraystretch}{1.5}
    \begin{tabular}{c||c|c|c}
      \hline
		  	& $O$ 				& $\Omega$ 				& $\Theta$ \\ \hline \hline
      {\it Best-case} 	& \teal{by example}	
			& \purple{``weakness'' of $\mathcal{A}$}
			& \violet{$O = \Omega$}    \\ \hline
    \end{tabular}
  \end{table}
\end{frame}
%%%%%%%%%%%%%%%

%%%%%%%%%%%%%%%
\begin{frame}{}
  \begin{exampleblock}{Average-case of \textsc{Heapsort}}
    Assume that all elements are distinct. 
    Show that the \red{average-case} running time of \textsc{Heapsort} is $\red{\Theta}(n \log n)$.
  \end{exampleblock}

  \vspace{0.50cm}
  \begin{columns}
    \column{0.50\textwidth}
      \pause
      \fignocaption{width = 0.75\textwidth}{figs/simple}
    \column{0.50\textwidth}
      \pause
      \fignocaption{width = 0.80\textwidth}{figs/simple-not-easy}
  \end{columns}
\end{frame}
%%%%%%%%%%%%%%%

%%%%%%%%%%%%%%%
\begin{frame}{}
  \centerline{\blue{\Large ``By a surprisingly short counting argument.''}}
  \begin{columns}
    \column{0.50\textwidth}
      \fignocaption{width = 0.50\textwidth}{figs/sedgewick}
      \centerline{\teal{Robert Sedgewick}}
    \pause
    \column{0.50\textwidth}
      \fignocaption{width = 0.60\textwidth}{figs/knuth}
      \centerline{\teal{D. E. Knuth}}
  \end{columns}

  \vspace{0.50cm}
  \centerline{\red{\Large ``It is elegant. \pause see exercise $30$.''}}
\end{frame}
%%%%%%%%%%%%%%%

%%%%%%%%%%%%%%%
\begin{frame}{}
  \[
    \red{\boxed{f(n) \triangleq \text{\# of heaps of } n \text{ distinct keys } [1 \cdots n]}}
  \]

  \pause
  \[
    f(n) = \binom{n-1}{m} f(m) f(n-1-m)
  \]

  \pause
  \vspace{0.50cm}
  \[
    \frac{f(n)}{n!} = \frac{1}{n} \frac{f(m)}{m!} \frac{f(n-1-m)}{(n-1-m)!}
  \]

  \pause
  \vspace{0.50cm}
  \[
    \teal{\boxed{f(n) = \frac{n!}{\prod_{1 \le i \le n} s_i}}}
  \]
  \[
    s_i \triangleq \text{ size of the subtree rooted at } i
  \]
\end{frame}
%%%%%%%%%%%%%%%

%%%%%%%%%%%%%%%
\begin{frame}{}
  \fignocaption{width = 0.60\textwidth}{figs/heap-subtrees}

  \[
    f(13) = \frac{13!}{13 \cdot 7 \cdot 5 \cdot 3 \cdot 3 \cdot 3} = 506880
  \]
\end{frame}
%%%%%%%%%%%%%%%