%%%%%%%%%%%%%%%
\begin{frame}{}
  \fignocaption{width = 0.40\textwidth}{figs/less-logo}
\end{frame}
%%%%%%%%%%%%%%%

%%%%%%%%%%%%%%%
\begin{frame}{}
  \begin{definition}[$|A| \le |B|$]
    $|A| \le |B|$ if there exists an \red{\it one-to-one} function $f$ from $A$ into $B$.
  \end{definition}

  \[
    \text{bijection} \;f: A \to f(A) \; (\blue{\subseteq B})
  \]

  \pause
  \vspace{0.80cm}
  \begin{center}
    {\red{\it $Q:$ What about onto function $f: A \to B$?}} \\[8pt]
    {\blue{$|B| \le |A|$ \pause {\footnotesize (Axiom of Choice)}}}
  \end{center}
\end{frame}
%%%%%%%%%%%%%%%

%%%%%%%%%%%%%%%
\begin{frame}{}
  \begin{definition}[$|A| < |B|$]
    $|A| < |B| \iff |A| \le |B| \land |A| \neq |B|$ 
  \end{definition}

  \pause
  \[
    |\N| < |\R|
  \]

  \[
    |X| < |2^{X}|
  \]
\end{frame}
%%%%%%%%%%%%%%%

%%%%%%%%%%%%%%%
\begin{frame}{}
  \begin{definition}[Countable Revisited]
    $X$ is countable:
    \[
      \exists n \in \N: |X| = n \red{\;\lor\;} |X| = |\N|
    \]
  \end{definition}

  \begin{exampleblock}{Proof for Countable (UD Exercise $22.5$)}
    $X$ is countable iff there exists a one-to-one function
    \[
      f: A \to \N.
    \]
    \pause
    $X$ is countable iff
    \[
      \blue{|X| \le |\N|.}
    \]
  \end{exampleblock}

  \pause
  \begin{exampleblock}{Subsets of Countable Set (UD $22.6$; UD Corollary $22.4$)}
    Every subset $B$ of a countable set $A$ is countable.
  \end{exampleblock}
\end{frame}
%%%%%%%%%%%%%%%

%%%%%%%%%%%%%%%
\begin{frame}{}
  \begin{exampleblock}{Set Union (UD $22.1$)}
    Give an example, if possible, of
    \begin{enumerate}[(a)]
      \setcounter{enumi}{2}
      \item a countably infinite collection of \blue{\it pairwise disjoint} nonempty sets whose union is finite.
      \setcounter{enumi}{1}
      \item a countably infinite collection of nonempty sets whose union is finite.
    \end{enumerate}
  \end{exampleblock}

  \[
    \Big(\set{A_i: i \in R} \quad A_{i} = \set{1}\Big) \;\red{= \set{\set{1}}}
  \]

  \pause
  \[
    |A| = n \implies |2^{A}| = 2^n
  \]
\end{frame}
%%%%%%%%%%%%%%%

%%%%%%%%%%%%%%%
\begin{frame}{}
  \begin{exampleblock}{Slope (UD $22.2 \;(e)$)}
    \begin{enumerate}[(a)]
      \setcounter{enumi}{4}
      \item the set of all lines with rational slopes
    \end{enumerate}
  \end{exampleblock}

  \pause
  \[
    (\Q, \;\red{\R})
  \]
\end{frame}
%%%%%%%%%%%%%%%

%%%%%%%%%%%%%%%
\begin{frame}{}
  \centerline{\large \red{\it $Q:$ Is ``$\le$'' a partial order?}}

  \pause
  \vspace{0.30cm}
  \begin{theorem}[Cantor-Schr\"{o}der–Bernstein (1887)]
    \[
      |X| \le |Y| \land |Y| \le |X| \implies |X| = |Y|
    \]
    \pause
    \[
      \exists\; \text{\blue{one-to-one} } f: A \to B \land g: B \to A \implies \exists\; \text{\blue{bijection} } h: A \to B
    \]
  \end{theorem}

  \pause
  \begin{columns}
    \column{0.30\textwidth}
      \fignocaption{width = 0.50\textwidth}{figs/proof-cantor-bernstein-book}
    \column{0.30\textwidth}
      \pause
      \fignocaption{width = 0.50\textwidth}{figs/fudan-set-theory-book}
    \column{0.30\textwidth}
      \pause
      \fignocaption{width = 0.50\textwidth}{figs/qrcode-cantor-bernstein-wiki}
  \end{columns}
\end{frame}
%%%%%%%%%%%%%%%

%%%%%%%%%%%%%%%
\begin{frame}{}
  \centerline{\large \red{\it $Q:$ Is ``$\le$'' a total order?}}

  \pause
  \vspace{0.50cm}
  \begin{theorem}[PCC]
    \begin{center}
      Principle of Cardinal Comparability (PCC) $\iff$ Axiom of Choice
    \end{center}
  \end{theorem}
\end{frame}
%%%%%%%%%%%%%%%
