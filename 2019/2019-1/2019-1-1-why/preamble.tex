% preamble.tex

\usepackage{lmodern}

% for chinese
\usepackage{xeCJK}

% theme
\usetheme{CambridgeUS} % try Madrid, Pittsburgh
\usecolortheme{beaver}
\usefonttheme[]{serif} % try "professionalfonts"

\setbeamertemplate{itemize items}[default]
\setbeamertemplate{enumerate items}[default]

% math
\usepackage{amsmath, amsfonts, latexsym, mathtools}

\newcommand{\nn}{\mathbb{N}}
\newcommand{\zz}{\mathbb{Z}}
\newcommand{\rr}{\mathbb{R}}

\newcommand{\set}[1]{\{#1\}}
\newcommand{\bset}[1]{\big\{ #1 \big\}}
\newcommand{\Bset}[1]{\Big\{ #1 \Big\}}

\DeclareMathOperator*{\argmin}{arg\,min}
\DeclareMathOperator*{\argmax}{arg\,max}

% colors
\newcommand{\red}[1]{\textcolor{red}{#1}}
\newcommand{\green}[1]{\textcolor{green}{#1}}
\newcommand{\blue}[1]{\textcolor{blue}{#1}}
\newcommand{\purple}[1]{\textcolor{purple}{#1}}

% colorded box
\newcommand{\rbox}[1]{\red{\boxed{#1}}}
\newcommand{\gbox}[1]{\green{\boxed{#1}}}
\newcommand{\bbox}[1]{\blue{\boxed{#1}}}
\newcommand{\pbox}[1]{\purple{\boxed{#1}}}

\usepackage{adjustbox}

% tikz
\usepackage{tikz}
\usetikzlibrary{shapes, positioning, fit}

% algorithm
\usepackage{algorithm}
\usepackage[noend]{algpseudocode}
\newcommand{\hStatex}[0]{\vspace{5pt}}

% for fig without caption: #1: width/size; #2: fig file
\newcommand{\fig}[2]{
  \begin{figure}[htp]
    \centering
      \includegraphics[#1]{#2}
  \end{figure}
}

% thankyou
\newcommand{\thankyou}{
  \begin{frame}[noframenumbering]{}
    \fig{width = 0.55\textwidth}{figs/thankyou}
  \end{frame}
}
%%%%%%%%%%%%%%%%%%%%%%%%%%%%%%%%%%%%%%%%%%%%%%%%%%%%%%%%%%%%%%
\newcommand{\titletext}{1-1: The Counterfeit Coin Problem}
