%%%%%%%%%%%%%%%
\begin{frame}{}
  \centerline{\LARGE Partition}

  \vspace{0.60cm}
  \fignocaption{width = 0.60\textwidth}{figs/partition}
\end{frame}
%%%%%%%%%%%%%%%

%%%%%%%%%%%%%%%
\begin{frame}{}
  \begin{definition}[Partition]
    A family of sets \red{$\set{A_{\alpha}: \alpha \in I}$} is a \blue{\it partition} of \purple{$X$} if

    \begin{enumerate}[(i)]
      \item 
	\[
	  \forall \alpha \in I: A_{\alpha} \neq \emptyset
	\]
	\[
	  \textcolor{teal}{\forall \alpha \in I \; \exists x \in X: x \in A_{\alpha}}
	\]
      \item 
	\[
	  \bigcup_{\alpha \in I} A_{\alpha} = X
	\]
	\[
	  \textcolor{teal}{\forall x \in X \; \exists \alpha \in I: x \in A_{\alpha}}
	\]
      \item 
	\[
	  \forall \alpha, \beta \in I: A_{\alpha} \cap A_{\beta} = \emptyset \lor A_{\alpha} = A_{\beta}
	\]
	\[
	  \textcolor{teal}{\forall \alpha, \beta \in I: A_{\alpha} \cap A_{\beta} \neq \emptyset \implies A_{\alpha} = A_{\beta}}
	\]
    \end{enumerate}
  \end{definition}
\end{frame}
%%%%%%%%%%%%%%%

%%%%%%%%%%%%%%%
\begin{frame}{}
  \begin{columns}
    \column{0.50\textwidth}
      \fignocaption{width = 0.85\textwidth}{figs/world-map}
    \column{0.50\textwidth}
      \pause
      \fignocaption{width = 0.80\textwidth}{figs/china-map}
  \end{columns}
\end{frame}
%%%%%%%%%%%%%%%

%%%%%%%%%%%%%%%
\begin{frame}{}
  \begin{exampleblock}{Partitions of $\mathbb{R}^{3}$ (UD $11.3$)}
    Is $\set{A_r: r \in \mathbb{R}}$ a partition of $\mathbb{R}^{3}$?

    \[
      A_{r} = \set{(x,y,z) \in \mathbb{R}^{3}: x + y + z = r}
    \]
  \end{exampleblock}

  \pause
  \fignocaption{width = 0.32\textwidth}{figs/xyzr}
\end{frame}
%%%%%%%%%%%%%%%

%%%%%%%%%%%%%%%
\begin{frame}{}
  \begin{exampleblock}{Partitions of $\mathbb{R}^{3}$ (UD $11.3$)}
    Is $\set{A_r: r \in \mathbb{R}}$ a partition of $\mathbb{R}^{3}$?

    \[
      A_{r} = \set{(x,y,z) \in \mathbb{R}^{3}: x^2 + y^2 + z^2 = r^2}
    \]
  \end{exampleblock}

  \pause
  \fignocaption{width = 0.35\textwidth}{figs/x2y2z2r2}
\end{frame}
%%%%%%%%%%%%%%%

%%%%%%%%%%%%%%%
\begin{frame}{}
  \begin{exampleblock}{Partitions of the Set of Polynomials (UD $11.7$)}
    \[
      p(x) = a_n x^n + a_{n-1} x^{n-1} + \cdots + a_1 x^1 + a_0 \quad (a_j \in \mathbb{R}, n \in \mathbb{N})
    \]
    \[
      \text{deg}(p = 0) = -\infty
    \]

    \begin{enumerate}[(a)]
      \item
	\[
	  A_m = \set{p: \text{deg}(p) = m} \quad m \in \mathbb{N}
	\]
      \setcounter{enumi}{2}
      \item 
	\[
	  A_q = \set{p: \exists r (p = qr)} \quad q \in P
	\]
	\pause
	\vspace{-0.20cm}
	\[
	  \textcolor{teal}{q \in A_q}
	\]
	\vspace{-0.30cm}
	\[
	  \textcolor{teal}{p \in A_p}
	\]
	\pause
	\vspace{-0.30cm}
	\[
	  \textcolor{teal}{p \neq q \land r = pq \implies (r \in A_q \cap A_q) \land (A_p \neq A_q)}
	\]
    \end{enumerate}
  \end{exampleblock}
\end{frame}
%%%%%%%%%%%%%%%

%%%%%%%%%%%%%%%
\begin{frame}{}
  \begin{exampleblock}{Partitions of the Set of Polynomials (UD $11.7$)}
    \[
      p(x) = a_n x^n + a_{n-1} x^{n-1} + \cdots + a_1 x^1 + a_0 \quad (a_j \in \mathbb{R}, n \in \mathbb{N})
    \]
    \[
      \text{deg}(p = 0) = -\infty
    \]

    \begin{enumerate}[(a)]
      \setcounter{enumi}{1}
      \item 
	\[
	  A_c = \set{p: p(0) = c} \quad c \in \mathbb{R}
	\]
      \setcounter{enumi}{3}
      \item 
	\[
	  A_c = \set{p: p(c) = 0} \quad c \in \mathbb{R}
	\]
	\pause
	\vspace{-0.20cm}
	\[
	  \textcolor{teal}{p(x) = x^2 + 1}
	\]
    \end{enumerate}
  \end{exampleblock}
\end{frame}
%%%%%%%%%%%%%%%

%%%%%%%%%%%%%%%
\begin{frame}{}
  \begin{exampleblock}{Subset and Partition (UD $11.9$)}
    $\set{A_{\alpha}: \alpha \in I}$ is a partition of $X \neq \emptyset$.
    \begin{enumerate}[(a)]
      \item 
	\[
	  B \subseteq X, \quad \forall \alpha \in I: A_{\alpha} \cap B \neq \emptyset
	\]

	To prove that
	\[
	  \set{A_{\alpha} \cap B: \alpha \in I} \text{ \red{\it is} a partition of } B.
	\]
    \end{enumerate}
  \end{exampleblock}

  \fignocaption{width = 0.50\textwidth}{figs/subset-partition}
\end{frame}
%%%%%%%%%%%%%%%
