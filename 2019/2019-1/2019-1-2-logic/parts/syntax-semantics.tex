% syntax-semantics.tex

%%%%%%%%%%%%%%%
\begin{frame}{}
  \begin{exampleblock}{UD 2.5: 命题逻辑公式的语义}
    \[
      P \to \lnot (Q \land \lnot P)
    \]

    \vspace{0.40cm}
    \begin{center}
      {\red{\large 真值表 (truth table)}}
    \end{center}
  \end{exampleblock}

  \vspace{0.60cm}
  \begin{center}
    命题逻辑公式的语义就是它的真值表,与原子命题的真假有关。
  \end{center}
\end{frame}
%%%%%%%%%%%%%%%

%%%%%%%%%%%%%%%
\begin{frame}{}
  \begin{exampleblock}{一阶谓词逻辑公式的语义}
    \[
      L = \set{<}
    \]
    \[
      \psi: \forall x \exists y \; (y < x)
    \]

    \pause
    \begin{center}
      {\red{$Q:$} $\psi$ 是真是假?}
    \end{center}
  \end{exampleblock}

  \pause
  \[
    \mathcal{U} = \mathbb{N}
  \]
  \pause
  \[
    \mathcal{U} = \mathbb{Z}
  \]

  \pause
  \begin{center}
    一阶谓词逻辑公式的语义与它的\red{\bf 结构 (Structure)}有关。
  \end{center}
\end{frame}
%%%%%%%%%%%%%%%

%%%%%%%%%%%%%%%
\begin{frame}{}
  \begin{center}
    一阶谓词逻辑中的重言式
  \end{center}

  \[
    \Big(\forall y \lnot P(y) \to \lnot P(x)\Big) \to \Big(P(x) \to \exists y P(y)\Big)
  \]

  \[
    \Big(\forall x (\alpha \to \beta)\Big) \to (\forall x \alpha \to \forall x \beta)
  \]
\end{frame}
%%%%%%%%%%%%%%%
