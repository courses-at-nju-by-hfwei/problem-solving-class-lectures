\documentclass{beamer}
\usepackage{lmodern}

\usetheme{CambridgeUS} % try Madrid, Pittsburgh
\usecolortheme{beaver}
\usefonttheme[onlymath]{serif} % try "professionalfonts"

% \setbeamertemplate{itemize items}[default]
% \setbeamertemplate{enumerate items}[default]

\usepackage{amsmath, amsfonts, latexsym}
\DeclareMathOperator*{\argmin}{arg\,min}
\DeclareMathOperator*{\argmax}{arg\,max}

% colors
\newcommand{\red}[1]{\textcolor{red}{#1}}
\newcommand{\green}[1]{\textcolor{green}{#1}}
\newcommand{\blue}[1]{\textcolor{blue}{#1}}
\newcommand{\purple}[1]{\textcolor{purple}{#1}}

\usepackage{tikz}
% see http://tex.stackexchange.com/a/7045/23098
\newcommand*\circled[1]{\tikz[baseline=(char.base)]{
            \node[shape=circle,draw,inner sep=2pt] (char) {#1};}}

\usepackage{pifont}

\usepackage[normalem]{ulem}
\newcommand{\middlewave}[1]{\raisebox{0.5em}{\uwave{\hspace{#1}}}}

\usepackage{graphicx, subcaption}

\usepackage{algorithm}
\usepackage[noend]{algpseudocode}

\newcommand{\pno}[1]{\textcolor{blue}{\scriptsize [Problem: #1]}}
\newcommand{\set}[1]{\{#1\}}

\newcommand{\cmark}{\textcolor{red}{\ding{51}}}
\newcommand{\xmark}{\textcolor{red}{\ding{55}}}
%%%%%%%%%%%%%%%%%%%%%%%%%%%%%%%%%%%%%%%%%%%%%%%%%%%%%%%%%%%%%%
% for fig without caption: #1: width/size; #2: fig file
\newcommand{\fignocaption}[2]{
  \begin{figure}[htp]
    \centering
      \includegraphics[#1]{#2}
  \end{figure}
}

% for fig with caption: #1: width/size; #2: fig file; #3: fig caption
\newcommand{\fig}[3]{
  \begin{figure}[htp]
    \centering
      \includegraphics[#1]{#2}
      \caption[labelInTOC]{#3}
  \end{figure}
}

\newcommand{\titletext}{Linear Programming}
%%%%%%%%%%%%%%%%%%%%
\title[\titletext]{\titletext}
\subtitle{}

\author[Hengfeng Wei]{Hengfeng Wei}
\institute{hengxin0912@gmail.com}
\date{Feb 23, 2017}

\AtBeginSection[]{
  \begin{frame}[noframenumbering, plain]
    \frametitle{\titletext}
    \tableofcontents[currentsection, sectionstyle=show/shaded, subsectionstyle=show/show/hide]
  \end{frame}
}

%%%%%%%%%%
\begin{document}
\maketitle

\section{Formulation}

%%%%%%%%%%%%%%%%%%%%
\begin{frame}{Linear programming}
  \begin{equation*}
	\setlength\arraycolsep{2.5pt}
	\begin{array}{l@{\quad}l}
	  \max/\min 	& \text{linear function } f \text{ on } x	\\[5pt]
	  \text{subject to} 	&	\\
					&	\text{linear constraints } (\geq, =, \leq)
	\end{array}
  \end{equation*}

  \vspace{0.80cm}

  Mathematical programming:
  \begin{itemize}
	\item multi-objective
	\item non-linear objective/constraints
	\item integral variables
  \end{itemize}
\end{frame}
%%%%%%%%%%%%%%%%%%%%
\begin{frame}{Linear programming}
  \begin{columns}
	\column{0.50\textwidth}
	  \begin{equation*}
		\setlength\arraycolsep{2.5pt}
		\begin{array}{l@{\quad}rccl}
		  \red{\boxed{\max}} 	& \sum_{j=1}^{n} c_j x_j 	&	&	&	\\[5pt]
		  \text{s.t.} 	&	&	&	&	\\[5pt]
		  &	\sum_{j=1}^{n} a_{ij} x_j &	\red{\boxed{\leq}}	&	b_i	&	i = 1 \dots m \\[10pt]
		  &	\red{\boxed{x_j}} 		& \geq 	&	0	&	j = 1 \dots n
		\end{array}
	  \end{equation*}
	\column{0.50\textwidth}
	  \begin{equation*}
		\setlength\arraycolsep{2.5pt}
		\begin{array}{l@{\quad}rcc}
		  \max 	& c^{T}x	&	&	\\[5pt]
		  \text{s.t.} 	&	&	&	\\[5pt]
				&	Ax 	&	\leq	&	b	\\[10pt]
				&	x	& 	\geq 	&	0	
		\end{array}
	  \end{equation*}
  \end{columns}

  \vspace{0.50cm}

  \[
	\sum_{j=1}^{n} a_{ij} x_j \leq b_i	\iff b_i - \sum_{j=1}^{n} a_{ij} x_j \geq 0
  \]

  \[
	x_{n+i} = b_i - \sum_{j=1}^{n} a_{ij} x_j \quad x_{n+i} \geq 0
  \]
\end{frame}
%%%%%%%%%%%%%%%%%%%%

\section{Primal and Dual}

%%%%%%%%%%%%%%%%%%%%
\begin{frame}{Primal-dual}
  \begin{columns}
	\column{0.50\textwidth}
	  \begin{equation*}
		\setlength\arraycolsep{2.5pt}
		\begin{array}{l@{\quad}rcc}
		  \max 	& c^{T}x	&	&	\\[5pt]
		  \text{s.t.} 	&	&	&	\\[5pt]
				&	Ax 	&	\leq	&	b	\\[10pt]
				&	x	& 	\geq 	&	0	
		\end{array}
	  \end{equation*}
	\column{0.50\textwidth}
	  \begin{equation*}
		\setlength\arraycolsep{2.5pt}
		\begin{array}{l@{\quad}rcc}
		  \min 	& b^{T}y	&	&	\\[5pt]
		  \text{s.t.} 	&	&	&	\\[5pt]
				&	A^{T}y 	&	\geq	&	c	\\[10pt]
				&	y	& 	\geq 	&	0	
		\end{array}
	  \end{equation*}
  \end{columns}
\end{frame}
%%%%%%%%%%%%%%%%%%%%
\begin{frame}{Primal-dual}
  \begin{equation*}
	\setlength\arraycolsep{2.5pt}
	\begin{array}{l@{\quad}rcrcrcr}
	  \max 	& 3x_1  &+&	x_2		&+&	2x_3	&	&	\\[5pt]
	  \text{s.t.} 	&	&	&	&	&	&	&	\\[5pt]
			& x_1   	&+&	x_2	&+&	3x_3	&\leq& 30 \\[10pt]
			& 2x_1	&+&	2x_2 	&+& 5x_3 	&\leq& 24 \\[10pt]
			& 4x_1   	&+&	x_2		&+&	2x_3	&\leq& 36	\\[10pt]
			& x_1,	&&	x_2,	&&	x_3	&\geq&	0
	\end{array}
  \end{equation*}

  \vspace{0.60cm}

  \[
	x^{\ast} = (8, 4, 0) \qquad\qquad\quad v^{\ast} = 28
  \]
\end{frame}
%%%%%%%%%%%%%%%%%%%%
\begin{frame}{The multiplier approach}
  \begin{equation*}
	\setlength\arraycolsep{5pt}
	\begin{array}{rll}
	  \circled{1} + \circled{2} &\Rightarrow	3x_1 + 3x_2 + 8x_3 & \leq 54 \\[5pt]
	  \circled{1} + \frac{1}{2} \times \circled{3} &\Rightarrow 3x_1 + \frac{3}{2}x_2 + 4x_3 & \leq 48 \\[5pt]
	  \circled{1} + \frac{1}{2} \times \circled{2} &\Rightarrow 2x_1 + 2x_2 + \frac{11}{2}x_3 & \text{\xmark{}} \\[5pt]
	  % \frac{1}{2} \times \circled{2} + \frac{1}{2} \times \circled{3}	&\Rightarrow \\
	  \red{0} \times \circled{1} + \red{\frac{1}{6}} \times \circled{2} + \red{\frac{2}{3}} \times \circled{3} &\Rightarrow 3x_1 + x_2 + \frac{13}{6}x_3 & \leq 28
	\end{array}
  \end{equation*}

  \begin{align*}
	3x_1 &+ x_2 + 2x_3	\\
	  &\rbox{\leq} y_1 \times \circled{1} + y_2 \times \circled{2} + y_3 \times \circled{3} \\
	  &= (y_1 + 2y_2 + 4y_3) x_1 + (y_1 + 2y_2 + y_3) x_2 + (3y_1 + 5y_2 + 2y_3)  \\
	  &\leq 30y_1 + 24y_2 + 36y_3
  \end{align*}
\end{frame}
%%%%%%%%%%%%%%%%%%%%
\begin{frame}{Primal-dual \pno{29.4-2}}
  \begin{columns}
	\column{0.50\textwidth}
	  \begin{equation*}
		\setlength\arraycolsep{2.5pt}
		\begin{array}{l@{\quad}rcrcrcr}
		  \max 	& 3x_1  &+&	x_2		&+&	2x_3	&	&	\\[5pt]
		  \text{s.t.} 	&	&	&	&	&	&	&	\\[5pt]
				& x_1   &+&	x_2		&+&	3x_3	&\leq& 30 \\[10pt]
				& 2x_1	&+&	2x_2 	&+& 5x_3 	&\geq& 24 \\[10pt]
				& 4x_1  &+&	x_2		&+&	2x_3	&=& 36	\\[10pt]
				& &&	&&	x_1 &\geq&	0	\\[10pt]
				& &&	&&	x_2 &\geq&	0
		\end{array}
	  \end{equation*}
	\column{0.50\textwidth}
	  \begin{equation*}
		\setlength\arraycolsep{2.5pt}
		\begin{array}{l@{\quad}rcrcrcr}
		  \min 	& 30y_1  &+&	24y_2		&+&	36y_3	&	&	\\[5pt]
		  \text{s.t.} 	&	&	&	&	&	&	&	\\[5pt]
				& y_1   &+&	2y_2	&+&	4y_3	&\geq& 3 \\[10pt]
				& y_1	&+&	2y_2 	&+& y_3 	&\geq& 1 \\[10pt]
				& 3y_1  &+&	5y_2	&+&	2y_3	&=& 2	\\[10pt]
				& &&	&&	y_1 &\geq&	0	\\[10pt]
				& &&	&&	y_2 &\leq&	0
		\end{array}
	  \end{equation*}
  \end{columns}
\end{frame}
%%%%%%%%%%%%%%%%%%%%
\begin{frame}{Weak/strong duality theorems}
  \begin{theorem}[Weak duality (29.8)]
	\[
	  c^{T} x \leq b^{T} y	\quad (\forall x, y)
	\]
  \end{theorem}

  \begin{corollary}[29.9]
	\[
	  c^{T} x \leq b^{T} y \Rightarrow x \text{ optimal to primal; } y \text{ optimal to dual.} 
	\]
  \end{corollary}

  \begin{theorem}[Strong duality (29.10)]
	If an LP has a \red{\boxed{bounded}} optimal solution $x^{\ast}$, then
	\begin{enumerate}
	  \item the dual has a bounded optimal solution $y^{\ast}$
	  \item $c^{T} x^{\ast} = b^{T} y^{\ast}$
	\end{enumerate}
  \end{theorem}

  \centerline{\purple{Remark:} P is unbounded $\Rightarrow$ D is infeasible.}
\end{frame}
%%%%%%%%%%%%%%%%%%%%
\begin{frame}{Linear-inequality feasibility \pno{29-1}}
  \[
	LP \Rightarrow LF
  \]

  \[
	\max 0	\quad \green{\smiley}	\qquad \qquad \max \; -x_0\; \text{(Ch 29.5)}
  \]

  \centerline{\rule{0.6\textwidth}{.4pt}}

  \[
	LF \Rightarrow LP
  \]

  \vspace{0.30cm}

  \begin{columns}
	\column{0.50\textwidth}
	  \begin{equation*}
		\setlength\arraycolsep{2.5pt}
		\begin{array}{l@{\quad}rcc}
		  \max 	& c^{T}x	&	&	\\[5pt]
		  \text{s.t.} 	&	&	&	\\[5pt]
		  &	Ax 	&	\leq	&	b	\\[10pt]
		  &	x	& 	\geq 	&	0	
		\end{array}
	  \end{equation*}
	\column{0.50\textwidth}
	  \begin{enumerate}
		\item feasible?
		\item unbounded?
		\item finite optimal
	  \end{enumerate}
  \end{columns}
\end{frame}
%%%%%%%%%%%%%%%%%%%%
\begin{frame}{Linear-inequality feasibility}
  Binary search from $c^{T} x = 0$:
  \[
	-2\quad -1\quad -\frac{1}{2}\quad \rbox{0}\quad \frac{1}{2}\quad 1\quad 2\quad 4
  \]

  \[
	c^{T} x \geq 0 \begin{cases}
	  c^T x \geq 2^0	\begin{cases}
		c^Tx \geq 2^1 \\
		c^Tx \geq 2^{-1}
	  \end{cases} \\
	  c^T x \geq -2^0	\begin{cases}
		c^Tx \geq -2^{-1} \\
		c^Tx \geq -2^1
	  \end{cases}
	\end{cases}
  \]

  \begin{center}
	Termination?\\
	Approximation?
  \end{center}
\end{frame}
%%%%%%%%%%%%%%%%%%%%
\begin{frame}{Linear-inequality feasibility}
  \begin{columns}
	\column{0.50\textwidth}
	  \begin{equation*}
		\setlength\arraycolsep{2.5pt}
		\begin{array}{l@{\quad}rcc}
		  \max 	& c^{T}x	&	&	\\[5pt]
		  \text{s.t.} 	&	&	&	\\[5pt]
				&	Ax 	&	\leq	&	b	\\[10pt]
				&	x	& 	\geq 	&	0	
		\end{array}
	  \end{equation*}
	\column{0.50\textwidth}
	  \begin{equation*}
		\setlength\arraycolsep{2.5pt}
		\begin{array}{l@{\quad}rcc}
		  \min 	& b^{T}y	&	&	\\[5pt]
		  \text{s.t.} 	&	&	&	\\[5pt]
				&	A^{T}y 	&	\geq	&	c	\\[10pt]
				&	y	& 	\geq 	&	0	
		\end{array}
	  \end{equation*}
  \end{columns}

  \begin{align*}
	b^{T} y &\red{\boxed{\leq}} c^{T} x	\\
	Ax \leq b	&	\quad	A^{T} y \geq c	\\
	x \geq 0	&	\quad 	y \geq 0
  \end{align*}

  \begin{center}
	\purple{Remark:} What if this LF is infeasible?
  \end{center}
\end{frame}
%%%%%%%%%%%%%%%%%%%%
%%%%%%%%%%%%%%%%%%%%
%%%%%%%%%%%%%%%%%%%%
%%%%%%%%%%%%%%%%%%%%

\section{SSSP}

%%%%%%%%%%%%%%%%%%%%
\begin{frame}{SPSP (Ch 29.2)}
  \begin{equation*}
	\setlength\arraycolsep{2.5pt}
	\begin{array}{l@{\quad}rcl}
	  \color{red}{\boxed{\max}} 	& d_t	& &	\\[5pt]
	  \text{s.t.} 	& &	\\[5pt]
			& d_v   &\leq& d_u + w(u,v) \quad \forall (u,v) \in E	\\[10pt]
			& d_s	&=& 0 
	\end{array}
  \end{equation*}

  \vspace{0.60cm}

  \begin{equation*}
	\setlength\arraycolsep{2.5pt}
	\begin{array}{l@{\quad}l}
	  Q_{1}:	& \min \; d_t	\\
	  Q_{2}:	& d_v \geq 0 \quad \forall v \in V \\
	  Q_{3}:	& d_v \leq d_u + w(u,v)
	\end{array}
  \end{equation*}
\end{frame}
%%%%%%%%%%%%%%%%%%%%
\begin{frame}{SPSP}
  \begin{columns}
	\column{0.50\textwidth}
	  \begin{equation*}
		\setlength\arraycolsep{2.5pt}
		\begin{array}{l@{\quad}l}
		  \min	& w(P)		\\[5pt]
		  \text{s.t.} 		\\[5pt]
		  & P: s \leadsto t
		\end{array}
	  \end{equation*}
	\column{0.50\textwidth}
	  \begin{equation*}
		\setlength\arraycolsep{2.5pt}
		\begin{array}{l@{\quad}l}
		  \min	& \sum_{(u,v) \in E} w_{uv} \cdot x_{uv}		\\[5pt]
		  \text{s.t.} 		\\[5pt]
		  & P: s \leadsto t	\\[10pt]
		  &	\red{\boxed{x_{uv} = \set{0,1}}}	\quad \forall (u,v) \in E
		\end{array}
	  \end{equation*}
  \end{columns}

  \vspace{0.80cm}

  \[
	 \text{in}(v) - \text{out}(v) 
	 =	\sum_{u} x_{uv} - \sum_{u} x_{vu} 
	 = \left\{\begin{array}{rl}
	  -1,	&	v = s	\\
	  1,	&	v = t	\\
	  0,	&	\text{o.w.}
	\end{array}\right.
  \]
\end{frame}
%%%%%%%%%%%%%%%%%%%%
\begin{frame}{SPSP}
  \[
	x_{12}\;\; x_{14}\;\;	x_{23}\;\;	x_{24}\;\;	x_{31}\;\;	x_{43}
  \]

  \begin{equation*}
	\begin{pmatrix}
	  -1	&	-1	&	&	&	1	&	\\
	  1		&		&	-1	&	-1	&	\\
	  		&		&	1	&	1	&	-1	\\
	 		&	1	&	&	1	&	&	-1
	\end{pmatrix}
	\begin{pmatrix}
	  x_{12}\\x_{14}\\x_{23}\\x_{24}\\x_{31}\\x_{43}
	\end{pmatrix}
	= 
	\begin{pmatrix}
	  -1\\0\\0\\1
	\end{pmatrix}
  \end{equation*}
\end{frame}
%%%%%%%%%%%%%%%%%%%%
\begin{frame}{SPSP}
  \begin{align*}
	\sum_{(u,v) \in E} w_{uv} \cdot x_{uv} & \red{\boxed{\geq}} (d_2 - d_s) x_{12} + (d_t - d_s) x_{14} + \dots	\\
	&	=	\sum_{(u,v) \in E} (d_v - d_u) x_{uv}	\\
	&	=	d_t - d_s
  \end{align*}

  \[
	d_v - d_u \leq w(u,v) \iff d_v \leq d_u + w(u,v)
  \]
\end{frame}
%%%%%%%%%%%%%%%%%%%%
\begin{frame}{SPSP: explanation}
  \begin{align*}
	&	d_v \leq d_u + w(u,v)	\quad \forall u: u \to v	\\[6pt]
	& 	\iff	d_v \leq \min_{u: u \to v} d_u + w(u,v)	\\[6pt]
	&	\xLeftrightarrow{\max d_v}	d_v = \min_{u: u \to v} d_u + w(u,v)
  \end{align*}

  \vspace{0.50cm}

  \begin{center}
	Physical ball-string model: \red{PULL} it!
  \end{center}
\end{frame}
%%%%%%%%%%%%%%%%%%%%
\begin{frame}{SSSP}
  \begin{equation*}
	\setlength\arraycolsep{2.5pt}
	\begin{array}{l@{\quad}rcl}
	  \color{red}{\boxed{\max}} 	& \sum_{t} d_t	& &	\\[5pt]
	  \text{s.t.} 	& &	\\[5pt]
			& d_v   &\leq& d_u + w(u,v) \quad \forall (u,v) \in E	\\[10pt]
			& d_s	&=& 0 
	\end{array}
  \end{equation*}

  \vspace{0.60cm}

  \[
	\max \sum_{t} d_t \iff \max \set{d_t \mid t \in V}
  \]

  \begin{Proof}
	\begin{itemize}
	  \item ``$\Rightarrow$:''
	  \item ``$\Leftarrow$:'' $\max d_i$ never forces us to decrease $d_j$.
	\end{itemize}
  \end{Proof}

\end{frame}
%%%%%%%%%%%%%%%%%%%%
\begin{frame}
  \[
	 \text{in}(v) - \text{out}(v) 
	 =	\sum_{u} x_{uv} - \sum_{u} x_{vu} 
	 = \left\{\begin{array}{rl}
	  -1,	&	v = s	\\
	  1,	&	v = t	\\
	  1,	&	\text{o.w.}
	\end{array}\right.
  \]
	
  \centerline{Simplex method vs. Dijkstra's alg \& Bellman-Ford alg?}
\end{frame}
%%%%%%%%%%%%%%%%%%%%
\section{Game}

%%%%%%%%%%%%%%%%%%%%
\begin{frame}{Game}
  \begin{description}
	\item[Alice:] (e)conomy, (m)edicine
	\item[Bob:] (t)ax, (i)mmigration
	\item[Votes:] Alice gains vs. Bob loses
	\item[Strategy:] pure vs. mixed
  \end{description}

  \vspace{0.20cm}

  \begin{center}
	$G=$ \begin{tabular}{|c|cc|}
	  \hline
	  &	$t\; (y_1)$	&	$i\; (y_2)$	\\ \hline
	  $e\; (x_1)$	&	$3$	&	$-1$	\\
	  $m\; (x_2)$	&	$-2$	&	$1$	\\ \hline
	\end{tabular}
  \end{center}

  \[
	\sum_{i,j} G_{ij} \cdot \mathbb{P}\set{A_i, B_j} = \sum_{i,j} G_{ij} x_i y_j
  \]

  \begin{center}
	\red{Q: Who Announce First?}\\[20pt]
  \end{center}
\end{frame}
%%%%%%%%%%%%%%%%%%%%
\begin{frame}{Game}
  \begin{columns}
	\column{0.50\textwidth}
	  \begin{equation*}
		\setlength\arraycolsep{2.5pt}
		\begin{array}{l@{\quad}r}
		  \max &	\boxed{\qquad}	\\
		  \text{s.t.}	&	\\
		  		&	x_1 + x_2 = 1 \\
				&	x_1 \geq 0	\\
				&	x_2 \geq 0
		\end{array}
	  \end{equation*}
	  \begin{align*}
		z &= \min \set{3x_1 - 2x_2, -x_1 + x_2} \\
		&\iff \left\{\begin{array}{l}
		  \max z \\
		  z \leq 3x_1 - 2x_2 \\
		  z \leq -x_1 + x_2
		\end{array}\right.
	  \end{align*}
	\column{0.50\textwidth}
	  \begin{equation*}
		\setlength\arraycolsep{2.5pt}
		\begin{array}{l@{\quad}r}
		  \min &	\boxed{\qquad}	\\
		  \text{s.t.}	&	\\
		  		&	y_1 + y_2 = 1 \\
				&	y_1 \geq 0	\\
				&	y_2 \geq 0
		\end{array}
	  \end{equation*}
	  \begin{align*}
		w &= \max \set{3y_1 - y_2, -2y_1 + y_2} \\
		&\iff \left\{\begin{array}{l}
		  \min w \\
		  w \geq 3y_1 - y_2 \\
		  w \geq -2y_1 + y_2
		\end{array}\right.
	  \end{align*}
  \end{columns}
\end{frame}
%%%%%%%%%%%%%%%%%%%%
\begin{frame}{Game}
  \begin{columns}
	\column{0.50\textwidth}
	  \begin{equation*}
		\setlength\arraycolsep{2.5pt}
		\begin{array}{l@{\quad}rcrcrcr}
		  \max 	& z &	&	&	&	&	&	\\[5pt]
		  \text{s.t.} 	&	&	&	&	&	&	&	\\[5pt]
		  &	-3x_1 &+&	2x_2	&+&	z	&\leq& 0 \\[5pt]
		  &	x_1	  &-&	x_2 	&+& z 	&\leq& 0 \\[5pt]
		  & x_1   &+&	x_2		& &		&=&	1 \\[5pt]
		  &		&&				&&	x_1 &\geq& 0 \\[5pt]
		  &		&&				&&	x_2 &\geq& 0
		\end{array}
	  \end{equation*}
	  \[
		x^{\ast} = (\frac{3}{7}, \frac{4}{7}), \quad z = \frac{1}{7}
	  \]
	\column{0.50\textwidth}
	  \begin{equation*}
		\setlength\arraycolsep{2.5pt}
		\begin{array}{l@{\quad}rcrcrcr}
		  \min 	& w &	&	&	&	&	&	\\[5pt]
		  \text{s.t.} 	&	&	&	&	&	&	&	\\[5pt]
		  &	-3y_1 &+&	y_2		&+&	w	&\geq& 0 \\[5pt]
		  &	2y_1  &-&	y_2 	&+& w 	&\geq& 0 \\[5pt]
		  & y_1   &+&	y_2		& &		&=&	1 \\[5pt]
		  &		&&				&&	y_1 &\geq& 0 \\[5pt]
		  &		&&				&&	y_2 &\geq& 0
		\end{array}
	  \end{equation*}
	  \[
		y^{\ast} = (\frac{2}{7}, \frac{5}{7}), \quad w = \frac{1}{7}
	  \]
  \end{columns}

  \vspace{0.50cm}
  \[
	\boxed{\max_{x} \min_{y} \sum_{i,j} G_{ij} x_i y_j \rbox{=} \min_{y} \max_{x} \sum_{i,j} G_{ij} x_i y_j}
  \]
\end{frame}
%%%%%%%%%%%%%%%%%%%%


\end{document}
%%%%%%%%%%