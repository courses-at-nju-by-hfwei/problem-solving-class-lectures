% file: sections/menger-theorem.tex

%%%%%%%%%%%%%%%%%%%%
\begin{frame}
  \begin{theorem}[Menger's Theorem (Theorem $5.16$)]
    \begin{center}
      Let $u$ and $v$ be \purple{nonadjacent} vertices in a graph $G$. \\[5pt]
      The \red{minimum number of vertices in a $u-v$ separating set} \\
      equals the \blue{maximum number of internally disjoint $u-v$ paths in $G$}.
    \end{center}
  \end{theorem}

  \pause
  \begin{center}
    How do \textsc{Case 1}, \textsc{Case 2}, and \textsc{Case 3} cover all possibilities? \\[8pt] \pause
    Are \textsc{Case 1} and \textsc{Case 2} mutually exclusive? \\[4pt] \pause
    What is the key to use the induction hypothesis in \textsc{Case 2}? \\[8pt] \pause
    Are \textsc{Case 1} and \textsc{Case 3} mutually exclusive? \\[4pt] \pause
    What will fail if we do not exclude \textsc{Case 1} from \textsc{Case 3}? \\[8pt] \pause
    Can you restate these three cases in terms of $N(u)$ and $N(v)$? \\[4pt] \pause
    Can you rearrange these three cases to make them (hopefully) easier to understand?
  \end{center}
\end{frame}
%%%%%%%%%%%%%%%%%%%%

%%%%%%%%%%%%%%%%%%%%
\begin{frame}
  \begin{center}
    \red{By induction on the number $m$ of edges of $G$.} \pause
  \end{center}

  \pause
  \begin{description}
    \item[\textsc{Case I}:] There exists a minimum $u-v$ separating set $W$ in $G$ 
      containing \blue{a vertex $x$ that is adjacent to both $u$ and $v$}.
  \end{description}

  \fig{width = 0.40\textwidth}{figs/menger-theorem-case-I}

  \pause
  \[
    W - \set{x} \text{ is a minimum } u-v \text{ separating set in } G - x
  \]
\end{frame}
%%%%%%%%%%%%%%%%%%%%

%%%%%%%%%%%%%%%%%%%%
\begin{frame}
  \begin{center}
    \red{By induction on the number $m$ of edges of $G$.}
  \end{center}

  \begin{description}
    \item[\textsc{Case II}:] There exists a minimum $u-v$ separating set $W$ in $G$
      containing \blue{a vertex in $W$ that is not adjacent to $u$} \\
      and \blue{a vertex in $W$ that is not adjacent to $v$}.
  \end{description}

  \begin{columns}
    \column{0.30\textwidth}
      \fig{width = 0.90\textwidth}{figs/menger-theorem-case-II}
    \column{0.40\textwidth}
      \pause
      \fig{width = 0.80\textwidth}{figs/menger-theorem-case-II-gv}
      \uncover<3->{
	\[ 
	  \red{\boxed{m(G'_v) < m(G)}}
	\]
      }
    \column{0.40\textwidth}
      \uncover<4->{
	\fig{width = 0.80\textwidth}{figs/menger-theorem-case-II-gu}
	\[ 
	  \red{\boxed{m(G'_u) < m(G)}}
	\]
      }
  \end{columns}
\end{frame}
%%%%%%%%%%%%%%%%%%%%

%%%%%%%%%%%%%%%%%%%%
\begin{frame}
  \begin{center}
    \red{By induction on the number $m$ of edges of $G$.}
  \end{center}

  \begin{description}
    \item[\textsc{Case III}:] For each minimum $u-v$ separating set $W$ in $G$,
      either \blue{every vertex of $W$ is adjacent to $u$ and not adjacent to $v$}
      or \blue{every vertex of $W$ is adjacent to $v$ and not adjacent to $u$}.
  \end{description}

  \begin{columns}
    \column{0.50\textwidth}
      \fig{width = 0.80\textwidth}{figs/menger-theorem-case-III}
    \column{0.50\textwidth}
      \pause
      \begin{center}
	\red{$\boxed{P = u, x, y, \ldots, v}$ \\[5pt] A $u-v$ shortest simple path in $G$}
      \end{center}
      \pause
      \[
	m(G - e) < m(G)
      \]
      \begin{center}
	A minimum $u-v$ separating set \\ in $G - e$ contains $k$ vertices.
      \end{center}
  \end{columns}
\end{frame}
%%%%%%%%%%%%%%%%%%%%

%%%%%%%%%%%%%%%%%%%%
\begin{frame}
  \begin{description}
    \item[\textsc{Case I}:] There exists a minimum $u-v$ separating set $W$ in $G$ 
      containing \purple{a vertex $x$ that is adjacent to both $u$ and $v$}.
      \only<2->{
	\[
	  \exists W: \exists x \in W: x-u \land x-v
	\]
      }
    \item[\textsc{Case II}:] There exists a minimum $u-v$ separating set $W$ in $G$
      containing \purple{a vertex in $W$ that is not adjacent to $u$} \\
      and \purple{a vertex in $W$ that is not adjacent to $v$}.
      \only<3->{
	\begin{align*}
	  \exists W:\; &\exists x \in W: x \nadj u \\
	  	     \land\; &\exists y \in W: y \nadj v
	\end{align*}
      }
    \item[\textsc{Case III}:] For each minimum $u-v$ separating set $W$ in $G$,
      either \purple{every vertex of $W$ is adjacent to $u$ and not adjacent to $v$}
      or \purple{every vertex of $W$ is adjacent to $v$ and not adjacent to $u$}.
      \only<4->{
	\begin{align*}
	  \forall W:\; &\forall x \in W: x - u \land x \nadj v \\
	  	     \lor\; &\forall x \in W: x - v \land x \nadj u 
	\end{align*}
      }
  \end{description}
\end{frame}
%%%%%%%%%%%%%%%%%%%%

%%%%%%%%%%%%%%%%%%%%
\begin{frame}
  \begin{columns}[t]
    \column{0.50\textwidth}
      \[
	\teal{\textsc{I}:}\; \exists W: \exists x \in W: x-u \land x-v
      \]

      \begin{align*}
	\teal{\textsc{II}:}\;
	\exists W:\; &\exists x \in W: x \nadj u \\
		   \land\; &\exists y \in W: y \nadj v
      \end{align*}


      \begin{align*}
	\teal{\textsc{III}:}\;
	\forall W:\; &\forall x \in W: x - u \land x \nadj v \\
		   \lor\; &\forall x \in W: x - v \land x \nadj u 
      \end{align*}
  \column{0.50\textwidth}
    \uncover<3->{
      \[
	\red{\textsc{I}':}\; \forall W: \forall x \in W: x \nadj u \lor x \nadj v
      \]
    }

    \uncover<2->{
      \begin{align*}
	\red{\textsc{II}':}\;
	\forall W:\; &\forall x \in W: x-u\\
		   \lor\; &\forall y \in W: y - v
      \end{align*}
    }

    \uncover<4->{
      \[
	\teal{\textsc{III}} \equiv \red{\textsc{II}'} \land \red{\textsc{I}'}
      \]
    }
  \end{columns}

  \vspace{-0.50cm}
  \only<5->{
    \begin{columns}
      \column{0.10\textwidth}
	\begin{description}[$\red{\textsc{II}'}$]
	  \item[$\teal{\textsc{II}}$]
	  \item[$\red{\textsc{II}'}$] \textcolor{white}{dummy}
	    \vspace{-1.00cm}
	    \[
	      \teal{\textsc{I}}
	    \]
	    \vspace{-0.60cm}
	    \[
	      \teal{\textsc{III}}
	    \]
	\end{description}
    \end{columns}
  }
\end{frame}
%%%%%%%%%%%%%%%%%%%%

%%%%%%%%%%%%%%%%%%%%
\begin{frame}
  \begin{columns}[t]
    \column{0.50\textwidth}
      \begin{description}[$\red{\textsc{II}'}$]
	\item[$\teal{\textsc{II}}$]
	  \begin{align*}
	    \teal{\textsc{II}:}\;
	    \exists W:\; &\exists x \in W: x \nadj u \\
		       \land\; &\exists y \in W: y \nadj v
	  \end{align*}
	\item[$\red{\textsc{II}'}$] \textcolor{white}{dummy}
	  \[
	    \teal{\textsc{I}:}\; \exists W: \exists x \in W: x-u \land x-v
	  \]

	  \begin{align*}
	    \teal{\textsc{III}:}\;
	    \forall W:\; &\forall x \in W: x - u \land x \nadj v \\
		       \lor\; &\forall x \in W: x - v \land x \nadj u 
	  \end{align*}
    \end{description}
    \column{0.50\textwidth}
      \pause
      \begin{description}[$\red{\textsc{II}'}$]
	\item[$\teal{\textsc{II}}$]
	  \begin{align*}
	    % \teal{\textsc{II}:}\; 
	    \exists W:\; &W \nsubseteq N(u) \\
		  \land\; &W \nsubseteq N(v)
	  \end{align*}
	\item[$\red{\textsc{II}'}$] \textcolor{white}{dummy}
	  \pause
	  \[
	    % \teal{\textsc{I}:}\; 
	    \exists W: \exists x \in W: x \in N(u) \cap N(v)
	  \]

	  \pause
	  \begin{align*}
	    % \teal{\textsc{III}:}\; 
	    \forall W: \;&W \subseteq N(u) \land W \cap N(v) = \emptyset \\
		  \lor\; &W \subseteq N(v) \land W \cap N(u) = \emptyset
	  \end{align*}
      \end{description}
  \end{columns}
\end{frame}
%%%%%%%%%%%%%%%%%%%%

%%%%%%%%%%%%%%%%%%%%
\begin{frame}
  \begin{description}[$\red{\textsc{II}'}$]
    \item[$\teal{\textsc{II}}$]
      \begin{align*}
	\teal{\textsc{II}:}\; 
	\exists W:\; &W \nsubseteq N(u) \\
	      \land\; &W \nsubseteq N(v)
      \end{align*}

      % \uncover<2->{
      %   \begin{center}
      %     \red{$Q:$ What is the key to use the induction hypothesis in \textsc{Case II}?}
      %   \end{center}
      % }
    \item[$\red{\textsc{II}'}$] \textcolor{white}{dummy}
      \[
	\teal{\textsc{I}:}\; 
	\exists W: \exists x \in W: x \in N(u) \cap N(v)
      \]

      \begin{align*}
	\teal{\textsc{III}:}\; 
	\forall W: \;&W \subseteq N(u) \land W \cap N(v) = \emptyset \\
	      \lor\; &W \subseteq N(v) \land W \cap N(u) = \emptyset
      \end{align*}

      % \uncover<3->{
      %   \begin{center}
      %     \red{$Q:$ What will fail if we do not exclude \textsc{Case I} from \textsc{Case III}?}
      %   \end{center}
      % }
  \end{description}
\end{frame}
%%%%%%%%%%%%%%%%%%%%

%%%%%%%%%%%%%%%%%%%%
\begin{frame}{}
  \begin{theorem}[Menger's Theorem for Edge-Connectivity (Theorem $5.21$)]
    \begin{center}
      For distinct vertices $u$ and $v$ in a graph $G$, \\[5pt]
      the \red{minimum number of edges of $G$ that separate $u$ and $v$} \\
      equals the \blue{maximum number of pairwise edge-disjoint $u-v$ paths in $G$}.
    \end{center}
  \end{theorem}

  \pause
  \begin{center}
    \red{Line Graph}
  \end{center}

  \vspace{-0.30cm}
  \begin{columns}
    \column{0.25\textwidth}
      \fig{width = 0.80\textwidth}{figs/line-graph-1}
    \column{0.25\textwidth}
      \fig{width = 0.80\textwidth}{figs/line-graph-2}
    \column{0.25\textwidth}
      \fig{width = 0.80\textwidth}{figs/line-graph-3}
    \column{0.25\textwidth}
      \fig{width = 0.80\textwidth}{figs/line-graph-4}
  \end{columns}

  \pause
  \begin{center}
    Definition $4.2.18$ \& Theorem $4.2.19$ \\
    of ``Introduction to Graph Theory'' by Douglas B. West
  \end{center}
\end{frame}
%%%%%%%%%%%%%%%%%%%%
