% file: parts/solving-recurrences.tex

%%%%%%%%%%%%%%%
\begin{frame}{}
  \centerline{\teal{\Large Solving Recurrence}}

  \fignocaption{width = 0.50\textwidth}{figs/pencil-paper}
\end{frame}
%%%%%%%%%%%%%%%

%%%%%%%%%%%%%%%
\begin{frame}{}
  \fignocaption{width = 0.60\textwidth}{figs/feedback}

  Why again?
\end{frame}
%%%%%%%%%%%%%%%

%%%%%%%%%%%%%%%
\begin{frame}{}
  \fignocaption{width = 0.60\textwidth}{figs/feedback}

  base cases
\end{frame}
%%%%%%%%%%%%%%%

%%%%%%%%%%%%%%%
\begin{frame}{}
  \begin{exampleblock}{First-order Linear Recurrence (CS $4.2-11$)}
    \[
      T(n) = \left\{\begin{array}{lr}
	1		& n = 0 \\
	2T(n-1) + n 2^n & n > 0 
      \end{array}\right.
    \]
  \end{exampleblock}

  \pause
  \vspace{0.50cm}
  \begin{theorem}[First-order Linear Recurrences with \red{Constant Coefficients} (CS Theorem $4.5$)]
    \[
      T(n) = \left\{\begin{array}{lr}
	a		& n = 0 \\
	\red{r}T(n-1) + g(n) & n > 0 
      \end{array}\right.
    \]

    \pause
    \vspace{0.30cm}
    \[
      \boxed{\teal{T(n) = r^n a + \sum_{i=1}^{n} r^{n-i} g(i)}}
    \]
  \end{theorem}
\end{frame}
%%%%%%%%%%%%%%%

%%%%%%%%%%%%%%%
\begin{frame}{}
  \begin{theorem}[First-order Linear Recurrences]
    \[
      T(n) = \red{x_n} T(n-1) + y_n  \quad \text{ for } n > 0 \text{ with } T(0) = 0
    \]

    \[
      \boxed{\teal{T(n) = y_0 + \sum_{1 \le j < n} y_j x_{j+1} x_{j+2} \cdots x_n}}
    \]
  \end{theorem}

  \pause
  \begin{proof}
    \[
      \frac{T(n)}{\underbrace{x_n x_{n-1} \cdots x_1}_{\text{\red{summation factor}}}}
      	= \frac{T(n-1)}{x_{n-1} \cdots x_1} + \frac{y_n}{x_n x_{n-1} \cdots x_1}
    \]

    \pause
    \[
      S(n) \triangleq \frac{T(n)}{x_n x_{n-1} \cdots x_1}
    \]
  \end{proof}
\end{frame}
%%%%%%%%%%%%%%%

%%%%%%%%%%%%%%%
\begin{frame}{}
  \[
    T(n) = (1 + \frac{1}{n}) T(n-1) + 2  \quad \text{ for } n > 1 \text{ with } T(1) = 0
  \]

  \pause
  \[
    \blue{x_n = 1 + \frac{1}{n}} \implies \pause x_n x_{n-1} \cdots x_1 = n + 1
  \]

  \pause
  \[
    \frac{T(n)}{n+1} = \frac{T(n-1)}{n} + \frac{2}{n+1} \quad \text{ for } n > 1
  \]

  \pause
  \vspace{0.30cm}
  \[
    \frac{T(n)}{n+1} = \frac{T(1)}{2} + 2 \sum_{3 \le k \le n+1} \frac{1}{k}
  \]

  \pause
  \vspace{0.30cm}
  \[
    \boxed{\teal{T(n) = 2(n+1) (H_{n+1} - \frac{3}{2})}}
  \]
\end{frame}
%%%%%%%%%%%%%%%

%%%%%%%%%%%%%%%
\begin{frame}{}
  \begin{exampleblock}{After-class Exercise}
    \[
      T(n) = T(n-1) - \frac{2T(n-1)}{n} + 2\left(1 - \frac{2T(n-1)}{n}\right), n > 0 \text{ with } T(0) = 0
    \]
  \end{exampleblock}

  \begin{columns}
    \pause
    \column{0.50\textwidth}
      \fignocaption{width = 0.60\textwidth}{figs/do-it-yourself}
    \column{0.50\textwidth}
      \fignocaption{width = 0.80\textwidth}{figs/maweidu}
  \end{columns}
\end{frame}
%%%%%%%%%%%%%%%

%%%%%%%%%%%%%%%
\begin{frame}{}
  \begin{theorem}[Linear Recurrences with Constant Coefficients]
    \[
      a_n = r_1 a_{n-1} + r_2 a_{n-2} + \cdots + r_t a_{n-t} \quad \text{for } n \ge t
    \]
    \[
      a_0, a_1, \cdots, a_{t-1}
    \]

    \pause
    \[
      \boxed{\teal{q(x) \;\red{\equiv}\; x^t - r_1 x^{t-1} - r_2 x^{t-2} - \cdots - r_t}}
    \]

    \pause
    \[
      \beta_1 (m_1), \beta_2 (m_2), \cdots, \beta_i (m_i), \cdots, \beta_{k} (m_k)
    \]
    \[
      m_1 + m_2 + \cdots + m_k = t
    \]

    \pause
    \[
      \boxed{\red{a_n = \sum_{0 \le j < m_1} c_{1j} n^j \beta_{1}^{n} + \sum_{0 \le j < m_2} c_{2j} n^j \beta_{2}^{n} + \cdots 
	+ \sum_{0 \le j < m_k} c_{kj} n^j \beta_{k}^{n}}}
    \]
  \end{theorem}
\end{frame}
%%%%%%%%%%%%%%%

%%%%%%%%%%%%%%%
\begin{frame}{}
  \[
    a_n = r_1 a_{n-1} + r_2 a_{n-2} + \cdots + r_t a_{n-t} \quad \text{for } n \ge t
  \]

  \[
    \boxed{\teal{q(x) \;\red{\equiv}\; x^t - r_1 x^{t-1} - r_2 x^{t-2} - \cdots - r_t}}
  \]

  \pause
  \begin{proof}
    \[
      \red{\beta \; (m = 2)}
    \]

    \pause
    \[
      \beta^n = r_1 \beta^{n-1} + r_2 \beta^{n-2} + \cdots + r_t \beta^{n-t} \quad \text{for } n \ge t
    \]
    \pause
    \vspace{-0.50cm}
    \[
      \beta^{n-t} q(\beta) = 0
    \]

    \pause
    \vspace{-0.50cm}
    \[
      n\beta^{n} = r_1 (n-1) \beta^{n-1} + r_2 (n-2) \beta^{n-2} + \cdots + r_t (n-t) \beta^{n-t} \quad \text{for } n \ge t
    \]
    \pause
    \vspace{-0.50cm}
    \[
      \beta^{n-t} \left(\left(n-t\right)q(\beta) + \beta q'(\beta)\right) = 0
    \]
  \end{proof}
\end{frame}
%%%%%%%%%%%%%%%

%%%%%%%%%%%%%%%
\begin{frame}{}
  \[
    a_n = 5a_{n-1} - 6a_{n-2} \; n \ge 2 \; (a_0 = 0, a_1 = 1)
  \]

  \pause
  \[
    \boxed{\teal{x^2 - 5x + 6 = (x-2) (x-3) = 0}} \implies x = 2, 3
  \]

  \pause
  \[
    \boxed{\teal{a_n = c_0 3^n + c_1 2^n}}
  \]
  
  \pause
  \begin{align*}
    a_0 &= 0 = c_0 + c_1 \\
    a_1 &= 1 = 3c_0 + 2c_1
  \end{align*}

  \pause
  \[
    \boxed{\red{a_n = 3^n - 2^n}}
  \]
\end{frame}
%%%%%%%%%%%%%%%

%%%%%%%%%%%%%%%
\begin{frame}{}
  \[
    a_n = 5 a_{n-1} - 8 a_{n-2} + 4 a_{n-3}, n \ge 3 \; (a_0 = 0, a_1 = 1, a_2 = 4)
  \]

  \pause
  \[
    \boxed{x^3 - 5 x^2 + 8 x - 4 = 0}
  \]

  \pause
  \[
    (x-1) (x-2)^2 = 0 \implies x_1 = 1, x_{2} = 2, x'_{2} = 2
  \]

  \pause
  \[
    a_n = c_1 \cdot 1^{n} + c_{2} \cdot 2^{n} + c'_{2} \cdot n 2^{n}
  \]

  \pause
  \vspace{-0.30cm}
  \begin{align*}
    a_0 &= 0 = c_1 + c_2 \\
    a_1 &= 1 = c_1 + 2c_2 + 2c'_{2} \\
    a_2 &= 4 = c_1 + 4c_2 + 8c'_{2}
  \end{align*}

  \pause
  \[
    \boxed{a_n = n 2^{n-1}}
  \]
\end{frame}
%%%%%%%%%%%%%%%

%%%%%%%%%%%%%%%
\begin{frame}{}
  \[
    a_n = 2a_{n-1} - a_{n-2} + 2a_{n-3},\; n \ge 3 \; (a_0 = 1, a_1 = 0, a_2 = -1)
  \]

  \pause
  \[
    \boxed{\teal{x^3 - 2x^2 + x - 2}} = (x^2 + 1) (x-2) = 0 \implies x = 2, \red{i}, \red{-i}
  \]

  \pause
  \[
    \boxed{\teal{a_n = c_1 2^n + c_2 i^n + c_3 (-i)^n}}
  \]

  \pause
  \vspace{-0.50cm}
  \begin{align*}
    a_0 &= 1 = c_1 + c_2 + c_3 \\
    a_1 &= 0 = 2c_1 + c_2 i - c_3 i \\
    a_2 &= -1 = 4c_1 - c_2 - c_3
  \end{align*}

  \pause
  \vspace{-0.30cm}
  \[
    \boxed{\red{a_n = \frac{1}{2} i^{n} \left(1 + (-1)^n \right)}}
  \]

  \pause
  \[
    \blue{1},\; \red{0},\; \blue{-1},\; \red{0},\; \blue{1},\; \red{0},\; \blue{-1},\; \red{0},\;
    \blue{1},\; \red{0},\; \blue{-1},\; \red{0} \cdots
  \]
\end{frame}
%%%%%%%%%%%%%%%

%%%%%%%%%%%%%%%
\begin{frame}{}
  \begin{exampleblock}{After-class Exercise}
    Given initial conditions $a_0, a_1$, and $a_2$ for which the growth rate of the solution to
    \[
      a_n = 2a_{n-1} + a_{n-2} - 2a_{n-3},\; n > 2
    \]
    is 
    \begin{enumerate}[(i)]
      \item constant,
      \item exponential,
      \item fluctuating in sign
    \end{enumerate}
  \end{exampleblock}

  \begin{columns}
    \pause
    \column{0.50\textwidth}
      \fignocaption{width = 0.60\textwidth}{figs/do-it-yourself}
    \column{0.50\textwidth}
      \fignocaption{width = 0.80\textwidth}{figs/maweidu}
  \end{columns}
\end{frame}
%%%%%%%%%%%%%%%

%%%%%%%%%%%%%%%
\begin{frame}{}
  \centerline{\teal{\large First-order Linear Non-homogeneous Recurrences}}

  \[
    a_n = r_1 a_{n-1} + r_2 a_{n-2} + \cdots + r_t a_{n-t} + \red{r} \quad \text{for } n \ge t
  \]

  \pause
  \fignocaption{width = 0.50\textwidth}{figs/fib-noip}
  \[
    F(n) = F(n-1) + F(n-2) + 2, \; n \ge 3\; (F(1) = F(2) = 0)
  \]

  \pause
  \vspace{-0.30cm}
  \[
    \boxed{T(n,k) = \left\{\begin{array}{lr}
      0, & k = 0 \lor n = k \\
      T(n-1, k) + T(n-1, k-1) + c, & \text{\it o.w.}
    \end{array}\right.}
  \]
\end{frame}
%%%%%%%%%%%%%%%

%%%%%%%%%%%%%%%
\begin{frame}{}
  \[
    a_n = 5a_{n-1} - 6a_{n-2} + 2, \; n \ge 2 \; (a_0 = 0, a_1 = 1)
  \]

  \pause
  \[
    \boxed{\teal{a_n = c_0 3^n + c_1 2^n + \red{c_2}}}
  \]

  \pause
  \[
    \boxed{\teal{c_2 = 5c_2 - 6c_2 + 2}} \implies c_2 = 1
  \]

  \pause
  \begin{align*}
    a_0 &= 0 = c_0 + c_1 + 1\\
    a_1 &= 1 = 3c_0 + 2c_1 + 1
  \end{align*}

  \pause
  \[
    \red{\boxed{a_n = 2 \cdot 3^n - 3 \cdot 2^n + 1}}
  \]
\end{frame}
%%%%%%%%%%%%%%%

%%%%%%%%%%%%%%%
\begin{frame}{}
  \centerline{\teal{\Large More Issues about Linear Recurrences}}
\end{frame}
%%%%%%%%%%%%%%%

%%%%%%%%%%%%%%%
\begin{frame}{}
  \[
    a_n = r_1 a_{n-1} + r_2 a_{n-2} + \cdots + r_t a_{n-t} + \red{g(n)} \quad \text{for } n \ge t
  \]

  \pause
\end{frame}
%%%%%%%%%%%%%%%

%%%%%%%%%%%%%%%
\begin{frame}{}
  \[
    a_n = r_1 a_{n-1} + r_2 a_{n-2} + \cdots + r_t a_{n-t} \quad \text{for } n \ge t
  \]

  \pause
  \[
    t \ge 5
  \]

  \pause
  \begin{columns}
    \column{0.50\textwidth}
    \column{0.50\textwidth}
  \end{columns}

  \pause
  \vspace{0.50cm}
  \centerline{\red{Generating Functions and Asymptotic Analysis}}
\end{frame}
%%%%%%%%%%%%%%%

%%%%%%%%%%%%%%%
\begin{frame}{}
  \[
    a_n = f_1(n) a_{n-1} + f_2(n) a_{n-2} + \cdots + f_t(n) a_{n-t} \quad \text{for } n \ge t
  \]

  \pause
  \vspace{0.80cm}
  \centerline{\red{Genearating Functions}}
\end{frame}
%%%%%%%%%%%%%%%

%%%%%%%%%%%%%%%
\begin{frame}{}
  \fignocaption{width = 0.40\textwidth}{figs/AoA}
\end{frame}
%%%%%%%%%%%%%%%
