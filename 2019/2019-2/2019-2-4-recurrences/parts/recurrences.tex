% recurrences.tex

%%%%%%%%%%%%%%%
\begin{frame}{}
  \centerline{\teal{\Large Recurrences}}
  \fig{width = 0.60\textwidth}{figs/fractal-tree}
\end{frame}
%%%%%%%%%%%%%%%

%%%%%%%%%%%%%%%
\begin{frame}{}
  \[
    T(n) = aT(n/b) + f(n) \quad (a > 0, b > 1) 
  \]

  \pause
  \centerline{\red{Assume that $T(n)$ is constant for sufficiently small $n$.}}

  \pause
  \vspace{0.30cm}
  \begin{equation*}
    \left.\begin{aligned}
      f(n) \\
      af(\frac{n}{b}) \\
      a^2f(\frac{n}{b^2}) \\
      \vdots \\
	  a^{\log_b n} \red{T}(1) = \Theta(n^{\log_b a})
    \end{aligned}\right\}
    \pause
    \red{\sum} \pause\;\; \blue{\overset{f(n) \text{\it vs. } n^{E}}{=}} \pause
	\left\{\begin{aligned}
      n^{\log_b a},		 \\
      n^{\log_b a}\log n, 	 \\
      f(n),		   \quad \\
    \end{aligned}\right.
    \quad \left.\begin{aligned}
      f(n) &= O(n^{E-\epsilon}) \\
      f(n) &= \Theta(n^E) \\
      f(n) &= \Omega(n^{E+\epsilon}) 
    \end{aligned}\right.
  \end{equation*}
\end{frame}
%%%%%%%%%%%%%%%

%%%%%%%%%%%%%%%
\begin{frame}{}
  \begin{columns}
    \column{0.30\textwidth}
      \begin{exampleblock}{Solving Recurrences (Problem $2.15$)}
	\begin{enumerate}[(1)]
	  \item $\Theta(n^{\log_3 2})$
	  \item \textcolor{red}{$\Theta(\log^2 n)$}
	  \item $\Theta(n)$
	  \item $\Theta(n \log n)$
	  \item \textcolor{red}{$\Theta(n \log^2 n)$}
	  \item $\Theta(n^2)$
	  \item $\Theta(n^{\frac{3}{2}}\log n)$
	  \item $\Theta(n)$
	  \item \teal{$\Theta(n^{c+1})$}
	  \item \teal{$\Theta(c^{n+1})$}
	  \item \blue{$\cdots$}
	\end{enumerate}
      \end{exampleblock}
    \column{0.65\textwidth}
      \[
		T(n) = T(n/2) + \log n
      \]
      \[
		T(n) = 2T(n/2) + n\log n
      \]

      \pause
      \vspace{0.20cm}
      \begin{alertblock}{Reference:}
		\vspace{-0.30cm}
		\[
		  f(n) = \Theta(n^{\log_b a} \log^{\red{k}}n) \implies \Theta(n^{\log_b a} \log^{\red{k+1}}n)
		\]
      \end{alertblock}

      \pause
      \vspace{0.50cm}
      \begin{exampleblock}{Gaps in Master Theorem (Problem $2.18$)}
		\pause
		\[
		  T(n) = 2T(n/2) + \frac{n}{\log n} \pause = \Theta(n \log \log n)
		\]
      \end{exampleblock}
  \end{columns}
\end{frame}
%%%%%%%%%%%%%%%

%%%%%%%%%%%%%%%
\begin{frame}{}
  \begin{columns}
    \column{0.30\textwidth}
      \begin{exampleblock}{Solving Recurrences (Problem $2.15$)}
	\begin{enumerate}[(1)]
	  \item $\Theta(n^{\log_3 2})$
	  \item \textcolor{red}{$\Theta(\log^2 n)$}
	  \item $\Theta(n)$
	  \item $\Theta(n \log n)$
	  \item \textcolor{red}{$\Theta(n \log^2 n)$}
	  \item $\Theta(n^2)$
	  \item $\Theta(n^{\frac{3}{2}}\log n)$
	  \item $\Theta(n)$
	  \item \teal{$\Theta(n^{c+1})$}
	  \item \teal{$\Theta(c^{n+1})$}
	  \item \purple{$\cdots$}
	\end{enumerate}
      \end{exampleblock}
    \column{0.65\textwidth}
      \[
		T(n) = T(n-1) + c^n \quad c > 1
      \]

      \[
		T(n) = T(n-1) + n^c \quad c \ge 1
      \]
	  \pause
	  \[
		\int
	  \]
	  \pause
	  \[
		(\frac{n}{2}) \cdot (\frac{n}{2})^{c} \le T(n) \le n \cdot n^c
	  \]
  \end{columns}
\end{frame}
%%%%%%%%%%%%%%%

%%%%%%%%%%%%%%%
\begin{frame}[fragile]{}
  \begin{exampleblock}{Solving Recurrences (Problem $2.15\; (11)$)}
    \[
      T(n) = T(n/2) + T(n/4) + T(n/8)
    \]
  \end{exampleblock}

  \pause
  \vspace{0.50cm}
  \fig{width = 0.40\textwidth}{figs/kidding}
  \[
    \red{\text{Where is } f(n)?}
  \]
\end{frame}
%%%%%%%%%%%%%%%

%%%%%%%%%%%%%%%
\begin{frame}[fragile]{}
  \begin{exampleblock}{Solving Recurrences (Problem $2.15\; (11)$)}
    \[
      T(n) = T(n/2) + T(n/4) + T(n/8)
    \]
  \end{exampleblock}

  \pause
  \vspace{0.30cm}
  \[
    \boxed{\teal{T(n) = \Theta(n^{0.879146})}}
  \]

  \pause
  \vspace{0.30cm}
  \[
    \boxed{\red{T(n) = \Theta(n^\alpha)}}
  \]

  \pause
  \vspace{0.30cm}
  \[
    2^{-\alpha} + 4^{-\alpha} + 8^{-\alpha} = 1
  \]

  \pause
  \vspace{0.30cm}
  \begin{center}
    \begin{verbatim}
      Solve[2^{-x} + 4^{-x} + 8^{-x} == 1, x] // N
    \end{verbatim}
  \end{center}
\end{frame}
%%%%%%%%%%%%%%%

%%%%%%%%%%%%%%%
\begin{frame}{}
  \begin{exampleblock}{Solving Recurrences (Problem $2.15\; (11)$)}
    \[
      T(n) = T(n/2) + T(n/4) + T(n/8) \;\red{+\; n}
    \]
  \end{exampleblock}

  \pause
  \vspace{0.30cm}
  \centerline{By recursion-tree.}
  \pause
  \[
    \boxed{\teal{T(n) = \Theta(n)}}
  \]

  \pause
  \vspace{0.50cm}
  \centerline{\red{Exercise: Prove it by mathematical induction.}}

  \pause
  \begin{alertblock}{Reference:}
    {\it ``On the Solution of Linear Recurrence Equations''} by Akra \& Bazzi, 1996.

    \[
      T(n) = \sum_{i=1}^{k} a_i T(n/b_i) + f(n)
    \]
  \end{alertblock}
\end{frame}
%%%%%%%%%%%%%%%

%%%%%%%%%%%%%%%
% \begin{frame}{Gaps (Problem 1.2.16)}
%   \[
%     T(n) = 2T(n/2) + \frac{n}{\log n} \pause = \Theta(n \log \log n)
%   \]
% 
%   \pause
% 
%   The regularity condition in Case 3:
%   \[
% 	bf(n/c) \le cf(n), \text{ for some } c < 1 \text{ and sufficiently large } n
%   \]
% 
%   \[
% 	T(n) = T(n/2) + n(2 - \cos n)
%   \]
% 
%   \[
% 	n^{E} = n^0 \quad f(n) = n(2 - \cos n) = \Omega(n^{0 + \epsilon})
%   \]
% 
%   \pause
%   \[
% 	n = 2\pi k (k \text{ odd}) \Rightarrow c \ge \frac{3}{2}
%   \]
% \end{frame}
%%%%%%%%%%%%%%%

%%%%%%%%%%%%%%%
\begin{frame}{}
  \begin{exampleblock}{Solving Recurrences (Problem $2.17$)}
    \begin{align*} 
      \text{T}(n) &= \sqrt{n}\ \text{T}(\sqrt{n})+n \\
	  &= \teal{n^{\frac{1}{2}}\ \text{T}\left(n^{\frac{1}{2}} \right)+n} \\
	  \onslide<2->{
		&= n^{\frac{1}{2}}\left( n^{\frac{1}{2^2}}\ \text{T}\left(n^{\frac{1}{2^2}} \right) + n^{\frac{1}{2}} \right)+n\\
		&= \teal{n^{\frac{1}{2}+\frac{1}{2^2}}\ \text{T}\left(n^{\frac{1}{2^2}}\right) + 2n} \\
	  }
	  \onslide<3->{
		&= n^{\frac{1}{2}+\frac{1}{2^2}}\left(n^{\frac{1}{2^3}}\ \text{T}\left(n^{\frac{1}{2^3}}\right) + n^{\frac{1}{2^2}} \right) + 2n\\
		&= \teal{n^{\frac{1}{2}+\frac{1}{2^2}+\frac{1}{2^3}}\ \text{T}\left(n^{\frac{1}{2^3}}\right) + 3n} \\ 
	  }
	  \onslide<4->{
		&= \cdots \\ 
		&= \red{n^{\sum_{i=1}^{k}\frac{1}{2^i}}\ \text{T}\left(n^{\frac{1}{2^k}}\right) + kn} \\ 
	  }
    \end{align*}
  \end{exampleblock}
\end{frame}
%%%%%%%%%%%%%%%

%%%%%%%%%%%%%%%
\begin{frame}{}
  \[
    \text{T}(n) = \red{n^{\sum_{i=1}^{k}\frac{1}{2^i}}\ \text{T}\left(n^{\frac{1}{2^k}}\right) + kn}
  \]

  \fig{width = 0.30\textwidth}{figs/slow-down-zoo}

  \pause
  \[
    \red{n^{\frac{1}{2^k}} = 1}
  \]
\end{frame}
%%%%%%%%%%%%%%%

%%%%%%%%%%%%%%%
\begin{frame}{}
  \[
    n^{\frac{1}{2^k}} = \red{2} \pause \implies k = \log\log n
  \]

  \pause
  \vspace{-0.30cm}
  \begin{align*} 
    \text{T}(n) &= \red{n^{\sum_{i=1}^{k}\frac{1}{2^i}}\ \text{T}\left(n^{\frac{1}{2^k}}\right) + kn}\\ 
	    &=n^{\sum_{i=1}^{\log\log n}\frac{1}{2^i}}\ \text{T}(2) + n \log\log n\\ 
  \end{align*}

  \pause
  \vspace{-0.30cm}
  \[
    \sum_{i=1}^{\log \log n}\frac{1}{2^i} < 1 \implies T(n) = \Theta(n \log \log n)
  \]

  \pause
  \vspace{0.40cm}
  \centerline{\red{Exercise: Prove it by mathematical induction.}}
\end{frame}
%%%%%%%%%%%%%%%

%%%%%%%%%%%%%%%
\begin{frame}{}
  \[
    \boxed{T(n) = \sqrt{n} T(\sqrt{n}) + n}
  \]

  \pause
  \[
    \frac{T(n)}{n} = \frac{T(\sqrt{n})}{\sqrt{n}} + 1
  \]

  \pause
  \[
    \red{n \leftrightarrow 2^m}
  \]

  \pause
  \[
    \frac{T(2^m)}{2^m} = \frac{T(2^{m/2})}{2^{m/2}} + 1
  \]

  \pause
  \[
    \red{S(m) \leftrightarrow \frac{T(2^m)}{2^m}}
  \]

  \pause
  \[
    S(m) = S(m/2) + 1 = \pause \Theta(\log m)
  \]

  \pause
  \vspace{-0.60cm}
  \[
    \boxed{\teal{T(n) = n \log \log n}}
  \]
\end{frame}
%%%%%%%%%%%%%%%