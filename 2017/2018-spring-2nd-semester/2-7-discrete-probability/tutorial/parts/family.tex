% file: parts/family.tex

%%%%%%%%%%%%%%%
\begin{frame}{}
  \centerline{\teal{\Large The Boy/Girl Puzzle}}
  \vspace{0.30cm}
  \fignocaption{width = 0.60\textwidth}{figs/family}
\end{frame}
%%%%%%%%%%%%%%%

%%%%%%%%%%%%%%%
\begin{frame}{}
  \begin{exampleblock}{Both Girls (CS Problem $5.3-12$)}
    Mr. and Mrs. Smith have two children of different ages,\\
    what is the probability that they has \red{two girls},
    \begin{enumerate}[(a)]
      \item given that \violet{one of the children} is a girl?
      \item given that \violet{the older child} is a girl?
    \end{enumerate}
  \end{exampleblock}

  \fignocaption{width = 0.35\textwidth}{figs/girl}
\end{frame}
%%%%%%%%%%%%%%%

%%%%%%%%%%%%%%%
\begin{frame}{}
  \[
    G_1: \text{the older child is a girl}
  \]
  \[
    G_2: \text{the younger child is a girl}
  \]

  \begin{align*}
    \onslide<2->{\teal{\Pr\set{G_1 \land G_2 \mid G_1 \lor G_2}}} 
    \onslide<3->{&= \frac{\Pr{\set{G_1 \land G_2}}}{\Pr{\set{G_1 \lor G_2}}} \\}
    \onslide<4->{&= \frac{\Pr\set{G_1 \land G_2}}{\Pr\set{G_1} + \Pr\set{G_2} - \Pr\set{G_1 \land G_2}} \\}
    \onslide<5->{&= \frac{1/4}{3/4} = \frac{1}{3}}
  \end{align*}

  \begin{align*}
    \onslide<6->{\teal{\Pr\set{G_1 \land G_2 \mid G_1}}}
    \onslide<7->{= \frac{\Pr{\set{G_1 \land G_2}}}{\Pr\set{G_1}}}
    \onslide<8->{= \frac{1/4}{1/2} = \frac{1}{2}}
  \end{align*}
\end{frame}
%%%%%%%%%%%%%%%

%%%%%%%%%%%%%%%
\begin{frame}{}
  \[
    \boxed{(G_1, G_2), (G_1, B_2), (B_1, G_2), (B_1, B_2)}
  \]

  \pause
  \[
    \teal{\Pr\set{G_1 \land G_2 \mid G_1 \lor G_2} = \frac{1}{3}}
  \]

  \pause
  \[
    \red{(G_1, G_2)}, (G_1, B_2), (B_1, G_2), \lgray{(B_1, B_2)}
  \]

  \pause
  \[
    \teal{\Pr\set{G_1 \land G_2 \mid G_1} = \frac{1}{2}}
  \]

  \pause
  \[
    \red{(G_1, G_2)}, (G_1, B_2), \lgray{(B_1, G_2)}, \lgray{(B_1, B_2)}
  \]
\end{frame}
%%%%%%%%%%%%%%%

%%%%%%%%%%%%%%%
\begin{frame}{}
  \[
    \teal{\Pr\set{G_1 \land G_2 \mid G_1 \lor G_2} = \frac{1}{3}}
  \]
  \[
    \teal{\Pr\set{G_1 \land G_2 \mid G_1} = \frac{1}{2}}
  \]

  \pause
  \[
    \Pr\set{G_1 \land G_2 \mid G_1 \lor G_2} = \red{\boxed{\frac{2}{3}}} \Pr\set{G_1 \land G_2 \mid G_1}
  \]

  \pause
  \[
    \Pr\set{G_1 \land G_2 \mid G_1 \lor G_2} = \red{\boxed{\Pr\set{G_1 \mid G_1 \lor G_2}}} \Pr\set{G_1 \land G_2 \mid G_1}
  \]

  \pause
  \[
    \Pr\set{G_1 \mid G_1 \lor G_2} = \frac{\Pr\set{G_1 \land (G_1 \lor G_2)}}{\Pr\set{G_1 \lor G_2}} 
    \pause = \frac{\Pr\set{G_1}}{\Pr\set{G_1 \land G_2}} = \frac{2}{3}
  \]
\end{frame}
%%%%%%%%%%%%%%%

%%%%%%%%%%%%%%%
\begin{frame}{}
  \fignocaption{width = 0.40\textwidth}{figs/easy}

  \pause
  \fignocaption{width = 0.25\textwidth}{figs/keep-calm-not-finished}
\end{frame}
%%%%%%%%%%%%%%%

%%%%%%%%%%%%%%%
\begin{frame}{}
  \begin{exampleblock}{Both Girls (CS Problem $5.3-12$)}
    Mr. and Mrs. Smith have two children of different ages,\\
    what is the probability that they has \red{two girls},
    \begin{enumerate}[(a)]
      \item given that \violet{one of the children} is a girl?
    \end{enumerate}
  \end{exampleblock}

  \pause
  \vspace{0.50cm}
  \centerline{\red{\large $Q:$ {\bf How} do you know that \violet{``one of the children is a girl''}?}}
\end{frame}
%%%%%%%%%%%%%%%

%%%%%%%%%%%%%%%
\begin{frame}{}
  \centerline{\red{\large $Q:$ {\bf How} do you know that \violet{``one of the children is a girl''}?}}

  \pause
  \vspace{0.30cm}
  \begin{columns}
    \column{0.25\textwidth}
      \fignocaption{width = 0.85\textwidth}{figs/house1}
    \column{0.25\textwidth}
      \fignocaption{width = 0.75\textwidth}{figs/house0}
    \column{0.25\textwidth}
      \fignocaption{width = 0.75\textwidth}{figs/house2}
    \column{0.25\textwidth}
      \fignocaption{width = 0.90\textwidth}{figs/house3}
  \end{columns}

  \pause
  \vspace{0.80cm}
  \begin{enumerate}[(I)]
    \setlength{\itemsep}{8pt}
    \item From all families with two children, at least one of whom is a girl, \\
      \red{a {\footnotesize (Smith's)} family is chosen at random}.
    \pause
    \item From all families with two children, \red{one child {\footnotesize (of Smith)} is selected} \\
      at random that happens to be a girl.
  \end{enumerate}
\end{frame}
%%%%%%%%%%%%%%%

%%%%%%%%%%%%%%%
\begin{frame}{}
  \centerline{\red{\large $Q:$ {\bf How} do you know that \violet{``one of the children is a girl''}?}}

  \pause
  \vspace{0.30cm}
  \fignocaption{width = 0.45\textwidth}{figs/two-rooms}

  \pause
  \begin{enumerate}[(I)]
    \setlength{\itemsep}{8pt}
    \item I \purple{\textsc{know}} them well and I tell you that ``one of the children is a girl''.
    \pause
    \item I \purple{\textsc{don't know}} them. I just open a room door and see a girl.
  \end{enumerate}
\end{frame}
%%%%%%%%%%%%%%%

%%%%%%%%%%%%%%%
\begin{frame}{}
  \centerline{\teal{\large The Monty-Hall Problem Comes Back}}

  \fignocaption{width = 0.45\textwidth}{figs/monty-hall-problem}
\end{frame}
%%%%%%%%%%%%%%%

%%%%%%%%%%%%%%%
\begin{frame}{}
  \centerline{\red{\large $Q:$ {\bf How} do you know that \violet{``one of the children is a girl''}?}}

  \vspace{0.60cm}
  \begin{enumerate}[(I)]
    \setcounter{enumi}{1}
    \item \red{$g:$} From all families with two children, \red{one child {\footnotesize (of Smith)} is selected} \\
      at random that happens to be a girl.
  \end{enumerate}

  \pause
  \vspace{0.50cm}
  \[
    \Pr\set{G_1 \land G_2 \mid g} = \frac{\set{G_1 \land G_2 \land g}}{\Pr\set{g}} \pause = \frac{1/4}{1/2} = \frac{1}{2}
  \]
\end{frame}
%%%%%%%%%%%%%%%

%%%%%%%%%%%%%%%
\begin{frame}{}
  \begin{exampleblock}{After-class Exercise:}
    A new couple, \teal{known to have two children}, has just moved into town. \\
    Suppose that the mother is \teal{encountered walking with one of her children}. \\
    If this child is a girl, what is the probability that \red{both children are girls}?
  \end{exampleblock}

  \vspace{0.50cm}
  \fignocaption{width = 0.40\textwidth}{figs/mother-girl-walking.png}
\end{frame}
%%%%%%%%%%%%%%%
