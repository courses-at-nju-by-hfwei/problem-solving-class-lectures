\section{Approximation Algorithms}

%%%%%%%%%%%%%%%%%%%%%%%%%%%%%%
% \subsection{Makespan Scheduling Problem}

%%%%%%%%%%%%%%%%%%%%
\begin{frame}{Makespan scheduling problem}
  \begin{exampleblock}{Makespan scheduling problem (\ms{})}
	\begin{itemize}
	  \item $n$ jobs: $J_1, \dots, J_n$
	  \item processing time: $p_1, \dots, p_n$
	  \item $m \ge 2$ machines: $M_1, \dots, M_m$
	 ­\item goal: minimize the makespan
	\end{itemize}
  \end{exampleblock}
\end{frame}
%%%%%%%%%%%%%%%%%%%%
\begin{frame}{\ms{} is \npc{}}
  \begin{definition}[Partition]
	\begin{description}
	  \item[Instance:] 
		\[
		  \forall a \in A: s(a) \in \mathbb{Z}^{+} 
		\]
	  \item[Question:] Is there a subset $A' \subseteq A$: 
		\[
		  \sum_{a \in A'} s(a) = \sum_{a \in A \setminus A'} s(a)
		\]
	\end{description}
  \end{definition}

  \[
	A = \set{5,1,3,4,8,2,7} \implies A' = \set{5,3,7}, A \setminus A' = \set{1,2,4,8}
  \]
\end{frame}
%%%%%%%%%%%%%%%%%%%%
\begin{frame}{\ms{} is strongly \npc{}}
  \begin{definition}[$3$-Partition]
	\begin{description}
	  \item[Instance:] 
		\begin{gather*}
		  |A| = 3m, B \in \mathbb{Z}^{+} \\
		  \forall a \in A: s(a) \in \mathbb{Z}^{+}, B/4 < s(a) < B/2
		\end{gather*}
	  \item[Question:] Can $A$ be partitioned into $m$ disjoint sets $S_1,\dots,S_m$:
		\[
		  \forall 1 \le i \le m: |S_i| = 3, \sum_{a \in S_i} s(a) = B
		\]
	\end{description}
  \end{definition}

  \[
	A = \set{1,2,2,3,3,4,6,7,8}, m = 3, B = 12 \implies \set{1,3,8; 2,4,6; 2,3,7}
  \]
\end{frame}
%%%%%%%%%%%%%%%%%%%%
\begin{frame}{List-Scheduling (LS) algorithm}
  \begin{exampleblock}{List-Scheduling algorithm (JH 4.2.1.4)}
	\begin{itemize}
	  \item online
	  \item assign job to the least heavily loaded
	\end{itemize}
  \end{exampleblock}

  \[
	J = {2,3,4,6,2,2} \qquad m = 3
  \]
\end{frame}
%%%%%%%%%%%%%%%%%%%%
\begin{frame}{LS is $2$-approx.}
  \[
	T \emph{ vs. } T^{\ast}: \frac{T}{T^{\ast}}
  \]

  \begin{align*}
	T^{\ast} &\ge \frac{1}{m} \sum_{j} t_j \\
	T^{\ast} &\ge \max_{j} t_j
  \end{align*}

  \begin{enumerate}
	\item $M_i$: with the maximum load
	\item $J_i$: the last job on $M_i$
  \end{enumerate}

  \[
	ms_i \le \sum_{j} t_j \implies s_i \le \frac{1}{m} \sum_{j} t_j \le T^{\ast}
  \]

  \[
	T = c_i = s_i + p_i \le T^{\ast} + T^{\ast} = 2T^{\ast}
  \]
\end{frame}
%%%%%%%%%%%%%%%%%%%%
\begin{frame}{$2$-approx. is (almost) tight}
  \[
	n = \underbrace{m(m - 1)}_{p_i = 1} + \underbrace{1}_{p_i = m}
  \]

  \[
	\frac{T}{T^{\ast}} = \frac{2m - 1}{m} = 2 - \frac{1}{m}
  \]
\end{frame}
%%%%%%%%%%%%%%%%%%%%
\begin{frame}{LS is $(2-\frac{1}{m})$-approx.}
  \begin{align*}
	m s_i &\le \sum_{j \neq i} p_j = \frac{1}{m} (\sum_{j} p_j - p_i) \\
	&= \frac{1}{m} \sum_{j} p_j - \frac{1}{m} p_i \\
	&\le T^{\ast} - \frac{1}{m} p_i
  \end{align*}

  \begin{align*}
	T &= c_i = s_i + p_i \\
	&\le T^{\ast} + (1 - \frac{1}{m}) p_i
  \end{align*}
\end{frame}
%%%%%%%%%%%%%%%%%%%%
\begin{frame}{Sorting-Scheduling algorithm}
  \begin{exampleblock}{Sorting-Scheduling algorithm (JH 4.2.1.5)}
	Longest Processing Time (LPT) rule:
	\begin{itemize}
	  \item sorting non-increasingly
	  \item applying LS
	\end{itemize}
  \end{exampleblock}

  \begin{enumerate}
	\item $M_i$: with the maximum load
	\item $J_i$: the last job on $M_i$
  \end{enumerate}

  \[
	|M_i| = 1 \implies T = T^{\ast}
  \]

  \begin{align*}
	|M_i| \ge 2 &\implies p_i \le \frac{1}{2} T^{\ast} \\
	& \implies T = s_i + p_i \le (\frac{3}{2} - \frac{1}{2m}) T^{\ast}
  \end{align*}
\end{frame}
%%%%%%%%%%%%%%%%%%%%
\begin{frame}{LPT rule is $(\frac{4}{3} - \frac{1}{3m})$-approx.}
  \[
	p_1 \ge p_2 \ge \cdots \ge p_n
  \]

  \begin{description}
	\item[CASE $p_i \le \frac{1}{3} T^{\ast}$:] 
	  \[
		T \le (\frac{4}{3} - \frac{1}{3m}) T^{\ast}
	  \]
	\item[CASE $p_i > \frac{1}{3} T^{\ast}$:]
	  \begin{align*}
		p_i &\equiv p_n \text{\small (\textcolor{red}{w.l.o.g}: $T$ unchanged; $T^{\ast}$ not smaller)} \\
		&\implies p_1 \ge p_2 \ge \cdots \ge p_n > \frac{1}{3} T^{\ast} \implies |M_i| \le 2 \\
		&\implies n \le 2m \implies n = 2m - h \xLongrightarrow[\text{\small argument}]{\text{\small exchange}} T = T^{\ast}
	  \end{align*}
  \end{description}
\end{frame}
%%%%%%%%%%%%%%%%%%%%
\begin{frame}{$(\frac{4}{3} - \frac{1}{3m})$-approx. is tight}
  \[
	n = 2m + 1
  \]

  \begin{align*}
	\text{LPT: } &J_1 = x, J_{2m} = y, J_{2m+1} = z \\
	\text{OPT: } &J_{2m-1} = J_{2m} = J_{2m+1} = y \\
	\text{\it vs. } &\frac{x + 2y}{3y} = \frac{4}{3} - \frac{1}{3m}
  \end{align*}

  \[
	J = \set{2m-1,2m-1,\dots,m+1,m+1,m,m,m}
  \]

  \begin{alertblock}{Reference}
	\begin{itemize}
	  \item ``Bounds on Multiprocessing Timing Anomalies'' by R.L.Graham, 1969
	  \item ``Approximation Algorithms for NP-Hard Problems'' edited by Dorit Hochbaum, 1996 ({\small Theorem 1.5})
	\end{itemize}
  \end{alertblock}
\end{frame}
%%%%%%%%%%%%%%%%%%%%
\begin{frame}{\ptas{} for \ms{}}
  \begin{enumerate}
	\item $M_i$: with the maximum load
	\item $J_i$: the last job on $M_i$
  \end{enumerate}

  \[
	T = s_i + p_i \le T^{\ast} + p_i
  \]

  \begin{enumerate}
	\item $J = J_{L} \triangleq \set{\text{long jobs}} \uplus J_{S} \triangleq \set{\text{short jobs}}$
	\item $S_L$: the optimal schedule for $J_L$
	\item $S$: apply List-Scheduling to $S_L$ and $J_S$
  \end{enumerate}

  \begin{alertblock}{Reference}
	\begin{itemize}
	  \item ``The Design of Approximation Algorithms'' by David P. Williamson and David Shmoys, 2011 ({\small Section 3.2}).
	\end{itemize}
  \end{alertblock}
\end{frame}
%%%%%%%%%%%%%%%%%%%%
\begin{frame}{\ptas{} for \ms{}}
  \begin{enumerate}
	\item Split $J$:
	  \begin{gather*}
		J_i \in J_{S} \iff p_i \le \epsilon \cdot \frac{1}{m} \sum_{j} p_j \\
		\implies |J_L| < \frac{1}{\epsilon} \cdot m
	  \end{gather*}
	\item Time for $S_L$ (\textcolor{red}{$m$ being a constant!}):
	  \[
		m^{\frac{1}{\epsilon} \cdot m} \cdot O(n)
	  \]
	\item Approx. ratio ($p_i \in J_{S}$ case):
	  \begin{align*}
		T &= s_i + p_i \le \frac{1}{m} \sum_{j} p_j + \epsilon \cdot \frac{1}{m} \sum_{j} p_j \\
		&= (1 + \epsilon) \frac{1}{m} \sum_{j} p_j \\
		&\le (1 + \epsilon) T^{\ast}
	  \end{align*}
  \end{enumerate}
\end{frame}
%%%%%%%%%%%%%%%%%%%%
\begin{frame}{No \fptas{} for \ms{}}
  \begin{theorem}[$\ms \in \ptas \setminus \fptas$]
	No \fptas{} for \ms{}.
  \end{theorem}

  \[
	\ms{} \text{ is strongly } \npc{} \implies \ms{} \text{ with } \max_{j} p_j \le q(n) \text{ is } \npc{}.
  \]

  \begin{alertblock}{Reference}
	\begin{itemize}
	  \item ``The Design of Approximation Algorithms'' by David P. Williamson and David Shmoys, 2011 ({\small Section 3.2}).
	\end{itemize}
  \end{alertblock}
\end{frame}
%%%%%%%%%%%%%%%%%%%%
\begin{frame}{No \fptas{} for \ms{}}
  \begin{theorem}[]
	\[
	  \exists \fptas{} \text{ for } \ms{} \implies \ms \in \p{}.
	\]
  \end{theorem}

  \[
	A_{\epsilon}: \epsilon = \frac{1}{\lceil 2n q(n) \rceil}
  \]

  \begin{align*}
	(1 + \epsilon) T^{\ast} &= T^{\ast} + \epsilon \cdot T^{\ast} \\
	&\le T^{\ast} + \frac{1}{\lceil 2n q(n) \rceil} \cdot n q(n) \\
	&\le T^{\ast} + \frac{1}{2}
  \end{align*}

  \[
	\text{Time: } \text{Poly}(\frac{1}{\epsilon}, n) = \text{Poly}(\lceil 2n q(n) \rceil, n) = \text{Poly}(n)
  \]
\end{frame}
%%%%%%%%%%%%%%%%%%%%
%%%%%%%%%%%%%%%%%%%%%%%%%%%%%%
