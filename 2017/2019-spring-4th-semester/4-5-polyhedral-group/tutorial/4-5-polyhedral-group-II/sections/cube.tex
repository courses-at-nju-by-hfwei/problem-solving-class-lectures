% cube.tex

%%%%%%%%%%%%%%%%%%%%
\begin{frame}{}
  \fig{width = 0.25\textwidth}{figs/cube}

  \[
	\sym(C) \cong S_4
  \]

  \pause
  \vspace{-0.30cm}
  \[
	\Bcard{\big\{H: H \le \sym(C)\big\}} = 30
  \]
\end{frame}
%%%%%%%%%%%%%%%%%%%%

%%%%%%%%%%%%%%%%%%%%
\begin{frame}{}
  \[
	\Bcard{\sym(C)} \le 24
  \]

  \pause
  \fig{width = 0.30\textwidth}{figs/cube-diag}
  \[
	\red{\sym(C) \le S_4}
  \]

  \pause
  \[
	\Bcard{\sym(C)} = \underbrace{6}_{\text{Facing Upward}} \times \underbrace{4}_{\text{Rotation}}
  \]
\end{frame}
%%%%%%%%%%%%%%%%%%%%

%%%%%%%%%%%%%%%%%%%%
\begin{frame}{}

  \begin{center}
	Order of $1$: id \blue{($\# = 1$)} \\[6pt]

	\fig{width = 0.25\textwidth}{figs/cube-diag}

	\pause
	Order of $4$: face-to-face \blue{($\# = 9$)}
	\begin{align*} 
	  f_{td} = (1\;2\;3\;4) \quad f_{td}^{2} = (1\;3) (2\;4) \quad f_{td}^{3} = (1\;4\;3\;2) \\
	  f_{lr} = (1\;3\;2\;4) \quad f_{lr}^{2} = (1\;2) (3\;4) \quad f_{lr}^{3} = (1\;4\;2\;3) \\
	  f_{fb} = (1\;2\;4\;3) \quad f_{fb}^{2} = (1\;4) (2\;3) \quad f_{fb}^{3} = (1\;3\;4\;2)
	\end{align*}
  \end{center}
\end{frame}
%%%%%%%%%%%%%%%%%%%%

%%%%%%%%%%%%%%%%%%%%
\begin{frame}{}
  \fig{width = 0.25\textwidth}{figs/cube-diag}

  \begin{center}
	Order of $3$: vertex-to-vertex \blue{($\# = 8$)}
	\begin{align*}
	  v_{1} = (2\;3\;4) \quad v_{1}^{2} = (2\;4\;3) \\
	  v_{2} = (1\;4\;3) \quad v_{2}^{2} = (1\;3\;4) \\
	  v_{3} = (1\;2\;4) \quad v_{3}^{2} = (1\;4\;2) \\
	  v_{4} = (1\;2\;3) \quad v_{4}^{2} = (1\;3\;2) \\
	\end{align*}
	\end{center}
\end{frame}
%%%%%%%%%%%%%%%%%%%%

%%%%%%%%%%%%%%%%%%%%
\begin{frame}{}
  \fig{width = 0.60\textwidth}{figs/cube-diag-transpose}

  \begin{center}
	Order of $2$: edge-to-edge \blue{($\# = 6$)}
	\begin{align*}
	  e_{12} = (1\;2) \quad e_{13} = (1\;3) \quad e_{14} = (1\;4) \\
	  e_{23} = (2\;3) \quad e_{24} = (2\;4) \quad e_{34} = (3\;4)
	\end{align*}
  \end{center}
\end{frame}
%%%%%%%%%%%%%%%%%%%%

%%%%%%%%%%%%%%%%%%%%
\begin{frame}{}
  \fig{width = 0.30\textwidth}{figs/flag}

  \[
	\Bcard{\big\{H: H \le \sym(C)\big\}} = 30
  \]
\end{frame}
%%%%%%%%%%%%%%%%%%%%

%%%%%%%%%%%%%%%%%%%%
\begin{frame}{}
  \only<1>{\fig{width = 0.50\textwidth}{figs/s4-subgroup}}
  \only<2>{\fig{width = 0.60\textwidth}{figs/s4-subgroup-diagram}}
\end{frame}
%%%%%%%%%%%%%%%%%%%%

%%%%%%%%%%%%%%%%%%%%
\begin{frame}{}
  \begin{center}
	\red{Order of $1$}: id \blue{($\# = 1$)} \\[6pt]

	\red{Order of $4$}: face-to-face \blue{($\# = 9$)}
	\begin{align*} 
	  f_{td} = (1\;2\;3\;4) \quad f_{td}^{2} = (1\;3) (2\;4) \quad f_{td}^{3} = (1\;4\;3\;2) \\
	  f_{lr} = (1\;3\;2\;4) \quad f_{lr}^{2} = (1\;2) (3\;4) \quad f_{lr}^{3} = (1\;4\;2\;3) \\
	  f_{fb} = (1\;2\;4\;3) \quad f_{fb}^{2} = (1\;4) (2\;3) \quad f_{fb}^{3} = (1\;3\;4\;2)
	\end{align*}
	\red{Order of $3$}: vertex-to-vertex \blue{($\# = 8$)}
	\begin{align*}
	  v_{1} = (2\;3\;4) \quad v_{1}^{2} = (2\;4\;3) \\
	  v_{2} = (1\;4\;3) \quad v_{2}^{2} = (1\;3\;4) \\
	  v_{3} = (1\;2\;4) \quad v_{3}^{2} = (1\;4\;2) \\
	  v_{4} = (1\;2\;3) \quad v_{4}^{2} = (1\;3\;2) \\
	\end{align*}

	\vspace{-0.50cm}
	\red{Order of $2$}: edge-to-edge \blue{($\# = 6$)}
	\begin{align*}
	  e_{12} = (1\;2) \quad e_{13} = (1\;3) \quad e_{14} = (1\;4) \\
	  e_{23} = (2\;3) \quad e_{24} = (2\;4) \quad e_{34} = (3\;4)
	\end{align*}
  \end{center}
\end{frame}
%%%%%%%%%%%%%%%%%%%%

%%%%%%%%%%%%%%%%%%%%
\begin{frame}{}
  \[
	H \le S_4 \red{\implies} \bcard{H} = 1,\quad 2,\quad 3,\quad 4,\quad 6,\quad 8,\quad 12,\quad 24
  \]

  \[
	\bcard{H} = \begin{cases}
	  1: & \text{id} \quad \blue{(\# = 1)} \\[3pt]
	  2: & \blue{(\# = 6 + 3 = 9)} \\[3pt]
	  3: & v_1, v_2, v_3, v_4 \quad \blue{(\# = 4)} \\[3pt]
	  4: & \red{(\# = 7)} \\[3pt]
	  6: & \red{(\# = 4)} \\[3pt]
	  8: & \red{(\# = 3)} \\[3pt]
	  12: & A_4 \quad \red{(\# = 1)} \\[3pt]
	  24: & S_4 \quad \blue{(\# = 1)} \\[3pt]
	\end{cases}
  \]
\end{frame}
%%%%%%%%%%%%%%%%%%%%
\begin{frame}{}
  \[
	\purple{\boxed{\bcard{G} = 4 \red{\implies} G \cong \z_{4} \lor G \cong K_4 \cong \z_{2} \times \z_{2}}}
  \]

  \pause
  \[
	H \cong \z_{4}: \pause f_{fd}, f_{lr}, f_{fb} \quad \blue{(\# = 3)}
  \]

  \pause
  \vspace{-0.80cm}
  \[
	H \cong K_{4} = \set{e, a, b, ab} \quad \purple{(a^2 = b^2 = e, ab = ba)}
  \]

  \pause
  \vspace{-0.50cm}
  \brown{
	\begin{align*}
	  e_{12} = (1\;2) \quad e_{13} = (1\;3) \quad e_{14} = (1\;4) \\
	  e_{23} = (2\;3) \quad e_{24} = (2\;4) \quad e_{34} = (3\;4)
	\end{align*}
	\[
	  f_{td}^{2} = (1\;3) (2\;4) \quad f_{lr}^{2} = (1\;2) (3\;4) \quad f_{fb}^{2} = (1\;4) (2\;3)
	\]
  }

  \pause
  \vspace{-0.60cm}
  \[ 
  	\set{(1), (1\; 2), (3\; 4), (1\; 2) (3\; 4)} 
  \]
  \[ 
  	\set{(1), (1\; 3), (2\; 4), (1\; 3) (2\; 4)} 
  \]
  \[ 
  	\set{(1), (1\; 4), (2\; 3), (1\; 4) (2\; 3)} 
  \]
  \pause
  \vspace{-0.40cm}
  \[ 
  	\textcolor{red}{\set{(1), (1\; 2) (3\; 4), (1\; 3) (2\; 4), (1\; 4) (2\; 3)}} 
  \]
\end{frame}
%%%%%%%%%%%%%%%%%%%%
\begin{frame}{}
  \[
	\purple{\boxed{\bcard{G} = 6 \red{\implies} G \cong \z_{6} \lor G \cong D_3}}
  \]

  \pause
  \[
    H \ncong \mathbb{Z}_{6}
  \]

  \pause
  \[
	H \cong D_3 = \set{1, r, r^2, s, rs, r^2s} \quad (r^3 = 1, s^2 = 1, \red{srs = r^{-1}})
  \]

  \pause
  \vspace{-0.60cm}
  \brown{
	\begin{align*}
	  \red{v_{1}} = (2\;3\;4) \quad v_{1}^{2} = (2\;4\;3) \\
	  \red{v_{2}} = (1\;4\;3) \quad v_{2}^{2} = (1\;3\;4) \\
	  \red{v_{3}} = (1\;2\;4) \quad v_{3}^{2} = (1\;4\;2) \\
	  \red{v_{4}} = (1\;2\;3) \quad v_{4}^{2} = (1\;3\;2) \\
	\end{align*}
  }

  \pause
  \vspace{-1.20cm}
  \begin{theorem}
	There are only 4 subgroups \red{$\cong D_3$} in $S_4$.
  \end{theorem}

  % \[
  %   r = (1\;3\;2), \quad s = (1\;3)
  % \]
  % \centerline{What does $srs = r^{-1}$ mean?}
\end{frame}
%%%%%%%%%%%%%%%%%%%%

%%%%%%%%%%%%%%%%%%%%
\begin{frame}
  \fig{width = 0.40\textwidth}{figs/triangles-in-cube}

  \begin{columns}
	\column{0.33\textwidth}
	  \pause
	  \fig{width = 0.80\textwidth}{figs/triangle-in-cube}
	\column{0.33\textwidth}
	  \pause
	  \fig{width = 1.00\textwidth}{figs/triangle-in-cube-rotate}
	\column{0.33\textwidth}
	  \pause
	  \fig{width = 1.00\textwidth}{figs/triangle-in-cube-reflect}
  \end{columns}
\end{frame}
%%%%%%%%%%%%%%%%%%%%

%%%%%%%%%%%%%%%%%%%%
\begin{frame}{}
  \[
	\purple{\boxed{\bcard{G} = 8 \red{\implies} G \cong \z_{8},\quad \z_2 \times \z_2 \times \z_2,\quad \z_4 \times \z_2,\quad D_4,\quad Q_8}}
  \]

  \pause
  \[
	Q_8 = \set{\pm 1, \pm I, \pm J, \pm K} \quad \blue{(\text{Example } 3.15)}
  \]

  \pause
  \[
    H \ncong \mathbb{Z}_{8}
  \]

  \pause
  \[
    H \ncong \mathbb{Z}_{2} \times \mathbb{Z}_{2} \times \mathbb{Z}_{2}
  \]

  \pause
  \[
    H \ncong \mathbb{Z}_{4} \times \mathbb{Z}_{2}
  \]

  \pause
  \[
	H \ncong Q_8 \implies |H| \ge 9
  \]
\end{frame}
%%%%%%%%%%%%%%%%%%%%

%%%%%%%%%%%%%%%%%%%%
\begin{frame}{}
  \[
	\purple{\boxed{\bcard{G} = 8 \red{\implies} G \cong \z_{8},\quad \z_2 \times \z_2 \times \z_2,\quad \z_4 \times \z_2,\quad D_4,\quad Q_8}}
  \]

  \[
    H \cong D_4 = \set{1, r, r^2, r^3, s, rs, r^2s, r^3s} \quad (r^4 = 1, s^2 = 1, srs = r^{-1})
  \]

  \pause
  \brown{
	\begin{align*} 
	  \red{f_{td}} = (1\;2\;3\;4) \quad f_{td}^{2} = (1\;3) (2\;4) \quad f_{td}^{3} = (1\;4\;3\;2) \\
	  \red{f_{lr}} = (1\;3\;2\;4) \quad f_{lr}^{2} = (1\;2) (3\;4) \quad f_{lr}^{3} = (1\;4\;2\;3) \\
	  \red{f_{fb}} = (1\;2\;4\;3) \quad f_{fb}^{2} = (1\;4) (2\;3) \quad f_{fb}^{3} = (1\;3\;4\;2)
	\end{align*}
  }

  \pause
  \begin{theorem}
	There are only $3$ subgroups \red{$\cong D_4$} of $S_4$.
  \end{theorem}
  % \[
  %   r = (1\;2\;4\;3), \quad s = (2\;3)
  % \]
  % \centerline{What does $srs = r^{-1}$ mean?}
\end{frame}
%%%%%%%%%%%%%%%%%%%%

%%%%%%%%%%%%%%%%%%%%
\begin{frame}{}
  \fig{width = 0.50\textwidth}{figs/d4-in-s4}

  \pause
  \vspace{-1.00cm}
  \begin{align*}
	H_{\rm \green{green}} &= \langle (1\ 2\ 3\ 4), (1\ 3) \rangle \\
	H_{\rm \blue{blue}}   &= \langle (1\ 2\ 4\ 3), (1\ 4) \rangle \\
	H_{\rm \red{red}}     &= \langle (1\ 3\ 2\ 4), (1\ 2) \rangle 
  \end{align*}
\end{frame}
%%%%%%%%%%%%%%%%%%%%

%%%%%%%%%%%%%%%%%%%%
\begin{frame}{}
  \[
	\purple{\boxed{\bcard{G} = 12 \red{\implies} G \cong \z_{12},\quad \z_6 \times \z_2,\quad D_6,\quad A_4,\quad \text{Dic}_{12}}}
  \]

  \pause
  \[
    H \cong A_4
  \]

  \pause
  \begin{theorem}
    There is only one subgroup of order $12$ in $S_4$.
  \end{theorem}
\end{frame}
%%%%%%%%%%%%%%%%%%%%

%%%%%%%%%%%%%%%%%%%%
\begin{frame}
  \fig{width = 0.30\textwidth}{figs/arthur-cayley}
  \begin{center}
	\teal{Arthur Cayley ($1821$ -- $1895$)}
  \end{center}
\end{frame}
%%%%%%%%%%%%%%%%%%%%

%%%%%%%%%%%%%%%%%%%%
\begin{frame}
  \fig{width = 0.60\textwidth}{figs/s4-cayley}

  \begin{center}
	$\sym(C) \cong S_4$ arranged on a \red{\it truncated} cube
  \end{center}
\end{frame}
%%%%%%%%%%%%%%%%%%%%

%%%%%%%%%%%%%%%%%%%%
\begin{frame}
  \fig{width = 0.50\textwidth}{figs/flag-art}
\end{frame}
%%%%%%%%%%%%%%%%%%%%
