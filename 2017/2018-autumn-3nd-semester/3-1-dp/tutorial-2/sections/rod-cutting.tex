% file: sections/rod-cutting.tex

%%%%%%%%%%%%%%%%%%%%
\begin{frame}{}
  \centerline{\LARGE \teal{Rod Cutting}}

  \vspace{0.50cm}
  \begin{columns}
    \column{0.50\textwidth}
      \fignocaption{width = 0.80\textwidth}{figs/rod-cutter}
    \column{0.50\textwidth}
      \fignocaption{width = 0.75\textwidth}{figs/rod-wukong}
  \end{columns}
\end{frame}
%%%%%%%%%%%%%%%%%%%%

%%%%%%%%%%%%%%%%%%%%
\begin{frame}{}
  \begin{exampleblock}{Optimal Substructure of Rod-Cutting (Problem $15.3$-$5$)}
    \[
      \text{\red{\it Limit }} l_i: \# \text{ of pieces of length } i, \quad 1 \le i \le n
    \]
  \end{exampleblock}

  \vspace{0.30cm}
  \begin{columns}
    \pause
    \column{0.50\textwidth}
      \[
	n = 4
      \]

      \begin{table}[]
	\centering
	\begin{tabular}{c||C|C|C|C}
	  \textbf{length $i$}  & 1 & 2 & 3 & 4 \\ \hline
	  \textbf{price $p_i$} & 1 & 1 & 1 & 1 
	\end{tabular}% 
      \end{table}

      \begin{table}[]
	\centering
	\begin{tabular}{c||C|C|C|C}
	  \textbf{length $i$}  & 1 & 2 & 3 & 4 \\ \hline
	  \textbf{limit $l_i$} & \red{2} & 1 & 1 & 1 
	\end{tabular}% 
      \end{table}
    \pause
    \column{0.50\textwidth}
      \fignocaption{width = 0.50\textwidth}{figs/rod-cutting-opt}
      \[
	R(4) = 3
      \]

      \pause
      \fignocaption{width = 0.50\textwidth}{figs/rod-cutting-substructure}
  \end{columns}
\end{frame}
%%%%%%%%%%%%%%%%%%%%

%%%%%%%%%%%%%%%%%%%%
\begin{frame}{}
  \fignocaption{width = 0.35\textwidth}{figs/well-done-dog}

  \pause
  \begin{quote}
    {\large \it ``Show that the optimal-substructure property \textcolor<3->{red}{described in Section $15.1$} no longer holds.''}
  \end{quote}
\end{frame}
%%%%%%%%%%%%%%%%%%%%

%%%%%%%%%%%%%%%%%%%%
\begin{frame}{}
  \[
    R(i, L): \max \text{\it revenue obtainable by cutting up a rod of length $i$}
  \]
  \[
    \hl{yellow}{\it with the length limit array $L$}
  \]

  \pause
  \vspace{0.30cm}
  \[
    \red{\fbox{\large \it Where is the \teal{leftmost} cut?}}
  \]

  \pause
  \vspace{0.50cm}
  \[
    R(i, L) = \max_{1 \le j \le i} \Big(p_j + R\big(i-j, \brown{L[j \mapsto L_j-1]}\big)\Big)
  \]
\end{frame}
%%%%%%%%%%%%%%%%%%%%
