% file: overview.tex

%%%%%%%%%%%%%%%
\begin{frame}{学生反馈}
  \centerline{\large \red{$Q:$} Assertion, Invariant, Loop invariant 之间是什么关系?}
\end{frame}
%%%%%%%%%%%%%%%

%%%%%%%%%%%%%%%
\begin{frame}{}
  \centerline{\large \red{$Q:$} How to prove a loop partially correct?}

  \begin{columns}
    \column{0.60\textwidth}
      \fignocaption{width = 0.60\textwidth}{figs/loop-invariant}
    \column{0.35\textwidth}
      \pause
      \[
	\teal{\set{P} \texttt{ loop } \set{Q}}
      \]
      \vspace{-0.20cm}
      \pause
      \begin{enumerate}[(1)]
	\setlength{\itemsep}{8pt}
	\item $\set{P} \texttt{ init } \set{\red{I}}$
	\item $\set{\red{I} \land C} \texttt{ body } \set{\red{I}}$
	\item $\set{\red{I} \land \lnot C} \implies \set{Q}$
      \end{enumerate}
  \end{columns}

  \pause
  \vspace{0.50cm}
  \centerline{$I$ is \emph{before} the loop.}
\end{frame}
%%%%%%%%%%%%%%%

%%%%%%%%%%%%%%%
\begin{frame}{}
  \centerline{\large \red{$Q:$} How to prove a loop totally correct?}

  \begin{columns}
    \column{0.50\textwidth}
      \fignocaption{width = 0.70\textwidth}{figs/loop-invariant}
    \column{0.50\textwidth}
      \[
	\teal{D(X)}
      \]
      \vspace{-0.20cm}
      \pause
      \begin{enumerate}[(1)]
	\setlength{\itemsep}{8pt}
	\item $\set{\red{I} \land C} \texttt{ body } \set{D(X') < D(X)}$
	\item $\set{\red{I} \land D(X) = \min} \implies \lnot C$
      \end{enumerate}
  \end{columns}
\end{frame}
%%%%%%%%%%%%%%%

%%%%%%%%%%%%%%%
\begin{frame}{}
  \centerline{\large \red{$Q:$} How to develop loop invariants?}

  \pause
  \fignocaption{width = 0.60\textwidth}{figs/loop-dynamics}

  \pause
  \vspace{-0.30cm}
  \[
    \text{\large $\red{I} \equiv \big(\teal{\texttt{totalWork} = \texttt{workDone} + \texttt{workToDo}}\big)$}
  \]
  \pause
  \[
    \texttt{workDone} \xLeftarrow{\texttt{data}} \texttt{workToDo}
  \]
\end{frame}
%%%%%%%%%%%%%%%
