%%%%%%%%%%%%%%%
\begin{frame}{}
  \begin{exampleblock}{$3.$ 集合的势 (Cardinality)}
    $A$ 是由所有半径为有理数、圆心在$x$轴 (实数轴) 上的圆组成的集合。

    请问$A$的势是什么, 并给出证明。
  \end{exampleblock}

  \pause
  \vspace{0.30cm}
  \[
    |\R| \le \teal{|\Q \times \R|} \le |\R \times \R| = |\R|
  \]
\end{frame}
%%%%%%%%%%%%%%%

%%%%%%%%%%%%%%%
\begin{frame}{}
  \uncover<2->{
    \begin{definition}[$|A| \le |B|$]
      $|A| \le |B|$ if there exists an \red{\it one-to-one} function $f$ from $A$ into $B$.
    \end{definition}
  }

  \uncover<3->{
    \vspace{0.50cm}
    \centerline{\large \red{\it $Q:$ Is ``$\le$'' a partial order?}}
  }

  \vspace{0.30cm}
  \begin{theorem}[Cantor-\textcolor{gray}{(Dedekind)}-Schr\"{o}der–Bernstein (1887)]
    \[
      |X| \le |Y| \land |Y| \le |X| \implies |X| = |Y|
    \]
    \uncover<4->{
      \[
	\exists\; f: A \blue{\;\xrightarrow{1-1}\;} B \land g: B \blue{\;\xrightarrow{1-1}\;} A \implies \exists\; h: A \red{\;\xleftrightarrow[onto]{1-1}\;} B
      \]
    }
  \end{theorem}
\end{frame}
%%%%%%%%%%%%%%%

%%%%%%%%%%%%%%%
\begin{frame}{}
  \begin{proof}[By Julius K\"{o}nig (1906)]
    \pause
    \begin{center}
      \teal{Suppose that $A$ and $B$ are disjoint.}
      \[
	A \uplus B
      \]
    \end{center}

    \pause
    \[
      \red{a \in A}: \purple{\cdots \to f^{-1}(g^{-1}(a)) \to g^{-1}(a) \to}\; \red{a} \;\blue{\to f(a) \to g(f(a)) \to \cdots}
    \]

    \begin{columns}
      \column{0.50\textwidth}
	\pause
	\begin{enumerate}[(i)]
	  \item $\cdots \leadsto \cdots$
	  \item $a \in A \leadsto \cdots$
	  \item $b \in B \leadsto \cdots$
	  \item $a \in A \leadsto a \in A$
	\end{enumerate}
      \column{0.30\textwidth}
        \pause
	\begin{center}
	  {\red{Partition} of $A \uplus B$}
	\end{center}
    \end{columns}
  \end{proof}
\end{frame}
%%%%%%%%%%%%%%%

%%%%%%%%%%%%%%%
\begin{frame}{}
  \fig{width = 0.60\textwidth}{figs/Cantor-Bernstein-Proof}
\end{frame}
%%%%%%%%%%%%%%%
