% file: parts/max-subarray.tex

%%%%%%%%%%%%%%%
\begin{frame}{}
  \begin{exampleblock}{Maximal Sum Subarray (Problem $4.1-5$)}
    \begin{itemize}
      \item array $A[1 \cdots n], a_{i} >=< 0$
      \item to find (the sum of) an MS in $A$
    \end{itemize}
	
    \[
      A[-2,1 ,-3, \boxed{\red{4,-1,2,1}} ,-5,4]
    \]
  \end{exampleblock}

  \pause
  \vspace{0.30cm}
  \begin{alertblock}{Trial and error.}
    \begin{itemize}
      \item try subproblem $\text{MSS}[i]$: the sum of the MS (\text{MS}[i]) in $A[1 \cdots i]$
      \item goal: $\text{mss} = \text{MSS}[n]$
	\pause
      \item question: Is $a_{i} \in \text{MS}[i]$?
      \item recurrence: 
	\[ 
	  \text{MSS}[i] = \max \set{\text{MSS}[i-1], \textcolor{red}{???}}
	\]
    \end{itemize}
  \end{alertblock}
\end{frame}
%%%%%%%%%%%%%%%

%%%%%%%%%%%%%%%
\begin{frame}{}
  \begin{block}{Solution.}
    \begin{itemize}
      \item subproblem $\text{MSS}[i]$: the sum of the MS \textcolor{red}{\it ending with} $a_{i}$ or 0
      \item goal: 
	\[
	  \boxed{\teal{\text{mss} = \max\limits_{1 \le i \le n} \text{MSS}[i]}}
	\]
      \item<2-> question: where does the $\text{MS}[i]$ start?
      \item<2-> recurrence: 
	\[ 
	  \boxed{\teal{\text{MSS}[i] = \max \left\{\text{MSS}[i-1] + a_{i}, 0\right\}}} \quad \text{\textcolor{red}{(prove it!)}}
	\]
      \item<3-> initialization: $\text{MSS}[0] = 0$
    \end{itemize}

    % \begin{displaymath}
    %   \text{MSS}[i] = \left\{ \begin{array}{ll}
    %     0 & i = 0 \\
    %     \max \set{\text{MSS}[i-1] + a_{i}, 0} & i > 0
    %   \end{array} \right.
    % \end{displaymath}
  \end{block}
\end{frame}
%%%%%%%%%%%%%%%

%%%%%%%%%%%%%%%
\begin{frame}{}
  % file: algs/max-subarray-origin.tex

\begin{algorithm}[H]
  % \caption{Max-sum subarray.}
  \begin{algorithmic}[1]
    \Procedure{MSS}{$A[1 \cdots n]$}
      \State $\text{MSS}[0] \gets 0$

      \hStatex
      \For{$i \gets 1 \;\text{\bf to } n$}
	\State $\text{MSS}[i] \gets \max\left\{\text{MSS}[i-1] + A[i], 0\right\}$
      \EndFor

      \hStatex
      \State \Return $\max\limits_{1 \le i \le n} \text{MSS}[i]$
    \EndProcedure
  \end{algorithmic}
\end{algorithm}


  \pause
  \vspace{-0.30cm}
  % file: parts/max-subarray.tex

%%%%%%%%%%%%%%%
\begin{frame}{}
  \begin{exampleblock}{Maximal Sum Subarray (Problem $4.1-5$)}
    \begin{itemize}
      \item Array $A[1 \cdots n], a_{i} >=< 0$
      \item To find (the sum of) an MS in $A$
    \end{itemize}
	
    \[
      A[-2,1 ,-3, \boxed{\red{4,-1,2,1}} ,-5,4]
    \]
  \end{exampleblock}
\end{frame}
%%%%%%%%%%%%%%%

%%%%%%%%%%%%%%%
\begin{frame}{}
  \centerline{$\text{MSS}[i]$: the sum of the MS ($\text{MS}[i]$) in $A[1 \cdots i]$}

  \pause
  \vspace{0.30cm}
  \[
    \text{mss} = \text{MSS}[n]
  \]

  \pause
  \vspace{0.60cm}
  \centerline{\red{$Q:$} Is $a_{i} \in \text{MS}[i]$?}
  
  \pause
  \vspace{0.30cm}
  \[ 
    \text{MSS}[i] = \max \set{\text{MSS}[i-1], \red{???}}
  \]
\end{frame}
%%%%%%%%%%%%%%%

%%%%%%%%%%%%%%%
\begin{frame}{}
  \centerline{$\text{MSS}[i]$: the sum of the MS \textcolor{red}{\it ending with} $a_{i}$ or 0}

  \pause
  \vspace{0.30cm}
  \[
    \boxed{\teal{\text{mss} = \max\limits_{1 \le i \le n} \text{MSS}[i]}}
  \]

  \vspace{0.60cm}
  \centerline{\red{$Q:$} where does the $\text{MS}[i]$ start?}

  \pause
  \vspace{0.30cm}
  \[ 
    \boxed{\red{\text{MSS}[i] = \max \left\{\text{MSS}[i-1] + a_{i}, 0\right\}}}
  \]

  \pause
  \vspace{0.30cm}
  \[
    \text{MSS}[0] = 0
  \]

    % \begin{displaymath}
    %   \text{MSS}[i] = \left\{ \begin{array}{ll}
    %     0 & i = 0 \\
    %     \max \set{\text{MSS}[i-1] + a_{i}, 0} & i > 0
    %   \end{array} \right.
    % \end{displaymath}
\end{frame}
%%%%%%%%%%%%%%%

%%%%%%%%%%%%%%%
\begin{frame}{}
  % file: algs/max-subarray-origin.tex

\begin{algorithm}[H]
  % \caption{Max-sum subarray.}
  \begin{algorithmic}[1]
    \Procedure{MSS}{$A[1 \cdots n]$}
      \State $\text{MSS}[0] \gets 0$

      \hStatex
      \For{$i \gets 1 \;\text{\bf to } n$}
	\State $\text{MSS}[i] \gets \max\left\{\text{MSS}[i-1] + A[i], 0\right\}$
      \EndFor

      \hStatex
      \State \Return $\max\limits_{1 \le i \le n} \text{MSS}[i]$
    \EndProcedure
  \end{algorithmic}
\end{algorithm}


  % \pause
  % \vspace{-0.30cm}
  % % file: parts/max-subarray.tex

%%%%%%%%%%%%%%%
\begin{frame}{}
  \begin{exampleblock}{Maximal Sum Subarray (Problem $4.1-5$)}
    \begin{itemize}
      \item Array $A[1 \cdots n], a_{i} >=< 0$
      \item To find (the sum of) an MS in $A$
    \end{itemize}
	
    \[
      A[-2,1 ,-3, \boxed{\red{4,-1,2,1}} ,-5,4]
    \]
  \end{exampleblock}
\end{frame}
%%%%%%%%%%%%%%%

%%%%%%%%%%%%%%%
\begin{frame}{}
  \centerline{$\text{MSS}[i]$: the sum of the MS ($\text{MS}[i]$) in $A[1 \cdots i]$}

  \pause
  \vspace{0.30cm}
  \[
    \text{mss} = \text{MSS}[n]
  \]

  \pause
  \vspace{0.60cm}
  \centerline{\red{$Q:$} Is $a_{i} \in \text{MS}[i]$?}
  
  \pause
  \vspace{0.30cm}
  \[ 
    \text{MSS}[i] = \max \set{\text{MSS}[i-1], \red{???}}
  \]
\end{frame}
%%%%%%%%%%%%%%%

%%%%%%%%%%%%%%%
\begin{frame}{}
  \centerline{$\text{MSS}[i]$: the sum of the MS \textcolor{red}{\it ending with} $a_{i}$ or 0}

  \pause
  \vspace{0.30cm}
  \[
    \boxed{\teal{\text{mss} = \max\limits_{1 \le i \le n} \text{MSS}[i]}}
  \]

  \vspace{0.60cm}
  \centerline{\red{$Q:$} where does the $\text{MS}[i]$ start?}

  \pause
  \vspace{0.30cm}
  \[ 
    \boxed{\red{\text{MSS}[i] = \max \left\{\text{MSS}[i-1] + a_{i}, 0\right\}}}
  \]

  \pause
  \vspace{0.30cm}
  \[
    \text{MSS}[0] = 0
  \]

    % \begin{displaymath}
    %   \text{MSS}[i] = \left\{ \begin{array}{ll}
    %     0 & i = 0 \\
    %     \max \set{\text{MSS}[i-1] + a_{i}, 0} & i > 0
    %   \end{array} \right.
    % \end{displaymath}
\end{frame}
%%%%%%%%%%%%%%%

%%%%%%%%%%%%%%%
\begin{frame}{}
  % file: algs/max-subarray-origin.tex

\begin{algorithm}[H]
  % \caption{Max-sum subarray.}
  \begin{algorithmic}[1]
    \Procedure{MSS}{$A[1 \cdots n]$}
      \State $\text{MSS}[0] \gets 0$

      \hStatex
      \For{$i \gets 1 \;\text{\bf to } n$}
	\State $\text{MSS}[i] \gets \max\left\{\text{MSS}[i-1] + A[i], 0\right\}$
      \EndFor

      \hStatex
      \State \Return $\max\limits_{1 \le i \le n} \text{MSS}[i]$
    \EndProcedure
  \end{algorithmic}
\end{algorithm}


  % \pause
  % \vspace{-0.30cm}
  % % file: parts/max-subarray.tex

%%%%%%%%%%%%%%%
\begin{frame}{}
  \begin{exampleblock}{Maximal Sum Subarray (Problem $4.1-5$)}
    \begin{itemize}
      \item Array $A[1 \cdots n], a_{i} >=< 0$
      \item To find (the sum of) an MS in $A$
    \end{itemize}
	
    \[
      A[-2,1 ,-3, \boxed{\red{4,-1,2,1}} ,-5,4]
    \]
  \end{exampleblock}
\end{frame}
%%%%%%%%%%%%%%%

%%%%%%%%%%%%%%%
\begin{frame}{}
  \centerline{$\text{MSS}[i]$: the sum of the MS ($\text{MS}[i]$) in $A[1 \cdots i]$}

  \pause
  \vspace{0.30cm}
  \[
    \text{mss} = \text{MSS}[n]
  \]

  \pause
  \vspace{0.60cm}
  \centerline{\red{$Q:$} Is $a_{i} \in \text{MS}[i]$?}
  
  \pause
  \vspace{0.30cm}
  \[ 
    \text{MSS}[i] = \max \set{\text{MSS}[i-1], \red{???}}
  \]
\end{frame}
%%%%%%%%%%%%%%%

%%%%%%%%%%%%%%%
\begin{frame}{}
  \centerline{$\text{MSS}[i]$: the sum of the MS \textcolor{red}{\it ending with} $a_{i}$ or 0}

  \pause
  \vspace{0.30cm}
  \[
    \boxed{\teal{\text{mss} = \max\limits_{1 \le i \le n} \text{MSS}[i]}}
  \]

  \vspace{0.60cm}
  \centerline{\red{$Q:$} where does the $\text{MS}[i]$ start?}

  \pause
  \vspace{0.30cm}
  \[ 
    \boxed{\red{\text{MSS}[i] = \max \left\{\text{MSS}[i-1] + a_{i}, 0\right\}}}
  \]

  \pause
  \vspace{0.30cm}
  \[
    \text{MSS}[0] = 0
  \]

    % \begin{displaymath}
    %   \text{MSS}[i] = \left\{ \begin{array}{ll}
    %     0 & i = 0 \\
    %     \max \set{\text{MSS}[i-1] + a_{i}, 0} & i > 0
    %   \end{array} \right.
    % \end{displaymath}
\end{frame}
%%%%%%%%%%%%%%%

%%%%%%%%%%%%%%%
\begin{frame}{}
  \input{algs/max-subarray-origin}

  % \pause
  % \vspace{-0.30cm}
  % \input{algs/max-subarray}
\end{frame}
%%%%%%%%%%%%%%%

%%%%%%%%%%%%%%%
\begin{frame}{}
  \begin{columns}
    \column{0.40\textwidth}
      \fignocaption{width = 0.60\textwidth}{figs/programming-pearls.jpg}
    \column{0.60\textwidth}
      \begin{description}[<+->][Michael Shamos]
	\setlength{\itemsep}{6pt}
	\item[Ulf Grenander] $O(n^3) \implies O(n^2)$
	\item[Michael Shamos] $O(n \log n)$, onenight
	\item[Jon Bentley] Conjecture: $\Omega(n \log n)$
	\item[Michael Shamos] Carnegie Mellon seminar
	\item[Jay Kadane] $O(n)$, \uncover<6->{\textcolor{red}{$\le 1$ minute}}
      \end{description}
  \end{columns}
\end{frame}
%%%%%%%%%%%%%%%

%%%%%%%%%%%%%%%
\begin{frame}{Maximum-product subarray}
  \begin{exampleblock}{Maximum-product subarray (Problem 7.4)}
    \begin{itemize}
      \item Array $A[1 \dots n]$
      \item Find maximum-product subarray of $A$
    \end{itemize}
  \end{exampleblock}
\end{frame}
%%%%%%%%%%%%%%%

%%%%%%%%%%%%%%%
\begin{frame}{}
  \centerline{\teal{\large Ending with $i$}}

  \begin{table}
    \renewcommand{\arraystretch}{1.8}
    \centering
    \begin{tabular}{|L||C|C|C|C|C|C|C|}
      \hline
      &	& \frac{1}{2} & 4 & -2 & \textcolor{red}{5} & \textcolor{red}{-\frac{1}{5}} & 8 \\ \hline \pause 
      \text{MaxP}[i] & 1	& \frac{1}{2} & 4 & -2 & 5 & 8 & 64 \\ \hline \pause
      \text{MinP}[i] & 1	& \frac{1}{2} & 2 & -8 & -40 & -1 & -8  \\ \hline
    \end{tabular}
  \end{table}

  \pause
  \begin{align*}
    \text{MaxP}[i] &= \max\set{\text{MaxP}[i-1] \cdot a_i, \text{MinP}[i-1] \cdot a_i, a_i} \\
    \text{MinP}[i] &= \min\set{\text{MaxP}[i-1] \cdot a_i, \text{MinP}[i-1] \cdot a_i, a_i}
  \end{align*}
\end{frame}
%%%%%%%%%%%%%%%

%%%%%%%%%%%%%%%
% \begin{frame}{}
%   2d
% \end{frame}
%%%%%%%%%%%%%%%

\end{frame}
%%%%%%%%%%%%%%%

%%%%%%%%%%%%%%%
\begin{frame}{}
  \begin{columns}
    \column{0.40\textwidth}
      \fignocaption{width = 0.60\textwidth}{figs/programming-pearls.jpg}
    \column{0.60\textwidth}
      \begin{description}[<+->][Michael Shamos]
	\setlength{\itemsep}{6pt}
	\item[Ulf Grenander] $O(n^3) \implies O(n^2)$
	\item[Michael Shamos] $O(n \log n)$, onenight
	\item[Jon Bentley] Conjecture: $\Omega(n \log n)$
	\item[Michael Shamos] Carnegie Mellon seminar
	\item[Jay Kadane] $O(n)$, \uncover<6->{\textcolor{red}{$\le 1$ minute}}
      \end{description}
  \end{columns}
\end{frame}
%%%%%%%%%%%%%%%

%%%%%%%%%%%%%%%
\begin{frame}{Maximum-product subarray}
  \begin{exampleblock}{Maximum-product subarray (Problem 7.4)}
    \begin{itemize}
      \item Array $A[1 \dots n]$
      \item Find maximum-product subarray of $A$
    \end{itemize}
  \end{exampleblock}
\end{frame}
%%%%%%%%%%%%%%%

%%%%%%%%%%%%%%%
\begin{frame}{}
  \centerline{\teal{\large Ending with $i$}}

  \begin{table}
    \renewcommand{\arraystretch}{1.8}
    \centering
    \begin{tabular}{|L||C|C|C|C|C|C|C|}
      \hline
      &	& \frac{1}{2} & 4 & -2 & \textcolor{red}{5} & \textcolor{red}{-\frac{1}{5}} & 8 \\ \hline \pause 
      \text{MaxP}[i] & 1	& \frac{1}{2} & 4 & -2 & 5 & 8 & 64 \\ \hline \pause
      \text{MinP}[i] & 1	& \frac{1}{2} & 2 & -8 & -40 & -1 & -8  \\ \hline
    \end{tabular}
  \end{table}

  \pause
  \begin{align*}
    \text{MaxP}[i] &= \max\set{\text{MaxP}[i-1] \cdot a_i, \text{MinP}[i-1] \cdot a_i, a_i} \\
    \text{MinP}[i] &= \min\set{\text{MaxP}[i-1] \cdot a_i, \text{MinP}[i-1] \cdot a_i, a_i}
  \end{align*}
\end{frame}
%%%%%%%%%%%%%%%

%%%%%%%%%%%%%%%
% \begin{frame}{}
%   2d
% \end{frame}
%%%%%%%%%%%%%%%

\end{frame}
%%%%%%%%%%%%%%%

%%%%%%%%%%%%%%%
\begin{frame}{}
  \begin{columns}
    \column{0.40\textwidth}
      \fignocaption{width = 0.60\textwidth}{figs/programming-pearls.jpg}
    \column{0.60\textwidth}
      \begin{description}[<+->][Michael Shamos]
	\setlength{\itemsep}{6pt}
	\item[Ulf Grenander] $O(n^3) \implies O(n^2)$
	\item[Michael Shamos] $O(n \log n)$, onenight
	\item[Jon Bentley] Conjecture: $\Omega(n \log n)$
	\item[Michael Shamos] Carnegie Mellon seminar
	\item[Jay Kadane] $O(n)$, \uncover<6->{\textcolor{red}{$\le 1$ minute}}
      \end{description}
  \end{columns}
\end{frame}
%%%%%%%%%%%%%%%

%%%%%%%%%%%%%%%
\begin{frame}{Maximum-product subarray}
  \begin{exampleblock}{Maximum-product subarray (Problem 7.4)}
    \begin{itemize}
      \item Array $A[1 \dots n]$
      \item Find maximum-product subarray of $A$
    \end{itemize}
  \end{exampleblock}
\end{frame}
%%%%%%%%%%%%%%%

%%%%%%%%%%%%%%%
\begin{frame}{}
  \centerline{\teal{\large Ending with $i$}}

  \begin{table}
    \renewcommand{\arraystretch}{1.8}
    \centering
    \begin{tabular}{|L||C|C|C|C|C|C|C|}
      \hline
      &	& \frac{1}{2} & 4 & -2 & \textcolor{red}{5} & \textcolor{red}{-\frac{1}{5}} & 8 \\ \hline \pause 
      \text{MaxP}[i] & 1	& \frac{1}{2} & 4 & -2 & 5 & 8 & 64 \\ \hline \pause
      \text{MinP}[i] & 1	& \frac{1}{2} & 2 & -8 & -40 & -1 & -8  \\ \hline
    \end{tabular}
  \end{table}

  \pause
  \begin{align*}
    \text{MaxP}[i] &= \max\set{\text{MaxP}[i-1] \cdot a_i, \text{MinP}[i-1] \cdot a_i, a_i} \\
    \text{MinP}[i] &= \min\set{\text{MaxP}[i-1] \cdot a_i, \text{MinP}[i-1] \cdot a_i, a_i}
  \end{align*}
\end{frame}
%%%%%%%%%%%%%%%

%%%%%%%%%%%%%%%
% \begin{frame}{}
%   2d
% \end{frame}
%%%%%%%%%%%%%%%

\end{frame}
%%%%%%%%%%%%%%%

%%%%%%%%%%%%%%%
\begin{frame}{}
  \begin{columns}
    \column{0.40\textwidth}
      \fignocaption{width = 0.60\textwidth}{figs/programming-pearls.jpg}
    \column{0.60\textwidth}
      \begin{description}[<+->][Michael Shamos]
	\setlength{\itemsep}{6pt}
	\item[Ulf Grenander] $O(n^3) \implies O(n^2)$
	\item[Michael Shamos] $O(n \log n)$, onenight
	\item[Jon Bentley] Conjecture: $\Omega(n \log n)$
	\item[Michael Shamos] Carnegie Mellon seminar
	\item[Jay Kadane] $O(n)$, \uncover<6->{\textcolor{red}{$\le 1$ minute}}
      \end{description}
  \end{columns}
\end{frame}
%%%%%%%%%%%%%%%

%%%%%%%%%%%%%%%
\begin{frame}{Maximum-product subarray}
  \begin{exampleblock}{Maximum-product subarray (Problem 7.4)}
    \begin{itemize}
      \item Array $A[1 \dots n]$
      \item Find maximum-product subarray of $A$
    \end{itemize}

    \begin{enumerate}[(1)]
      \item $a_i \in \mathbb{N}$
      \item $a_i \in \mathbb{Z}$
      \item $a_i \in \mathbb{R}$ 
    \end{enumerate}
  \end{exampleblock}

  \pause
  \vspace{0.60cm}
  \centerline{sum \emph{vs.} product}
\end{frame}
%%%%%%%%%%%%%%%

%%%%%%%%%%%%%%%
\begin{frame}{Maximum-product subarray}
  \centerline{Subproblem: $\text{MaxP}[i], \text{MinP}[i]$}

  \begin{table}
    \renewcommand{\arraystretch}{1.5}
    \centering
    \begin{tabular}{|L||C|C|C|C|C|C|C|}
      \hline
      &	& \frac{1}{2} & 4 & -2 & \textcolor{red}{5} & \textcolor{red}{-\frac{1}{5}} & 8 \\ \hline
      \text{MaxP}[i] & 1	& \frac{1}{2} & 4 & -2 & 5 & 8 & 64 \\ \hline
      \text{MinP}[i] & 1	& \frac{1}{2} & 2 & -8 & -40 & -1 & -8  \\ \hline
    \end{tabular}
  \end{table}

  \begin{align*}
    \text{MaxP}[i] &= \max\set{\text{MaxP}[i-1] \cdot a_i, \text{MinP}[i-1] \cdot a_i, a_i} \\
    \text{MinP}[i] &= \min\set{\text{MaxP}[i-1] \cdot a_i, \text{MinP}[i-1] \cdot a_i, a_i}
  \end{align*}
\end{frame}
%%%%%%%%%%%%%%%

%%%%%%%%%%%%%%%
\begin{frame}{}
  2d
\end{frame}
%%%%%%%%%%%%%%%
