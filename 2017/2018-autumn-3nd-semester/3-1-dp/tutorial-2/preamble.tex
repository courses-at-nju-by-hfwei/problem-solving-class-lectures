% File: preamble.tex

\usepackage{lmodern}

\usepackage{CJKutf8}
% \usepackage{xeCJK}
\usetheme{CambridgeUS} % try Pittsburgh
\usecolortheme{beaver}
\usefonttheme[onlymath]{serif} % try "professionalfonts"

\setbeamertemplate{itemize items}[default]
\setbeamertemplate{enumerate items}[default]

\usepackage{animate}
\usepackage{xmpmulti}

\usepackage{amsmath, amsfonts, latexsym, mathtools, tabu}

\usepackage{amstext} % for \text macro
\usepackage{array}   % for \newcolumntype macro
\newcolumntype{C}{>{$}c<{$}} % math-mode version of "l" column type

\newcommand{\set}[1]{\left\{#1\right\}}
\newcommand{\ps}[1]{\mathcal{P}(#1)}
\usepackage{bm}
\DeclareMathOperator*{\argmin}{\arg\!\min}
\DeclareMathOperator*{\E}{\mathbb{E}}

\usepackage{algorithm}
\usepackage[noend]{algpseudocode}
% \algrenewcommand\alglinenumber[1]{$#1:$}
\algnewcommand{\IfThen}[2]{% \IfThenElse{<if>}{<then>}
  \State \algorithmicif\ #1\ \algorithmicthen\ #2}
\newcommand{\hStatex}[0]{\vspace{5pt}}

% colors
\newcommand{\red}[1]{\textcolor{red}{#1}}
\newcommand{\redoverlay}[2]{\textcolor<#2>{red}{#1}}
\newcommand{\green}[1]{\textcolor{green}{#1}}
\newcommand{\greenoverlay}[2]{\textcolor<#2>{green}{#1}}
\newcommand{\blue}[1]{\textcolor{blue}{#1}}
\newcommand{\blueoverlay}[2]{\textcolor<#2>{blue}{#1}}
\newcommand{\purple}[1]{\textcolor{purple}{#1}}
\newcommand{\cyan}[1]{\textcolor{cyan}{#1}}
\newcommand{\violet}[1]{\textcolor{violet}{#1}}
\newcommand{\lgray}[1]{\textcolor{lightgray}{#1}}
\newcommand{\teal}[1]{\textcolor{teal}{#1}}

% colorded box
\newcommand{\rbox}[1]{\red{\boxed{#1}}}
\newcommand{\gbox}[1]{\green{\boxed{#1}}}
\newcommand{\bbox}[1]{\blue{\boxed{#1}}}
\newcommand{\pbox}[1]{\purple{\boxed{#1}}}

\usepackage[linewidth = 1pt, framemethod = TikZ]{mdframed}
\mdfsetup{frametitlealignment=\center}

% #1: color; #2: text
\newcommand{\hl}[2]{\fcolorbox{#1}{#1!50}{#2}}

\usepackage{pifont}
\usepackage{wasysym}

\newcommand{\cmark}{\green{\ding{51}}}
\newcommand{\xmark}{\red{\ding{55}}}
%%%%%%%%%%%%%%%%%%%%%%%%%%%%%%%%%%%%%%%%%%%%%%%%%%%%%%%%%%%%%%
% for fig without caption: #1: width/size; #2: fig file
\newcommand{\fignocaption}[2]{
  \begin{figure}[htp]
    \centering
      \includegraphics[#1]{#2}
  \end{figure}
}

\usepackage{adjustbox}
\usepackage{tikz}
\usepackage{tikz-qtree}
\usetikzlibrary{shapes}

\newcommand{\push}{\texttt{Push}}
\newcommand{\pop}{\texttt{Pop}}

\newcommand{\titletext}{$3$-$1$ Dynamic Programming}
 
\newcommand{\thankyou}{
  \begin{frame}[noframenumbering]{}
    \fignocaption{width = 0.55\textwidth}{figs/thankyou}
  \end{frame}
}
