% induction.tex

%%%%%%%%%%%%%%%
\begin{frame}{}
  \begin{theorem}[First Principle of Mathematical Induction (Theorem 18.1)]
    For an integer $n$, let $P(n)$ denote an assertion. Suppose that
    \begin{enumerate}[(i)]
      \item $P(1)$ is true, and
      \item for all positive integers $n$, if $P(n)$ is true, then $P(n+1)$ is true.
    \end{enumerate}
    Then $P(n)$ holds for all positive integers $n$.
  \end{theorem}

  \pause
  \vspace{0.60cm}
  \[
    \uncover<3>{\red{\forall P:}}\; 
    \Big[ P(1) \land \forall n \in \mathbb{N}^{+} \big(P(n) \to P(n+1) \big) \Big]
      \to \forall n \in \mathbb{N}^{+} P(n).
  \]
\end{frame}
%%%%%%%%%%%%%%%

%%%%%%%%%%%%%%%
\begin{frame}{}
  \begin{theorem}[Second Principle of Mathematical Induction (Theorem 18.9)]
    For an integer $n$, let $Q(n)$ denote an assertion. Suppose that
    \begin{enumerate}[(i)]
      \item $Q(1)$ is true, and
      \item for all positive integers $n$, if $Q(1), \cdots, Q(n)$ are true,
	then $Q(n+1)$ is true.
    \end{enumerate}
    Then $Q(n)$ holds for all positive integers $n$.
  \end{theorem}

  \pause
  \vspace{0.30cm}
  \[
    \red{\forall Q:}\; 
    \Big[ Q(1) \land \forall n \in \mathbb{N}^{+} \Big(\big(Q(1) \land \cdots \land Q(n)\big) \to Q(n+1) \Big) \Big] 
      \to \forall n \in \mathbb{N}^{+} Q(n).
  \]
\end{frame}
%%%%%%%%%%%%%%%

%%%%%%%%%%%%%%%
\begin{frame}{}
  \[
    \teal{\text{PMI(II)} \leftrightarrow \text{PMI(I)}}
  \]

  \[
    \red{\forall P:}\; 
    \Big[ P(1) \land \forall n \in \mathbb{N}^{+} \big(P(n) \to P(n+1) \big) \Big]
      \to \forall n \in \mathbb{N}^{+} P(n).
  \]

  \[
    \red{\forall Q:}\; 
    \Big[ Q(1) \land \forall n \in \mathbb{N}^{+} \Big(\big(Q(1) \land \cdots \land Q(n)\big) \to Q(n+1) \Big) \Big] 
      \to \forall n \in \mathbb{N}^{+} Q(n).
  \]

  \pause
  \vspace{0.50cm}
  \begin{center}
    {\it \red{\Large Let us calculate [calculemus].}}
  \end{center}
\end{frame}
%%%%%%%%%%%%%%%

%%%%%%%%%%%%%%%
\begin{frame}{}
  \[
    \teal{\text{PMI(II)} \to \text{PMI(I)}}
  \]

  \[
    \red{\forall Q:}\; 
    \Big[ Q(1) \land \forall n \in \mathbb{N}^{+} \Big(\big(Q(1) \land \cdots \land Q(n)\big) \to Q(n+1) \Big) \Big] 
      \to \forall n \in \mathbb{N}^{+} Q(n).
  \]

  \[
    \red{\forall P:}\; 
    \Big[ P(1) \land \forall n \in \mathbb{N}^{+} \big(P(n) \to P(n+1) \big) \Big]
      \to \forall n \in \mathbb{N}^{+} P(n).
  \]

  \pause
  \[
    \blue{Q(n) \triangleq P(n)}
  \]
\end{frame}
%%%%%%%%%%%%%%%

%%%%%%%%%%%%%%%
\begin{frame}{}
  \[
    \teal{\text{PMI(I)} \to \text{PMI(II)}}
  \]

  \[
    \red{\forall P:}\; 
    \Big[ P(1) \land \forall n \in \mathbb{N}^{+} \big(P(n) \to P(n+1) \big) \Big]
      \to \forall n \in \mathbb{N}^{+} P(n).
  \]

  \[
    \red{\forall Q:}\; 
    \Big[ Q(1) \land \forall n \in \mathbb{N}^{+} \Big(\big(Q(1) \land \cdots \land Q(n)\big) \to Q(n+1) \Big) \Big] 
      \to \forall n \in \mathbb{N}^{+} Q(n).
  \]

  \pause
  \[
    \blue{P(n) \triangleq Q(1) \land \cdots \land Q(n)}
  \]
\end{frame}
%%%%%%%%%%%%%%%

%%%%%%%%%%%%%%%
\begin{frame}{}
  \fig{width = 0.50\textwidth}{figs/wait}{\centerline{\large 说好的数学归纳法呢?}}
\end{frame}
%%%%%%%%%%%%%%%

%%%%%%%%%%%%%%%
% \begin{frame}{}
%   \begin{columns}
%     \column{0.50\textwidth}
%       \begin{proof}
% 	\[
% 	  P(n) \triangleq Q(1) \land \cdots \land Q(n)
% 	\]
% 
% 	\uncover<3->{
% 	  \blue{用\red{(第一)数学归纳法}证明 $P(n)$ 对一切正整数都成立。}
% 	}
%       \end{proof}
%     \pause
%     \column{0.50\textwidth}
%       \fig{width = 0.60\textwidth}{figs/de-morgan.jpg}{\centerline{De Morgan (1806-1871)}}
%   \end{columns}
% 
%   \vspace{0.50cm}
%   \begin{quote}
%     {\large ``$\ldots$ introduced the term mathematical induction, making its idea rigorous.''} \hfill --- (1938)
%   \end{quote}
% \end{frame}
%%%%%%%%%%%%%%%

%%%%%%%%%%%%%%%
\begin{frame}{}
  \begin{exampleblock}{$\text{PMI(I)} \to \text{PMI(II)}$ (``标准''证明示例)}
    \[
      P(n) \triangleq Q(1) \land \cdots \land Q(n)
    \]

    \blue{用\red{第一数学归纳法}证明 $\forall n \in \mathbb{N}^{+}: P(n)$。}
  \end{exampleblock}

  \vspace{0.60cm}
  \pause
  \begin{proof}
    By mathematical induction on $\mathbb{N}^{+}$.

    \begin{description}[Inductive Hypothesis]
      \item[Basis Step:] $P(1)$
      \item[\textcolor{cyan}{Inductive Hypothesis:}] $P(n)$
      \item[Inductive Step:] $P(n) \to P(n+1)$
    \end{description}

    Therefore, $P(n)$ holds for all positive integers.
  \end{proof}
\end{frame}
%%%%%%%%%%%%%%%

%%%%%%%%%%%%%%%
\begin{frame}{}
  \begin{theorem}[Second Principle of Mathematical Induction]
    For an integer $n$, let $Q(n)$ denote an assertion. Suppose that
    \begin{enumerate}[(i)]
      \item $Q(1)$ is true, and
      \item for all positive integers $n$, if $Q(1), \cdots, Q(n)$ are true,
	then $Q(n+1)$ is true.
    \end{enumerate}
    Then $Q(n)$ holds for all positive integers $n$.
  \end{theorem}

  \begin{theorem}[Well-ordering Principle of $\mathbb{N}$]
    Every non-empty subset of the natural numbers contains a minimum.
  \end{theorem}

  \pause
  \begin{center}
    \red{By contradiction.}
    \[
      \exists S \neq \emptyset: S \text{ has no minimum element.}
    \]
    \pause
    \vspace{-0.30cm}
    \[
      \purple{Q(n) \triangleq n \notin S}
    \]
  \end{center}
\end{frame}
%%%%%%%%%%%%%%%

%%%%%%%%%%%%%%%
\begin{frame}{}
  \begin{theorem}[First Principle of Mathematical Induction]
  \end{theorem}

  \begin{theorem}[Well-ordering Principle of $\mathbb{N}$]
    Every non-empty subset of $\mathbb{N}$ contains a minimum.
  \end{theorem}

  \pause
  \begin{center}
    \red{By mathematical induction on the size $n$ of non-empty subsets of $\mathbb{N}$.}

    \vspace{-0.30cm}
    \[
      P(k): \text{All subsets of size $k$ contain a minimum.}
    \]

    \pause
    \begin{description}[Inductive Hypothesis:]
      \item[Basis Step:] $P(1)$
      \item[\textcolor{cyan}{Inductive Hypothesis:}] $P(n)$
      \item[Inductive Step:] $P(n) \to P(n+1)$
	\only<4>{
	  \begin{itemize}
	    \item $A' \gets A \setminus {a}$
	    \item $x \gets \min A'$
	    \item Compare $x$ with $a$
	  \end{itemize}
	}
    \end{description}

    \only<5>{
      \vspace{-0.50cm}
      \[
	\red{\forall n \in \mathbb{N}: P(n) \quad \text{\it vs. } \quad P(\infty)}
      \]
    }
  \end{center}
\end{frame}
%%%%%%%%%%%%%%%
