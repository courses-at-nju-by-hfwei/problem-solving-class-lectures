% file: sections/cut-property.tex

%%%%%%%%%%%%
\begin{frame}{}
  \centerline{\teal{\Large Cut Property}}
\end{frame}
%%%%%%%%%%%%%

%%%%%%%%%%%%%
\begin{frame}{}
  \begin{exampleblock}{Cut Property (Version I)}
    \begin{description}
      \item[$X:$] A part of some MST $T$ of $G$
      \item[$(S, V \setminus S):$] A \teal{\it cut} such that $X$ does \teal{\it not} cross $(S, V \setminus S)$
    ­ \item[$e:$] \red{A} lightest edge across $(S, V \setminus S)$
    \end{description}

    \pause
    \vspace{0.30cm}
    \centerline{Then \blue{$X \cup \set{e}$} is a part of \red{some} MST $T'$ of $G$.}
  \end{exampleblock}

  \pause
  \vspace{0.60cm}
  \centerline{\red{\large Correctness of Prim's and Kruskal's algorithms.}}
\end{frame}
%%%%%%%%%%%%%

%%%%%%%%%%%%%
\begin{frame}{}
  \centerline{\teal{\large By Exchange Argument.}}

  \pause
  \vspace{0.30cm}
  \fignocaption{width = 0.50\textwidth}{figs/cut-property}

  \pause
  \[
    T' = \underbrace{T + \set{e}}_{\red{\text{if } e \;\notin\; T}} - \set{e'}
  \]

  \pause
  \centerline{``a'' $\to$ ``the'' \red{$\implies$} ``some'' $\to$ ``all''}
\end{frame}
%%%%%%%%%%%%%

%%%%%%%%%%%%%
\begin{frame}{}
  \begin{exampleblock}{Cut Property (Version II)}
    \begin{center}
      A cut $(S, V \setminus S)$ \\[6pt]
      Let $e = (u,v)$ be \red{\emph{a}} lightest edge across $(S, V \setminus S)$
      \[
	\purple{\red{\exists} \text{ MST $T$ of } G: e \in T}
      \]
    \end{center}
  \end{exampleblock}

  \fignocaption{width = 0.40\textwidth}{figs/cut-property-no-name}

  \pause
  \vspace{-0.50cm}
  \[
    T' = \underbrace{T + \set{e}}_{\red{\text{if } e \;\notin\; T}} - \set{e'}
  \]
  
  \pause
  \centerline{``a'' $\to$ ``the'' \red{$\implies$} ``$\exists$'' $\to$ ``$\forall$''}
\end{frame}
%%%%%%%%%%%%%

%%%%%%%%%%%%%
\begin{frame}<presentation:0>[noframenumbering]
  \begin{exampleblock}{Converse of Cut Property (II)}
    \begin{center}
      $e = (u,v) \in \exists$ MST $T$ of $G$ \\[6pt]
      $\implies$ \\[6pt]
      $e$ is a lightest edge across some cut $(S, V \setminus S)$
    \end{center}
  \end{exampleblock}

  \pause
  \fignocaption{width = 0.40\textwidth}{figs/cut-property-no-name}
  \[
    T' = \underbrace{T - \set{e}}_{\teal{\text{to find } (S, V \setminus S)}} +\; \underbrace{\set{e'}}_{\red{\exists?}}
  \]
\end{frame}
%%%%%%%%%%%%%

%%%%%%%%%%%%%
\begin{frame}{}
  \begin{exampleblock}{Application of Cut Property}
    \centerline{$e = (u,v) \in G$ is a lightest edge $\implies$ $e \in \exists$ MST of $G$}
  \end{exampleblock}

  \uncover<2->{
    \[
      \Big(S = \set{u}, V \setminus S\Big)
    \]
  }

  \begin{exampleblock}{Application of Cut Property}
    \centerline{$e = (u,v) \in G$ is the unique lightest edge $\implies$ $e \in \forall$ MST}
  \end{exampleblock}
\end{frame}
%%%%%%%%%%%%%

%%%%%%%%%%%%%
\begin{frame}{}
  \begin{exampleblock}{Wrong Divide\&Conquer Algorithm for MST}
    \[
      (V_{1}, V_{2}): \Big\rvert |V_{1}| - |V_{2}| \Big\rvert \le 1
    \]
    
    \[
      T_{1} + T_{2} + \set{e}: e \text{ is a lightest edge across } (V_{1}, V_{2})
    \]
  \end{exampleblock}

  \pause
  \vspace{0.30cm}
  \fignocaption{width = 0.40\textwidth}{figs/divide-conquer-mst-counterexample.pdf}
\end{frame}
%%%%%%%%%%%%%
