% reduction.tex

%%%%%%%%%%%%%%%%%%%%
\begin{frame}
  \begin{definition}[Polynomial-time Reduction]
	\begin{center}
	  $L_1 \pr L_2$ if $\exists$ \red{poly.} time function $f$ such that
	\end{center}
	\[
	  \forall x: x \in L_1 \iff f(x) \in L_2.
	\]
  \end{definition}

  \pause
  \vspace{0.30cm}
  \[
	\forall L \in \np, \red{L \pr L'} \implies L' \text{ is } \nph
  \]

  \[
	\npc = \np \cap \nph
  \]
\end{frame}
%%%%%%%%%%%%%%%%%%%%

%%%%%%%%%%%%%%%%%%%%
\begin{frame}
  \[
	\unsat = \Big\{ \varphi: \varphi \text{ is unsatisfiable.} \Big\}
  \]

  \begin{center}
	\red{\large $Q:$ Is \unsat{} \nph?}
  \end{center}

  \pause
  \begin{proof}
	\begin{columns}
	  \column{0.50\textwidth}
		\pause
		\[
		  \sat \pr \unsat
		\]

		\pause
		\[
		  x \in \sat \iff x \notin \unsat
		\]
	  \column{0.50\textwidth}
		\pause
		\fig{width = 0.80\textwidth}{figs/wrong}
	\end{columns}
  \end{proof}

  \pause
  \[
	\red{\boxed{\forall x: x \in L_1 \iff f(x) \in L_2}}
  \]
\end{frame}
%%%%%%%%%%%%%%%%%%%%

%%%%%%%%%%%%%%%%%%%%
\begin{frame}
  \begin{exampleblock}{CLRS $34.5$-$6$}
	\[
	  \text{HAM-PATH is } \npc
	\]
  \end{exampleblock}

  \pause
  \[
	\red{\text{HAM-CYCLE} \pr \text{HAM-PATH}}
  \]

  \pause
  \begin{center}
	\resizebox{0.50\textwidth}{!}{% hc2hp.tex

\begin{tikzpicture}[n/.style = {draw, circle, minimum size = 8pt}, 
  every edge/.style = {draw, very thick}, node distance = 0.6cm and 1.0cm]
  \node (v) [n] {$v$};

  \node (u1) [n, left = of v] {};
  \node (u2) [n, above left = of v] {};
  \node (u3) [n, below left = of v] {};

  \path (v) edge (u1)
  			edge (u2)
  			edge (u3);

  \node (g) [draw, dashed, blue, rectangle, rounded corners, fit = (v) (u1) (u2) (u3), label = {[blue] above : $G$}] {};

  \pause
  \node (v') [n, red, below right = 0.6cm and 0.4cm of v] {$v'$};
  \path (v') edge[red, dashed] (u1)
			 edge[red, dashed] (u2)
			 edge[red, dashed] (u3);

  \pause
  \node (s) [n, teal, above right = 0.2cm and 1.0cm of v] {$s$};
  \node (t) [n, teal, above right = 0.1cm and 0.8cm of v'] {$t$};
  \path (s) edge[teal, dashed] (v)
		(t) edge[teal, dashed] (v');
\end{tikzpicture}
}
  \end{center}

  \pause
  \vspace{-0.30cm}
  \[
	G \in \text{HAM-CYCLE} \iff G' \in \text{HAM-PATH}
  \]
\end{frame}
%%%%%%%%%%%%%%%%%%%%

%%%%%%%%%%%%%%%%%%%%
\begin{frame}
  \[
	\red{\text{HAM-CYCLE} \pr \text{HAM-PATH}}
  \]

  \only<2-4>{
  \[
	\blue{\forall e \in G: \text{Construct } G_e}
  \]
  \begin{center}
	\resizebox{0.40\textwidth}{!}{% hc2hp-2.tex

\begin{tikzpicture}[n/.style = {draw, circle, minimum size = 8pt}, 
  every edge/.style = {draw, very thick}]
  \node (u) [n] {$u$};
  \node (v) [n, below right = 1.0cm and 0.2cm of u] {$v$};

  \path (u) edge node [midway, above = 3pt] {$e$} (v);

  \node (g) [draw, dashed, blue, circle, minimum size = 20pt, fit = (u) (v), label = {[blue] above : $G$}] {};

  \pause
  \node (s) [n, teal, right = 1.3cm of u] {$s$};
  \node (t) [n, teal, right = of v] {$t$};
  \path (s) edge[teal, dashed] (u)
		(t) edge[teal, dashed] (v);
\end{tikzpicture}
}
  \end{center}}
  \only<5->{\fig{width = 0.60\textwidth}{figs/wrong}}

  \uncover<3->{
	\vspace{-0.30cm}
	\[
	  G \text{ has a HC containing } e \iff G_e \text{ has a HP}
	\]
  }

  \uncover<4->{
	\vspace{-0.60cm}
	\[
	  \red{G \in \text{HAM-CYCLE} \iff \exists G_e: G_e \text{ has a HP}}
	\]
  }
\end{frame}
%%%%%%%%%%%%%%%%%%%%

%%%%%%%%%%%%%%%%%%%%
\begin{frame}
  \begin{definition}[Polynomial-time Reduction]
	\begin{center}
	  $L_1 \pr L_2$ if $\exists$ \red{poly.} time function $f$ such that
	\end{center}
	\[
	  \forall x: x \in L_1 \iff f(x) \in L_2.
	\]
  \end{definition}

  \begin{columns}
	\column{0.60\textwidth}
	  \begin{center}
		\pause
		\[
		  x \text{ for } L_1 \mapsto x' = f(x) \text{ for L2}
		\]
		\pause
		\[
		  \text{Call the oracle } O_2 \text{ for } L_2 \text{ once}
		\]
		\pause
		\[
		  \text{Answer whatever } O_2 \text{ returns}
		\]
	  \end{center}
	\column{0.40\textwidth}
	  \pause
	  \fig{width = 0.70\textwidth}{figs/that-is-all}
  \end{columns}
\end{frame}
%%%%%%%%%%%%%%%%%%%%

%%%%%%%%%%%%%%%%%%%%
\begin{frame}
  \begin{definition}[Polynomial-time Reduction]
	\begin{center}
	  $L_1 \pr L_2$ if $\exists$ \red{poly.} time function $f$ such that
	\end{center}
	\[
	  \forall x: x \in L_1 \iff f(x) \in L_2.
	\]
  \end{definition}

  \pause
  \begin{center}
	\red{\large Karp Reduction}
  \end{center}

  \pause
  \begin{columns}
	\column{0.50\textwidth}
	  \fig{width = 0.60\textwidth}{figs/karp}
	  \begin{center}
		\href{https://en.wikipedia.org/wiki/Richard\_M.\_Karp}{Richard M. Karp ($1935 \sim$)}
	  \end{center}
	\column{0.50\textwidth}
	  \fig{width = 1.00\textwidth}{figs/karp-paper}
	  \begin{center}
		\href{https://people.eecs.berkeley.edu/~luca/cs172/karp.pdf}{$(1972)$}
	  \end{center}
  \end{columns}
\end{frame}
%%%%%%%%%%%%%%%%%%%%

%%%%%%%%%%%%%%%%%%%%
\begin{frame}
  \begin{center}
	\red{\large Cook Reduction}
  \end{center}

  \begin{columns}
	\column{0.50\textwidth}
	  \fig{width = 0.60\textwidth}{figs/cook}
	  \begin{center}
		\href{https://en.wikipedia.org/wiki/Stephen\_Cook}{Stephen Cook ($1939 \sim$)}
	  \end{center}
	\column{0.50\textwidth}
	  \fig{width = 1.00\textwidth}{figs/cook-paper}
	  \begin{center}
		\href{http://www.cs.toronto.edu/~sacook/homepage/1971.pdf}{$(1971)$}
	  \end{center}
  \end{columns}
\end{frame}
%%%%%%%%%%%%%%%%%%%%

%%%%%%%%%%%%%%%%%%%%
% \begin{frame}
%   \[
% 	\text{HAM-PATH} \pr \text{HAM-CYCLE}
%   \]
% 
%   \centerline{$\pr$: add $v'; (v', v), \forall v \in V $}
% \end{frame}
%%%%%%%%%%%%%%%%%%%%
