% file: sections/splay-tree.tex

%%%%%%%%%%%%%%%
\begin{frame}{}
  \begin{quote}
    \centering
    \purple{\Large What work are you proudest of?} \\[12pt]

    \fignocaption{width = 0.25\textwidth}{figs/qrcode-tarjan-interview}

    \pause
    \vspace{0.60cm}
    \brown{\large Proudest? It’s hard to choose. \\[5pt] \pause
      I like the \red{\Large self-adjusting search tree} data structure \\[4pt]
      that Danny Sleator and I developed. \\[5pt]
      % That’s a nice one.
    }
  \end{quote}
\end{frame}
%%%%%%%%%%%%%%%

%%%%%%%%%%%%%%%
\begin{frame}{}
  \fignocaption{width = 0.75\textwidth}{figs/self-adjusting-bst-paper}

  \begin{quote}
    \centering
    \teal{``Self-Adjusting Binary Search Trees -- \red{\it Splay Tree}'', JACM, $1985$}
  \end{quote}
\end{frame}
%%%%%%%%%%%%%%%

%%%%%%%%%%%%%%%
\begin{frame}{}
  \centerline{\large \teal{Self-Adjusting Binary Search Trees}}

  \fignocaption{width = 0.35\textwidth}{figs/self-adjusting-tree}
\end{frame}
%%%%%%%%%%%%%%%

%%%%%%%%%%%%%%%
\begin{frame}{}
  \begin{center}
    $\splay(x)$:  \\[8pt]
    Moving node $x$ to the root of the tree by performing a sequence of \red{rotations}
    along the path from $x$ to the root.
  \end{center}

  \pause
  \fignocaption{width = 0.75\textwidth}{figs/rotation}
\end{frame}
%%%%%%%%%%%%%%%

%%%%%%%%%%%%%%%
\begin{frame}{}
  \fignocaption{width = 0.40\textwidth}{figs/splay-tree-case-zero}

  \vspace{0.30cm}
  \centerline{\textsc{Case $0$}: $x$ is the root}
\end{frame}
%%%%%%%%%%%%%%%

%%%%%%%%%%%%%%%
\begin{frame}{}
  \fignocaption{width = 0.80\textwidth}{figs/splay-tree-case-zig}

  \vspace{0.30cm}
  \centerline{\textsc{Case $1$}: zig}
  \[
    y = p(x) \text{ is the root}
  \]
\end{frame}
%%%%%%%%%%%%%%%

%%%%%%%%%%%%%%%
\begin{frame}{}
  \fignocaption{width = 0.85\textwidth}{figs/splay-tree-case-zigzig}

  \vspace{0.30cm}
  \centerline{\textsc{Case $2$}: zig-zig}
  \[
    y = p(x) \qquad z = p(y)
  \]
  \[
    x = lc(y) \qquad y = lc(z)
  \]

  \pause
  \[
    \red{(1): y - z \qquad (2): x - y}
  \]
\end{frame}
%%%%%%%%%%%%%%%

%%%%%%%%%%%%%%%
\begin{frame}{}
  \fignocaption{width = 0.80\textwidth}{figs/splay-tree-case-zigzag}

  \vspace{0.30cm}
  \centerline{\textsc{Case $3$}: zig-zag}
  \[
    y = p(x) \qquad z = p(y)
  \]
  \[
    x = rc(y) \qquad y = lc(z)
  \]

  \pause
  \[
    \red{(1): x - y \qquad (2): x - z}
  \]
\end{frame}
%%%%%%%%%%%%%%%

%%%%%%%%%%%%%%%
\begin{frame}{}
  \fignocaption{width = 0.95\textwidth}{figs/splay-at-1}{\centerline{$\textsc{Splay}(1)$}}

  \pause
  \fignocaption{width = 0.50\textwidth}{figs/splay-at-2}{\centerline{$\textsc{Splay}(2)$}}
\end{frame}
%%%%%%%%%%%%%%%

%%%%%%%%%%%%%%%
\begin{frame}{}
  \begin{center}
    \teal{\Large Amortized analysis of \textsc{Splay}} \\[30pt] \pause

    A splay tree $T$ of \red{$n$}-node \\[6pt]
    An arbitrary sequence of \red{$m$} \textsc{Splay} operations
  \end{center}

  \pause
  \[
    \# \text{ of rotations}
  \]

  \pause
  \vspace{0.50cm}
  \begin{theorem}
    \[
      \hat{c}_{\splay} = O(\log n).
    \]
  \end{theorem}
\end{frame}
%%%%%%%%%%%%%%%

%%%%%%%%%%%%%%%
\begin{frame}{}
  \[
    \Phi_{0}\;\; \splay_{1} \;\;
    \Phi_{1}\;\; \splay_{2} \;\;
    \Phi_{2}\;
    \cdots\;
    \underbrace{\Phi_{i-1}\;\; \splay_{i}\;\; \Phi_{i}\;}_{\purple{\text{the $i$-th \textsc{Splay}}}}
    \cdots\;
    \splay_{m}\;\; 
    \Phi_{m}
  \]

  \[
    \hat{c}_{\splay_{i}} =  c_{\splay_{i}} + (\Phi_{\splay_{i}} - \Phi_{\splay_{i-1}})
  \]

  \pause
  \vspace{0.60cm}
  \begin{center}
    \red{\Large \it How to define $\Phi$?}
  \end{center}
\end{frame}
%%%%%%%%%%%%%%%

%%%%%%%%%%%%%%%
\begin{frame}{}
  \[
    s(x): \# \text{ of nodes in the subtree rooted at } x
  \]

  \begin{columns}
    \column{0.50\textwidth}
      \pause
      \[
	r(x) = \log s(x)
      \]

      \pause
      \[
	\purple{\Phi = \sum_{x \in T} r(x)}
      \]
    \column{0.50\textwidth}
      \pause
      \fignocaption{width = 0.50\textwidth}{figs/magic}
  \end{columns}

  \pause
  \vspace{0.50cm}
  \[
    \hat{c}_{\splay_{i}} =  c_{\splay_{i}} + (\Phi_{\splay_{i}} - \Phi_{\splay_{i-1}})
  \]

  \begin{center}
    \red{\large \it How to calculate $(\Phi_{\splay_{i}} - \Phi_{\splay_{i-1}})$ and $c_{\splay_{i}}$?}
  \end{center}
\end{frame}
%%%%%%%%%%%%%%%

%%%%%%%%%%%%%%%
\begin{frame}{}
  \[
    \Phi_{0}\;\; \splay_{1} \;\;
    \Phi_{1}\;\; \splay_{2} \;\;
    \Phi_{2}\;
    \cdots\;
    \underbrace{\Phi_{i-1}\;\; \splay_{i}\;\; \Phi_{i}\;}_{\purple{\text{the $i$-th \textsc{Splay}}}}
    \cdots\;
    \splay_{m}\;\; 
    \Phi_{m}
  \]

  \pause
  \begin{gather*}
    \underbrace{\Phi_{i-1}\;\; \splay_{i}\;\; \Phi_{i}}_{\purple{\text{the $i$-th \textsc{Splay}}}}: \\
    \Phi_{i-1} \;\triangleq\; \Phi_{0'}\; \iter_{1}\; \Phi_{1'}\; 
    \cdots\;
    \underbrace{\Phi_{k-1}\; \iter_{k}\; \Phi_{k}\;}_{\purple{\text{the $k$-th \textsc{Iteration}}}}
    \cdots\;
    \iter_{l}\; \Phi_{l} \;\triangleq\; \Phi_{i}\;
  \end{gather*}

  \pause
  \begin{align*}
    \hat{c}_{\splay_{i}} &= \sum_{1 \le j \le l} \hat{c}_{\iter_{j}} \\
      &= \sum_{1 \le j \le l} c_{\iter_{j}} + (\Phi_{\iter_{j}} - \Phi_{\iter_{j-1}})
  \end{align*}
\end{frame}
%%%%%%%%%%%%%%%

%%%%%%%%%%%%%%%
\begin{frame}{}
  \[
    \hat{c}_{\iter_{j}} =  c_{\iter_{j}} + (\Phi_{\iter_{j}} - \Phi_{\iter_{j-1}})
  \]

  \pause
  \vspace{0.60cm}
  \center{\large \it \red{By Case Analysis.}}

  \pause
  \[
    \hat{c}_{j} =  c_{j} + (\Phi_{j} - \Phi_{j-1})
  \]

  \vspace{0.30cm}
  \centerline{\blue{Remember:} \iter{}}
\end{frame}
%%%%%%%%%%%%%%%

%%%%%%%%%%%%%%%
\begin{frame}{}
  \fignocaption{width = 0.40\textwidth}{figs/splay-tree-case-zero}

  \centerline{\textsc{Case $0$}}

  \pause
  \begin{align*}
    \hat{c}_{j} &=  c_{j} + (\Phi_{j} - \Phi_{j-1}) \\
	\onslide<3->{
	  &= 0 + 0 \\
	  &= 0
	}
  \end{align*}
\end{frame}
%%%%%%%%%%%%%%%

%%%%%%%%%%%%%%%
\begin{frame}{}
  \fignocaption{width = 0.80\textwidth}{figs/splay-tree-case-zig}

  \centerline{\textsc{Case $1$}: zig}

  \begin{align*}
    \hat{c}_{j} &=  c_{j} + (\Phi_{j} - \Phi_{j-1}) \\
      \onslide<2->{
	&= 1 + r_{j}(x) + r_{j}(y) - r_{j-1}(x) - r_{j-1}(y) \\
      }
      \onslide<3->{
	&\le 1 + r_{j}(x) - r_{j-1}(x) \\
      }
      \onslide<4->{
	&\red{\;\le\;} 1 + 3\big(r_{j}(x) - r_{j-1}(x)\big)
      }
  \end{align*}
\end{frame}
%%%%%%%%%%%%%%%

%%%%%%%%%%%%%%%
\begin{frame}{}
  \fignocaption{width = 0.80\textwidth}{figs/splay-tree-case-zigzig}

  \centerline{\textsc{Case $2$}: zig-zig}

  \begin{align*}
    \hat{c}_{j} &=  c_{j} + (\Phi_{j} - \Phi_{j-1}) \\
      \onslide<2->{
	&= 2 + r_{j}(x) + r_{j}(y) + r_{j}(y)- r_{j-1}(x) - r_{j-1}(y) - r_{j-1}(z) \\
      }
      \onslide<3->{
	&= 2 + r_{j}(y) + r_{j}(y)- r_{j-1}(x) - r_{j-1}(y) \\
      }
      \onslide<4->{
	&\le 2 + r_{j}(x) + r_{j}(z) - 2r_{j-1}(x) \\
      }
      \onslide<5->{
	&\red{\;\le\;} 3\big(r_{j}(x) - r_{j-1}(x)\big)
      }
  \end{align*}
\end{frame}
%%%%%%%%%%%%%%%

%%%%%%%%%%%%%%%
\begin{frame}{}
  \fignocaption{width = 0.80\textwidth}{figs/splay-tree-case-zigzag}

  \centerline{\textsc{Case $3$}: zig-zag}

  \begin{align*}
    \hat{c}_{j} &=  c_{j} + (\Phi_{j} - \Phi_{j-1}) \\
      \onslide<2->{
	&= 2 + r_{j}(x) + r_{j}(y) + r_{j}(y)- r_{j-1}(x) - r_{j-1}(y) - r_{j-1}(z) \\
      }
      \onslide<3->{
	&\le 2 + r_{j}(y) + r_{j}(z) - 2r_{j-1}(x) \\
      }
      \onslide<4->{
	&\red{\;\le\;} 3\big(r_{j}(x) - r_{j-1}(x)\big)
      }
  \end{align*}
\end{frame}
%%%%%%%%%%%%%%%

%%%%%%%%%%%%%%%
\begin{frame}{}
  \[
    \hat{c}_{\iter_{j}} \le \left.
    \begin{cases}
      0, & \text{\textsc{Case} $0$} \\
      1 + 3\big(r_{j}(x) - r_{j-1}(x)\big), & \text{\textsc{Case} $1$} \\
      3\big(r_{j}(x) - r_{j-1}(x)\big), & \text{\textsc{Case} $2$} \\
      3\big(r_{j}(x) - r_{j-1}(x)\big), & \text{\textsc{Case} $3$} \\
    \end{cases} \right.
  \]

  \pause
  \begin{align*}
    \hat{c}_{\splay_{i}} &= \sum_{1 \le j \le l} \hat{c}_{\iter_{j}} \\
      &= \sum_{1 \le j \le l} c_{\iter_{j}} + (\Phi_{\iter_{j}} - \Phi_{\iter_{j-1}}) \\
    \onslide<3->{
      &\le 3\big(r_{\iter_{l}}(x) - r_{\iter_{0}}(x)\big) \red{\; + 1} \\
    }
    \onslide<4->{
      &= 3\big(\log n - r_{\iter_{0}}(x)\big) + 1 \\
    }
    \onslide<5->{
      &\le 3 \log n + 1 \\
      &= O(\log n)
    }
  \end{align*}
\end{frame}
%%%%%%%%%%%%%%%

%%%%%%%%%%%%%%%
\begin{frame}{}
  \begin{theorem}[\textsc{Balance Theorem}]
    \[
      \sum_{1 \le i \le m} c_{\splay_{i}} = O\Big((m + n) \log n\Big)
    \]
  \end{theorem}

  \pause
  \vspace{0.50cm}
  \begin{proof}
    \begin{align*}
      \sum_{1 \le i \le m} c_{\splay_{i}} &= \sum_{1 \le i \le m} \hat{c}_{\splay_{i}} + \Big(\Phi_{\splay_{0}} - \Phi_{\splay_{m}}\Big) \\
      \onslide<3->{
        &\le m \log n + \;\red{n \log n} \\
        &= (m + n) \log n
      }
    \end{align*}
  \end{proof}
\end{frame}
%%%%%%%%%%%%%%%

%%%%%%%%%%%%%%%
\begin{frame}{}
  \fignocaption{width = 0.80\textwidth}{figs/splay-tree-case-zigzig}

  \pause
  \vspace{0.30cm}
  \begin{center}
    \teal{MTR (Move To Root) heuristic:} \\[6pt]
    Keeping rotate the edge joining $x$ to its parent.
  \end{center}

  \pause
  \vspace{0.30cm}
  \centerline{\large \red{Does this work?}}
\end{frame}
%%%%%%%%%%%%%%%

%%%%%%%%%%%%%%%
\begin{frame}{}
  \begin{columns}
    \column{0.50\textwidth}
      \[
	\purple{\Phi = \sum_{x \in T} r(x)}
      \]

      \fignocaption{width = 0.80\textwidth}{figs/idea}
    \column{0.50\textwidth}
      \pause
      \fignocaption{width = 0.60\textwidth}{figs/qrcode-cstheory-splay-tree-potential}
  \end{columns}
\end{frame}
%%%%%%%%%%%%%%%

%%%%%%%%%%%%%%%
\begin{frame}{}
  \begin{columns}
    \column{0.40\textwidth}
      \begin{center}
	\[
	  \red{\textsc{Splay}(x)}
	\]
	\pause
	\[
	  \textsc{Search}(x, t)
	\]
	\[
	  \textsc{Insert}(x, t)
	\]
	\[
	  \textsc{Delete}(x, t)
	\]
	\[
	  \textsc{Join}(t_1, t_2)
	\]
	\[
	  \textsc{Split}(x, t)
	\]
      \end{center}
    \column{0.60\textwidth}
      \pause
      \fignocaption{width = 0.80\textwidth}{figs/iceberg}
      \pause
      \fignocaption{width = 0.85\textwidth}{figs/self-adjusting-bst-paper}
  \end{columns}
\end{frame}
%%%%%%%%%%%%%%%
