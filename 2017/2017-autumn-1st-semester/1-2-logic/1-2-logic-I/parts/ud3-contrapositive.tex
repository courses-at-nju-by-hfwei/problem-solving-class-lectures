%%%%%%%%%%%%%%%
\begin{frame}{}
  \centerline{\LARGE 命题逻辑部分习题选讲}
  \vspace{0.50cm}
  \centerline{\large 第三章 \; 逆否命题与逆命题}
\end{frame}
%%%%%%%%%%%%%%%

%%%%%%%%%%%%%%%
\begin{frame}{}
  \begin{exampleblock}{题目 3.7:四类命题}
    \[
      p \to q
    \]
    \begin{enumerate}
      \item 逆否命题 (contrapositive)
	\[
	  \lnot q \to \lnot p
	\]
      \item 逆命题 (converse)
	\[
	  q \to p
	\]
      \item 否命题 (inverse)
	\[
	  \lnot p \to \lnot q
	\]
      \item 命题的否定 (negated)
	\[
	  p \land \lnot q
	\]
    \end{enumerate}
  \end{exampleblock}
\end{frame}
%%%%%%%%%%%%%%%

%%%%%%%%%%%%%%%
\begin{frame}{}
  \begin{exampleblock}{题目 3.6:Breakfast}
    Matilda always eats at least one of the following for breakfast: \\
    \begin{enumerate}
      \item cereal, bread, or yogurt. 
    \end{enumerate}
    On Monday, she is especially picky.

    \begin{enumerate}
      \setcounter{enumi}{1}
      \item If she eats cereal and bread, she also eats yogurt.
      \item If she eats bread or yogurt, she also eats cereal.
      \item She never eats both cereal and yogurt.
      \item She always eats bread or cereal.
    \end{enumerate}

    Can you say what Matilda eats on Monday? If so, what does she eat?
  \end{exampleblock}
\end{frame}
%%%%%%%%%%%%%%%

%%%%%%%%%%%%%%%
\begin{frame}{}
  引入命题符号:你觉得这有什么问题吗?
  \begin{columns}
    \column{0.50\textwidth}
      \begin{description}
	\item[$A:$] Cereal
	\item[$B:$] Bread
	\item[$C:$] Yogurt
      \end{description}
    \column{0.50\textwidth}
      \begin{description}
	\item[$P:$] Cereal
	\item[$Q:$] Bread
	\item[$R:$] Yogurt
      \end{description}
  \end{columns}

  \vspace{0.30cm}
  \pause
  \fignocaption{width = 0.50\textwidth}{figs/text-color-puzzle.jpg}
\end{frame}
%%%%%%%%%%%%%%%

%%%%%%%%%%%%%%%
\begin{frame}{}
  这是一个有效的推理, 但是你觉得它有什么问题吗?
  
  \fignocaption{width = 0.70\textwidth}{figs/breakfast-reasoning}
\end{frame}
%%%%%%%%%%%%%%%

%%%%%%%%%%%%%%%
\begin{frame}{}
  \begin{quote}
    \centerline{\red{\LARGE Let us calculate [calculemus].}}
  \end{quote}
\end{frame}
%%%%%%%%%%%%%%%

%%%%%%%%%%%%%%%
\begin{frame}{}
  \begin{exampleblock}{题目 3.9:巧克力蛋糕配方}
    \begin{enumerate}
      \item \textcolor<3->{red}{Exactly three} use $\cdots$
      \item $A$ \& $G$ use the \textcolor<3->{red}{same} amount of $f$.
      \item $A$ \& $G$ use \textcolor<3->{red}{different} kind of $c$.
    \end{enumerate}
  \end{exampleblock}

  \begin{table}
    \renewcommand{\arraystretch}{1.5}
    \begin{tabular}{|c||c|c|c|c|}
      \hline
			      & French (F) & Swiss (S) & German (G) & American (A) \\ \hline \hline
      Semisweet choco. (c) 	& \cmark & \cmark & \xmark & \cmark \\ \hline
      Very little flour (f) 	& \cmark & \xmark & \cmark & \cmark \\ \hline
      $<\frac{1}{4}$ cup sugar (s) 	& \cmark & \cmark & \cmark & \xmark \\ \hline
    \end{tabular}
  \end{table}

  \pause
  \vspace{0.30cm}
  \centerline{\large 你没有勇气来\red{``算一算''}? 如何选取命题符号? 如何形式化上述条件?}
\end{frame}
%%%%%%%%%%%%%%%

%%%%%%%%%%%%%%%
\begin{frame}{}
  \begin{exampleblock}{题目 3.10:利用逆否命题作证明}
    Let $n$ be an integer. Prove that if $3n$ is odd, then $n$ is odd.
  \end{exampleblock}

  \vspace{0.60cm}

  \begin{exampleblock}{题目 3.11:利用逆否命题作证明}
    Prove that if $x$ is odd, then $\sqrt{2x}$ is not an integer.
  \end{exampleblock}

  \pause
  \[
    \sqrt{2x} = k \implies 2x = k^2 \implies k \text{ is even} \implies x \text{ is even} 
  \]
\end{frame}
%%%%%%%%%%%%%%%
