% decide.tex

%%%%%%%%%%%%%%%%%%%%
\begin{frame}
  \begin{columns}
	\column{0.50\textwidth}
	  \fig{width = 0.80\textwidth}{figs/decide}
	\column{0.50\textwidth}
	  \fig{width = 0.65\textwidth}{figs/accept}
  \end{columns}

  \begin{columns}
	\column{0.50\textwidth}
	  \pause
	  \fig{width = 0.75\textwidth}{figs/decide-yes-no}

	  \pause
	  \begin{center}
		\red{\large Always terminate.}
	  \end{center}
	\column{0.50\textwidth}
	  \pause
	  \fig{width = 0.70\textwidth}{figs/loop}
	  \begin{center}
		\red{\large May loop forever for ``NO'' instance.}
	  \end{center}
  \end{columns}
\end{frame}
%%%%%%%%%%%%%%%%%%%%

%%%%%%%%%%%%%%%%%%%%
\begin{frame}
  \begin{definition}[Halting Problem]
	\begin{description}
	  \item[Input:] An arbitrary program and input
	  \item[Output:] Will the program eventually halt?
	\end{description}
  \end{definition}

  \begin{columns}
	\column{0.50\textwidth}
	  \fig{width = 0.50\textwidth}{figs/turing-16}
	\column{0.50\textwidth}
	  \fig{width = 0.80\textwidth}{figs/halting-problem}
  \end{columns}

  \pause
  \begin{center}
	\red{\large Undecidable} \pause \\[4pt] 
    \blue{\large But Acceptable (Semi-decidable)}
  \end{center}
\end{frame}
%%%%%%%%%%%%%%%%%%%%

%%%%%%%%%%%%%%%%%%%%
\begin{frame}
  \[
	\p = \Big\{ L: L \text{ is \red{\large decided} by a poly. time algorithm} \Big\}
  \]

  \pause
  \vspace{0.30cm}
  \begin{theorem}[Theorem $34.2$]
	\[
	  \p = \Big\{ L: L \text{ is \red{\large accepted} by a poly. time algorithm} \Big\}
	\]
  \end{theorem}

  \pause
  \vspace{0.60cm}
  \begin{quote}
	\centering
	\purple{\large You can safely forget ``semi-decidable'' \\ in computational complexity theory.}
  \end{quote}
\end{frame}
%%%%%%%%%%%%%%%%%%%%
